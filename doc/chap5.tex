%\chapter{LEAST SQUARES REFINEMENT USING CCSL}
%\begin{htmlonly}
%\documentclass[a4paper]{CCSLman}
%\usepackage{html,makeidx,color}
%%\chapter{LEAST SQUARES REFINEMENT USING CCSL}
%\begin{htmlonly}
%\documentclass[a4paper]{CCSLman}
%\usepackage{html,makeidx,color}
%%\chapter{LEAST SQUARES REFINEMENT USING CCSL}
%\begin{htmlonly}
%\documentclass[a4paper]{CCSLman}
%\usepackage{html,makeidx,color}
%%\chapter{LEAST SQUARES REFINEMENT USING CCSL}
%\begin{htmlonly}
%\documentclass[a4paper]{CCSLman}
%\usepackage{html,makeidx,color}
%\input{chap5.ptr}
%\end{htmlonly}
%\internal{c2}
%\internal{c3}
%\internal{c4}
%\internal{c1}
%\internal{c6}
%\internal{c7}
\startdocument
%\htmladdtonavigation{\htmladdnormallink
%  {\htmladdimg{../icons/appenx.gif}}
%  {../appenx/appendix}}
\label{chap:5}
\markboth{Least Squares Refinement}{}
\section{What We Mean by LSQ}\markright{What We Mean by LSQ}
% 
The term LSQ is used here to mean \ital{Least Squares Refinement}, 
which uses
observations of a function to improve the values of some set of
parameters on which that function depends.
\p 
Refinement involving a crystal structure could be either \ital{standard} single 
crystal LSQ
refinement, in which each observation depends on a single
structure factor, or \ital{profile refinement} (PR) for which
several structure factors contribute to one observation.
\p 
The PR routines are written to use many of the same CCSL routines as do 
standard LSQ
main programs.  The PR programs and Libraries are available separately, with 
a separate Manual.
\p 
CCSL also deals with simpler LSQ problems.  For example, the main program
FWLSQ fits the 5,7 or 9 parameters of the exponential approximation to a 
scattering factor curve. The simplicity in this case is that no crystal structure is
involved.
Another simple case is the refinement of (up to) 6 cell parameters,
given an observed list of d-spacings (actually d* squared values, in
the main program DSLSQ).  This must deal with
the constraints which are
necessarily imposed on the cell parameters by the space group symmetry.
\p 
The essentials of a LSQ problem are:\p
\begin{list} {} {\setlength{\labelwidth}{8mm}
  \setlength{\parsep}{-1ex}
  \setlength{\leftmargin}{\labelwidth}
 \addtolength{\leftmargin}{4mm}}
\item[a) \hfill] a set of \ital{observations} of something, 
(the \ital{observed function})
usually with their \ital{estimated standard deviations,} $\sigma$'s,
measured at different values of some argument \ital{ARG};
\ital{ARG} may be $h,k,l$ (for standard LSQ), $\sin\theta/\lambda$, 
$\theta$ (for
Rietveld PR) or $\lambda$, or energy, or time of flight; it is a
quantity which takes different values, and will identify the
observation.\\[1mm]
For crystallographic applications the observations are often all of the
same physical thing, but this is not necessary.  For example, geometric
constraints may be introduced by giving bond lengths and/or angles as
additional types of observation, with $\sigma$'s.
\item[b) \hfill]  some \ital{calculated function} involving \ital{ARG},
which defines a
mathematical model to be compared with an observation.  This calculated
function depends on \ital{parameters}, things which may possibly be varied in order
to improve the fit of the function to its related observations.\\[1mm]
By \ital{fit} we mean minimisation of the weighted sum of squares of 
differences
between observed and calculated function values (where the summing is
over the different values of \ital{ARG}).  It is desirable to use 
statistical weights ($1/\sigma^2$)
for the differences.\\[1mm]
The theory may be viewed as using the beginning of a Taylor series
expansion, and therefore requires that the parameters be close to
their \ital{correct} values, making the required shifts so small that their
squares may be neglected.\end{list}
 
\section{Parameters and Variables}\markright{Parameters and Variables}
% 
\subsection{Parameters}
In LSQ a \ital{parameter} is something which it is sensible to try to
adjust. It represents some physical thing like ``the x coordinate of the 3rd
atom'', ``an overall temperature factor'' or ``the scale factor for all
observations coded number 2''.
\p 
A \ital{parameter} is also something with respect to which the function may be
differentiated.  Differentiation may be analytic, if it is possible (and
sensible) to write down such derivatives algebraically.  It may also be
numerical in awkward cases, using a simple approximation to the
derivative involving function values.
\p 
For any particular run of a LSQ refinement program, it is unlikely that
the user will want to vary every parameter.  The subset of parameters
which are actually to be varied (i.e. those for which shifts are
required) we call \ital{variables}.
\p 
Parameters are thus either \ital{fixed} or \ital{varied}.
For a fixed parameter,
there is no need for a derivative; but for all parameters to be
varied, derivatives will be needed.
% 
\subsection{Basic Variables}
% 
It is often the case that parameters are related, either inherently by
the symmetry of the problem, or by a strict constraint applied by the user.  If
two variables are related, they must not both be refined by LSQ;  the
process expects the shifts to be independent.  Every constraint or
relation reduces by 1 the number of variables to be refined.
\p 
The subset of variables for which shifts are actually calculated
we call \ital{basic variables}, or basics for short.
\p 
Those variables which are not basic we call \ital{redundant}. 
Thus if a problem has $n_p$ parameters, $n_f$ of 
which are fixed and $n_v$ varied,
               $$n_p = n_f + n_v$$
and if, of those $n_v$, $n_b$ are basic and $n_r$ are redundant,
              $$n_v = n_b + n_r$$
% 
\subsection{Strict Constraints}
% 
The existence of $n_r$ redundant variables implies that there are $n_r$
strict constraints on the problem. Derivatives are calculated 
(and shifts are required) for all
variables, not just the basics.
\p
Those constraints which must be imposed because of the crystallographic
symmetry are generated automatically by CCSL, and need not be provided by
the user.  For example, the  special position $(x,x,x)$ will give rise to
one basic variable, $x$, with two redundant variables $y$ and $z$ related to
it by two constraints, but the user will not need to tell the system
this.\p
The user may impose additional strict constraints.  Two constraint types
are at present available.  The simpler is type 1:
             $$a_1\Delta p_1 = a_2\Delta p_2$$
that is, constant $a_1$ times the shift in parameter 1 = constant $a_2$ times
the shift in parameter 2.\p
Type 2 constraints involve a linear combination of parameters, with
constant coefficients, thus:
 $$  a_1\Delta p_1 + a_2\Delta p_2 + a_3\Delta p3 + \ldots\ \mbox{etc}\
   = 0$$
Note that the way these expressions are written involves differentiating
the original constraint.  For example, if a type 2 constraint is
needed to fix the sum of three parameters to be 3, the constraint
$$p_1 + p_2 + p_3 = 3\quad\mbox{becomes}\quad \Delta p_1 + 
\Delta p_2 + \Delta p_3 = 0$$
Other more complicated types of constraint could be added if needed.
%
\subsection{Parameter Names}
% 
The need to fix, vary or relate parameters implies the need to call
these parameters by name.  The CCSL LSQ facilities allow each problem to
be set up individually.  Names for the parameters of a problem are
given in its main program;  the user needs to know what these names are
in order to refer to the parameters in his Crystal Data.
\p
Examples of names used in various standard main programs are:
\ssp
\verb| Na2 X  Ca B11  SCAL 2  P23 ITF   A*|
\ssp 
Most names have two parts, which we call the genus and species names;  the
genus names above are \exac{\ Na2, Ca, SCAL, P23}, with corresponding
species names \exac{\ X, B11, 2} and \exac{ITF}.  The name \exac{A*} 
is simply a species name.
\p 
In LSQ programs which involve structure factors,
those structure parameters which belong to a particular named atom
have the atom name as genus name, and one of:
\ssp 
\verb|X   Y   Z   B11   B12   B13   B22   B23   B33   ITF  SITE  SCAT|
\ssp 
as species name.  
(\exac{SCAT} means the scattering factor, when this is represented 
by one number and it is sensible to refine it).
\p
MAGLSQ has the additional species names:
\ssp
\exac{MU   MU1   THET   THE1   PHI   PHI1   PSI1   PSI2   PSI3   PSI4}.
\p
When LSQ programs refer to parameters in output for shifts, correlations etc,
they use these parameter names.
\p 
More details about parameter
names are to be found in Chapter 3 under \cds{\htmlref{L FIX}{ss:fix}}.  
Note the availability
of words like \bd{XYZ} to mean ``all three x,y,z coordinates", 
\bd{CELL} to mean ``all six
cell parameters", etc.  One may also use the word \bd{ALL} 
to mean ``all the members
of this genus" (as in \exac{ALL Na3} for ``all the parameters of atom Na3")
 or ``all
the genera with this species" (as in \exac{ALL X}, or \exac{ALL SITE}, or even 
\exac{ALL XYZ}).
%
\subsection{Families of Parameters}
% 
Structure parameters such as those introduced in the previous section,
being a commonly occurring group in LSQ, are said in CCSL to belong to 
\ital{family
2} of the parameters.  \ital{Family~1} contains all other
parameters in the simple LSQ applications we are considering here.  Other
families are used in PR, and in other LSQ programs.
\p 
Family 1, genus 1 is treated by CCSL as special.  It
contains the parameters for which the species name by itself is enough,
like \exac{A*} above;  another example is an overall temperature factor for a
structure, known as \exac{TFAC}.
\p 
Genera 2, 3 etc of family 1 may in general be given any genus name.  Within
such a genus the species name could be simply an integer, or species could have 
individual names.  This choice is
for the writer of the main program.  In \mlink{sflsq}{SFLSQ}\ there are two genera in
family 1, the second being named \exac{SCAL} with species numbered 1,2 etc.
\p 
Unless the user is writing or modifying a main program, he need not be
concerned with the detailed mechanism for numbering parameters.  He only
needs to know what names he is allowed to use on his Crystal Data
for the programs he runs.
\p 
\section{Examples of Least squares programs}
\markright{Examples of Least squares programs}
%
We illustrate these general ideas by, first, a simple example of LSQ
which does not involve structure parameters, and, second, notes on the
use of the standard single crystal structure refinement program, SFLSQ.
\p
There are two fundamental questions to be asked of a main LSQ program:
\begin{enumerate}
\item on which calculated function is it refining 
\item which parameters may be refined.
\end{enumerate}
%
\subsection{DSLSQ}
% 
DSLSQ is a main program which determines the values of up to 6 cell
parameters by LSQ fitting, using values of d spacings observed at 
different values of h,k,l.

The calculated function used is the expression for the square of d*, 
in reciprocal space:
$$d^*(h,k,l)^2 = (h^2a^{*2} + k^2b^{*2} + l^2c^{*2} +
         2kb^*l c^*\cos\alpha^* + 2lc^*ha^*\cos\beta^*
 + 2ha^*kb^*\cos\gamma^*)
$$
The parameters to be varied are the reciprocal cell quadratic products
A*, B*, C*, D*, E* and F*, where A*=$a^{*2}$, D*=$b^*c^*\cos\alpha^*$, etc.
\p
The author of the main program had the choice of names for these
parameters, and decided to make them members of family 1, genus 1,
having species names A*, B* etc.  Alternatively, she could have made 
them members of
family 1, genus 2, with genus name CELL, and species names the integers
1 to 6.  There are no other parameters in this simple example.
\p 
The calculated function $G_{calc}(h,k,l)$ has been programmed into a routine
CALCDS, which is given the values of h,k,l, and returns the value of 
$G_{calc}$ as
defined above, together with all derivatives of $G_{calc}$ with respect to
the relevant cell parameters. If a user happened to have data which were
observations of, say, d rather than d* squared, CALCDS could be simply
rewritten to calculate instead $G_{calc}(h,k,l)=d$ and its derivatives.
\p 
Other routines which have been written for this specific application are:\\
\stlink{a}{APSHDS} to apply shifts to the parameters,\\
\stlink{n}{NWINDS} to output new Crystal Data containing the shifted parameters,\\
\stlink{p}{PARSDS} to decide which are the parameters of the problem, and\\
\stlink{v}{VARSDS} to set up which parameters are variables.
\p 
The routines APSHDS, CALCDS, NWINDS, PARSDS and VARSDS may be inspected
in a listing of CCSL if further detail is sought.
% 
\subsection{Structure Factor LSQ}
\mlink{sflsq}{SFLSQ}\ (standard LSQ refinement with possible geometric constraints),
GRLSQ (taking groups of input reflections together) and \mlink{maglsq}{MAGLSQ}\ 
(for magnetic structures) are all examples of CCSL single-crystal LSQ programs.
\p 
Crystallographic LSQ involving structure factors follows much the same
pattern as the DSLSQ example above.  The calculated function involves a
structure factor computation by a routine such as LFCALC, during which 
derivatives are made with respect to all structure parameters which are
variables.
\p 
For details of the \hyperlink{card:L}{\cds{L}} which drive such LSQ programs the
user  should consult the specification in Chapter 3.  An example of the Crystal
Data  for \mlink{sflsq}{SFLSQ}\ is as follows: 
\p\ssk\begin{verbatim} 
I NCYC 5   PRIN 3   CONV 0.02
L RELA 2   2 Co SITE  1 Mn SITE  1 Sn SITE
L VARY   ALL SITE
L MODE 5     WGHT 1
L SCAL     1    1    1
L FIX  DOMR
N Co2Mnsn at room temp
S     -X,    -Y,     -Z
S  1/2+X, 1/2+Y,      Z
S      Y,     Z,      X
S     -Y,     X,     -Z
A Co   1/4   1/4   1/4    0.3115
A Mn     0     0     0    0.2842
A Sn   1/2   1/2   1/2    0.3386
F Co    1    0.2500
F Mn    1   -0.3730
F Sn    1    0.6228
C 6
E    1    100.    0.05
\end{verbatim}
\par   
The \hyperlink{card:C}{C}, \hyperlink{card:S}{\bd{S}}, 
\hyperlink{card:A}{\bd{A}}, \hyperlink{card:F}{\bd{F}}, and
\hyperlink{card:E}{\cds{E}}are the same 
as for other programs, giving 3
atoms in special positions in a cubic space group, with neutron nuclear
scattering factors, and type 1 extinction corrections on the structure
factors.
\p 
The \hyperlink{card:I}{\cd{I}} requests 5 cycles of refinement, with printing of 
the observed and calculated values of the function on the first 
and last cycles, and terminating before the
5th cycle if the largest shift/$\sigma$ is smaller than 0.02 (rather than the
default of 0.01).
\p
\hyperlink{card:L}{\cds{L}} are needed where defaults need to be changed, so \bd{L WGHT} 
asks for
unit weights rather than statistical weights, \bd{L MODE} asks for the special
\ital{extinction correction} input mode rather than the standard format for
observed input files, and so on.
\p 
Each LSQ main program has its own defaults built in as to whether a
given parameter should be fixed or varied.  In general most structure
parameters are by default varied, but site occupation factors are
by default fixed, so the \exac{L VARY ALL SITE} card is needed here.
\p 
It is required to relate all three site occupation factors by the
linear relation expressed on the \cd{L RELA}.
%
\subsection{Interpretation}
% 
The main program \mlink{sflsq}{SFLSQ} interprets the above Crystal Data and deduces
that all three atomic positions are so special that all their
coordinates are fixed.  It makes 9 basic variables, being:
\ssp 
\exac{MOSC, SCAL 1, SCAL 2, SCAL 3} (having deduced that there are 3 scale
zones from the \cd{L SCAL}) \exac{Co ITF, Mn ITF, Sn ITF} (the isotropic
temperature factors, varied by default) and \exac{Mn SITE} and \exac{Sn SITE}.
\p 
it makes 10 variables, being all the above plus \exac{Co SITE} (a redundant
variable), and records the constraint:
\ssp 
\cl{\exac{2* Co SITE + Mn SITE + Sn SITE = constant}}
\ssk 
By default, as there is no \cd{L REFI}, the program refines on the
modulus of the structure factor.  Another input dataset is needed, in the
format indicated by the card saying \exac{L MODE 5};
  the name of the file containing
this data set is requested interactively.
% 
\section{Making Other Least Squares Programs}
\markright{Making Other Least Squares Programs}
%
The existing main programs can be adapted to refine different parameters,
or to use a different calculated function in the refinement.  Examples
which already exist are:
\p
\mlink{mplsq}{MPLSQ}, multipole LSQ, in which the scattering density is expressed as a sum
of spherical harmonic (multipole) functions, and the coefficients of these
functions  may be refined.  A new family 5 has been defined for
them, as there may be different numbers of multipole coefficients for 
different atoms.
\p
\mlink{palsq}{PALSQ}, polarization analysis LSQ, in which the observations to be fitted
are the components of the scattered neutron polarization for a given input
polarization.
\p
Recently (April 2001) some changes have been made to make it easier to introduce new parameters
   into a least squares refinement based on structure factors. The author of the
   new program must provide:
\begin{enumerate}   
\item   		 An augmented main program in which the new parameters are defined
             this should be based on \mlink{sflsq}{SFLSQ}.
             The names of the new parameters (A4) must be added to ISFWRD
             and their specifications (family, genus, and species) to ISWDSP
\item       A subroutine called say DOXXX1 with 1 dummy agument MODE.
             When called with MODE=1 it should read all the data from the
             CDF which are needed to give values to the new parameters.
             When called with MODE=2 it should check that enough data have
             been read, and do any necessary tidying up.
             For every new parameter which can be refined there should be a
             corresponding integer pointer. This pointer will be set to the
             number  of the of the variable to which ithe parameter is
             assigned, or zero if it is fixed.
             DOXXX1 should be called in the main program with MODE=1 just after
             \stlink{s}{SETFC}, and with MODE=2 just before \stlink{p}{PARSSF}.
\item       A logical function called say DOXXX2 with two dummy arguments
             NN and MODE
\begin{description}             
\item[MODE = 1] Determine symmetry constraints on new parameters
                      NN is not used.
\item[MODE = 2] Set Variable pointers to new parameters
                 NN is the variable number
\item[MODE = 3] Apply shifts, one parameter per entry.
                      NN is not used
\item[MODE = 4] Write new parameter card (one per entry)
                      NN is the serial number of the letter on the card to
                      rewrite
\item [MODE = 5] Do any calculation required because of parameter changes 
                      at the end of a least squares cycle.
\item [MODE = 6] Interpret new words used to fix or vary groups of parameters
 when they apply to a particular genus.
 Such words should be defined in the main program as belonging to families
 with  negative numbers (<-10).
\item [MODE = 7] Interpret new words used to fix or vary groups of parameters
 when they apply to all such parameters (all genera or genus not applicable).
\end{description}     
\item  A subroutine called say LXXCAL to calculate the structure factors
             and their derivatives with respect to all the structural parameters
             (Only needed if the new parameters modify the structure factor)
\item   A subroutine called say CALXXX to calculate the measurement (GCALC)
             to be fitted, in terms of the structure factor and other
             experimental parameters.
             (Only needed if the new parameters change how this is done)
\item   A logical function called say DFTXXX which will be called  with
             the dummy arguments set to the family, genus and species of a 
             parameter. It should return TRUE if the parameter is to be varied.
\end{enumerate}             
        In the EXTERNAL declaration in the main program:

\begin{varindent}{3 cm}
                       LFCALC should be replaced by LXXCAL\\ 
                       DOTHER should be replaced by DOXXX2\\ 
                       DEFALT should be replaced by DFTXXX\\
\end{varindent} 
\par                      
        A call to CALXXX should replace that to \stlink{c}{CALCSF} inside the least squares
        cycles. It has two dummy arguments: H giving the reflection indices, 
        and the second LXXCAL being the structure factor calculation routine.      
        If, as for magnetic structures, two types of structure factors are to
        be calculated a third parameter gives the name of the required routine.
        It should be declared EXTERNAL. 
        For example for magnetic least squares the call becomes:
        call CALCMG(H,LFCALC,LMCALC)\\
        \stlink{l}{LFCALC} calculates the nuclear structure factors and their
        derivatives.\\
        \stlink{l}{LMCALC} calculates the magnetic structure factors and their
        derivatives.\\
        \stlink{c}{CALCMG} combines them into GCALC to match the observed quantity GOBS.\\ 
\p        
        Note. It is convenient to group the main program and the 
        special subroutines in a single file to be included in the main program
        section of the master file.
%        
\section{Structure Factor Least Squares with multiple Data Sources}
\markright{ Multiple Data Sources}
As of Version 4.4 of CCSL the set of structure factor least squares programs have been reorganised to allow
data from more than one source to be refined concurrently. To facilitate this,
provision has been made to attach source dependent information such as crystal orientation, wavelength, neutron polarisation etc, to the files containing the experimental data. This information is found in lines at the start of the data files  beginning with a \#  followed by a word indicating the type of information and the information itself. The words following the \# which can currently be recognised are: 
\begin{description}
\item{Polarisation} followed by 5 numbers: the Up and Down polarising efficiencies
		and the polarisation direction in orthogonal crystal coordinates.
\item{Wavelength} followed by  its value and, if needed, the absorption 
		coefficient.
\item{Orientation} followed by 9 components of the matrix giving the directions of the diffractometer axes in orthogonal crystal coordinates.
\end{description}
Only the first 5 characters of the words are matched, their case is unimportant.

The CCSL  programs \mlink{absmsf}{ABSMSF}, \mlink{sorgam}{SORGAM}, \mlink{sorasy}{SORASY} etc which make 
files  to use as input to least
squares programs now automatically add these lines.
This information is used when necessary to calculate the parameters used in
Becker Coppens extinction so these no longer need be included with the data -
(L MODE=5 and L MODE=7 are now obsolete).\\
The data are now read only once, rather than once per least squares cycle as
in versions prior to 4.4.

The least squares programs \mlink{sflsq}{SFLSQ}, \mlink{mplsq}{MPLSQ}, \mlink{maglsq}{MAGLSQ}, \mlink{mmplsq}{MMPLSQ}, \mlink{chilsq}{CHILSQ}, \mlink{mpclsq}{MPCLSQ} and \mlink{mgtlsq}{MGTLSQ}
all now use this new structure. 
\begin{itemize}
\item \mlink{sflsq}{SFLSQ}  and \mlink{mplsq}{MPLSQ}  can only read structure factor type data (REFI 1-4)
\item \mlink{chilsq}{CHILSQ} and \mlink{mpclsq}{MPCLSQ} normally only read polarised neutron data (REFI=5 or 7)
but in both cases data measured with different orientations or wavelengths
can be refined together. It is sometimes useful to include unpolarised structure factor data
measured with an applied field as well. \\
Update 4.4.17 allows polarised neutron data from  polycrystalline samples of anisotropic
paramagnets to be refined (MODE 11, REFI 9)

\item \mlink{maglsq}{MAGLSQ}, \mlink{mmplsq}{MMPLSQ} and \mlink{mgtlsq}{MGTLSQ} all allow integrated intensity and polarised neutron data to be combined. (REFI=1-4,5,and 7)
\end{itemize}
When there are multiple data sets, each is presented in a separate file and the 
values of MODE and REFI for each as well as its allocated weight may be given at the same time as the file-name. 
In this case the request for the data file name has the form:\\
{\tt Name of data set, its MODE, REFI and weight? :}\\
If the reply given is just the filename then the program assumes there is just a
 single source of data: \bd{MODE} and \bd{REFI} are taken from the crystal data file 
 as in the pre-4.4 versions.
 \subsection{SNPLSQ}
 In the new program \mlink{snplsq}{SNPLSQ}, which allows SNP and structure factor data to be refined
 together, provision has to be made for different scale factors and different
 magnetic domain populations to be associated with each data-set. These quantities must 
 also be identified with valid least squares parameters. Each data set is therefore identified by a
 {\em name} of up to 4 characters eg \bd{PA27} which is given on a 
 \hyperlink{Q:sorc}{\cd{L SORC}}. \bd{SORC}
 may be followed by the filename given with a path which is either absolute (starts with ``/")
 or relative to the current working directory. Environment variables are recognised.
 If no filename follows the identifier, the file name will be requested interactively.
 For each data source identified by {\em name}  there should be one or more \cds{L SORC {\em name}} 
 giving further information specific to these data: \bd{REFI}, \bd{MODE} and \bd{WGHT} are obligatory. The
integers following REFI and MODE have their usual meanings but apply only to this particular data-set. The number following \bd{WGHT} gives the weight for these data. 
\p
There should also
be \cds{L SORC SCAL} and \cds{L SORC DPOP} giving the scale factors and domain populations
for the data set. The number of scale factors needed is the number of scaling zones
given in the data file. The number of domains whose populations must be given depends
on the relative symmetries of the nuclear and magnetic structures and on the type of refinement.  
The new least squares parameters associated with the scales and domain populations are identified by the genus {\em name} and a composite species name starting with either \bd{SC} for a scale factor
or \bd{DP} for a domain population and ending with 2 digits identifying its number.
\subsection{SNPLSQ: Example L DATA and L SORC Cards}  
\p\ssk\begin{verbatim} 
L DATA PA27 $TT/khe27k.pal 
L SORC PA27 MODE 9 REFI 8 WGHT 1.0 
L SORC PA27 DPOP  0.18  0.07  0.07 0.18  0.18 0.07 0.07 0.18   
L DATA SF30 $TT/khe30k.sf 
L SORC SF30 MODE 7 REFI 1 WGHT 0.3 SCAL 7.16
L SORC SF30 DPOP 0.25 0.25 0.25 0.25
\end{verbatim}
The data are for a helical structure with 4 orientation domains. PA27 contains
SNP data (REFI 8) which is sensitive to the chirality so 8 domain populations are needed.
SF27 is a set of structure factor amplitudes (REFI 1) insensitive to chirality so only 4
domain populations can be refined. Pairs of chiral domains are numbered consecutively
so only odd-numbered domains are defined for the SF27 data. Constraints might therefore be 
given as follows:
\begin{verbatim} 
Z Domain constraints for PA data
L RELA 1 1 PA27 DP08 1 PA27 DP01
L RELA 1 1 PA27 DP07 1 PA27 DP02
L RELA 1 1 PA27 DP06 1 PA27 DP03
L RELA 1 1 PA27 DP05 1 PA27 DP04
L RELA 2 1 PA27 DP01 1 PA27 DP02 1 PA27 DP03 1 PA27 DP04 
Z Domain constraints for SF data
L RELA 2 1 SF30 DP01 1 SF30 DP03 1 SF30 DP05 1 SF30 DP07  
\end{verbatim}
No scale factors are needed for the SNP data; one is needed for the structure amplitudes.
It could be fixed using:
\begin{verbatim} 
L FIX  SF30 SC01
\end{verbatim}
\p
 For least squares refinements using different types of data, the weight given
 to each is important. A rough rule of thumb is to make the relative weights
 inversely proportional to the residual $\chi^2$ of each when refined separately.
%
\finchapter
%\end{document}

%\end{htmlonly}
%\internal{c2}
%\internal{c3}
%\internal{c4}
%\internal{c1}
%\internal{c6}
%\internal{c7}
\startdocument
%\htmladdtonavigation{\htmladdnormallink
%  {\htmladdimg{../icons/appenx.gif}}
%  {../appenx/appendix}}
\label{chap:5}
\markboth{Least Squares Refinement}{}
\section{What We Mean by LSQ}\markright{What We Mean by LSQ}
% 
The term LSQ is used here to mean \ital{Least Squares Refinement}, 
which uses
observations of a function to improve the values of some set of
parameters on which that function depends.
\p 
Refinement involving a crystal structure could be either \ital{standard} single 
crystal LSQ
refinement, in which each observation depends on a single
structure factor, or \ital{profile refinement} (PR) for which
several structure factors contribute to one observation.
\p 
The PR routines are written to use many of the same CCSL routines as do 
standard LSQ
main programs.  The PR programs and Libraries are available separately, with 
a separate Manual.
\p 
CCSL also deals with simpler LSQ problems.  For example, the main program
FWLSQ fits the 5,7 or 9 parameters of the exponential approximation to a 
scattering factor curve. The simplicity in this case is that no crystal structure is
involved.
Another simple case is the refinement of (up to) 6 cell parameters,
given an observed list of d-spacings (actually d* squared values, in
the main program DSLSQ).  This must deal with
the constraints which are
necessarily imposed on the cell parameters by the space group symmetry.
\p 
The essentials of a LSQ problem are:\p
\begin{list} {} {\setlength{\labelwidth}{8mm}
  \setlength{\parsep}{-1ex}
  \setlength{\leftmargin}{\labelwidth}
 \addtolength{\leftmargin}{4mm}}
\item[a) \hfill] a set of \ital{observations} of something, 
(the \ital{observed function})
usually with their \ital{estimated standard deviations,} $\sigma$'s,
measured at different values of some argument \ital{ARG};
\ital{ARG} may be $h,k,l$ (for standard LSQ), $\sin\theta/\lambda$, 
$\theta$ (for
Rietveld PR) or $\lambda$, or energy, or time of flight; it is a
quantity which takes different values, and will identify the
observation.\\[1mm]
For crystallographic applications the observations are often all of the
same physical thing, but this is not necessary.  For example, geometric
constraints may be introduced by giving bond lengths and/or angles as
additional types of observation, with $\sigma$'s.
\item[b) \hfill]  some \ital{calculated function} involving \ital{ARG},
which defines a
mathematical model to be compared with an observation.  This calculated
function depends on \ital{parameters}, things which may possibly be varied in order
to improve the fit of the function to its related observations.\\[1mm]
By \ital{fit} we mean minimisation of the weighted sum of squares of 
differences
between observed and calculated function values (where the summing is
over the different values of \ital{ARG}).  It is desirable to use 
statistical weights ($1/\sigma^2$)
for the differences.\\[1mm]
The theory may be viewed as using the beginning of a Taylor series
expansion, and therefore requires that the parameters be close to
their \ital{correct} values, making the required shifts so small that their
squares may be neglected.\end{list}
 
\section{Parameters and Variables}\markright{Parameters and Variables}
% 
\subsection{Parameters}
In LSQ a \ital{parameter} is something which it is sensible to try to
adjust. It represents some physical thing like ``the x coordinate of the 3rd
atom'', ``an overall temperature factor'' or ``the scale factor for all
observations coded number 2''.
\p 
A \ital{parameter} is also something with respect to which the function may be
differentiated.  Differentiation may be analytic, if it is possible (and
sensible) to write down such derivatives algebraically.  It may also be
numerical in awkward cases, using a simple approximation to the
derivative involving function values.
\p 
For any particular run of a LSQ refinement program, it is unlikely that
the user will want to vary every parameter.  The subset of parameters
which are actually to be varied (i.e. those for which shifts are
required) we call \ital{variables}.
\p 
Parameters are thus either \ital{fixed} or \ital{varied}.
For a fixed parameter,
there is no need for a derivative; but for all parameters to be
varied, derivatives will be needed.
% 
\subsection{Basic Variables}
% 
It is often the case that parameters are related, either inherently by
the symmetry of the problem, or by a strict constraint applied by the user.  If
two variables are related, they must not both be refined by LSQ;  the
process expects the shifts to be independent.  Every constraint or
relation reduces by 1 the number of variables to be refined.
\p 
The subset of variables for which shifts are actually calculated
we call \ital{basic variables}, or basics for short.
\p 
Those variables which are not basic we call \ital{redundant}. 
Thus if a problem has $n_p$ parameters, $n_f$ of 
which are fixed and $n_v$ varied,
               $$n_p = n_f + n_v$$
and if, of those $n_v$, $n_b$ are basic and $n_r$ are redundant,
              $$n_v = n_b + n_r$$
% 
\subsection{Strict Constraints}
% 
The existence of $n_r$ redundant variables implies that there are $n_r$
strict constraints on the problem. Derivatives are calculated 
(and shifts are required) for all
variables, not just the basics.
\p
Those constraints which must be imposed because of the crystallographic
symmetry are generated automatically by CCSL, and need not be provided by
the user.  For example, the  special position $(x,x,x)$ will give rise to
one basic variable, $x$, with two redundant variables $y$ and $z$ related to
it by two constraints, but the user will not need to tell the system
this.\p
The user may impose additional strict constraints.  Two constraint types
are at present available.  The simpler is type 1:
             $$a_1\Delta p_1 = a_2\Delta p_2$$
that is, constant $a_1$ times the shift in parameter 1 = constant $a_2$ times
the shift in parameter 2.\p
Type 2 constraints involve a linear combination of parameters, with
constant coefficients, thus:
 $$  a_1\Delta p_1 + a_2\Delta p_2 + a_3\Delta p3 + \ldots\ \mbox{etc}\
   = 0$$
Note that the way these expressions are written involves differentiating
the original constraint.  For example, if a type 2 constraint is
needed to fix the sum of three parameters to be 3, the constraint
$$p_1 + p_2 + p_3 = 3\quad\mbox{becomes}\quad \Delta p_1 + 
\Delta p_2 + \Delta p_3 = 0$$
Other more complicated types of constraint could be added if needed.
%
\subsection{Parameter Names}
% 
The need to fix, vary or relate parameters implies the need to call
these parameters by name.  The CCSL LSQ facilities allow each problem to
be set up individually.  Names for the parameters of a problem are
given in its main program;  the user needs to know what these names are
in order to refer to the parameters in his Crystal Data.
\p
Examples of names used in various standard main programs are:
\ssp
\verb| Na2 X  Ca B11  SCAL 2  P23 ITF   A*|
\ssp 
Most names have two parts, which we call the genus and species names;  the
genus names above are \exac{\ Na2, Ca, SCAL, P23}, with corresponding
species names \exac{\ X, B11, 2} and \exac{ITF}.  The name \exac{A*} 
is simply a species name.
\p 
In LSQ programs which involve structure factors,
those structure parameters which belong to a particular named atom
have the atom name as genus name, and one of:
\ssp 
\verb|X   Y   Z   B11   B12   B13   B22   B23   B33   ITF  SITE  SCAT|
\ssp 
as species name.  
(\exac{SCAT} means the scattering factor, when this is represented 
by one number and it is sensible to refine it).
\p
MAGLSQ has the additional species names:
\ssp
\exac{MU   MU1   THET   THE1   PHI   PHI1   PSI1   PSI2   PSI3   PSI4}.
\p
When LSQ programs refer to parameters in output for shifts, correlations etc,
they use these parameter names.
\p 
More details about parameter
names are to be found in Chapter 3 under \cds{\htmlref{L FIX}{ss:fix}}.  
Note the availability
of words like \bd{XYZ} to mean ``all three x,y,z coordinates", 
\bd{CELL} to mean ``all six
cell parameters", etc.  One may also use the word \bd{ALL} 
to mean ``all the members
of this genus" (as in \exac{ALL Na3} for ``all the parameters of atom Na3")
 or ``all
the genera with this species" (as in \exac{ALL X}, or \exac{ALL SITE}, or even 
\exac{ALL XYZ}).
%
\subsection{Families of Parameters}
% 
Structure parameters such as those introduced in the previous section,
being a commonly occurring group in LSQ, are said in CCSL to belong to 
\ital{family
2} of the parameters.  \ital{Family~1} contains all other
parameters in the simple LSQ applications we are considering here.  Other
families are used in PR, and in other LSQ programs.
\p 
Family 1, genus 1 is treated by CCSL as special.  It
contains the parameters for which the species name by itself is enough,
like \exac{A*} above;  another example is an overall temperature factor for a
structure, known as \exac{TFAC}.
\p 
Genera 2, 3 etc of family 1 may in general be given any genus name.  Within
such a genus the species name could be simply an integer, or species could have 
individual names.  This choice is
for the writer of the main program.  In \mlink{sflsq}{SFLSQ}\ there are two genera in
family 1, the second being named \exac{SCAL} with species numbered 1,2 etc.
\p 
Unless the user is writing or modifying a main program, he need not be
concerned with the detailed mechanism for numbering parameters.  He only
needs to know what names he is allowed to use on his Crystal Data
for the programs he runs.
\p 
\section{Examples of Least squares programs}
\markright{Examples of Least squares programs}
%
We illustrate these general ideas by, first, a simple example of LSQ
which does not involve structure parameters, and, second, notes on the
use of the standard single crystal structure refinement program, SFLSQ.
\p
There are two fundamental questions to be asked of a main LSQ program:
\begin{enumerate}
\item on which calculated function is it refining 
\item which parameters may be refined.
\end{enumerate}
%
\subsection{DSLSQ}
% 
DSLSQ is a main program which determines the values of up to 6 cell
parameters by LSQ fitting, using values of d spacings observed at 
different values of h,k,l.

The calculated function used is the expression for the square of d*, 
in reciprocal space:
$$d^*(h,k,l)^2 = (h^2a^{*2} + k^2b^{*2} + l^2c^{*2} +
         2kb^*l c^*\cos\alpha^* + 2lc^*ha^*\cos\beta^*
 + 2ha^*kb^*\cos\gamma^*)
$$
The parameters to be varied are the reciprocal cell quadratic products
A*, B*, C*, D*, E* and F*, where A*=$a^{*2}$, D*=$b^*c^*\cos\alpha^*$, etc.
\p
The author of the main program had the choice of names for these
parameters, and decided to make them members of family 1, genus 1,
having species names A*, B* etc.  Alternatively, she could have made 
them members of
family 1, genus 2, with genus name CELL, and species names the integers
1 to 6.  There are no other parameters in this simple example.
\p 
The calculated function $G_{calc}(h,k,l)$ has been programmed into a routine
CALCDS, which is given the values of h,k,l, and returns the value of 
$G_{calc}$ as
defined above, together with all derivatives of $G_{calc}$ with respect to
the relevant cell parameters. If a user happened to have data which were
observations of, say, d rather than d* squared, CALCDS could be simply
rewritten to calculate instead $G_{calc}(h,k,l)=d$ and its derivatives.
\p 
Other routines which have been written for this specific application are:\\
\stlink{a}{APSHDS} to apply shifts to the parameters,\\
\stlink{n}{NWINDS} to output new Crystal Data containing the shifted parameters,\\
\stlink{p}{PARSDS} to decide which are the parameters of the problem, and\\
\stlink{v}{VARSDS} to set up which parameters are variables.
\p 
The routines APSHDS, CALCDS, NWINDS, PARSDS and VARSDS may be inspected
in a listing of CCSL if further detail is sought.
% 
\subsection{Structure Factor LSQ}
\mlink{sflsq}{SFLSQ}\ (standard LSQ refinement with possible geometric constraints),
GRLSQ (taking groups of input reflections together) and \mlink{maglsq}{MAGLSQ}\ 
(for magnetic structures) are all examples of CCSL single-crystal LSQ programs.
\p 
Crystallographic LSQ involving structure factors follows much the same
pattern as the DSLSQ example above.  The calculated function involves a
structure factor computation by a routine such as LFCALC, during which 
derivatives are made with respect to all structure parameters which are
variables.
\p 
For details of the \hyperlink{card:L}{\cds{L}} which drive such LSQ programs the
user  should consult the specification in Chapter 3.  An example of the Crystal
Data  for \mlink{sflsq}{SFLSQ}\ is as follows: 
\p\ssk\begin{verbatim} 
I NCYC 5   PRIN 3   CONV 0.02
L RELA 2   2 Co SITE  1 Mn SITE  1 Sn SITE
L VARY   ALL SITE
L MODE 5     WGHT 1
L SCAL     1    1    1
L FIX  DOMR
N Co2Mnsn at room temp
S     -X,    -Y,     -Z
S  1/2+X, 1/2+Y,      Z
S      Y,     Z,      X
S     -Y,     X,     -Z
A Co   1/4   1/4   1/4    0.3115
A Mn     0     0     0    0.2842
A Sn   1/2   1/2   1/2    0.3386
F Co    1    0.2500
F Mn    1   -0.3730
F Sn    1    0.6228
C 6
E    1    100.    0.05
\end{verbatim}
\par   
The \hyperlink{card:C}{C}, \hyperlink{card:S}{\bd{S}}, 
\hyperlink{card:A}{\bd{A}}, \hyperlink{card:F}{\bd{F}}, and
\hyperlink{card:E}{\cds{E}}are the same 
as for other programs, giving 3
atoms in special positions in a cubic space group, with neutron nuclear
scattering factors, and type 1 extinction corrections on the structure
factors.
\p 
The \hyperlink{card:I}{\cd{I}} requests 5 cycles of refinement, with printing of 
the observed and calculated values of the function on the first 
and last cycles, and terminating before the
5th cycle if the largest shift/$\sigma$ is smaller than 0.02 (rather than the
default of 0.01).
\p
\hyperlink{card:L}{\cds{L}} are needed where defaults need to be changed, so \bd{L WGHT} 
asks for
unit weights rather than statistical weights, \bd{L MODE} asks for the special
\ital{extinction correction} input mode rather than the standard format for
observed input files, and so on.
\p 
Each LSQ main program has its own defaults built in as to whether a
given parameter should be fixed or varied.  In general most structure
parameters are by default varied, but site occupation factors are
by default fixed, so the \exac{L VARY ALL SITE} card is needed here.
\p 
It is required to relate all three site occupation factors by the
linear relation expressed on the \cd{L RELA}.
%
\subsection{Interpretation}
% 
The main program \mlink{sflsq}{SFLSQ} interprets the above Crystal Data and deduces
that all three atomic positions are so special that all their
coordinates are fixed.  It makes 9 basic variables, being:
\ssp 
\exac{MOSC, SCAL 1, SCAL 2, SCAL 3} (having deduced that there are 3 scale
zones from the \cd{L SCAL}) \exac{Co ITF, Mn ITF, Sn ITF} (the isotropic
temperature factors, varied by default) and \exac{Mn SITE} and \exac{Sn SITE}.
\p 
it makes 10 variables, being all the above plus \exac{Co SITE} (a redundant
variable), and records the constraint:
\ssp 
\cl{\exac{2* Co SITE + Mn SITE + Sn SITE = constant}}
\ssk 
By default, as there is no \cd{L REFI}, the program refines on the
modulus of the structure factor.  Another input dataset is needed, in the
format indicated by the card saying \exac{L MODE 5};
  the name of the file containing
this data set is requested interactively.
% 
\section{Making Other Least Squares Programs}
\markright{Making Other Least Squares Programs}
%
The existing main programs can be adapted to refine different parameters,
or to use a different calculated function in the refinement.  Examples
which already exist are:
\p
\mlink{mplsq}{MPLSQ}, multipole LSQ, in which the scattering density is expressed as a sum
of spherical harmonic (multipole) functions, and the coefficients of these
functions  may be refined.  A new family 5 has been defined for
them, as there may be different numbers of multipole coefficients for 
different atoms.
\p
\mlink{palsq}{PALSQ}, polarization analysis LSQ, in which the observations to be fitted
are the components of the scattered neutron polarization for a given input
polarization.
\p
Recently (April 2001) some changes have been made to make it easier to introduce new parameters
   into a least squares refinement based on structure factors. The author of the
   new program must provide:
\begin{enumerate}   
\item   		 An augmented main program in which the new parameters are defined
             this should be based on \mlink{sflsq}{SFLSQ}.
             The names of the new parameters (A4) must be added to ISFWRD
             and their specifications (family, genus, and species) to ISWDSP
\item       A subroutine called say DOXXX1 with 1 dummy agument MODE.
             When called with MODE=1 it should read all the data from the
             CDF which are needed to give values to the new parameters.
             When called with MODE=2 it should check that enough data have
             been read, and do any necessary tidying up.
             For every new parameter which can be refined there should be a
             corresponding integer pointer. This pointer will be set to the
             number  of the of the variable to which ithe parameter is
             assigned, or zero if it is fixed.
             DOXXX1 should be called in the main program with MODE=1 just after
             \stlink{s}{SETFC}, and with MODE=2 just before \stlink{p}{PARSSF}.
\item       A logical function called say DOXXX2 with two dummy arguments
             NN and MODE
\begin{description}             
\item[MODE = 1] Determine symmetry constraints on new parameters
                      NN is not used.
\item[MODE = 2] Set Variable pointers to new parameters
                 NN is the variable number
\item[MODE = 3] Apply shifts, one parameter per entry.
                      NN is not used
\item[MODE = 4] Write new parameter card (one per entry)
                      NN is the serial number of the letter on the card to
                      rewrite
\item [MODE = 5] Do any calculation required because of parameter changes 
                      at the end of a least squares cycle.
\item [MODE = 6] Interpret new words used to fix or vary groups of parameters
 when they apply to a particular genus.
 Such words should be defined in the main program as belonging to families
 with  negative numbers (<-10).
\item [MODE = 7] Interpret new words used to fix or vary groups of parameters
 when they apply to all such parameters (all genera or genus not applicable).
\end{description}     
\item  A subroutine called say LXXCAL to calculate the structure factors
             and their derivatives with respect to all the structural parameters
             (Only needed if the new parameters modify the structure factor)
\item   A subroutine called say CALXXX to calculate the measurement (GCALC)
             to be fitted, in terms of the structure factor and other
             experimental parameters.
             (Only needed if the new parameters change how this is done)
\item   A logical function called say DFTXXX which will be called  with
             the dummy arguments set to the family, genus and species of a 
             parameter. It should return TRUE if the parameter is to be varied.
\end{enumerate}             
        In the EXTERNAL declaration in the main program:

\begin{varindent}{3 cm}
                       LFCALC should be replaced by LXXCAL\\ 
                       DOTHER should be replaced by DOXXX2\\ 
                       DEFALT should be replaced by DFTXXX\\
\end{varindent} 
\par                      
        A call to CALXXX should replace that to \stlink{c}{CALCSF} inside the least squares
        cycles. It has two dummy arguments: H giving the reflection indices, 
        and the second LXXCAL being the structure factor calculation routine.      
        If, as for magnetic structures, two types of structure factors are to
        be calculated a third parameter gives the name of the required routine.
        It should be declared EXTERNAL. 
        For example for magnetic least squares the call becomes:
        call CALCMG(H,LFCALC,LMCALC)\\
        \stlink{l}{LFCALC} calculates the nuclear structure factors and their
        derivatives.\\
        \stlink{l}{LMCALC} calculates the magnetic structure factors and their
        derivatives.\\
        \stlink{c}{CALCMG} combines them into GCALC to match the observed quantity GOBS.\\ 
\p        
        Note. It is convenient to group the main program and the 
        special subroutines in a single file to be included in the main program
        section of the master file.
%        
\section{Structure Factor Least Squares with multiple Data Sources}
\markright{ Multiple Data Sources}
As of Version 4.4 of CCSL the set of structure factor least squares programs have been reorganised to allow
data from more than one source to be refined concurrently. To facilitate this,
provision has been made to attach source dependent information such as crystal orientation, wavelength, neutron polarisation etc, to the files containing the experimental data. This information is found in lines at the start of the data files  beginning with a \#  followed by a word indicating the type of information and the information itself. The words following the \# which can currently be recognised are: 
\begin{description}
\item{Polarisation} followed by 5 numbers: the Up and Down polarising efficiencies
		and the polarisation direction in orthogonal crystal coordinates.
\item{Wavelength} followed by  its value and, if needed, the absorption 
		coefficient.
\item{Orientation} followed by 9 components of the matrix giving the directions of the diffractometer axes in orthogonal crystal coordinates.
\end{description}
Only the first 5 characters of the words are matched, their case is unimportant.

The CCSL  programs \mlink{absmsf}{ABSMSF}, \mlink{sorgam}{SORGAM}, \mlink{sorasy}{SORASY} etc which make 
files  to use as input to least
squares programs now automatically add these lines.
This information is used when necessary to calculate the parameters used in
Becker Coppens extinction so these no longer need be included with the data -
(L MODE=5 and L MODE=7 are now obsolete).\\
The data are now read only once, rather than once per least squares cycle as
in versions prior to 4.4.

The least squares programs \mlink{sflsq}{SFLSQ}, \mlink{mplsq}{MPLSQ}, \mlink{maglsq}{MAGLSQ}, \mlink{mmplsq}{MMPLSQ}, \mlink{chilsq}{CHILSQ}, \mlink{mpclsq}{MPCLSQ} and \mlink{mgtlsq}{MGTLSQ}
all now use this new structure. 
\begin{itemize}
\item \mlink{sflsq}{SFLSQ}  and \mlink{mplsq}{MPLSQ}  can only read structure factor type data (REFI 1-4)
\item \mlink{chilsq}{CHILSQ} and \mlink{mpclsq}{MPCLSQ} normally only read polarised neutron data (REFI=5 or 7)
but in both cases data measured with different orientations or wavelengths
can be refined together. It is sometimes useful to include unpolarised structure factor data
measured with an applied field as well. \\
Update 4.4.17 allows polarised neutron data from  polycrystalline samples of anisotropic
paramagnets to be refined (MODE 11, REFI 9)

\item \mlink{maglsq}{MAGLSQ}, \mlink{mmplsq}{MMPLSQ} and \mlink{mgtlsq}{MGTLSQ} all allow integrated intensity and polarised neutron data to be combined. (REFI=1-4,5,and 7)
\end{itemize}
When there are multiple data sets, each is presented in a separate file and the 
values of MODE and REFI for each as well as its allocated weight may be given at the same time as the file-name. 
In this case the request for the data file name has the form:\\
{\tt Name of data set, its MODE, REFI and weight? :}\\
If the reply given is just the filename then the program assumes there is just a
 single source of data: \bd{MODE} and \bd{REFI} are taken from the crystal data file 
 as in the pre-4.4 versions.
 \subsection{SNPLSQ}
 In the new program \mlink{snplsq}{SNPLSQ}, which allows SNP and structure factor data to be refined
 together, provision has to be made for different scale factors and different
 magnetic domain populations to be associated with each data-set. These quantities must 
 also be identified with valid least squares parameters. Each data set is therefore identified by a
 {\em name} of up to 4 characters eg \bd{PA27} which is given on a 
 \hyperlink{Q:sorc}{\cd{L SORC}}. \bd{SORC}
 may be followed by the filename given with a path which is either absolute (starts with ``/")
 or relative to the current working directory. Environment variables are recognised.
 If no filename follows the identifier, the file name will be requested interactively.
 For each data source identified by {\em name}  there should be one or more \cds{L SORC {\em name}} 
 giving further information specific to these data: \bd{REFI}, \bd{MODE} and \bd{WGHT} are obligatory. The
integers following REFI and MODE have their usual meanings but apply only to this particular data-set. The number following \bd{WGHT} gives the weight for these data. 
\p
There should also
be \cds{L SORC SCAL} and \cds{L SORC DPOP} giving the scale factors and domain populations
for the data set. The number of scale factors needed is the number of scaling zones
given in the data file. The number of domains whose populations must be given depends
on the relative symmetries of the nuclear and magnetic structures and on the type of refinement.  
The new least squares parameters associated with the scales and domain populations are identified by the genus {\em name} and a composite species name starting with either \bd{SC} for a scale factor
or \bd{DP} for a domain population and ending with 2 digits identifying its number.
\subsection{SNPLSQ: Example L DATA and L SORC Cards}  
\p\ssk\begin{verbatim} 
L DATA PA27 $TT/khe27k.pal 
L SORC PA27 MODE 9 REFI 8 WGHT 1.0 
L SORC PA27 DPOP  0.18  0.07  0.07 0.18  0.18 0.07 0.07 0.18   
L DATA SF30 $TT/khe30k.sf 
L SORC SF30 MODE 7 REFI 1 WGHT 0.3 SCAL 7.16
L SORC SF30 DPOP 0.25 0.25 0.25 0.25
\end{verbatim}
The data are for a helical structure with 4 orientation domains. PA27 contains
SNP data (REFI 8) which is sensitive to the chirality so 8 domain populations are needed.
SF27 is a set of structure factor amplitudes (REFI 1) insensitive to chirality so only 4
domain populations can be refined. Pairs of chiral domains are numbered consecutively
so only odd-numbered domains are defined for the SF27 data. Constraints might therefore be 
given as follows:
\begin{verbatim} 
Z Domain constraints for PA data
L RELA 1 1 PA27 DP08 1 PA27 DP01
L RELA 1 1 PA27 DP07 1 PA27 DP02
L RELA 1 1 PA27 DP06 1 PA27 DP03
L RELA 1 1 PA27 DP05 1 PA27 DP04
L RELA 2 1 PA27 DP01 1 PA27 DP02 1 PA27 DP03 1 PA27 DP04 
Z Domain constraints for SF data
L RELA 2 1 SF30 DP01 1 SF30 DP03 1 SF30 DP05 1 SF30 DP07  
\end{verbatim}
No scale factors are needed for the SNP data; one is needed for the structure amplitudes.
It could be fixed using:
\begin{verbatim} 
L FIX  SF30 SC01
\end{verbatim}
\p
 For least squares refinements using different types of data, the weight given
 to each is important. A rough rule of thumb is to make the relative weights
 inversely proportional to the residual $\chi^2$ of each when refined separately.
%
\finchapter
%\end{document}

%\end{htmlonly}
%\internal{c2}
%\internal{c3}
%\internal{c4}
%\internal{c1}
%\internal{c6}
%\internal{c7}
\startdocument
%\htmladdtonavigation{\htmladdnormallink
%  {\htmladdimg{../icons/appenx.gif}}
%  {../appenx/appendix}}
\label{chap:5}
\markboth{Least Squares Refinement}{}
\section{What We Mean by LSQ}\markright{What We Mean by LSQ}
% 
The term LSQ is used here to mean \ital{Least Squares Refinement}, 
which uses
observations of a function to improve the values of some set of
parameters on which that function depends.
\p 
Refinement involving a crystal structure could be either \ital{standard} single 
crystal LSQ
refinement, in which each observation depends on a single
structure factor, or \ital{profile refinement} (PR) for which
several structure factors contribute to one observation.
\p 
The PR routines are written to use many of the same CCSL routines as do 
standard LSQ
main programs.  The PR programs and Libraries are available separately, with 
a separate Manual.
\p 
CCSL also deals with simpler LSQ problems.  For example, the main program
FWLSQ fits the 5,7 or 9 parameters of the exponential approximation to a 
scattering factor curve. The simplicity in this case is that no crystal structure is
involved.
Another simple case is the refinement of (up to) 6 cell parameters,
given an observed list of d-spacings (actually d* squared values, in
the main program DSLSQ).  This must deal with
the constraints which are
necessarily imposed on the cell parameters by the space group symmetry.
\p 
The essentials of a LSQ problem are:\p
\begin{list} {} {\setlength{\labelwidth}{8mm}
  \setlength{\parsep}{-1ex}
  \setlength{\leftmargin}{\labelwidth}
 \addtolength{\leftmargin}{4mm}}
\item[a) \hfill] a set of \ital{observations} of something, 
(the \ital{observed function})
usually with their \ital{estimated standard deviations,} $\sigma$'s,
measured at different values of some argument \ital{ARG};
\ital{ARG} may be $h,k,l$ (for standard LSQ), $\sin\theta/\lambda$, 
$\theta$ (for
Rietveld PR) or $\lambda$, or energy, or time of flight; it is a
quantity which takes different values, and will identify the
observation.\\[1mm]
For crystallographic applications the observations are often all of the
same physical thing, but this is not necessary.  For example, geometric
constraints may be introduced by giving bond lengths and/or angles as
additional types of observation, with $\sigma$'s.
\item[b) \hfill]  some \ital{calculated function} involving \ital{ARG},
which defines a
mathematical model to be compared with an observation.  This calculated
function depends on \ital{parameters}, things which may possibly be varied in order
to improve the fit of the function to its related observations.\\[1mm]
By \ital{fit} we mean minimisation of the weighted sum of squares of 
differences
between observed and calculated function values (where the summing is
over the different values of \ital{ARG}).  It is desirable to use 
statistical weights ($1/\sigma^2$)
for the differences.\\[1mm]
The theory may be viewed as using the beginning of a Taylor series
expansion, and therefore requires that the parameters be close to
their \ital{correct} values, making the required shifts so small that their
squares may be neglected.\end{list}
 
\section{Parameters and Variables}\markright{Parameters and Variables}
% 
\subsection{Parameters}
In LSQ a \ital{parameter} is something which it is sensible to try to
adjust. It represents some physical thing like ``the x coordinate of the 3rd
atom'', ``an overall temperature factor'' or ``the scale factor for all
observations coded number 2''.
\p 
A \ital{parameter} is also something with respect to which the function may be
differentiated.  Differentiation may be analytic, if it is possible (and
sensible) to write down such derivatives algebraically.  It may also be
numerical in awkward cases, using a simple approximation to the
derivative involving function values.
\p 
For any particular run of a LSQ refinement program, it is unlikely that
the user will want to vary every parameter.  The subset of parameters
which are actually to be varied (i.e. those for which shifts are
required) we call \ital{variables}.
\p 
Parameters are thus either \ital{fixed} or \ital{varied}.
For a fixed parameter,
there is no need for a derivative; but for all parameters to be
varied, derivatives will be needed.
% 
\subsection{Basic Variables}
% 
It is often the case that parameters are related, either inherently by
the symmetry of the problem, or by a strict constraint applied by the user.  If
two variables are related, they must not both be refined by LSQ;  the
process expects the shifts to be independent.  Every constraint or
relation reduces by 1 the number of variables to be refined.
\p 
The subset of variables for which shifts are actually calculated
we call \ital{basic variables}, or basics for short.
\p 
Those variables which are not basic we call \ital{redundant}. 
Thus if a problem has $n_p$ parameters, $n_f$ of 
which are fixed and $n_v$ varied,
               $$n_p = n_f + n_v$$
and if, of those $n_v$, $n_b$ are basic and $n_r$ are redundant,
              $$n_v = n_b + n_r$$
% 
\subsection{Strict Constraints}
% 
The existence of $n_r$ redundant variables implies that there are $n_r$
strict constraints on the problem. Derivatives are calculated 
(and shifts are required) for all
variables, not just the basics.
\p
Those constraints which must be imposed because of the crystallographic
symmetry are generated automatically by CCSL, and need not be provided by
the user.  For example, the  special position $(x,x,x)$ will give rise to
one basic variable, $x$, with two redundant variables $y$ and $z$ related to
it by two constraints, but the user will not need to tell the system
this.\p
The user may impose additional strict constraints.  Two constraint types
are at present available.  The simpler is type 1:
             $$a_1\Delta p_1 = a_2\Delta p_2$$
that is, constant $a_1$ times the shift in parameter 1 = constant $a_2$ times
the shift in parameter 2.\p
Type 2 constraints involve a linear combination of parameters, with
constant coefficients, thus:
 $$  a_1\Delta p_1 + a_2\Delta p_2 + a_3\Delta p3 + \ldots\ \mbox{etc}\
   = 0$$
Note that the way these expressions are written involves differentiating
the original constraint.  For example, if a type 2 constraint is
needed to fix the sum of three parameters to be 3, the constraint
$$p_1 + p_2 + p_3 = 3\quad\mbox{becomes}\quad \Delta p_1 + 
\Delta p_2 + \Delta p_3 = 0$$
Other more complicated types of constraint could be added if needed.
%
\subsection{Parameter Names}
% 
The need to fix, vary or relate parameters implies the need to call
these parameters by name.  The CCSL LSQ facilities allow each problem to
be set up individually.  Names for the parameters of a problem are
given in its main program;  the user needs to know what these names are
in order to refer to the parameters in his Crystal Data.
\p
Examples of names used in various standard main programs are:
\ssp
\verb| Na2 X  Ca B11  SCAL 2  P23 ITF   A*|
\ssp 
Most names have two parts, which we call the genus and species names;  the
genus names above are \exac{\ Na2, Ca, SCAL, P23}, with corresponding
species names \exac{\ X, B11, 2} and \exac{ITF}.  The name \exac{A*} 
is simply a species name.
\p 
In LSQ programs which involve structure factors,
those structure parameters which belong to a particular named atom
have the atom name as genus name, and one of:
\ssp 
\verb|X   Y   Z   B11   B12   B13   B22   B23   B33   ITF  SITE  SCAT|
\ssp 
as species name.  
(\exac{SCAT} means the scattering factor, when this is represented 
by one number and it is sensible to refine it).
\p
MAGLSQ has the additional species names:
\ssp
\exac{MU   MU1   THET   THE1   PHI   PHI1   PSI1   PSI2   PSI3   PSI4}.
\p
When LSQ programs refer to parameters in output for shifts, correlations etc,
they use these parameter names.
\p 
More details about parameter
names are to be found in Chapter 3 under \cds{\htmlref{L FIX}{ss:fix}}.  
Note the availability
of words like \bd{XYZ} to mean ``all three x,y,z coordinates", 
\bd{CELL} to mean ``all six
cell parameters", etc.  One may also use the word \bd{ALL} 
to mean ``all the members
of this genus" (as in \exac{ALL Na3} for ``all the parameters of atom Na3")
 or ``all
the genera with this species" (as in \exac{ALL X}, or \exac{ALL SITE}, or even 
\exac{ALL XYZ}).
%
\subsection{Families of Parameters}
% 
Structure parameters such as those introduced in the previous section,
being a commonly occurring group in LSQ, are said in CCSL to belong to 
\ital{family
2} of the parameters.  \ital{Family~1} contains all other
parameters in the simple LSQ applications we are considering here.  Other
families are used in PR, and in other LSQ programs.
\p 
Family 1, genus 1 is treated by CCSL as special.  It
contains the parameters for which the species name by itself is enough,
like \exac{A*} above;  another example is an overall temperature factor for a
structure, known as \exac{TFAC}.
\p 
Genera 2, 3 etc of family 1 may in general be given any genus name.  Within
such a genus the species name could be simply an integer, or species could have 
individual names.  This choice is
for the writer of the main program.  In \mlink{sflsq}{SFLSQ}\ there are two genera in
family 1, the second being named \exac{SCAL} with species numbered 1,2 etc.
\p 
Unless the user is writing or modifying a main program, he need not be
concerned with the detailed mechanism for numbering parameters.  He only
needs to know what names he is allowed to use on his Crystal Data
for the programs he runs.
\p 
\section{Examples of Least squares programs}
\markright{Examples of Least squares programs}
%
We illustrate these general ideas by, first, a simple example of LSQ
which does not involve structure parameters, and, second, notes on the
use of the standard single crystal structure refinement program, SFLSQ.
\p
There are two fundamental questions to be asked of a main LSQ program:
\begin{enumerate}
\item on which calculated function is it refining 
\item which parameters may be refined.
\end{enumerate}
%
\subsection{DSLSQ}
% 
DSLSQ is a main program which determines the values of up to 6 cell
parameters by LSQ fitting, using values of d spacings observed at 
different values of h,k,l.

The calculated function used is the expression for the square of d*, 
in reciprocal space:
$$d^*(h,k,l)^2 = (h^2a^{*2} + k^2b^{*2} + l^2c^{*2} +
         2kb^*l c^*\cos\alpha^* + 2lc^*ha^*\cos\beta^*
 + 2ha^*kb^*\cos\gamma^*)
$$
The parameters to be varied are the reciprocal cell quadratic products
A*, B*, C*, D*, E* and F*, where A*=$a^{*2}$, D*=$b^*c^*\cos\alpha^*$, etc.
\p
The author of the main program had the choice of names for these
parameters, and decided to make them members of family 1, genus 1,
having species names A*, B* etc.  Alternatively, she could have made 
them members of
family 1, genus 2, with genus name CELL, and species names the integers
1 to 6.  There are no other parameters in this simple example.
\p 
The calculated function $G_{calc}(h,k,l)$ has been programmed into a routine
CALCDS, which is given the values of h,k,l, and returns the value of 
$G_{calc}$ as
defined above, together with all derivatives of $G_{calc}$ with respect to
the relevant cell parameters. If a user happened to have data which were
observations of, say, d rather than d* squared, CALCDS could be simply
rewritten to calculate instead $G_{calc}(h,k,l)=d$ and its derivatives.
\p 
Other routines which have been written for this specific application are:\\
\stlink{a}{APSHDS} to apply shifts to the parameters,\\
\stlink{n}{NWINDS} to output new Crystal Data containing the shifted parameters,\\
\stlink{p}{PARSDS} to decide which are the parameters of the problem, and\\
\stlink{v}{VARSDS} to set up which parameters are variables.
\p 
The routines APSHDS, CALCDS, NWINDS, PARSDS and VARSDS may be inspected
in a listing of CCSL if further detail is sought.
% 
\subsection{Structure Factor LSQ}
\mlink{sflsq}{SFLSQ}\ (standard LSQ refinement with possible geometric constraints),
GRLSQ (taking groups of input reflections together) and \mlink{maglsq}{MAGLSQ}\ 
(for magnetic structures) are all examples of CCSL single-crystal LSQ programs.
\p 
Crystallographic LSQ involving structure factors follows much the same
pattern as the DSLSQ example above.  The calculated function involves a
structure factor computation by a routine such as LFCALC, during which 
derivatives are made with respect to all structure parameters which are
variables.
\p 
For details of the \hyperlink{card:L}{\cds{L}} which drive such LSQ programs the
user  should consult the specification in Chapter 3.  An example of the Crystal
Data  for \mlink{sflsq}{SFLSQ}\ is as follows: 
\p\ssk\begin{verbatim} 
I NCYC 5   PRIN 3   CONV 0.02
L RELA 2   2 Co SITE  1 Mn SITE  1 Sn SITE
L VARY   ALL SITE
L MODE 5     WGHT 1
L SCAL     1    1    1
L FIX  DOMR
N Co2Mnsn at room temp
S     -X,    -Y,     -Z
S  1/2+X, 1/2+Y,      Z
S      Y,     Z,      X
S     -Y,     X,     -Z
A Co   1/4   1/4   1/4    0.3115
A Mn     0     0     0    0.2842
A Sn   1/2   1/2   1/2    0.3386
F Co    1    0.2500
F Mn    1   -0.3730
F Sn    1    0.6228
C 6
E    1    100.    0.05
\end{verbatim}
\par   
The \hyperlink{card:C}{C}, \hyperlink{card:S}{\bd{S}}, 
\hyperlink{card:A}{\bd{A}}, \hyperlink{card:F}{\bd{F}}, and
\hyperlink{card:E}{\cds{E}}are the same 
as for other programs, giving 3
atoms in special positions in a cubic space group, with neutron nuclear
scattering factors, and type 1 extinction corrections on the structure
factors.
\p 
The \hyperlink{card:I}{\cd{I}} requests 5 cycles of refinement, with printing of 
the observed and calculated values of the function on the first 
and last cycles, and terminating before the
5th cycle if the largest shift/$\sigma$ is smaller than 0.02 (rather than the
default of 0.01).
\p
\hyperlink{card:L}{\cds{L}} are needed where defaults need to be changed, so \bd{L WGHT} 
asks for
unit weights rather than statistical weights, \bd{L MODE} asks for the special
\ital{extinction correction} input mode rather than the standard format for
observed input files, and so on.
\p 
Each LSQ main program has its own defaults built in as to whether a
given parameter should be fixed or varied.  In general most structure
parameters are by default varied, but site occupation factors are
by default fixed, so the \exac{L VARY ALL SITE} card is needed here.
\p 
It is required to relate all three site occupation factors by the
linear relation expressed on the \cd{L RELA}.
%
\subsection{Interpretation}
% 
The main program \mlink{sflsq}{SFLSQ} interprets the above Crystal Data and deduces
that all three atomic positions are so special that all their
coordinates are fixed.  It makes 9 basic variables, being:
\ssp 
\exac{MOSC, SCAL 1, SCAL 2, SCAL 3} (having deduced that there are 3 scale
zones from the \cd{L SCAL}) \exac{Co ITF, Mn ITF, Sn ITF} (the isotropic
temperature factors, varied by default) and \exac{Mn SITE} and \exac{Sn SITE}.
\p 
it makes 10 variables, being all the above plus \exac{Co SITE} (a redundant
variable), and records the constraint:
\ssp 
\cl{\exac{2* Co SITE + Mn SITE + Sn SITE = constant}}
\ssk 
By default, as there is no \cd{L REFI}, the program refines on the
modulus of the structure factor.  Another input dataset is needed, in the
format indicated by the card saying \exac{L MODE 5};
  the name of the file containing
this data set is requested interactively.
% 
\section{Making Other Least Squares Programs}
\markright{Making Other Least Squares Programs}
%
The existing main programs can be adapted to refine different parameters,
or to use a different calculated function in the refinement.  Examples
which already exist are:
\p
\mlink{mplsq}{MPLSQ}, multipole LSQ, in which the scattering density is expressed as a sum
of spherical harmonic (multipole) functions, and the coefficients of these
functions  may be refined.  A new family 5 has been defined for
them, as there may be different numbers of multipole coefficients for 
different atoms.
\p
\mlink{palsq}{PALSQ}, polarization analysis LSQ, in which the observations to be fitted
are the components of the scattered neutron polarization for a given input
polarization.
\p
Recently (April 2001) some changes have been made to make it easier to introduce new parameters
   into a least squares refinement based on structure factors. The author of the
   new program must provide:
\begin{enumerate}   
\item   		 An augmented main program in which the new parameters are defined
             this should be based on \mlink{sflsq}{SFLSQ}.
             The names of the new parameters (A4) must be added to ISFWRD
             and their specifications (family, genus, and species) to ISWDSP
\item       A subroutine called say DOXXX1 with 1 dummy agument MODE.
             When called with MODE=1 it should read all the data from the
             CDF which are needed to give values to the new parameters.
             When called with MODE=2 it should check that enough data have
             been read, and do any necessary tidying up.
             For every new parameter which can be refined there should be a
             corresponding integer pointer. This pointer will be set to the
             number  of the of the variable to which ithe parameter is
             assigned, or zero if it is fixed.
             DOXXX1 should be called in the main program with MODE=1 just after
             \stlink{s}{SETFC}, and with MODE=2 just before \stlink{p}{PARSSF}.
\item       A logical function called say DOXXX2 with two dummy arguments
             NN and MODE
\begin{description}             
\item[MODE = 1] Determine symmetry constraints on new parameters
                      NN is not used.
\item[MODE = 2] Set Variable pointers to new parameters
                 NN is the variable number
\item[MODE = 3] Apply shifts, one parameter per entry.
                      NN is not used
\item[MODE = 4] Write new parameter card (one per entry)
                      NN is the serial number of the letter on the card to
                      rewrite
\item [MODE = 5] Do any calculation required because of parameter changes 
                      at the end of a least squares cycle.
\item [MODE = 6] Interpret new words used to fix or vary groups of parameters
 when they apply to a particular genus.
 Such words should be defined in the main program as belonging to families
 with  negative numbers (<-10).
\item [MODE = 7] Interpret new words used to fix or vary groups of parameters
 when they apply to all such parameters (all genera or genus not applicable).
\end{description}     
\item  A subroutine called say LXXCAL to calculate the structure factors
             and their derivatives with respect to all the structural parameters
             (Only needed if the new parameters modify the structure factor)
\item   A subroutine called say CALXXX to calculate the measurement (GCALC)
             to be fitted, in terms of the structure factor and other
             experimental parameters.
             (Only needed if the new parameters change how this is done)
\item   A logical function called say DFTXXX which will be called  with
             the dummy arguments set to the family, genus and species of a 
             parameter. It should return TRUE if the parameter is to be varied.
\end{enumerate}             
        In the EXTERNAL declaration in the main program:

\begin{varindent}{3 cm}
                       LFCALC should be replaced by LXXCAL\\ 
                       DOTHER should be replaced by DOXXX2\\ 
                       DEFALT should be replaced by DFTXXX\\
\end{varindent} 
\par                      
        A call to CALXXX should replace that to \stlink{c}{CALCSF} inside the least squares
        cycles. It has two dummy arguments: H giving the reflection indices, 
        and the second LXXCAL being the structure factor calculation routine.      
        If, as for magnetic structures, two types of structure factors are to
        be calculated a third parameter gives the name of the required routine.
        It should be declared EXTERNAL. 
        For example for magnetic least squares the call becomes:
        call CALCMG(H,LFCALC,LMCALC)\\
        \stlink{l}{LFCALC} calculates the nuclear structure factors and their
        derivatives.\\
        \stlink{l}{LMCALC} calculates the magnetic structure factors and their
        derivatives.\\
        \stlink{c}{CALCMG} combines them into GCALC to match the observed quantity GOBS.\\ 
\p        
        Note. It is convenient to group the main program and the 
        special subroutines in a single file to be included in the main program
        section of the master file.
%        
\section{Structure Factor Least Squares with multiple Data Sources}
\markright{ Multiple Data Sources}
As of Version 4.4 of CCSL the set of structure factor least squares programs have been reorganised to allow
data from more than one source to be refined concurrently. To facilitate this,
provision has been made to attach source dependent information such as crystal orientation, wavelength, neutron polarisation etc, to the files containing the experimental data. This information is found in lines at the start of the data files  beginning with a \#  followed by a word indicating the type of information and the information itself. The words following the \# which can currently be recognised are: 
\begin{description}
\item{Polarisation} followed by 5 numbers: the Up and Down polarising efficiencies
		and the polarisation direction in orthogonal crystal coordinates.
\item{Wavelength} followed by  its value and, if needed, the absorption 
		coefficient.
\item{Orientation} followed by 9 components of the matrix giving the directions of the diffractometer axes in orthogonal crystal coordinates.
\end{description}
Only the first 5 characters of the words are matched, their case is unimportant.

The CCSL  programs \mlink{absmsf}{ABSMSF}, \mlink{sorgam}{SORGAM}, \mlink{sorasy}{SORASY} etc which make 
files  to use as input to least
squares programs now automatically add these lines.
This information is used when necessary to calculate the parameters used in
Becker Coppens extinction so these no longer need be included with the data -
(L MODE=5 and L MODE=7 are now obsolete).\\
The data are now read only once, rather than once per least squares cycle as
in versions prior to 4.4.

The least squares programs \mlink{sflsq}{SFLSQ}, \mlink{mplsq}{MPLSQ}, \mlink{maglsq}{MAGLSQ}, \mlink{mmplsq}{MMPLSQ}, \mlink{chilsq}{CHILSQ}, \mlink{mpclsq}{MPCLSQ} and \mlink{mgtlsq}{MGTLSQ}
all now use this new structure. 
\begin{itemize}
\item \mlink{sflsq}{SFLSQ}  and \mlink{mplsq}{MPLSQ}  can only read structure factor type data (REFI 1-4)
\item \mlink{chilsq}{CHILSQ} and \mlink{mpclsq}{MPCLSQ} normally only read polarised neutron data (REFI=5 or 7)
but in both cases data measured with different orientations or wavelengths
can be refined together. It is sometimes useful to include unpolarised structure factor data
measured with an applied field as well. \\
Update 4.4.17 allows polarised neutron data from  polycrystalline samples of anisotropic
paramagnets to be refined (MODE 11, REFI 9)

\item \mlink{maglsq}{MAGLSQ}, \mlink{mmplsq}{MMPLSQ} and \mlink{mgtlsq}{MGTLSQ} all allow integrated intensity and polarised neutron data to be combined. (REFI=1-4,5,and 7)
\end{itemize}
When there are multiple data sets, each is presented in a separate file and the 
values of MODE and REFI for each as well as its allocated weight may be given at the same time as the file-name. 
In this case the request for the data file name has the form:\\
{\tt Name of data set, its MODE, REFI and weight? :}\\
If the reply given is just the filename then the program assumes there is just a
 single source of data: \bd{MODE} and \bd{REFI} are taken from the crystal data file 
 as in the pre-4.4 versions.
 \subsection{SNPLSQ}
 In the new program \mlink{snplsq}{SNPLSQ}, which allows SNP and structure factor data to be refined
 together, provision has to be made for different scale factors and different
 magnetic domain populations to be associated with each data-set. These quantities must 
 also be identified with valid least squares parameters. Each data set is therefore identified by a
 {\em name} of up to 4 characters eg \bd{PA27} which is given on a 
 \hyperlink{Q:sorc}{\cd{L SORC}}. \bd{SORC}
 may be followed by the filename given with a path which is either absolute (starts with ``/")
 or relative to the current working directory. Environment variables are recognised.
 If no filename follows the identifier, the file name will be requested interactively.
 For each data source identified by {\em name}  there should be one or more \cds{L SORC {\em name}} 
 giving further information specific to these data: \bd{REFI}, \bd{MODE} and \bd{WGHT} are obligatory. The
integers following REFI and MODE have their usual meanings but apply only to this particular data-set. The number following \bd{WGHT} gives the weight for these data. 
\p
There should also
be \cds{L SORC SCAL} and \cds{L SORC DPOP} giving the scale factors and domain populations
for the data set. The number of scale factors needed is the number of scaling zones
given in the data file. The number of domains whose populations must be given depends
on the relative symmetries of the nuclear and magnetic structures and on the type of refinement.  
The new least squares parameters associated with the scales and domain populations are identified by the genus {\em name} and a composite species name starting with either \bd{SC} for a scale factor
or \bd{DP} for a domain population and ending with 2 digits identifying its number.
\subsection{SNPLSQ: Example L DATA and L SORC Cards}  
\p\ssk\begin{verbatim} 
L DATA PA27 $TT/khe27k.pal 
L SORC PA27 MODE 9 REFI 8 WGHT 1.0 
L SORC PA27 DPOP  0.18  0.07  0.07 0.18  0.18 0.07 0.07 0.18   
L DATA SF30 $TT/khe30k.sf 
L SORC SF30 MODE 7 REFI 1 WGHT 0.3 SCAL 7.16
L SORC SF30 DPOP 0.25 0.25 0.25 0.25
\end{verbatim}
The data are for a helical structure with 4 orientation domains. PA27 contains
SNP data (REFI 8) which is sensitive to the chirality so 8 domain populations are needed.
SF27 is a set of structure factor amplitudes (REFI 1) insensitive to chirality so only 4
domain populations can be refined. Pairs of chiral domains are numbered consecutively
so only odd-numbered domains are defined for the SF27 data. Constraints might therefore be 
given as follows:
\begin{verbatim} 
Z Domain constraints for PA data
L RELA 1 1 PA27 DP08 1 PA27 DP01
L RELA 1 1 PA27 DP07 1 PA27 DP02
L RELA 1 1 PA27 DP06 1 PA27 DP03
L RELA 1 1 PA27 DP05 1 PA27 DP04
L RELA 2 1 PA27 DP01 1 PA27 DP02 1 PA27 DP03 1 PA27 DP04 
Z Domain constraints for SF data
L RELA 2 1 SF30 DP01 1 SF30 DP03 1 SF30 DP05 1 SF30 DP07  
\end{verbatim}
No scale factors are needed for the SNP data; one is needed for the structure amplitudes.
It could be fixed using:
\begin{verbatim} 
L FIX  SF30 SC01
\end{verbatim}
\p
 For least squares refinements using different types of data, the weight given
 to each is important. A rough rule of thumb is to make the relative weights
 inversely proportional to the residual $\chi^2$ of each when refined separately.
%
\finchapter
%\end{document}

%\end{htmlonly}
%\internal{c2}
%\internal{c3}
%\internal{c4}
%\internal{c1}
%\internal{c6}
%\internal{c7}
\startdocument
%\htmladdtonavigation{\htmladdnormallink
%  {\htmladdimg{../icons/appenx.gif}}
%  {../appenx/appendix}}
\label{chap:5}
\markboth{Least Squares Refinement}{}
\section{What We Mean by LSQ}\markright{What We Mean by LSQ}
% 
The term LSQ is used here to mean \ital{Least Squares Refinement}, 
which uses
observations of a function to improve the values of some set of
parameters on which that function depends.
\p 
Refinement involving a crystal structure could be either \ital{standard} single 
crystal LSQ
refinement, in which each observation depends on a single
structure factor, or \ital{profile refinement} (PR) for which
several structure factors contribute to one observation.
\p 
The PR routines are written to use many of the same CCSL routines as do 
standard LSQ
main programs.  The PR programs and Libraries are available separately, with 
a separate Manual.
\p 
CCSL also deals with simpler LSQ problems.  For example, the main program
FWLSQ fits the 5,7 or 9 parameters of the exponential approximation to a 
scattering factor curve. The simplicity in this case is that no crystal structure is
involved.
Another simple case is the refinement of (up to) 6 cell parameters,
given an observed list of d-spacings (actually d* squared values, in
the main program DSLSQ).  This must deal with
the constraints which are
necessarily imposed on the cell parameters by the space group symmetry.
\p 
The essentials of a LSQ problem are:\p
\begin{list} {} {\setlength{\labelwidth}{8mm}
  \setlength{\parsep}{-1ex}
  \setlength{\leftmargin}{\labelwidth}
 \addtolength{\leftmargin}{4mm}}
\item[a) \hfill] a set of \ital{observations} of something, 
(the \ital{observed function})
usually with their \ital{estimated standard deviations,} $\sigma$'s,
measured at different values of some argument \ital{ARG};
\ital{ARG} may be $h,k,l$ (for standard LSQ), $\sin\theta/\lambda$, 
$\theta$ (for
Rietveld PR) or $\lambda$, or energy, or time of flight; it is a
quantity which takes different values, and will identify the
observation.\\[1mm]
For crystallographic applications the observations are often all of the
same physical thing, but this is not necessary.  For example, geometric
constraints may be introduced by giving bond lengths and/or angles as
additional types of observation, with $\sigma$'s.
\item[b) \hfill]  some \ital{calculated function} involving \ital{ARG},
which defines a
mathematical model to be compared with an observation.  This calculated
function depends on \ital{parameters}, things which may possibly be varied in order
to improve the fit of the function to its related observations.\\[1mm]
By \ital{fit} we mean minimisation of the weighted sum of squares of 
differences
between observed and calculated function values (where the summing is
over the different values of \ital{ARG}).  It is desirable to use 
statistical weights ($1/\sigma^2$)
for the differences.\\[1mm]
The theory may be viewed as using the beginning of a Taylor series
expansion, and therefore requires that the parameters be close to
their \ital{correct} values, making the required shifts so small that their
squares may be neglected.\end{list}
 
\section{Parameters and Variables}\markright{Parameters and Variables}
% 
\subsection{Parameters}
In LSQ a \ital{parameter} is something which it is sensible to try to
adjust. It represents some physical thing like ``the x coordinate of the 3rd
atom'', ``an overall temperature factor'' or ``the scale factor for all
observations coded number 2''.
\p 
A \ital{parameter} is also something with respect to which the function may be
differentiated.  Differentiation may be analytic, if it is possible (and
sensible) to write down such derivatives algebraically.  It may also be
numerical in awkward cases, using a simple approximation to the
derivative involving function values.
\p 
For any particular run of a LSQ refinement program, it is unlikely that
the user will want to vary every parameter.  The subset of parameters
which are actually to be varied (i.e. those for which shifts are
required) we call \ital{variables}.
\p 
Parameters are thus either \ital{fixed} or \ital{varied}.
For a fixed parameter,
there is no need for a derivative; but for all parameters to be
varied, derivatives will be needed.
% 
\subsection{Basic Variables}
% 
It is often the case that parameters are related, either inherently by
the symmetry of the problem, or by a strict constraint applied by the user.  If
two variables are related, they must not both be refined by LSQ;  the
process expects the shifts to be independent.  Every constraint or
relation reduces by 1 the number of variables to be refined.
\p 
The subset of variables for which shifts are actually calculated
we call \ital{basic variables}, or basics for short.
\p 
Those variables which are not basic we call \ital{redundant}. 
Thus if a problem has $n_p$ parameters, $n_f$ of 
which are fixed and $n_v$ varied,
               $$n_p = n_f + n_v$$
and if, of those $n_v$, $n_b$ are basic and $n_r$ are redundant,
              $$n_v = n_b + n_r$$
% 
\subsection{Strict Constraints}
% 
The existence of $n_r$ redundant variables implies that there are $n_r$
strict constraints on the problem. Derivatives are calculated 
(and shifts are required) for all
variables, not just the basics.
\p
Those constraints which must be imposed because of the crystallographic
symmetry are generated automatically by CCSL, and need not be provided by
the user.  For example, the  special position $(x,x,x)$ will give rise to
one basic variable, $x$, with two redundant variables $y$ and $z$ related to
it by two constraints, but the user will not need to tell the system
this.\p
The user may impose additional strict constraints.  Two constraint types
are at present available.  The simpler is type 1:
             $$a_1\Delta p_1 = a_2\Delta p_2$$
that is, constant $a_1$ times the shift in parameter 1 = constant $a_2$ times
the shift in parameter 2.\p
Type 2 constraints involve a linear combination of parameters, with
constant coefficients, thus:
 $$  a_1\Delta p_1 + a_2\Delta p_2 + a_3\Delta p3 + \ldots\ \mbox{etc}\
   = 0$$
Note that the way these expressions are written involves differentiating
the original constraint.  For example, if a type 2 constraint is
needed to fix the sum of three parameters to be 3, the constraint
$$p_1 + p_2 + p_3 = 3\quad\mbox{becomes}\quad \Delta p_1 + 
\Delta p_2 + \Delta p_3 = 0$$
Other more complicated types of constraint could be added if needed.
%
\subsection{Parameter Names}
% 
The need to fix, vary or relate parameters implies the need to call
these parameters by name.  The CCSL LSQ facilities allow each problem to
be set up individually.  Names for the parameters of a problem are
given in its main program;  the user needs to know what these names are
in order to refer to the parameters in his Crystal Data.
\p
Examples of names used in various standard main programs are:
\ssp
\verb| Na2 X  Ca B11  SCAL 2  P23 ITF   A*|
\ssp 
Most names have two parts, which we call the genus and species names;  the
genus names above are \exac{\ Na2, Ca, SCAL, P23}, with corresponding
species names \exac{\ X, B11, 2} and \exac{ITF}.  The name \exac{A*} 
is simply a species name.
\p 
In LSQ programs which involve structure factors,
those structure parameters which belong to a particular named atom
have the atom name as genus name, and one of:
\ssp 
\verb|X   Y   Z   B11   B12   B13   B22   B23   B33   ITF  SITE  SCAT|
\ssp 
as species name.  
(\exac{SCAT} means the scattering factor, when this is represented 
by one number and it is sensible to refine it).
\p
MAGLSQ has the additional species names:
\ssp
\exac{MU   MU1   THET   THE1   PHI   PHI1   PSI1   PSI2   PSI3   PSI4}.
\p
When LSQ programs refer to parameters in output for shifts, correlations etc,
they use these parameter names.
\p 
More details about parameter
names are to be found in Chapter 3 under \cds{\htmlref{L FIX}{ss:fix}}.  
Note the availability
of words like \bd{XYZ} to mean ``all three x,y,z coordinates", 
\bd{CELL} to mean ``all six
cell parameters", etc.  One may also use the word \bd{ALL} 
to mean ``all the members
of this genus" (as in \exac{ALL Na3} for ``all the parameters of atom Na3")
 or ``all
the genera with this species" (as in \exac{ALL X}, or \exac{ALL SITE}, or even 
\exac{ALL XYZ}).
%
\subsection{Families of Parameters}
% 
Structure parameters such as those introduced in the previous section,
being a commonly occurring group in LSQ, are said in CCSL to belong to 
\ital{family
2} of the parameters.  \ital{Family~1} contains all other
parameters in the simple LSQ applications we are considering here.  Other
families are used in PR, and in other LSQ programs.
\p 
Family 1, genus 1 is treated by CCSL as special.  It
contains the parameters for which the species name by itself is enough,
like \exac{A*} above;  another example is an overall temperature factor for a
structure, known as \exac{TFAC}.
\p 
Genera 2, 3 etc of family 1 may in general be given any genus name.  Within
such a genus the species name could be simply an integer, or species could have 
individual names.  This choice is
for the writer of the main program.  In \mlink{sflsq}{SFLSQ}\ there are two genera in
family 1, the second being named \exac{SCAL} with species numbered 1,2 etc.
\p 
Unless the user is writing or modifying a main program, he need not be
concerned with the detailed mechanism for numbering parameters.  He only
needs to know what names he is allowed to use on his Crystal Data
for the programs he runs.
\p 
\section{Examples of Least squares programs}
\markright{Examples of Least squares programs}
%
We illustrate these general ideas by, first, a simple example of LSQ
which does not involve structure parameters, and, second, notes on the
use of the standard single crystal structure refinement program, SFLSQ.
\p
There are two fundamental questions to be asked of a main LSQ program:
\begin{enumerate}
\item on which calculated function is it refining 
\item which parameters may be refined.
\end{enumerate}
%
\subsection{DSLSQ}
% 
DSLSQ is a main program which determines the values of up to 6 cell
parameters by LSQ fitting, using values of d spacings observed at 
different values of h,k,l.

The calculated function used is the expression for the square of d*, 
in reciprocal space:
$$d^*(h,k,l)^2 = (h^2a^{*2} + k^2b^{*2} + l^2c^{*2} +
         2kb^*l c^*\cos\alpha^* + 2lc^*ha^*\cos\beta^*
 + 2ha^*kb^*\cos\gamma^*)
$$
The parameters to be varied are the reciprocal cell quadratic products
A*, B*, C*, D*, E* and F*, where A*=$a^{*2}$, D*=$b^*c^*\cos\alpha^*$, etc.
\p
The author of the main program had the choice of names for these
parameters, and decided to make them members of family 1, genus 1,
having species names A*, B* etc.  Alternatively, she could have made 
them members of
family 1, genus 2, with genus name CELL, and species names the integers
1 to 6.  There are no other parameters in this simple example.
\p 
The calculated function $G_{calc}(h,k,l)$ has been programmed into a routine
CALCDS, which is given the values of h,k,l, and returns the value of 
$G_{calc}$ as
defined above, together with all derivatives of $G_{calc}$ with respect to
the relevant cell parameters. If a user happened to have data which were
observations of, say, d rather than d* squared, CALCDS could be simply
rewritten to calculate instead $G_{calc}(h,k,l)=d$ and its derivatives.
\p 
Other routines which have been written for this specific application are:\\
\stlink{a}{APSHDS} to apply shifts to the parameters,\\
\stlink{n}{NWINDS} to output new Crystal Data containing the shifted parameters,\\
\stlink{p}{PARSDS} to decide which are the parameters of the problem, and\\
\stlink{v}{VARSDS} to set up which parameters are variables.
\p 
The routines APSHDS, CALCDS, NWINDS, PARSDS and VARSDS may be inspected
in a listing of CCSL if further detail is sought.
% 
\subsection{Structure Factor LSQ}
\mlink{sflsq}{SFLSQ}\ (standard LSQ refinement with possible geometric constraints),
GRLSQ (taking groups of input reflections together) and \mlink{maglsq}{MAGLSQ}\ 
(for magnetic structures) are all examples of CCSL single-crystal LSQ programs.
\p 
Crystallographic LSQ involving structure factors follows much the same
pattern as the DSLSQ example above.  The calculated function involves a
structure factor computation by a routine such as LFCALC, during which 
derivatives are made with respect to all structure parameters which are
variables.
\p 
For details of the \hyperlink{card:L}{\cds{L}} which drive such LSQ programs the
user  should consult the specification in Chapter 3.  An example of the Crystal
Data  for \mlink{sflsq}{SFLSQ}\ is as follows: 
\p\ssk\begin{verbatim} 
I NCYC 5   PRIN 3   CONV 0.02
L RELA 2   2 Co SITE  1 Mn SITE  1 Sn SITE
L VARY   ALL SITE
L MODE 5     WGHT 1
L SCAL     1    1    1
L FIX  DOMR
N Co2Mnsn at room temp
S     -X,    -Y,     -Z
S  1/2+X, 1/2+Y,      Z
S      Y,     Z,      X
S     -Y,     X,     -Z
A Co   1/4   1/4   1/4    0.3115
A Mn     0     0     0    0.2842
A Sn   1/2   1/2   1/2    0.3386
F Co    1    0.2500
F Mn    1   -0.3730
F Sn    1    0.6228
C 6
E    1    100.    0.05
\end{verbatim}
\par   
The \hyperlink{card:C}{C}, \hyperlink{card:S}{\bd{S}}, 
\hyperlink{card:A}{\bd{A}}, \hyperlink{card:F}{\bd{F}}, and
\hyperlink{card:E}{\cds{E}}are the same 
as for other programs, giving 3
atoms in special positions in a cubic space group, with neutron nuclear
scattering factors, and type 1 extinction corrections on the structure
factors.
\p 
The \hyperlink{card:I}{\cd{I}} requests 5 cycles of refinement, with printing of 
the observed and calculated values of the function on the first 
and last cycles, and terminating before the
5th cycle if the largest shift/$\sigma$ is smaller than 0.02 (rather than the
default of 0.01).
\p
\hyperlink{card:L}{\cds{L}} are needed where defaults need to be changed, so \bd{L WGHT} 
asks for
unit weights rather than statistical weights, \bd{L MODE} asks for the special
\ital{extinction correction} input mode rather than the standard format for
observed input files, and so on.
\p 
Each LSQ main program has its own defaults built in as to whether a
given parameter should be fixed or varied.  In general most structure
parameters are by default varied, but site occupation factors are
by default fixed, so the \exac{L VARY ALL SITE} card is needed here.
\p 
It is required to relate all three site occupation factors by the
linear relation expressed on the \cd{L RELA}.
%
\subsection{Interpretation}
% 
The main program \mlink{sflsq}{SFLSQ} interprets the above Crystal Data and deduces
that all three atomic positions are so special that all their
coordinates are fixed.  It makes 9 basic variables, being:
\ssp 
\exac{MOSC, SCAL 1, SCAL 2, SCAL 3} (having deduced that there are 3 scale
zones from the \cd{L SCAL}) \exac{Co ITF, Mn ITF, Sn ITF} (the isotropic
temperature factors, varied by default) and \exac{Mn SITE} and \exac{Sn SITE}.
\p 
it makes 10 variables, being all the above plus \exac{Co SITE} (a redundant
variable), and records the constraint:
\ssp 
\cl{\exac{2* Co SITE + Mn SITE + Sn SITE = constant}}
\ssk 
By default, as there is no \cd{L REFI}, the program refines on the
modulus of the structure factor.  Another input dataset is needed, in the
format indicated by the card saying \exac{L MODE 5};
  the name of the file containing
this data set is requested interactively.
% 
\section{Making Other Least Squares Programs}
\markright{Making Other Least Squares Programs}
%
The existing main programs can be adapted to refine different parameters,
or to use a different calculated function in the refinement.  Examples
which already exist are:
\p
\mlink{mplsq}{MPLSQ}, multipole LSQ, in which the scattering density is expressed as a sum
of spherical harmonic (multipole) functions, and the coefficients of these
functions  may be refined.  A new family 5 has been defined for
them, as there may be different numbers of multipole coefficients for 
different atoms.
\p
\mlink{palsq}{PALSQ}, polarization analysis LSQ, in which the observations to be fitted
are the components of the scattered neutron polarization for a given input
polarization.
\p
Recently (April 2001) some changes have been made to make it easier to introduce new parameters
   into a least squares refinement based on structure factors. The author of the
   new program must provide:
\begin{enumerate}   
\item   		 An augmented main program in which the new parameters are defined
             this should be based on \mlink{sflsq}{SFLSQ}.
             The names of the new parameters (A4) must be added to ISFWRD
             and their specifications (family, genus, and species) to ISWDSP
\item       A subroutine called say DOXXX1 with 1 dummy agument MODE.
             When called with MODE=1 it should read all the data from the
             CDF which are needed to give values to the new parameters.
             When called with MODE=2 it should check that enough data have
             been read, and do any necessary tidying up.
             For every new parameter which can be refined there should be a
             corresponding integer pointer. This pointer will be set to the
             number  of the of the variable to which ithe parameter is
             assigned, or zero if it is fixed.
             DOXXX1 should be called in the main program with MODE=1 just after
             \stlink{s}{SETFC}, and with MODE=2 just before \stlink{p}{PARSSF}.
\item       A logical function called say DOXXX2 with two dummy arguments
             NN and MODE
\begin{description}             
\item[MODE = 1] Determine symmetry constraints on new parameters
                      NN is not used.
\item[MODE = 2] Set Variable pointers to new parameters
                 NN is the variable number
\item[MODE = 3] Apply shifts, one parameter per entry.
                      NN is not used
\item[MODE = 4] Write new parameter card (one per entry)
                      NN is the serial number of the letter on the card to
                      rewrite
\item [MODE = 5] Do any calculation required because of parameter changes 
                      at the end of a least squares cycle.
\item [MODE = 6] Interpret new words used to fix or vary groups of parameters
 when they apply to a particular genus.
 Such words should be defined in the main program as belonging to families
 with  negative numbers (<-10).
\item [MODE = 7] Interpret new words used to fix or vary groups of parameters
 when they apply to all such parameters (all genera or genus not applicable).
\end{description}     
\item  A subroutine called say LXXCAL to calculate the structure factors
             and their derivatives with respect to all the structural parameters
             (Only needed if the new parameters modify the structure factor)
\item   A subroutine called say CALXXX to calculate the measurement (GCALC)
             to be fitted, in terms of the structure factor and other
             experimental parameters.
             (Only needed if the new parameters change how this is done)
\item   A logical function called say DFTXXX which will be called  with
             the dummy arguments set to the family, genus and species of a 
             parameter. It should return TRUE if the parameter is to be varied.
\end{enumerate}             
        In the EXTERNAL declaration in the main program:

\begin{varindent}{3 cm}
                       LFCALC should be replaced by LXXCAL\\ 
                       DOTHER should be replaced by DOXXX2\\ 
                       DEFALT should be replaced by DFTXXX\\
\end{varindent} 
\par                      
        A call to CALXXX should replace that to \stlink{c}{CALCSF} inside the least squares
        cycles. It has two dummy arguments: H giving the reflection indices, 
        and the second LXXCAL being the structure factor calculation routine.      
        If, as for magnetic structures, two types of structure factors are to
        be calculated a third parameter gives the name of the required routine.
        It should be declared EXTERNAL. 
        For example for magnetic least squares the call becomes:
        call CALCMG(H,LFCALC,LMCALC)\\
        \stlink{l}{LFCALC} calculates the nuclear structure factors and their
        derivatives.\\
        \stlink{l}{LMCALC} calculates the magnetic structure factors and their
        derivatives.\\
        \stlink{c}{CALCMG} combines them into GCALC to match the observed quantity GOBS.\\ 
\p        
        Note. It is convenient to group the main program and the 
        special subroutines in a single file to be included in the main program
        section of the master file.
%        
\section{Structure Factor Least Squares with multiple Data Sources}
\markright{ Multiple Data Sources}
As of Version 4.4 of CCSL the set of structure factor least squares programs have been reorganised to allow
data from more than one source to be refined concurrently. To facilitate this,
provision has been made to attach source dependent information such as crystal orientation, wavelength, neutron polarisation etc, to the files containing the experimental data. This information is found in lines at the start of the data files  beginning with a \#  followed by a word indicating the type of information and the information itself. The words following the \# which can currently be recognised are: 
\begin{description}
\item{Polarisation} followed by 5 numbers: the Up and Down polarising efficiencies
		and the polarisation direction in orthogonal crystal coordinates.
\item{Wavelength} followed by  its value and, if needed, the absorption 
		coefficient.
\item{Orientation} followed by 9 components of the matrix giving the directions of the diffractometer axes in orthogonal crystal coordinates.
\end{description}
Only the first 5 characters of the words are matched, their case is unimportant.

The CCSL  programs \mlink{absmsf}{ABSMSF}, \mlink{sorgam}{SORGAM}, \mlink{sorasy}{SORASY} etc which make 
files  to use as input to least
squares programs now automatically add these lines.
This information is used when necessary to calculate the parameters used in
Becker Coppens extinction so these no longer need be included with the data -
(L MODE=5 and L MODE=7 are now obsolete).\\
The data are now read only once, rather than once per least squares cycle as
in versions prior to 4.4.

The least squares programs \mlink{sflsq}{SFLSQ}, \mlink{mplsq}{MPLSQ}, \mlink{maglsq}{MAGLSQ}, \mlink{mmplsq}{MMPLSQ}, \mlink{chilsq}{CHILSQ}, \mlink{mpclsq}{MPCLSQ} and \mlink{mgtlsq}{MGTLSQ}
all now use this new structure. 
\begin{itemize}
\item \mlink{sflsq}{SFLSQ}  and \mlink{mplsq}{MPLSQ}  can only read structure factor type data (REFI 1-4)
\item \mlink{chilsq}{CHILSQ} and \mlink{mpclsq}{MPCLSQ} normally only read polarised neutron data (REFI=5 or 7)
but in both cases data measured with different orientations or wavelengths
can be refined together. It is sometimes useful to include unpolarised structure factor data
measured with an applied field as well. \\
Update 4.4.17 allows polarised neutron data from  polycrystalline samples of anisotropic
paramagnets to be refined (MODE 11, REFI 9)

\item \mlink{maglsq}{MAGLSQ}, \mlink{mmplsq}{MMPLSQ} and \mlink{mgtlsq}{MGTLSQ} all allow integrated intensity and polarised neutron data to be combined. (REFI=1-4,5,and 7)
\end{itemize}
When there are multiple data sets, each is presented in a separate file and the 
values of MODE and REFI for each as well as its allocated weight may be given at the same time as the file-name. 
In this case the request for the data file name has the form:\\
{\tt Name of data set, its MODE, REFI and weight? :}\\
If the reply given is just the filename then the program assumes there is just a
 single source of data: \bd{MODE} and \bd{REFI} are taken from the crystal data file 
 as in the pre-4.4 versions.
 \subsection{SNPLSQ}
 In the new program \mlink{snplsq}{SNPLSQ}, which allows SNP and structure factor data to be refined
 together, provision has to be made for different scale factors and different
 magnetic domain populations to be associated with each data-set. These quantities must 
 also be identified with valid least squares parameters. Each data set is therefore identified by a
 {\em name} of up to 4 characters eg \bd{PA27} which is given on a 
 \hyperlink{Q:sorc}{\cd{L SORC}}. \bd{SORC}
 may be followed by the filename given with a path which is either absolute (starts with ``/")
 or relative to the current working directory. Environment variables are recognised.
 If no filename follows the identifier, the file name will be requested interactively.
 For each data source identified by {\em name}  there should be one or more \cds{L SORC {\em name}} 
 giving further information specific to these data: \bd{REFI}, \bd{MODE} and \bd{WGHT} are obligatory. The
integers following REFI and MODE have their usual meanings but apply only to this particular data-set. The number following \bd{WGHT} gives the weight for these data. 
\p
There should also
be \cds{L SORC SCAL} and \cds{L SORC DPOP} giving the scale factors and domain populations
for the data set. The number of scale factors needed is the number of scaling zones
given in the data file. The number of domains whose populations must be given depends
on the relative symmetries of the nuclear and magnetic structures and on the type of refinement.  
The new least squares parameters associated with the scales and domain populations are identified by the genus {\em name} and a composite species name starting with either \bd{SC} for a scale factor
or \bd{DP} for a domain population and ending with 2 digits identifying its number.
\subsection{SNPLSQ: Example L DATA and L SORC Cards}  
\p\ssk\begin{verbatim} 
L DATA PA27 $TT/khe27k.pal 
L SORC PA27 MODE 9 REFI 8 WGHT 1.0 
L SORC PA27 DPOP  0.18  0.07  0.07 0.18  0.18 0.07 0.07 0.18   
L DATA SF30 $TT/khe30k.sf 
L SORC SF30 MODE 7 REFI 1 WGHT 0.3 SCAL 7.16
L SORC SF30 DPOP 0.25 0.25 0.25 0.25
\end{verbatim}
The data are for a helical structure with 4 orientation domains. PA27 contains
SNP data (REFI 8) which is sensitive to the chirality so 8 domain populations are needed.
SF27 is a set of structure factor amplitudes (REFI 1) insensitive to chirality so only 4
domain populations can be refined. Pairs of chiral domains are numbered consecutively
so only odd-numbered domains are defined for the SF27 data. Constraints might therefore be 
given as follows:
\begin{verbatim} 
Z Domain constraints for PA data
L RELA 1 1 PA27 DP08 1 PA27 DP01
L RELA 1 1 PA27 DP07 1 PA27 DP02
L RELA 1 1 PA27 DP06 1 PA27 DP03
L RELA 1 1 PA27 DP05 1 PA27 DP04
L RELA 2 1 PA27 DP01 1 PA27 DP02 1 PA27 DP03 1 PA27 DP04 
Z Domain constraints for SF data
L RELA 2 1 SF30 DP01 1 SF30 DP03 1 SF30 DP05 1 SF30 DP07  
\end{verbatim}
No scale factors are needed for the SNP data; one is needed for the structure amplitudes.
It could be fixed using:
\begin{verbatim} 
L FIX  SF30 SC01
\end{verbatim}
\p
 For least squares refinements using different types of data, the weight given
 to each is important. A rough rule of thumb is to make the relative weights
 inversely proportional to the residual $\chi^2$ of each when refined separately.
%
\finchapter
%\end{document}
