\documentclass[onecolumn,12pt,a4paper,]{report}
\usepackage{CCSLman,makeidx,color,ifthen,bm}
\makeatletter
\makeatletter
\@ifpackageloaded{tex4ht}
  {\let\iftexforht\@firstoftwo}
  {\let\iftexforht\@secondoftwo}
\makeatother
%
\iftexforht{\usepackage{here}}{\usepackage[style=plaintop]{floatrow}}
\usepackage{hyperref}
\iftexforht{\newcommand{\tablesize}{\normalsize}}{\newcommand{\tablesize}{\footnotesize}}
\usepackage{hyperref}

%
\newenvironment{clist}{\begin{list}{}{\setlength{\labelwidth}{2.5cm}
  \setlength{\itemsep}{-4ex}
  \setlength{\leftmargin}{\labelwidth}
 \addtolength{\leftmargin}{2cm}}}{\end{list}}

\pagestyle{empty}\usepackage{CCSLman}

%\input{versiondate}
%\newcommand{\versiondate}{\version}
%
%
%\html{\ccslstyles{../CCSLstyles.css}}
\title{Analytical approximations to Magnetic Form Factors}
\author{P.J. Brown}
\makeindex
\setlength{\topmargin}{0mm}
\def\kpv{{\bf{q}}}
\def\kp{q}
\def\kpu{\hat{\bf{q}}}
\def\rv{{\bf r}}
\def\ru{\hat{\bf r}}
\def\hf{\frac{1}{2}}
\def\Mv{{\bf M}}
\def\mv{{\bf m}}
\def\stl{$\sin\theta/\lambda$}
\def\inA{\AA{$^{-1}$}}
\def\n{\relax}
\def\p{\par}
\begin{document}
\maketitle
\chapter{Magnetic Form Factors}
\section{Definition of form factors}
The magnetic form factor  $f(\kpv)$ is obtained from the fourier transform of
the  magnetisation distribution of a single magnetic atom. Assuming that it
has a unique magnetisation direction it can be written
\begin{equation}
\Mv\int 
m(\rv)e^{i\kpv\cdot\rv}\,dr=\Mv f(\kpv)
\end{equation}
where \Mv\ gives the magnitude and direction of the moment and $m(\rv)$ is
a normalised scalar function  which describes how the intensity of magnetisation 
varies over the volume of the atom.
When the magnetisation arises from electrons in a single open shell the magnetic
form factor can be calculated from the radial distribution of the electrons in
that shell. The integrals from which the form factors are obtained have the form
\begin{equation}
\left\langle j_L(\kp)\right\rangle= \int 
U^2(r)j_l(\kp r)4\pi r^2\,dr
\end{equation}
The $j_l$ are the spherical Bessel functions defined by
\begin{equation}
j_l(x)={\sqrt{\pi\over2x}}J_{l+\frac12}(x)
\end{equation}
If the open shell has orbital quantum number $l$ the form factor 
for spin moment is 
\begin{equation}
f_s(\kpv)=\frac1{\mathrm{M}_S}\sum_{L=0}^{2l}i^L\left\langle 
j_L(\kp)\right\rangle
\sum_{M=-L}^L
S_{LM}Y^L_M\left({\kpu}\right)
\end{equation}
and that for orbital moment 
\begin{equation}
f_o(\kpv)=\frac1{\mathrm{M}_L}\sum_{L=0,2,\ldots}^{2l}
\left(\left\langle j_L(\kp)\right\rangle +
\left\langle j_{L+2}(\kp)\right\rangle\right)
\sum_{M=-L}^LB_{LM}Y^L_M(\kpu)
\end{equation}
\n
The coefficients $S_{LM}$, $B_{LM}$ have to be computed from the  orbital
wave-function \cite{Marshall}. The total spin moment ${\mathrm{M}_S}$ is  
given by $S_{00}$ 
and the orbital moment  ${\mathrm{M}_L}$ by $B_{00}$. For small $\kp$ the dipole
approximation 
\begin{equation}
f(\kpv)=({\mathrm{L+2S})}\left\langle j_0(\kp)\right\rangle
{\mathrm{L}}\left\langle j_2(\kp)\right\rangle
\end{equation}
can be used.
\\[2ex]
\section{Analytical approximation}
The tables which follow give the coefficients of an analytic approximation 
to the $\langle j_0\rangle$ integrals for the d electrons in ions of the 
\htmlref{3d}{3dj0} 
and \htmlref{4d}{4dj0}  and \htmlref{5d}{5dj0} series, the \htmlref{4f}{re4fj0} and \htmlref{5d}{re5dj0}
electrons of some rare earth ions and  the \htmlref{5f electrons }{acj0} of some actinide ions. 
For the 3d, 4d, 4f and 5f form factors the approximation has the form
\begin{equation}
\langle j_0(s)\rangle = A \exp(-as^2) + B \exp(-bs^2) + C \exp(-cs^2) + D
\end{equation}
For the data in the tables $s = \sin\theta/\lambda$ in units of \inA.
\\[2ex]
For the transition metal series the fits were made with 
form factor integrals  calculated from Hartree-Fock wave-functions \cite{clementi}. 
For
the rare-earth and actinide series the fits were with
Dirac-Fock form factors \cite{FandD,DandF}. 
There are some more recent calculations of form-factors for the 5d electrons
of the third group transition metals and of the rare-earth elements
\cite{kob:11,kob:12}. These authors used an expansion with seven rather than five terms
to fit the calculated form factors:
\begin{eqnarray}  
\langle j_0(s)\rangle& = &A \exp(-as^2) + B \exp(-bs^2) + C \exp(-cs^2)\nonumber\\
& &+ D \exp(-ds^2) + E\\
\end{eqnarray}
In the tables the number 
following the atom symbol indicates the ionisation state of the atom. 
Thus the coefficients following Fe0 are for a neutral iron atom and 
those following Fe2 are for Fe$^{2+}$.
\p
The integrals $\langle j_L\rangle$ with $L\ne0$ are zero when $s=0$ and have been
fitted with the forms 
\begin{equation}
\langle j_{L\ne 0}(s)\rangle = \left(A\exp(-as^2) + B\exp(-bs^2) + 
C\exp(-cs^2) + D\right)s^2
\end{equation}
or 
\begin{eqnarray}  
\langle j_{L\ne 0}(s)\rangle&= &\left(A\exp(-as^2) + B\exp(-bs^2) + C\exp(-cs^2)\right.\nonumber\\
& &  \left. + D\exp(-ds^2) + E \right)s^2
\end{eqnarray}

The second set of tables give the coefficients obtained for  
$\langle j2\rangle$ for \htmlref{3d}{3dj2}, \htmlref{4d}{4dj2} and \htmlref{5d}{5dj2} transition metals, 
\htmlref{4f}{re4fj2} and \htmlref{5d}{re5dj2} electrons of rare earths  and \htmlref{actinides}{acj2};
 $\langle j4\rangle$ for  \htmlref{3d}{3dj4}, \htmlref{4d}{4dj4} and \htmlref{5d}{5dj4} transition metals, 
\htmlref{rare earths}{rej4} and \htmlref{actinides}{acj4};
and $\langle j6\rangle$ for \htmlref{rare earths}{rej6} and \htmlref{actinides}{acj6}. 
\chapter{Tables of Form Factors}
\newpage
\section{{\Large$\langle j_0\rangle$ }Form factors for 3d transition elements and their ions}
\begin{table}[H]
\caption{\noindent $\langle j_0\rangle$ Form factors for 3d transition elements and their ions\hfill}
\label{3dj0}
\vspace{2mm}
{\tablesize
\begin{tabular}{lrrrrrrr}
\hline
Ion&
\multicolumn{1}{c}{A}&\multicolumn{1}{c}{a}&
\multicolumn{1}{c}{B}&\multicolumn{1}{c}{b}&
\multicolumn{1}{c}{C}&\multicolumn{1}{c}{c}&\multicolumn{1}{c}{D}\\
\hline\\[-2ex]
Sc0 &$0.2512$ &$90.0296$ &$0.3290$ &$39.4021$ &$0.4235$ &$14.3222$ &$-0.0043$ \\
Sc1 &$0.4889$ &$51.1603$ &$0.5203$ &$14.0764$ &$-0.0286$ &$0.1792$ &$0.0185$ \\
Sc2 &$0.5048$ &$31.4035$ &$0.5186$ &$10.9897$ &$-0.0241$ &$1.1831$ &$0.0000$ \\
Ti0 &$0.4657$ &$33.5898$ &$0.5490$ &$9.8791$ &$-0.0291$ &$0.3232$ &$0.0123$ \\
Ti1 &$0.5093$ &$36.7033$ &$0.5032$ &$10.3713$ &$-0.0263$ &$0.3106$ &$0.0116$ \\
Ti2 &$0.5091$ &$24.9763$ &$0.5162$ &$8.7569$ &$-0.0281$ &$0.9160$ &$0.0015$ \\
Ti3 &$0.3571$ &$22.8413$ &$0.6688$ &$8.9306$ &$-0.0354$ &$0.4833$ &$0.0099$ \\
V0 &$0.4086$ &$28.8109$ &$0.6077$ &$8.5437$ &$-0.0295$ &$0.2768$ &$0.0123$ \\
V1 &$0.4444$ &$32.6479$ &$0.5683$ &$9.0971$ &$-0.2285$ &$0.0218$ &$0.2150$ \\
V2 &$0.4085$ &$23.8526$ &$0.6091$ &$8.2456$ &$-0.1676$ &$0.0415$ &$0.1496$ \\
V3 &$0.3598$ &$19.3364$ &$0.6632$ &$7.6172$ &$-0.3064$ &$0.0296$ &$0.2835$ \\
V4 &$0.3106$ &$16.8160$ &$0.7198$ &$7.0487$ &$-0.0521$ &$0.3020$ &$0.0221$ \\
Cr0 &$0.1135$ &$45.1990$ &$0.3481$ &$19.4931$ &$0.5477$ &$7.3542$ &$-0.0092$ \\
Cr1 &$-0.0977$ &$0.0470$ &$0.4544$ &$26.0054$ &$0.5579$ &$7.4892$ &$0.0831$ \\
Cr2 &$1.2024$ &$-0.0055$ &$0.4158$ &$20.5475$ &$0.6032$ &$6.9560$ &$-1.2218$ \\
Cr3 &$-0.3094$ &$0.0274$ &$0.3680$ &$17.0355$ &$0.6559$ &$6.5236$ &$0.2856$ \\
Cr4 &$-0.2320$ &$0.0433$ &$0.3101$ &$14.9518$ &$0.7182$ &$6.1726$ &$0.2042$ \\
Mn0 &$0.2438$ &$24.9629$ &$0.1472$ &$15.6728$ &$0.6189$ &$6.5403$ &$-0.0105$ \\
Mn1 &$-0.0138$ &$0.4213$ &$0.4231$ &$24.6680$ &$0.5905$ &$6.6545$ &$-0.0010$ \\
Mn2 &$0.4220$ &$17.6840$ &$0.5948$ &$6.0050$ &$0.0043$ &$-0.6090$ &$-0.0219$ \\
Mn3 &$0.4198$ &$14.2829$ &$0.6054$ &$5.4689$ &$0.9241$ &$-0.0088$ &$-0.9498$ \\
Mn4 &$0.3760$ &$12.5661$ &$0.6602$ &$5.1329$ &$-0.0372$ &$0.5630$ &$0.0011$ \\
Fe0 &$0.0706$ &$35.0085$ &$0.3589$ &$15.3583$ &$0.5819$ &$5.5606$ &$-0.0114$ \\
Fe1 &$0.1251$ &$34.9633$ &$0.3629$ &$15.5144$ &$0.5223$ &$5.5914$ &$-0.0105$ \\
Fe2 &$0.0263$ &$34.9597$ &$0.3668$ &$15.9435$ &$0.6188$ &$5.5935$ &$-0.0119$ \\
Fe3 &$0.3972$ &$13.2442$ &$0.6295$ &$4.9034$ &$-0.0314$ &$0.3496$ &$0.0044$ \\
Fe4 &$0.3782$ &$11.3800$ &$0.6556$ &$4.5920$ &$-0.0346$ &$0.4833$ &$0.0005$ \\
Co0 &$0.4139$ &$16.1616$ &$0.6013$ &$4.7805$ &$-0.1518$ &$0.0210$ &$0.1345$ \\
Co1 &$0.0990$ &$33.1252$ &$0.3645$ &$15.1768$ &$0.5470$ &$5.0081$ &$-0.0109$ \\
Co2 &$0.4332$ &$14.3553$ &$0.5857$ &$4.6077$ &$-0.0382$ &$0.1338$ &$0.0179$ \\
Co3 &$0.3902$ &$12.5078$ &$0.6324$ &$4.4574$ &$-0.1500$ &$0.0343$ &$0.1272$ \\
Co4 &$0.3515$ &$10.7785$ &$0.6778$ &$4.2343$ &$-0.0389$ &$0.2409$ &$0.0098$ \\
Ni0 &$-0.0172$ &$35.7392$ &$0.3174$ &$14.2689$ &$0.7136$ &$4.5661$ &$-0.0143$ \\
Ni1 &$0.0705$ &$35.8561$ &$0.3984$ &$13.8042$ &$0.5427$ &$4.3965$ &$-0.0118$ \\
Ni2 &$0.0163$ &$35.8826$ &$0.3916$ &$13.2233$ &$0.6052$ &$4.3388$ &$-0.0133$ \\
Ni3 &$0.0012$ &$34.9998$ &$0.3468$ &$11.9874$ &$0.6667$ &$4.2518$ &$-0.0148 $ \\
Ni4 &$-0.0090$ &$35.8614$ &$0.2776$ &$11.7904$ &$0.7474$ &$4.2011$ &$-0.0163$ \\
Cu0 &$0.0909$ &$34.9838$ &$0.4088$ &$11.4432$ &$0.5128$ &$3.8248$ &$-0.0124$ \\
Cu1 &$0.0749$ &$34.9656$ &$0.4147$ &$11.7642$ &$0.5238$ &$3.8497$ &$-0.0127$ \\
Cu2 &$0.0232$ &$34.9686$ &$0.4023$ &$11.5640$ &$0.5882$ &$3.8428$ &$-0.0137$ \\
Cu3 &$0.0031$ &$34.9074$ &$0.3582$ &$10.9138$ &$0.6531$ &$3.8279$ &$-0.0147$ \\
Cu4 &$-0.0132$ &$30.6817$ &$0.2801$ &$11.1626$ &$0.7490$ &$3.8172$ &$-0.0165$ \\
\hline\\[-2ex]
\end{tabular}
}
\end{table}
\section{{\large$\langle j_0\rangle$ }Form factors for 4d transition elements and their ions}
\begin{table}[H]
\caption{
$\langle j_0\rangle$ form factors for 4d atoms and ions}\vspace{2mm}
\label{4dj0}
{\tablesize
\begin{tabular}{lrrrrrrr}
\hline
Ion&
\multicolumn{1}{c}{A}&\multicolumn{1}{c}{a}&
\multicolumn{1}{c}{B}&\multicolumn{1}{c}{b}&
\multicolumn{1}{c}{C}&\multicolumn{1}{c}{c}&\multicolumn{1}{c}{D}\\
\hline\\[-2ex]
Y0 &$0.5915$ &$67.6081$ &$1.5123$ &$17.9004$ &$-1.1130$ &$14.1359$ &$0.0080$ \\
Zr0 &$0.4106$ &$59.9961$ &$1.0543$ &$18.6476$ &$-0.4751$ &$10.5400$ &$0.0106$ \\
Zr1 &$0.4532$ &$59.5948$ &$0.7834$ &$21.4357$ &$-0.2451$ &$9.0360$ &$0.0098$ \\
Nb0 &$0.3946$ &$49.2297$ &$1.3197$ &$14.8216$ &$-0.7269$ &$9.6156$ &$0.0129$ \\
Nb1 &$0.4572$ &$49.9182$ &$1.0274$ &$15.7256$ &$-0.4962$ &$9.1573$ &$0.0118$ \\
Mo0 &$0.1806$ &$49.0568$ &$1.2306$ &$14.7859$ &$-0.4268$ &$6.9866$ &$0.0171$ \\
Mo1 &$0.3500$ &$48.0354$ &$1.0305$ &$15.0604$ &$-0.3929$ &$7.4790$ &$0.0139$ \\
Tc0 &$0.1298$ &$49.6611$ &$1.1656$ &$14.1307$ &$-0.3134$ &$5.5129$ &$0.0195$ \\
Tc1 &$0.2674$ &$48.9566$ &$0.9569$ &$15.1413$ &$-0.2387$ &$5.4578$ &$0.0160$ \\
Ru0 &$0.1069$ &$49.4238$ &$1.1912$ &$12.7417$ &$-0.3176$ &$4.9125$ &$0.0213$ \\
Ru1 &$0.4410$ &$33.3086$ &$1.4775$ &$9.5531$ &$-0.9361$ &$6.7220$ &$0.0176$ \\
Rh0 &$0.0976$ &$49.8825$ &$1.1601$ &$11.8307$ &$-0.2789$ &$4.1266$ &$0.0234$ \\
Rh1 &$0.3342$ &$29.7564$ &$1.2209$ &$9.4384$ &$-0.5755$ &$5.3320$ &$0.0210$ \\
Pd0 &$0.2003$ &$29.3633$ &$1.1446$ &$9.5993$ &$-0.3689$ &$4.0423$ &$0.0251$ \\
Pd1 &$0.5033$ &$24.5037$ &$1.9982$ &$6.9082$ &$-1.5240$ &$5.5133$ &$0.0213$ \\
\hline\\[-2ex]
\end{tabular}
}
\end{table}
\section{{\large$\langle j_0\rangle$ }Form factors for 5d transition elements and their ions}
\begin{table}[H]
 \caption{$\langle j_0\rangle$ form factors for the 5d electrons of transition atoms and ions from Hf to Re.\cite{kob:11}}
\label{5dj0} \vspace{1ex}
{\tablesize
\begin{tabular}{llrrrrrrrrr}
\hline
\multicolumn{1}{c}{ Ion}&\multicolumn{1}{c}{ Config}&\multicolumn{1}{c}{ A }&\multicolumn{1}{c}{  a }&\multicolumn{1}{c}{B }&\multicolumn{1}{c}{ b }&\multicolumn{1}{c}{ C }&\multicolumn{1}{c}{ c }&\multicolumn{1}{c}{ D }&\multicolumn{1}{c}{ d }&\multicolumn{1}{c}{E }\\
\hline
Hf2 & 6s05d2 &$0.4229$ &$50.465$ &$0.7333$ &$23.865$ &$-0.3798$ &$4.051$ &$0.2252$ &$2.497$ &$-0.0018$ \\
Hf3 & 6s05d1 &$0.3555$ &$40.954$ &$0.8483$ &$21.726$ &$-0.4116$ &$4.305$ &$0.2101$ &$2.349$ &$-0.0023$ \\
Ta2 & 6s05d3 &$0.3976$ &$45.095$ &$0.7746$ &$21.028$ &$-0.6098$ &$3.471$ &$0.4395$ &$2.570$ &$-0.0020$ \\
Ta3 & 6s05d2 &$0.3611$ &$36.921$ &$0.8579$ &$19.195$ &$-0.4945$ &$3.857$ &$0.2781$ &$2.303$ &$-0.0026$ \\
Ta4 & 6s05d1 &$0.3065$ &$31.817$ &$0.9611$ &$17.749$ &$-0.5463$ &$3.979$ &$0.2816$ &$2.232$ &$-0.0030$ \\
W0 & 6s05d6 &$0.3990$ &$73.810$ &$0.7138$ &$22.815$ &$-2.0436$ &$2.710$ &$1.9319$ &$2.559$ &$-0.0023$ \\
W0 & 6s15d5 &$0.3811$ &$62.707$ &$0.7523$ &$21.434$ &$-12.5449$ &$2.702$ &$12.4130$ &$2.674$ &$-0.0023$ \\
W0 & 6s25d4 &$0.3653$ &$53.965$ &$0.7926$ &$20.078$ &$-0.8142$ &$3.030$ &$0.6581$ &$2.476$ &$-0.0023$ \\
W0 & 6s05d5 &$0.4077$ &$51.367$ &$0.7436$ &$20.256$ &$-9.8283$ &$2.780$ &$9.6788$ &$2.740$ &$-0.0021$ \\
W0 & 6s15d4 &$0.3834$ &$46.233$ &$0.7890$ &$19.278$ &$-1.4650$ &$2.947$ &$1.2945$ &$2.628$ &$-0.0022$ \\
W2 & 6s05d4 &$0.3876$ &$40.340$ &$0.8008$ &$18.621$ &$-1.3911$ &$2.995$ &$1.2048$ &$2.627$ &$-0.0023$ \\
W3 & 6s05d3 &$0.3610$ &$33.519$ &$0.8717$ &$17.176$ &$-0.6183$ &$3.445$ &$0.3883$ &$2.276$ &$-0.0028$ \\
W4 & 6s05d2 &$0.3221$ &$29.047$ &$0.9574$ &$15.979$ &$-0.6287$ &$3.597$ &$0.3525$ &$2.174$ &$-0.0033$ \\
W5 & 6s05d1 &$0.2725$ &$25.966$ &$1.0558$ &$14.954$ &$-0.6990$ &$3.643$ &$0.3745$ &$2.145$ &$-0.0037$ \\
Re0 & 6s05d7 &$0.3807$ &$63.042$ &$0.7497$ &$19.967$ &$-6.5300$ &$2.501$ &$6.4013$ &$2.451$ &$-0.0028$ \\
Re0 & 6s15d6 &$0.3691$ &$53.934$ &$0.7837$ &$18.790$ &$-9.1491$ &$2.558$ &$8.9983$ &$2.517$ &$-0.0027$ \\
Re0 & 6s25d5 &$0.3548$ &$47.108$ &$0.8210$ &$17.769$ &$-9.8674$ &$2.599$ &$9.6938$ &$2.556$ &$-0.0027$ \\
Re0 & 6s05d6 &$0.3944$ &$45.427$ &$0.7742$ &$17.948$ &$-3.1692$ &$2.653$ &$3.0028$ &$2.521$ &$-0.0026$ \\
Re0 & 6s15d5 &$0.3736$ &$41.151$ &$0.8160$ &$17.158$ &$-7.0396$ &$2.642$ &$6.8523$ &$2.577$ &$-0.0026$ \\
Re2 & 6s05d5 &$0.3825$ &$36.336$ &$0.8218$ &$16.636$ &$-8.7220$ &$2.657$ &$8.5201$ &$2.601$ &$-0.0026$ \\
Re3 & 6s05d4 &$0.3585$ &$30.671$ &$0.8863$ &$15.527$ &$-0.8682$ &$3.047$ &$0.6263$ &$2.280$ &$-0.0030$ \\
Re4 & 6s05d3 &$0.2974$ &$27.372$ &$0.9826$ &$14.807$ &$-1.8869$ &$2.840$ &$1.6100$ &$2.476$ &$-0.0031$ \\
Re5 & 6s05d2 &$0.3143$ &$23.522$ &$1.0276$ &$13.505$ &$-0.7438$ &$3.393$ &$0.4059$ &$2.030$ &$-0.0041$ \\
Re6 & 6s05d1 &$0.2146$ &$22.496$ &$1.1616$ &$13.064$ &$-1.0455$ &$3.162$ &$0.6734$ &$2.196$ &$-0.0041$ \\
\hline
\end{tabular}
}
\end{table}
\begin{table}[H]
 \caption{$\langle j_0\rangle$ form factors for the 5d electrons of transition atoms and ions from Os to Au.\cite{kob:11}}
 \label{5dj0b}
 \vspace{1ex}
{ \tablesize
\begin{tabular}{llrrrrrrrrr}
\hline
\multicolumn{1}{c}{ Ion}&\multicolumn{1}{c}{ Config}&\multicolumn{1}{c}{ A }&\multicolumn{1}{c}{  a }&\multicolumn{1}{c}{B }&\multicolumn{1}{c}{ b }&\multicolumn{1}{c}{ C }&\multicolumn{1}{c}{ c }&\multicolumn{1}{c}{ D }&\multicolumn{1}{c}{ d }&\multicolumn{1}{c}{E}\\
\hline
Os0 & 6s05d8 &$0.3676$ &$54.835$ &$0.7793$ &$17.716$ &$-2.0669$ &$2.418$ &$1.9224$ &$2.247$ &$-0.0034$ \\
Os0 & 6s15d7 &$0.3571$ &$47.458$ &$0.8123$ &$16.770$ &$-1.2072$ &$2.556$ &$1.0404$ &$2.211$ &$-0.0033$ \\
Os0 & 6s25d6 &$0.3467$ &$41.778$ &$0.8458$ &$15.918$ &$-5.6370$ &$2.459$ &$5.4472$ &$2.381$ &$-0.0032$ \\
Os0 & 6s05d7 &$0.3837$ &$40.665$ &$0.8006$ &$16.096$ &$-3.5305$ &$2.487$ &$3.3488$ &$2.366$ &$-0.0030$ \\
Os0 & 6s15d6 &$0.3666$ &$36.997$ &$0.8390$ &$15.425$ &$-2.6944$ &$2.537$ &$2.4916$ &$2.360$ &$-0.0031$ \\
Os2 & 6s05d6 &$0.3786$ &$33.005$ &$0.8412$ &$14.990$ &$-7.0632$ &$2.503$ &$6.8462$ &$2.433$ &$-0.0030$ \\
Os3 & 6s05d5 &$0.3557$ &$28.222$ &$0.9002$ &$14.140$ &$-2.5972$ &$2.601$ &$2.3444$ &$2.376$ &$-0.0032$ \\
Os4 & 6s05d4 &$0.3337$ &$24.723$ &$0.9655$ &$13.288$ &$-0.9653$ &$2.906$ &$0.6698$ &$2.117$ &$-0.0037$ \\
Os5 & 6s05d3 &$0.3055$ &$22.152$ &$1.0395$ &$12.529$ &$-0.9158$ &$3.016$ &$0.5750$ &$2.032$ &$-0.0042$ \\
Os6 & 6s05d2 &$0.2714$ &$20.218$ &$1.1211$ &$11.851$ &$-0.9773$ &$3.050$ &$0.5894$ &$2.005$ &$-0.0046$ \\
Os7 & 6s05d1 &$0.2101$ &$19.108$ &$1.2240$ &$11.347$ &$-1.2543$ &$2.933$ &$0.8250$ &$2.088$ &$-0.0048$ \\
Ir0 & 6s05d9 &$0.3564$ &$48.464$ &$0.8049$ &$15.923$ &$-2.5258$ &$2.265$ &$2.3675$ &$2.121$ &$-0.0040$ \\
Ir0 & 6s15d8 &$0.3492$ &$42.195$ &$0.8350$ &$15.113$ &$-5.1496$ &$2.279$ &$4.9686$ &$2.201$ &$-0.0038$ \\
Ir0 & 6s25d7 &$0.3400$ &$37.499$ &$0.8675$ &$14.402$ &$-2.3703$ &$2.370$ &$2.1661$ &$2.177$ &$-0.0037$ \\
Ir0 & 6s05d8 &$0.3744$ &$36.764$ &$0.8240$ &$14.576$ &$-8.8616$ &$2.303$ &$8.6664$ &$2.255$ &$-0.0035$ \\
Ir0 & 6s15d7 &$0.3604$ &$33.570$ &$0.8597$ &$13.993$ &$-2.1686$ &$2.412$ &$1.9518$ &$2.188$ &$-0.0036$ \\
Ir2 & 6s05d7 &$0.3802$ &$30.032$ &$0.8550$ &$13.567$ &$-1.6185$ &$2.488$ &$1.3866$ &$2.162$ &$-0.0035$ \\
Ir3 & 6s05d6 &$0.3678$ &$25.828$ &$0.9065$ &$12.788$ &$-0.8587$ &$2.745$ &$0.5883$ &$1.960$ &$-0.0040$ \\
Ir4 & 6s05d5 &$0.3969$ &$22.050$ &$0.9310$ &$11.768$ &$-0.7090$ &$3.017$ &$0.3857$ &$1.778$ &$-0.0047$ \\
\hline
\end{tabular}
}
\end{table}
\section{{\large$\langle j_0\rangle$ }Form factors for Rare earth ions}
\begin{table}[H]
\caption{$\langle j_0\rangle$ form factors for the 4f electrons of rare earth ions}
\label{re4fj0}
{\tablesize
\begin{tabular}{lrrrrrrr}
\hline
Ion&
\multicolumn{1}{c}{A}&\multicolumn{1}{c}{a}&
\multicolumn{1}{c}{B}&\multicolumn{1}{c}{b}&
\multicolumn{1}{c}{C}&\multicolumn{1}{c}{c}&\multicolumn{1}{c}{D}\\
\hline\\[-2ex]
Ce2 &$0.2953$ &$17.6846$ &$0.2923$ &$6.7329$ &$0.4313$ &$5.3827$ &$-0.0194$ \\
Nd2 &$0.1645$ &$25.0453$ &$0.2522$ &$11.9782$ &$0.6012$ &$4.9461$ &$-0.0180$ \\
Nd3 &$0.0540$ &$25.0293$ &$0.3101$ &$12.1020$ &$0.6575$ &$4.7223$ &$-0.0216$ \\
Sm2 &$0.0909$ &$25.2032$ &$0.3037$ &$11.8562$ &$0.6250$ &$4.2366$ &$-0.0200$ \\
Sm3 &$0.0288$ &$25.2068$ &$0.2973$ &$11.8311$ &$0.6954$ &$4.2117$ &$-0.0213$ \\
Eu2 &$0.0755$ &$25.2960$ &$0.3001$ &$11.5993$ &$0.6438$ &$4.0252$ &$-0.0196$ \\
Eu3 &$0.0204$ &$25.3078$ &$0.3010$ &$11.4744$ &$0.7005$ &$3.9420$ &$-0.0220$ \\
Gd2 &$0.0636$ &$25.3823$ &$0.3033$ &$11.2125$ &$0.6528$ &$3.7877$ &$-0.0199$ \\
Gd3 &$0.0186$ &$25.3867$ &$0.2895$ &$11.1421$ &$0.7135$ &$3.7520$ &$-0.0217$ \\
Tb2 &$0.0547$ &$25.5086$ &$0.3171$ &$10.5911$ &$0.6490$ &$3.5171$ &$-0.0212$ \\
Tb3 &$0.0177$ &$25.5095$ &$0.2921$ &$10.5769$ &$0.7133$ &$3.5122$ &$-0.0231$ \\
Dy2 &$0.1308$ &$18.3155$ &$0.3118$ &$7.6645$ &$0.5795$ &$3.1469$ &$-0.0226$ \\
Dy3 &$0.1157$ &$15.0732$ &$0.3270$ &$6.7991$ &$0.5821$ &$3.0202$ &$-0.0249$ \\
Ho2 &$0.0995$ &$18.1761$ &$0.3305$ &$7.8556$ &$0.5921$ &$2.9799$ &$-0.0230$ \\
Ho3 &$0.0566$ &$18.3176$ &$0.3365$ &$7.6880$ &$0.6317$ &$2.9427$ &$-0.0248$ \\
Er2 &$0.1122$ &$18.1223$ &$0.3462$ &$6.9106$ &$0.5649$ &$2.7614$ &$-0.0235$ \\
Er3 &$0.0586$ &$17.9802$ &$0.3540$ &$7.0964$ &$0.6126$ &$2.7482$ &$-0.0251$ \\
Tm2 &$0.0983$ &$18.3236$ &$0.3380$ &$6.9178$ &$0.5875$ &$2.6622$ &$-0.0241$ \\
Tm3 &$0.0581$ &$15.0922$ &$0.2787$ &$7.8015$ &$0.6854$ &$2.7931$ &$-0.0224$ \\
Yb2 &$0.0855$ &$18.5123$ &$0.2943$ &$7.3734$ &$0.6412$ &$2.6777$ &$-0.0213$ \\
Yb3 &$0.0416$ &$16.0949$ &$0.2849$ &$7.8341$ &$0.6961$ &$2.6725$ &$-0.0229$ \\
Pr3 &$0.0504$ &$24.9989$ &$0.2572$ &$12.0377$ &$0.7142$ &$5.0039$ &$-0.0219$ \\
\hline\\[-2ex]
\end{tabular}
}
\vspace{2mm}
\end{table}
\begin{table}[H]
 \caption{$\langle j_0\rangle$ form factors for the 5d electrons of rare earth ions.\cite{kob:12}}
 \vspace{1ex}
 \label{re5dj0}
{\tablesize
\begin{tabular}{llrrrrrrrrr}
\hline
\multicolumn{1}{c}{ Ion}&\multicolumn{1}{c}{ Config}&\multicolumn{1}{c}{ A }&\multicolumn{1}{c}{  a }&\multicolumn{1}{c}{B }&\multicolumn{1}{c}{ b }&\multicolumn{1}{c}{ C }&\multicolumn{1}{c}{ c }&\multicolumn{1}{c}{ D }&\multicolumn{1}{c}{ d }&\multicolumn{1}{c}{E}\\
\hline
La2 & 4f05d1 &$0.5488$ &$63.822$ &$0.7238$ &$34.429$ &$-6.0375$ &$7.092$ &$5.7655$ &$6.839$ &$-0.0008$ \\
Ce2 & 4f15d1 &$0.4959$ &$63.797$ &$0.7571$ &$34.334$ &$-5.9903$ &$6.595$ &$5.7381$ &$6.370$ &$-0.0009$ \\
Ce3 & 4f05d1 &$1.2395$ &$35.447$ &$-1.6420$ &$4.939$ &$1.9467$ &$3.715$ &$-0.5481$ &$2.671$ &$0.0020$ \\
Pr2 & 4f25d1 &$0.4568$ &$63.765$ &$0.7795$ &$34.094$ &$-5.1096$ &$6.186$ &$4.8743$ &$5.950$ &$-0.0011$ \\
Pr3 & 4f15d1 &$0.7735$ &$40.670$ &$0.6284$ &$21.775$ &$-7.2962$ &$6.114$ &$6.8955$ &$5.866$ &$-0.0015$ \\
Pr4 & 4f05d1 &$1.3076$ &$27.762$ &$-3.2363$ &$4.278$ &$4.3424$ &$3.471$ &$-1.4168$ &$2.758$ &$0.0018$ \\
Nd2 & 4f35d1 &$1.2972$ &$27.071$ &$-2.4336$ &$4.208$ &$3.2415$ &$3.271$ &$-1.1081$ &$2.568$ &$0.0016$ \\
Nd3 & 4f25d1 &$0.6179$ &$41.924$ &$0.7344$ &$23.743$ &$-5.2140$ &$5.726$ &$4.8631$ &$5.427$ &$-0.0016$ \\
Nd4 & 4f15d1 &$1.2972$ &$27.071$ &$-2.4336$ &$4.208$ &$3.2415$ &$3.271$ &$-1.1081$ &$2.568$ &$0.0016$ \\
Pm3 & 4f35d1 &$0.5149$ &$42.825$ &$0.8077$ &$24.559$ &$-3.6851$ &$5.432$ &$3.3642$ &$5.053$ &$-0.0018$ \\
Sm2 & 4f55d1 &$0.3908$ &$63.383$ &$0.8062$ &$32.974$ &$-5.5271$ &$5.174$ &$5.3313$ &$5.015$ &$-0.0013$ \\
Sm3 & 4f45d1 &$0.4725$ &$42.826$ &$0.8301$ &$24.562$ &$-3.0608$ &$5.170$ &$2.7600$ &$4.755$ &$-0.0019$ \\
Eu2 & 4f65d1 &$0.3827$ &$63.023$ &$0.8044$ &$32.481$ &$-1.8540$ &$5.066$ &$1.6682$ &$4.618$ &$-0.0014$ \\
Eu3 & 4f55d1 &$0.3784$ &$44.430$ &$0.9025$ &$25.126$ &$-3.1521$ &$4.871$ &$2.8731$ &$4.511$ &$-0.0020$ \\
Gd0 & 4f85d1 &$0.4982$ &$104.440$ &$0.5928$ &$44.451$ &$-14.5061$ &$4.692$ &$14.4155$ &$4.664$ &$-0.0008$ \\
Gd0 & 6s14f75d1 &$0.4132$ &$75.211$ &$0.7389$ &$33.606$ &$-1.8951$ &$4.755$ &$1.7441$ &$4.411$ &$-0.0013$ \\
Gd2 & 4f75d1 &$0.3745$ &$62.755$ &$0.8026$ &$32.071$ &$-8.6327$ &$4.644$ &$8.4569$ &$4.560$ &$-0.0014$ \\
Gd3 & 4f65d1 &$0.3587$ &$44.300$ &$0.9082$ &$24.875$ &$-2.6769$ &$4.664$ &$2.4120$ &$4.274$ &$-0.0020$ \\
Gd4 & 4f55d1 &$0.4512$ &$32.712$ &$0.9183$ &$19.602$ &$-2.8665$ &$4.569$ &$2.4997$ &$4.098$ &$-0.0027$ \\
Tb0 & 4f95d1 &$0.5057$ &$104.335$ &$0.5796$ &$43.869$ &$-1.6061$ &$4.572$ &$1.5212$ &$4.339$ &$-0.0008$ \\
Tb0 & 6s14f95d1 &$0.4131$ &$74.925$ &$0.7311$ &$33.131$ &$-1.2648$ &$4.612$ &$1.1216$ &$4.133$ &$-0.0013$ \\
Tb2 & 4f85d1 &$0.3734$ &$62.276$ &$0.7955$ &$31.562$ &$-3.4780$ &$4.492$ &$3.3105$ &$4.298$ &$-0.0014$ \\
Tb3 & 4f75d1 &$0.3223$ &$44.892$ &$0.9303$ &$24.826$ &$-3.9948$ &$4.375$ &$3.7443$ &$4.140$ &$-0.0021$ \\
Tb4 & 4f65d1 &$0.3614$ &$33.730$ &$0.9844$ &$19.945$ &$-2.6536$ &$4.355$ &$2.3105$ &$3.892$ &$-0.0028$ \\
Dy2 & 4f95d1 &$0.3754$ &$61.715$ &$0.7859$ &$31.049$ &$-1.7452$ &$4.388$ &$1.5853$ &$4.024$ &$-0.0014$ \\
Dy3 & 4f85d1 &$0.3014$ &$45.110$ &$0.9392$ &$24.622$ &$-3.2549$ &$4.203$ &$3.0163$ &$3.936$ &$-0.0021$ \\
Ho3 & 4f95d1 &$0.2974$ &$44.755$ &$0.9335$ &$24.262$ &$-2.3286$ &$4.079$ &$2.0998$ &$3.728$ &$-0.0021$ \\
Er3 & 4f105d1 &$0.2916$ &$44.550$ &$0.9304$ &$23.933$ &$-1.3854$ &$4.035$ &$1.1655$ &$3.468$ &$-0.0021$ \\
Tm2 & 4f125d1 &$0.4067$ &$59.305$ &$0.7386$ &$29.168$ &$-0.3795$ &$4.593$ &$0.2356$ &$2.956$ &$-0.0015$ \\
Tm3 & 4f115d1 &$0.3302$ &$42.750$ &$0.8905$ &$23.057$ &$-0.5894$ &$4.397$ &$0.3708$ &$2.923$ &$-0.0022$ \\
Tm4 & 4f105d1 &$0.2406$ &$35.005$ &$1.0453$ &$19.560$ &$-4.5727$ &$3.572$ &$4.2895$ &$3.383$ &$-0.0027$ \\
Yb2 & 4f135d1 &$0.4170$ &$58.627$ &$0.7233$ &$28.615$ &$-0.3308$ &$4.553$ &$0.1919$ &$2.742$ &$-0.0015$ \\
Yb3 & 4f125d1 &$0.3498$ &$41.749$ &$0.8666$ &$22.455$ &$-0.4923$ &$4.434$ &$0.2780$ &$2.671$ &$-0.0022$ \\
Lu2 & 4f145d1 &$0.4216$ &$58.262$ &$0.7131$ &$28.218$ &$-0.3048$ &$4.446$ &$0.1715$ &$2.593$ &$-0.0015$ \\
Lu3 & 4f135d1 &$0.3217$ &$42.404$ &$0.8845$ &$22.448$ &$-0.4726$ &$4.258$ &$0.2686$ &$2.566$ &$-0.0022$ \\
\hline
\end{tabular}
}
\end{table}
\section{{\large$\langle j_0\rangle$ }Form factors for Actinide ions}
\begin{table}[H]
\caption{$\langle j_0\rangle$ Form factors for 5f electrons of actinide ions}\vspace{2mm}
\label{acj0}
{\tablesize
\begin{tabular}{lrrrrrrr}
\hline
Ion&
\multicolumn{1}{c}{A}&\multicolumn{1}{c}{a}&
\multicolumn{1}{c}{B}&\multicolumn{1}{c}{b}&
\multicolumn{1}{c}{C}&\multicolumn{1}{c}{c}&\multicolumn{1}{c}{D}\\
\hline\\[-2ex]
U3 &$0.5058$ &$23.2882$ &$1.3464$ &$7.0028$ &$-0.8724$ &$4.8683$ &$0.0192$ \\
U4 &$0.3291$ &$23.5475$ &$1.0836$ &$8.4540$ &$-0.4340$ &$4.1196$ &$0.0214$ \\
U5 &$0.3650$ &$19.8038$ &$3.2199$ &$6.2818$ &$-2.6077$ &$5.3010$ &$0.0233$ \\
Np3 &$0.5157$ &$20.8654$ &$2.2784$ &$5.8930$ &$-1.8163$ &$4.8457$ &$0.0211$ \\
Np4 &$0.4206$ &$19.8046$ &$2.8004$ &$5.9783$ &$-2.2436$ &$4.9848$ &$0.0228$ \\
Np5 &$0.3692$ &$18.1900$ &$3.1510$ &$5.8500$ &$-2.5446$ &$4.9164$ &$0.0248$ \\
Np6 &$0.2929$ &$17.5611$ &$3.4866$ &$5.7847$ &$-2.8066$ &$4.8707$ &$0.0267$ \\
Pu3 &$0.3840$ &$16.6793$ &$3.1049$ &$5.4210$ &$-2.5148$ &$4.5512$ &$0.0263$ \\
Pu4 &$0.4934$ &$16.8355$ &$1.6394$ &$5.6384$ &$-1.1581$ &$4.1399$ &$0.0248$ \\
Pu5 &$0.3888$ &$16.5592$ &$2.0362$ &$5.6567$ &$-1.4515$ &$4.2552$ &$0.0267$ \\
Pu6 &$0.3172$ &$16.0507$ &$3.4654$ &$5.3507$ &$-2.8102$ &$4.5133$ &$0.0281$ \\
Am2 &$0.4743$ &$21.7761$ &$1.5800$ &$5.6902$ &$-1.0779$ &$4.1451$ &$0.0218$ \\
Am3 &$0.4239$ &$19.5739$ &$1.4573$ &$5.8722$ &$-0.9052$ &$3.9682$ &$0.0238$ \\
Am4 &$0.3737$ &$17.8625$ &$1.3521$ &$6.0426$ &$-0.7514$ &$3.7199$ &$0.0258$ \\
Am5 &$0.2956$ &$17.3725$ &$1.4525$ &$6.0734$ &$-0.7755$ &$3.6619$ &$0.0277$ \\
Am6 &$0.2302$ &$16.9533$ &$1.4864$ &$6.1159$ &$-0.7457$ &$3.5426$ &$0.0294$ \\
Am7 &$0.3601$ &$12.7299$ &$1.9640$ &$5.1203$ &$-1.3560$ &$3.7142$ &$0.0316$ \\
\hline\\[-2ex]
\end{tabular}
}
\end{table}
\section{{\large $\langle j_2\rangle$} Form factors for 3d transition elements and their ions}
\label{3dj2}
\begin{table}[H]
\caption{
$\langle j_2\rangle$ form factors for 3d transition elements and their ions}
 \vspace{2mm}
{\tablesize
\begin{tabular}{lrrrrrrr}
\hline
Ion&
\multicolumn{1}{c}{A}&\multicolumn{1}{c}{a}&
\multicolumn{1}{c}{B}&\multicolumn{1}{c}{b}&
\multicolumn{1}{c}{C}&\multicolumn{1}{c}{c}&\multicolumn{1}{c}{D}\\
\hline\\[-2ex]
Sc0 &$10.8172$ &$54.3270$ &$4.7353$ &$14.8471$ &$0.6071$ &$4.2180$ &$0.0011$ \\
Sc1 &$8.5021$ &$34.2851$ &$3.2116$ &$10.9940$ &$0.4244$ &$3.6055$ &$0.0009$ \\
Sc2 &$4.3683$ &$28.6544$ &$3.7231$ &$10.8233$ &$0.6074$ &$3.6678$ &$0.0014$ \\
Ti0 &$4.3583$ &$36.0556$ &$3.8230$ &$11.1328$ &$0.6855$ &$3.4692$ &$0.0020$ \\
Ti1 &$6.1567$ &$27.2754$ &$2.6833$ &$8.9827$ &$0.4070$ &$3.0524$ &$0.0011$ \\
Ti2 &$4.3107$ &$18.3484$ &$2.0960$ &$6.7970$ &$0.2984$ &$2.5476$ &$0.0007$ \\
Ti3 &$3.3717$ &$14.4441$ &$1.8258$ &$5.7126$ &$0.2470$ &$2.2654$ &$0.0005$ \\
V0 &$3.8099$ &$21.8313$ &$2.4026$ &$7.5458$ &$0.4464$ &$2.6628$ &$0.0017$ \\
V1 &$4.7474$ &$23.3226$ &$2.3609$ &$7.8082$ &$0.4105$ &$2.7063$ &$0.0014$ \\
V2 &$3.4386$ &$16.5303$ &$1.9638$ &$6.1415$ &$0.2997$ &$2.2669$ &$0.0009$ \\
V3 &$2.3005$ &$14.6821$ &$2.0364$ &$6.1304$ &$0.4099$ &$2.3815$ &$0.0014$ \\
V4 &$1.8377$ &$12.2668$ &$1.8247$ &$5.4578$ &$0.3979$ &$2.2483$ &$0.0012$ \\
Cr0 &$3.7600$ &$20.1267$ &$2.1006$ &$6.8020$ &$0.4266$ &$2.3941$ &$0.0019$ \\
Cr1 &$3.7768$ &$20.3456$ &$2.1028$ &$6.8926$ &$0.4010$ &$2.4114$ &$0.0017$ \\
Cr2 &$2.6422$ &$16.0598$ &$1.9198$ &$6.2531$ &$0.4446$ &$2.3715$ &$0.0020$ \\
Cr3 &$1.6262$ &$15.0656$ &$2.0618$ &$6.2842$ &$0.5281$ &$2.3680$ &$0.0023$ \\
Cr4 &$1.0293$ &$13.9498$ &$1.9933$ &$6.0593$ &$0.5974$ &$2.3457$ &$0.0027$ \\
Mn0 &$2.6681$ &$16.0601$ &$1.7561$ &$5.6396$ &$0.3675$ &$2.0488$ &$0.0017$ \\
Mn1 &$3.2953$ &$18.6950$ &$1.8792$ &$6.2403$ &$0.3927$ &$2.2006$ &$0.0022$ \\
Mn2 &$2.0515$ &$15.5561$ &$1.8841$ &$6.0625$ &$0.4787$ &$2.2323$ &$0.0027$ \\
Mn3 &$1.2427$ &$14.9966$ &$1.9567$ &$6.1181$ &$0.5732$ &$2.2577$ &$0.0031$ \\
Mn4 &$0.7879$ &$13.8857$ &$1.8717$ &$5.7433$ &$0.5981$ &$2.1818$ &$0.0034$ \\
Fe0 &$1.9405$ &$18.4733$ &$1.9566$ &$6.3234$ &$0.5166$ &$2.1607$ &$0.0036$ \\
Fe1 &$2.6290$ &$18.6598$ &$1.8704$ &$6.3313$ &$0.4690$ &$2.1628$ &$0.0031$ \\
Fe2 &$1.6490$ &$16.5593$ &$1.9064$ &$6.1325$ &$0.5206$ &$2.1370$ &$0.0035$ \\
Fe3 &$1.3602$ &$11.9976$ &$1.5188$ &$5.0025$ &$0.4705$ &$1.9914$ &$0.0038$ \\
Fe4 &$1.5582$ &$8.2750$ &$1.1863$ &$3.2794$ &$0.1366$ &$1.1068$ &$-0.0022$ \\
Co0 &$1.9678$ &$14.1699$ &$1.4911$ &$4.9475$ &$0.3844$ &$1.7973$ &$0.0027$ \\
Co1 &$2.4097$ &$16.1608$ &$1.5780$ &$5.4604$ &$0.4095$ &$1.9141$ &$0.0031$ \\
Co2 &$1.9049$ &$11.6444$ &$1.3159$ &$4.3574$ &$0.3146$ &$1.6453$ &$0.0017$ \\
Co3 &$1.7058$ &$8.8595$ &$1.1409$ &$3.3086$ &$0.1474$ &$1.0899$ &$-0.0025$ \\
Co4 &$1.3110$ &$8.0252$ &$1.1551$ &$3.1792$ &$0.1608$ &$1.1301$ &$-0.0011$ \\
Ni0 &$1.0302$ &$12.2521$ &$1.4669$ &$4.7453$ &$0.4521$ &$1.7437$ &$0.0036$ \\
Ni1 &$2.1040$ &$14.8655$ &$1.4302$ &$5.0714$ &$0.4031$ &$1.7784$ &$0.0034$ \\
Ni2 &$1.7080$ &$11.0160$ &$1.2147$ &$4.1031$ &$0.3150$ &$1.5334$ &$0.0018$ \\
Ni3 &$1.4683$ &$8.6713$ &$1.1068$ &$3.2574$ &$0.1794$ &$1.1058$ &$-0.0023$ \\
Ni4 &$1.1612$ &$7.7000$ &$1.0027$ &$3.2628$ &$0.2719$ &$1.3780$ &$0.0025$ \\
Cu0 &$1.9182$ &$14.4904$ &$1.3329$ &$4.7301$ &$0.3842$ &$1.6394$ &$0.0035$ \\
Cu1 &$1.8814$ &$13.4333$ &$1.2809$ &$4.5446$ &$0.3646$ &$1.6022$ &$0.0033$ \\
Cu2 &$1.5189$ &$10.4779$ &$1.1512$ &$3.8132$ &$0.2918$ &$1.3979$ &$0.0017$ \\
Cu3 &$1.2797$ &$8.4502$ &$1.0315$ &$3.2796$ &$0.2401$ &$1.2498$ &$0.0015$ \\
Cu4 &$0.9568$ &$7.4481$ &$0.9099$ &$3.3964$ &$0.3729$ &$1.4936$ &$0.0049$ \\
\hline\\[-2ex]
\end{tabular}
}
\end{table}
\section{{\large $\langle j_2\rangle$} Form factors for 4d transition elements}
\begin{table}[H]
\caption{$\langle j_2\rangle$ form factors for 4d atoms and ions}\vspace{2mm}
\label{4dj2}
{\tablesize
\begin{tabular}{lrrrrrrr}
\hline
Ion&
\multicolumn{1}{c}{A}&\multicolumn{1}{c}{a}&
\multicolumn{1}{c}{B}&\multicolumn{1}{c}{b}&
\multicolumn{1}{c}{C}&\multicolumn{1}{c}{c}&\multicolumn{1}{c}{D}\\
\hline\\[-2ex]
Y0 &$14.4084$ &$44.6577$ &$5.1045$ &$14.9043$ &$-0.0535$ &$3.3189$ &$0.0028$ \\
Zr0 &$10.1378$ &$35.3372$ &$4.7734$ &$12.5453$ &$-0.0489$ &$2.6721$ &$0.0036$ \\
Zr1 &$11.8722$ &$34.9200$ &$4.0502$ &$12.1266$ &$-0.0632$ &$2.8278$ &$0.0034$ \\
Nb0 &$7.4796$ &$33.1789$ &$5.0884$ &$11.5708$ &$-0.0281$ &$1.5635$ &$0.0047$ \\
Nb1 &$8.7735$ &$33.2848$ &$4.6556$ &$11.6046$ &$-0.0268$ &$1.5389$ &$0.0044$ \\
Mo0 &$5.1180$ &$23.4217$ &$4.1809$ &$9.2080$ &$-0.0505$ &$1.7434$ &$0.0053$ \\
Mo1 &$7.2367$ &$28.1282$ &$4.0705$ &$9.9228$ &$-0.0317$ &$1.4552$ &$0.0049$ \\
Tc0 &$4.2441$ &$21.3974$ &$3.9439$ &$8.3753$ &$-0.0371$ &$1.1870$ &$0.0066$ \\
Tc1 &$6.4056$ &$24.8243$ &$3.5400$ &$8.6112$ &$-0.0366$ &$1.4846$ &$0.0044$ \\
Ru0 &$3.7445$ &$18.6128$ &$3.4749$ &$7.4201$ &$-0.0363$ &$1.0068$ &$0.0073$ \\
Ru1 &$5.2826$ &$23.6832$ &$3.5813$ &$8.1521$ &$-0.0257$ &$0.4255$ &$0.0131$ \\
Rh0 &$3.3651$ &$17.3444$ &$3.2121$ &$6.8041$ &$-0.0350$ &$0.5031$ &$0.0146$ \\
Rh1 &$4.0260$ &$18.9497$ &$3.1663$ &$6.9998$ &$-0.0296$ &$0.4862$ &$0.0127$ \\
Pd0 &$3.3105$ &$14.7265$ &$2.6332$ &$5.8618$ &$-0.0437$ &$1.1303$ &$0.0053$ \\
Pd1 &$4.2749$ &$17.9002$ &$2.7021$ &$6.3541$ &$-0.0258$ &$0.6999$ &$0.0071$ \\
\hline\\[-2ex]
\end{tabular}
}
\end{table}
\section{{\large $\langle j_2\rangle$} Form factors for 5d Transition elements}
\begin{table}[H]
 \caption{$\langle j_2\rangle$ form factors for the 5d electrons of transition atoms and ions from Hf to Re.\cite{kob:11}}
 \label{5dj2}
 \vspace{1ex}
 {\tablesize
\begin{tabular}{llrrrrrrrrr}
\hline
\multicolumn{1}{c}{ Ion}&\multicolumn{1}{c}{ Config}&\multicolumn{1}{c}{ A }&\multicolumn{1}{c}{  a }&\multicolumn{1}{c}{B }&\multicolumn{1}{c}{ b }&\multicolumn{1}{c}{ C }&\multicolumn{1}{c}{ c }&\multicolumn{1}{c}{ D }&\multicolumn{1}{c}{ d }&\multicolumn{1}{c}{E}\\
\hline
Hf2 & 6s05d2 &$9.6670$ &$33.435$ &$5.2429$ &$13.529$ &$-0.5533$ &$1.402$ &$0.4934$ &$1.254$ &$-0.0033$ \\
Hf3 & 6s05d1 &$7.5646$ &$27.367$ &$5.0743$ &$12.402$ &$-0.4133$ &$1.742$ &$0.3163$ &$1.437$ &$-0.0012$ \\
Ta2 & 6s05d3 &$8.1746$ &$29.871$ &$4.9405$ &$12.188$ &$-1.1294$ &$1.254$ &$1.0658$ &$1.181$ &$-0.0046$ \\
Ta3 & 6s05d2 &$6.5998$ &$25.026$ &$4.7913$ &$11.303$ &$-0.7801$ &$1.529$ &$0.6820$ &$1.382$ &$-0.0019$ \\
Ta4 & 6s05d1 &$5.5048$ &$21.620$ &$4.6734$ &$10.556$ &$-0.2181$ &$2.100$ &$0.0810$ &$1.049$ &$-0.0016$ \\
W0 & 6s05d6 &$8.4216$ &$64.652$ &$7.6983$ &$23.124$ &$2.3605$ &$8.583$ &$-0.3876$ &$4.620$ &$0.0033$ \\
W0 & 6s15d5 &$6.9289$ &$53.425$ &$7.2157$ &$20.016$ &$5.7754$ &$6.543$ &$-4.2114$ &$5.854$ &$0.0036$ \\
W0 & 6s25d4 &$5.4017$ &$47.110$ &$6.9113$ &$18.760$ &$7.0586$ &$6.363$ &$-5.4525$ &$5.802$ &$0.0040$ \\
W0 & 6s05d5 &$5.6854$ &$44.743$ &$6.8903$ &$18.753$ &$7.8523$ &$6.308$ &$-6.3447$ &$5.842$ &$0.0038$ \\
W0 & 6s15d4 &$4.5118$ &$41.281$ &$6.7136$ &$17.964$ &$7.6331$ &$6.269$ &$-6.0365$ &$5.758$ &$0.0042$ \\
W2 & 6s05d4 &$7.0301$ &$26.990$ &$4.6604$ &$11.083$ &$-0.8220$ &$1.145$ &$0.7581$ &$1.042$ &$-0.0068$ \\
W3 & 6s05d3 &$5.8036$ &$22.969$ &$4.5243$ &$10.361$ &$-0.7897$ &$1.388$ &$0.6927$ &$1.248$ &$-0.0031$ \\
W4 & 6s05d2 &$4.9006$ &$20.117$ &$4.4360$ &$9.765$ &$-2.0009$ &$1.497$ &$1.8714$ &$1.431$ &$-0.0014$ \\
W5 & 6s05d1 &$4.1973$ &$17.967$ &$4.3791$ &$9.255$ &$-1.8830$ &$1.603$ &$1.7205$ &$1.518$ &$-0.0004$ \\
Re0 & 6s05d7 &$6.7574$ &$55.529$ &$6.7931$ &$20.125$ &$2.3113$ &$7.529$ &$-0.5004$ &$4.412$ &$0.0037$ \\
Re0 & 6s15d6 &$5.5830$ &$46.852$ &$6.4516$ &$17.855$ &$5.0609$ &$6.037$ &$-3.5427$ &$5.312$ &$0.0040$ \\
Re0 & 6s25d5 &$4.4322$ &$41.798$ &$6.2063$ &$16.844$ &$9.8763$ &$5.751$ &$-8.3294$ &$5.406$ &$0.0043$ \\
Re0 & 6s05d6 &$4.7231$ &$40.130$ &$6.2139$ &$16.902$ &$8.8240$ &$5.766$ &$-7.3608$ &$5.396$ &$0.0042$ \\
Re0 & 6s15d5 &$3.7875$ &$37.254$ &$6.0587$ &$16.264$ &$6.9896$ &$5.786$ &$-5.4457$ &$5.279$ &$0.0045$ \\
Re2 & 6s05d5 &$3.0708$ &$33.896$ &$5.9941$ &$15.817$ &$6.7816$ &$5.768$ &$-5.2056$ &$5.231$ &$0.0047$ \\
Re3 & 6s05d4 &$1.7870$ &$31.248$ &$5.9068$ &$15.160$ &$17.4262$ &$5.551$ &$-15.6856$ &$5.339$ &$0.0053$ \\
Re4 & 6s05d3 &$4.4033$ &$18.733$ &$4.1971$ &$9.047$ &$-2.3106$ &$1.352$ &$2.1866$ &$1.299$ &$-0.0025$ \\
Re5 & 6s05d2 &$3.8162$ &$16.843$ &$4.1400$ &$8.605$ &$-4.0882$ &$1.443$ &$3.9333$ &$1.408$ &$-0.0012$ \\
Re6 & 6s05d1 &$3.3349$ &$15.341$ &$4.1015$ &$8.213$ &$-2.3511$ &$1.531$ &$2.1651$ &$1.459$ &$-0.0003$ \\
\hline
\end{tabular}
}
\end{table}
\begin{table}[H]
 \caption{$\langle j_\rangle$ form factors for the 5d electrons of transition atoms and ions from Os to Au.\cite{kob:11}}
\label{5dj2b} \vspace{1ex}
{\tablesize
\begin{tabular}{llrrrrrrrrr}
\hline
\multicolumn{1}{c}{ Ion}&\multicolumn{1}{c}{ Config}&\multicolumn{1}{c}{ A }&\multicolumn{1}{c}{  a }&\multicolumn{1}{c}{B }&\multicolumn{1}{c}{ b }&\multicolumn{1}{c}{ C }&\multicolumn{1}{c}{ c }&\multicolumn{1}{c}{ D }&\multicolumn{1}{c}{ d }&\multicolumn{1}{c}{E }\\
\hline
Os0 & 6s05d8 &$5.5418$ &$48.893$ &$6.0803$ &$17.984$ &$2.2542$ &$6.853$ &$-0.5285$ &$4.095$ &$0.0040$ \\
Os0 & 6s15d7 &$4.6511$ &$41.610$ &$5.8194$ &$16.062$ &$6.7172$ &$5.447$ &$-5.2689$ &$4.986$ &$0.0043$ \\
Os0 & 6s25d6 &$3.7421$ &$37.491$ &$5.6137$ &$15.256$ &$10.0488$ &$5.293$ &$-8.5683$ &$4.991$ &$0.0046$ \\
Os0 & 6s05d7 &$4.0236$ &$36.272$ &$5.6349$ &$15.338$ &$8.6568$ &$5.318$ &$-7.2505$ &$4.980$ &$0.0044$ \\
Os0 & 6s15d6 &$3.2611$ &$33.818$ &$5.4945$ &$14.808$ &$5.7610$ &$5.391$ &$-4.2811$ &$4.827$ &$0.0048$ \\
Os2 & 6s05d6 &$2.7289$ &$30.900$ &$5.4357$ &$14.396$ &$18.6546$ &$5.151$ &$-17.1708$ &$4.993$ &$0.0050$ \\
Os3 & 6s05d5 &$4.5913$ &$19.692$ &$4.0615$ &$8.862$ &$-0.9950$ &$1.086$ &$0.9081$ &$0.986$ &$-0.0077$ \\
Os4 & 6s05d4 &$3.9724$ &$17.514$ &$3.9817$ &$8.421$ &$-0.7719$ &$1.264$ &$0.6552$ &$1.107$ &$-0.0042$ \\
Os5 & 6s05d3 &$3.4764$ &$15.826$ &$3.9241$ &$8.032$ &$-0.6854$ &$1.397$ &$0.5394$ &$1.184$ &$-0.0024$ \\
Os6 & 6s05d2 &$3.0642$ &$14.479$ &$3.8847$ &$7.686$ &$-4.7235$ &$1.384$ &$4.5485$ &$1.352$ &$-0.0012$ \\
Os7 & 6s05d1 &$2.7164$ &$13.366$ &$3.8554$ &$7.370$ &$-5.0211$ &$1.439$ &$4.8166$ &$1.405$ &$-0.0004$ \\
Ir0 & 6s05d9 &$4.6102$ &$43.878$ &$5.4892$ &$16.394$ &$2.1707$ &$6.423$ &$-0.4761$ &$3.722$ &$0.0043$ \\
Ir0 & 6s15d8 &$3.9372$ &$37.508$ &$5.2846$ &$14.661$ &$3.7267$ &$5.263$ &$-2.3158$ &$4.416$ &$0.0045$ \\
Ir0 & 6s25d7 &$3.2263$ &$33.922$ &$5.1086$ &$13.921$ &$6.5993$ &$4.978$ &$-5.1841$ &$4.549$ &$0.0048$ \\
Ir0 & 6s05d8 &$3.4956$ &$32.991$ &$5.1369$ &$13.998$ &$8.3991$ &$4.924$ &$-7.0561$ &$4.612$ &$0.0047$ \\
Ir0 & 6s15d7 &$2.8732$ &$30.809$ &$5.0094$ &$13.522$ &$6.8656$ &$4.933$ &$-5.4669$ &$4.526$ &$0.0050$ \\
Ir2 & 6s05d7 &$2.4419$ &$28.356$ &$4.9470$ &$13.222$ &$4.7478$ &$5.029$ &$-3.3259$ &$4.393$ &$0.0052$ \\
Ir3 & 6s05d6 &$1.5883$ &$25.969$ &$4.8472$ &$12.711$ &$5.6507$ &$4.949$ &$-4.1190$ &$4.388$ &$0.0056$ \\
Ir4 & 6s05d5 &$3.5964$ &$16.439$ &$3.7872$ &$7.873$ &$-1.2303$ &$1.092$ &$1.1232$ &$1.003$ &$-0.0068$ \\
\hline
\end{tabular}
}
\end{table}
\section{{\large $\langle j_2\rangle$} Form factors for 4f electrons of rare earth ions}
\begin{table}[H]
\caption{$\langle j_2\rangle$ form factors for 4f electrons of rare-earth ions}
\vspace{2mm}
\label{re4fj2}
{\tablesize 
\begin{tabular}{lrrrrrrr}
\hline
Ion&
\multicolumn{1}{c}{A}&\multicolumn{1}{c}{a}&
\multicolumn{1}{c}{B}&\multicolumn{1}{c}{b}&
\multicolumn{1}{c}{C}&\multicolumn{1}{c}{c}&\multicolumn{1}{c}{D}\\
\hline\\[-2ex]
Ce2 &$0.9809$ &$18.0630$ &$1.8413$ &$7.7688$ &$0.9905$ &$2.8452$ &$0.0120$ \\
Nd2 &$1.4530$ &$18.3398$ &$1.6196$ &$7.2854$ &$0.8752$ &$2.6224$ &$0.0126$ \\
Nd3 &$0.6751$ &$18.3421$ &$1.6272$ &$7.2600$ &$0.9644$ &$2.6016$ &$0.0150$ \\
Sm2 &$1.0360$ &$18.4249$ &$1.4769$ &$7.0321$ &$0.8810$ &$2.4367$ &$0.0152$ \\
Sm3 &$0.4707$ &$18.4301$ &$1.4261$ &$7.0336$ &$0.9574$ &$2.4387$ &$0.0182$ \\
Eu2 &$0.8970$ &$18.4429$ &$1.3769$ &$7.0054$ &$0.9060$ &$2.4213$ &$0.0190$ \\
Eu3 &$0.3985$ &$18.4514$ &$1.3307$ &$6.9556$ &$0.9603$ &$2.3780$ &$0.0197$ \\
Gd2 &$0.7756$ &$18.4695$ &$1.3124$ &$6.8990$ &$0.8956$ &$2.3383$ &$0.0199$ \\
Gd3 &$0.3347$ &$18.4758$ &$1.2465$ &$6.8767$ &$0.9537$ &$2.3184$ &$0.0217$ \\
Tb2 &$0.6688$ &$18.4909$ &$1.2487$ &$6.8219$ &$0.8888$ &$2.2751$ &$0.0215$ \\
Tb3 &$0.2892$ &$18.4973$ &$1.1678$ &$6.7972$ &$0.9437$ &$2.2573$ &$0.0232$ \\
Dy2 &$0.5917$ &$18.5114$ &$1.1828$ &$6.7465$ &$0.8801$ &$2.2141$ &$0.0229$ \\
Dy3 &$0.2523$ &$18.5172$ &$1.0914$ &$6.7362$ &$0.9345$ &$2.2082$ &$0.0250$ \\
Ho2 &$0.5094$ &$18.5155$ &$1.1234$ &$6.7060$ &$0.8727$ &$2.1589$ &$0.0242$ \\
Ho3 &$0.2188$ &$18.5157$ &$1.0240$ &$6.7070$ &$0.9251$ &$2.1614$ &$0.0268$ \\
Er2 &$0.4693$ &$18.5278$ &$1.0545$ &$6.6493$ &$0.8679$ &$2.1201$ &$0.0261$ \\
Er3 &$0.1710$ &$18.5337$ &$0.9879$ &$6.6246$ &$0.9044$ &$2.1004$ &$0.0278$ \\
Tm2 &$0.4198$ &$18.5417$ &$0.9959$ &$6.6002$ &$0.8593$ &$2.0818$ &$0.0284$ \\
Tm3 &$0.1760$ &$18.5417$ &$0.9105$ &$6.5787$ &$0.8970$ &$2.0622$ &$0.0294$ \\
Yb2 &$0.3852$ &$18.5497$ &$0.9415$ &$6.5507$ &$0.8492$ &$2.0425$ &$0.0301$ \\
Yb3 &$0.1570$ &$18.5553$ &$0.8484$ &$6.5403$ &$0.8880$ &$2.0367$ &$0.0318$ \\
Pr3 &$0.8734$ &$18.9876$ &$1.5594$ &$6.0872$ &$0.8142$ &$2.4150$ &$0.0111$ \\
\hline\\[-2ex]
\end{tabular}
}
\end{table}
%
\section{{\large $\langle j_2\rangle$} Form factors for 5d electrons of rare earth elements}
\begin{table}[H]
 \caption{$\langle j_2\rangle$ form factors of the 5d electrons of rare earth ions.\cite{kob:12}}
 \vspace{1ex}
 \label{re5dj2}
{\tablesize
\begin{tabular}{llrrrrrrrrr}
\hline
\multicolumn{1}{c}{ Ion}&\multicolumn{1}{c}{ Config}&\multicolumn{1}{c}{ A }&\multicolumn{1}{c}{  a }&\multicolumn{1}{c}{B }&\multicolumn{1}{c}{ b }&\multicolumn{1}{c}{ C }&\multicolumn{1}{c}{ c }&\multicolumn{1}{c}{ D }&\multicolumn{1}{c}{ d }&\multicolumn{1}{c}{E}\\
\hline
La2 & 4f05d1 &$12.9681$ &$48.694$ &$9.8018$ &$22.831$ &$-0.5443$ &$4.354$ &$0.2584$ &$2.743$ &$-0.0001$ \\
Ce2 & 4f15d1 &$12.6550$ &$47.222$ &$9.3025$ &$21.878$ &$-0.6896$ &$3.762$ &$0.4405$ &$2.849$ &$0.0000$ \\
Ce3 & 4f05d1 &$8.7013$ &$35.824$ &$8.5894$ &$18.242$ &$-0.6110$ &$4.306$ &$0.2181$ &$2.336$ &$-0.0003$ \\
Pr2 & 4f25d1 &$12.4487$ &$45.891$ &$8.8239$ &$21.010$ &$-17.0630$ &$3.120$ &$16.8417$ &$3.095$ &$0.0001$ \\
Pr3 & 4f15d1 &$8.5203$ &$34.877$ &$8.2069$ &$17.602$ &$-0.5841$ &$3.958$ &$0.2347$ &$2.302$ &$-0.0002$ \\
Pr4 & 4f05d1 &$6.3910$ &$28.309$ &$7.6746$ &$15.204$ &$-0.6973$ &$4.010$ &$0.2304$ &$2.119$ &$-0.0004$ \\
Nd2 & 4f35d1 &$6.2640$ &$27.628$ &$7.3661$ &$14.731$ &$-0.6560$ &$3.743$ &$0.2354$ &$2.073$ &$-0.0003$ \\
Nd3 & 4f25d1 &$8.3541$ &$34.036$ &$7.8675$ &$17.031$ &$-0.5998$ &$3.584$ &$0.2896$ &$2.326$ &$0.0000$ \\
Nd4 & 4f15d1 &$6.2640$ &$27.628$ &$7.3661$ &$14.731$ &$-0.6560$ &$3.743$ &$0.2354$ &$2.073$ &$-0.0003$ \\
Pm3 & 4f35d1 &$8.2028$ &$33.286$ &$7.5637$ &$16.518$ &$-0.8756$ &$3.105$ &$0.6007$ &$2.486$ &$0.0002$ \\
Sm2 & 4f55d1 &$12.0956$ &$42.678$ &$7.6076$ &$18.857$ &$-1.4355$ &$2.749$ &$1.2720$ &$2.547$ &$0.0001$ \\
Sm3 & 4f45d1 &$8.1212$ &$32.502$ &$7.2340$ &$15.990$ &$-5.8639$ &$2.698$ &$5.6142$ &$2.630$ &$0.0003$ \\
Eu2 & 4f65d1 &$12.0033$ &$41.860$ &$7.2896$ &$18.289$ &$-1.4882$ &$2.592$ &$1.3411$ &$2.425$ &$0.0000$ \\
Eu3 & 4f55d1 &$8.0653$ &$31.753$ &$6.9151$ &$15.484$ &$-5.5701$ &$2.578$ &$5.3406$ &$2.515$ &$0.0003$ \\
Gd0 & 4f85d1 &$22.2510$ &$65.913$ &$7.5925$ &$23.045$ &$-3.5803$ &$1.952$ &$3.5355$ &$1.930$ &$-0.0010$ \\
Gd0 & 6s14f75d1 &$14.0295$ &$48.586$ &$7.3596$ &$18.927$ &$-5.1053$ &$2.242$ &$5.0055$ &$2.211$ &$-0.0004$ \\
Gd2 & 4f75d1 &$11.9202$ &$41.129$ &$7.0021$ &$17.772$ &$-5.3982$ &$2.392$ &$5.2663$ &$2.354$ &$-0.0001$ \\
Gd3 & 4f65d1 &$8.0082$ &$31.074$ &$6.6281$ &$15.022$ &$-1.9932$ &$2.515$ &$1.7825$ &$2.354$ &$0.0003$ \\
Gd4 & 4f55d1 &$5.8839$ &$25.328$ &$6.3297$ &$13.143$ &$-1.1650$ &$2.597$ &$0.8829$ &$2.219$ &$0.0004$ \\
Tb0 & 4f95d1 &$22.4050$ &$65.581$ &$7.3158$ &$22.565$ &$-4.9513$ &$1.745$ &$4.9149$ &$1.732$ &$-0.0014$ \\
Tb0 & 6s14f85d1 &$14.0011$ &$48.000$ &$7.0994$ &$18.468$ &$-0.6304$ &$2.188$ &$0.5431$ &$1.960$ &$-0.0007$ \\
Tb2 & 4f85d1 &$11.8504$ &$40.480$ &$6.7412$ &$17.303$ &$-11.6794$ &$2.248$ &$11.5615$ &$2.233$ &$-0.0002$ \\
Tb3 & 4f75d1 &$7.9534$ &$30.456$ &$6.3665$ &$14.596$ &$-0.8418$ &$2.508$ &$0.6482$ &$2.145$ &$0.0002$ \\
Tb4 & 4f65d1 &$5.8406$ &$24.788$ &$6.0831$ &$12.775$ &$-2.0431$ &$2.400$ &$1.7828$ &$2.221$ &$0.0004$ \\
Dy2 & 4f95d1 &$11.7905$ &$39.898$ &$6.5024$ &$16.874$ &$-11.5993$ &$2.115$ &$11.4943$ &$2.102$ &$-0.0003$ \\
Dy3 & 4f85d1 &$7.8866$ &$29.902$ &$6.1361$ &$14.212$ &$-12.5093$ &$2.229$ &$12.3329$ &$2.210$ &$0.0002$ \\
Ho3 & 4f95d1 &$7.8279$ &$29.392$ &$5.9224$ &$13.854$ &$-3.0002$ &$2.151$ &$2.8393$ &$2.080$ &$0.0001$ \\
Er3 & 4f105d1 &$7.7700$ &$28.922$ &$5.7263$ &$13.523$ &$-4.1104$ &$2.035$ &$3.9640$ &$1.990$ &$0.0000$ \\
Tm2 & 4f125d1 &$11.6672$ &$38.468$ &$5.8888$ &$15.774$ &$-4.8013$ &$1.717$ &$4.7296$ &$1.697$ &$-0.0012$ \\
Tm3 & 4f115d1 &$7.7142$ &$28.488$ &$5.5450$ &$13.214$ &$-5.7987$ &$1.923$ &$5.6660$ &$1.895$ &$-0.0002$ \\
Tm4 & 4f105d1 &$5.6664$ &$23.000$ &$5.2790$ &$11.534$ &$-3.2389$ &$1.988$ &$3.0496$ &$1.919$ &$0.0002$ \\
Yb2 & 4f135d1 &$11.6437$ &$38.074$ &$5.7103$ &$15.455$ &$-2.9992$ &$1.582$ &$2.9369$ &$1.554$ &$-0.0017$ \\
Yb3 & 4f125d1 &$7.6609$ &$28.088$ &$5.3775$ &$12.926$ &$-5.9455$ &$1.817$ &$5.8255$ &$1.793$ &$-0.0004$ \\
Lu2 & 4f145d1 &$11.6308$ &$37.717$ &$5.5427$ &$15.158$ &$-6.0682$ &$1.432$ &$6.0147$ &$1.420$ &$-0.0024$ \\
Lu3 & 4f135d1 &$7.6106$ &$27.715$ &$5.2212$ &$12.656$ &$-2.1969$ &$1.729$ &$2.0887$ &$1.671$ &$-0.0007$ \\
\hline
\end{tabular}
}
\end{table}
\section{{\large $\langle j_2\rangle$} Form factors for 5f electrons  of actinide ions}
\begin{table}[H]
\caption{$\langle j_2\rangle$ Form factors for actinide ions}\vspace{2mm}
\label{acj2}
{\tablesize
\begin{tabular}{lrrrrrrr}
\hline
Ion&
\multicolumn{1}{c}{A}&\multicolumn{1}{c}{a}&
\multicolumn{1}{c}{B}&\multicolumn{1}{c}{b}&
\multicolumn{1}{c}{C}&\multicolumn{1}{c}{c}&\multicolumn{1}{c}{D}\\
\hline\\[-2ex]
U3 &$4.1582$ &$16.5336$ &$2.4675$ &$5.9516$ &$-0.0252$ &$0.7646$ &$0.0057$ \\
U4 &$3.7449$ &$13.8944$ &$2.6453$ &$4.8634$ &$-0.5218$ &$3.1919$ &$0.0009$ \\
U5 &$3.0724$ &$12.5460$ &$2.3076$ &$5.2314$ &$-0.0644$ &$1.4738$ &$0.0035$ \\
Np3 &$3.7170$ &$15.1333$ &$2.3216$ &$5.5025$ &$-0.0275$ &$0.7996$ &$0.0052$ \\
Np4 &$2.9203$ &$14.6463$ &$2.5979$ &$5.5592$ &$-0.0301$ &$0.3669$ &$0.0141$ \\
Np5 &$2.3308$ &$13.6540$ &$2.7219$ &$5.4935$ &$-0.1357$ &$0.0493$ &$0.1224$ \\
Np6 &$1.8245$ &$13.1803$ &$2.8508$ &$5.4068$ &$-0.1579$ &$0.0444$ &$0.1438$ \\
Pu3 &$2.0885$ &$12.8712$ &$2.5961$ &$5.1896$ &$-0.1465$ &$0.0393$ &$0.1343$ \\
Pu4 &$2.7244$ &$12.9262$ &$2.3387$ &$5.1633$ &$-0.1300$ &$0.0457$ &$0.1177$ \\
Pu5 &$2.1409$ &$12.8319$ &$2.5664$ &$5.1522$ &$-0.1338$ &$0.0457$ &$0.1210$ \\
Pu6 &$1.7262$ &$12.3240$ &$2.6652$ &$5.0662$ &$-0.1695$ &$0.0406$ &$0.1550$ \\
Am2 &$3.5237$ &$15.9545$ &$2.2855$ &$5.1946$ &$-0.0142$ &$0.5853$ &$0.0033$ \\
Am3 &$2.8622$ &$14.7328$ &$2.4099$ &$5.1439$ &$-0.1326$ &$0.0309$ &$0.1233$ \\
Am4 &$2.4141$ &$12.9478$ &$2.3687$ &$4.9447$ &$-0.2490$ &$0.0215$ &$0.2371$ \\
Am5 &$2.0109$ &$12.0534$ &$2.4155$ &$4.8358$ &$-0.2264$ &$0.0275$ &$0.2128$ \\
Am6 &$1.6778$ &$11.3372$ &$2.4531$ &$4.7247$ &$-0.2043$ &$0.0337$ &$0.1892$ \\
Am7 &$1.8845$ &$9.1606$ &$2.0746$ &$4.0422$ &$-0.1318$ &$1.7227$ &$0.0020$ \\
\hline\\[-2ex]
\end{tabular}
}
\end{table}
\section{{\large$\langle j_4\rangle$} Form factors for 3d atoms and ions}
\begin{table}[H]
\caption{$\langle j_4\rangle$ form factors for 3d atoms and ions}\vspace{2mm}
\label{3dj4}
{\tablesize
\begin{tabular}{lrrrrrrr}
\hline
Ion&
\multicolumn{1}{c}{A}&\multicolumn{1}{c}{a}&
\multicolumn{1}{c}{B}&\multicolumn{1}{c}{b}&
\multicolumn{1}{c}{C}&\multicolumn{1}{c}{c}&\multicolumn{1}{c}{D}\\
\hline\\[-2ex]
Sc0 &$1.3420$ &$10.2000$ &$0.3837$ &$3.0786$ &$0.0468$ &$0.1178$ &$-0.0328$ \\
Sc1 &$7.1167$ &$15.4872$ &$-6.6671$ &$18.2692$ &$0.4900$ &$2.9917$ &$0.0047$ \\
Sc2 &$-1.6684$ &$15.6475$ &$1.7742$ &$9.0624$ &$0.4075$ &$2.4116$ &$0.0042$ \\
Ti0 &$-2.1515$ &$11.2705$ &$2.5149$ &$8.8590$ &$0.3555$ &$2.1491$ &$0.0045$ \\
Ti1 &$-1.0383$ &$16.1899$ &$1.4699$ &$8.9239$ &$0.3631$ &$2.2834$ &$0.0044$ \\
Ti2 &$-1.3242$ &$15.3096$ &$1.2042$ &$7.8994$ &$0.3976$ &$2.1562$ &$0.0051$ \\
Ti3 &$-1.1117$ &$14.6349$ &$0.7689$ &$6.9267$ &$0.4385$ &$2.0886$ &$0.0060$ \\
V0 &$-0.9633$ &$15.2729$ &$0.9274$ &$7.7315$ &$0.3891$ &$2.0530$ &$0.0063$ \\
V1 &$-0.9606$ &$15.5451$ &$1.1278$ &$8.1182$ &$0.3653$ &$2.0973$ &$0.0056$ \\
V2 &$-1.1729$ &$14.9732$ &$0.9092$ &$7.6131$ &$0.4105$ &$2.0391$ &$0.0067$ \\
V3 &$-0.9417$ &$14.2045$ &$0.5284$ &$6.6071$ &$0.4411$ &$1.9672$ &$0.0076$ \\
V4 &$-0.7654$ &$13.0970$ &$0.3071$ &$5.6739$ &$0.4476$ &$1.8707$ &$0.0081$ \\
Cr0 &$-0.6670$ &$19.6128$ &$0.5342$ &$6.4779$ &$0.3641$ &$1.9045$ &$0.0073$ \\
Cr1 &$-0.8309$ &$18.0428$ &$0.7252$ &$7.5313$ &$0.3828$ &$2.0032$ &$0.0073$ \\
Cr2 &$-0.8930$ &$15.6641$ &$0.5590$ &$7.0333$ &$0.4093$ &$1.9237$ &$0.0081$ \\
Cr3 &$-0.7327$ &$14.0727$ &$0.3268$ &$5.6741$ &$0.4114$ &$1.8101$ &$0.0085$ \\
Cr4 &$-0.6748$ &$12.9462$ &$0.1805$ &$6.7527$ &$0.4526$ &$1.7999$ &$0.0098$ \\
Mn0 &$-0.5452$ &$15.4713$ &$0.4406$ &$4.9024$ &$0.2884$ &$1.5430$ &$0.0059$ \\
Mn1 &$-0.7947$ &$17.8673$ &$0.6078$ &$7.7044$ &$0.3798$ &$1.9045$ &$0.0087$ \\
Mn2 &$-0.7416$ &$15.2555$ &$0.3831$ &$6.4693$ &$0.3935$ &$1.7997$ &$0.0093$ \\
Mn3 &$-0.6603$ &$13.6066$ &$0.2322$ &$6.2175$ &$0.4104$ &$1.7404$ &$0.0101$ \\
Mn4 &$-0.5127$ &$13.4613$ &$0.0313$ &$7.7631$ &$0.4282$ &$1.7006$ &$0.0113$ \\
Fe0 &$-0.5029$ &$19.6768$ &$0.2999$ &$3.7762$ &$0.2576$ &$1.4241$ &$0.0071$ \\
Fe1 &$-0.5109$ &$19.2501$ &$0.3896$ &$4.8913$ &$0.2810$ &$1.5265$ &$0.0069$ \\
Fe2 &$-0.5401$ &$17.2268$ &$0.2865$ &$3.7422$ &$0.2658$ &$1.4238$ &$0.0076$ \\
Fe3 &$-0.5507$ &$11.4929$ &$0.2153$ &$4.9063$ &$0.3468$ &$1.5230$ &$0.0095$ \\
Fe4 &$-0.5352$ &$9.5068$ &$0.1783$ &$5.1750$ &$0.3584$ &$1.4689$ &$0.0097$ \\
Co0 &$-0.4221$ &$14.1952$ &$0.2900$ &$3.9786$ &$0.2469$ &$1.2859$ &$0.0063$ \\
Co1 &$-0.4115$ &$14.5615$ &$0.3580$ &$4.7170$ &$0.2644$ &$1.4183$ &$0.0074$ \\
Co2 &$-0.4759$ &$14.0462$ &$0.2747$ &$3.7306$ &$0.2458$ &$1.2504$ &$0.0057$ \\
Co3 &$-0.4466$ &$13.3912$ &$0.1419$ &$3.0110$ &$0.2773$ &$1.3351$ &$0.0093$ \\
Co4 &$-0.4091$ &$13.1937$ &$-0.0194$ &$3.4169$ &$0.3534$ &$1.4214$ &$0.0112$ \\
Ni0 &$-0.4428$ &$14.4850$ &$0.0870$ &$3.2345$ &$0.2932$ &$1.3305$ &$0.0096$ \\
Ni1 &$-0.3836$ &$13.4246$ &$0.3116$ &$4.4619$ &$0.2471$ &$1.3088$ &$0.0079$ \\
Ni2 &$-0.3803$ &$10.4033$ &$0.2838$ &$3.3780$ &$0.2108$ &$1.1036$ &$0.0050$ \\
Ni3 &$-0.4014$ &$ 9.0462$ &$0.2314$ &$3.0753$ &$0.2192$ &$1.0838$ &$0.0060$ \\
Ni4 &$-0.3509$ &$8.1572$ &$0.2220$ &$2.1063$ &$0.1567$ &$0.9253$ &$0.0065$ \\
Cu0 &$-0.3204$ &$15.1324$ &$0.2335$ &$4.0205$ &$0.2312$ &$1.1957$ &$0.0068$ \\
Cu1 &$-0.3572$ &$15.1251$ &$0.2336$ &$3.9662$ &$0.2315$ &$1.1967$ &$0.0070$ \\
Cu2 &$-0.3914$ &$14.7400$ &$0.1275$ &$3.3840$ &$0.2548$ &$1.2552$ &$0.0103$ \\
Cu3 &$-0.3671$ &$14.0816$ &$-0.0078$ &$3.3149$ &$0.3154$ &$1.3767$ &$0.0132$ \\
Cu4 &$-0.2915$ &$14.1243$ &$-0.1065$ &$4.2008$ &$0.3247$ &$1.3516$ &$0.0148$ \\
\hline\\[-2ex]
\end{tabular}
}
\end{table}
\section{{\large$\langle j_4\rangle$} Form factors 4d atoms and ions}
\begin{table}[H]
\noindent\caption{$\langle j_4\rangle$ form factors for 4d atoms and ions}\
\label{4dj4}
 \vspace{1ex}
{\tablesize
\begin{tabular}{lrrrrrrr}
\hline
Ion&
\multicolumn{1}{c}{A}&\multicolumn{1}{c}{a}&
\multicolumn{1}{c}{B}&\multicolumn{1}{c}{b}&
\multicolumn{1}{c}{C}&\multicolumn{1}{c}{c}&\multicolumn{1}{c}{D}\\
\hline\\[-2ex]
Y0 &$-8.0767$ &$32.2014$ &$7.9197$ &$25.1563$ &$1.4067$ &$6.8268$ &$-0.0001$ \\
Zr0 &$-5.2697$ &$32.8680$ &$4.1930$ &$24.1833$ &$1.5202$ &$6.0481$ &$-0.0002$ \\
Zr1 &$-5.6384$ &$33.6071$ &$4.6729$ &$22.3383$ &$1.3258$ &$5.9245$ &$-0.0003$ \\
Nb0 &$-3.1377$ &$25.5948$ &$2.3411$ &$16.5686$ &$1.2304$ &$4.9903$ &$-0.0005$ \\
Nb1 &$-3.3598$ &$25.8202$ &$2.8297$ &$16.4273$ &$1.1203$ &$4.9824$ &$-0.0005$ \\
Mo0 &$-2.8860$ &$20.5717$ &$1.8130$ &$14.6281$ &$1.1899$ &$4.2638$ &$-0.0008$ \\
Mo1 &$-3.2618$ &$25.4862$ &$2.3596$ &$16.4622$ &$1.1164$ &$4.4913$ &$-0.0007$ \\
Tc0 &$-2.7975$ &$20.1589$ &$1.6520$ &$16.2609$ &$1.1726$ &$3.9427$ &$-0.0008$ \\
Tc1 &$-2.0470$ &$19.6830$ &$1.6306$ &$11.5925$ &$0.8698$ &$3.7689$ &$-0.0010$ \\
Ru0 &$-1.5042$ &$17.9489$ &$0.6027$ &$9.9608$ &$0.9700$ &$3.3927$ &$-0.0010$ \\
Ru1 &$-1.6278$ &$18.5063$ &$1.1828$ &$10.1886$ &$0.8138$ &$3.4180$ &$-0.0009$ \\
Rh0 &$-1.3492$ &$17.5766$ &$0.4527$ &$10.5066$ &$0.9285$ &$3.1555$ &$-0.0009$ \\
Rh1 &$-1.4673$ &$17.9572$ &$0.7381$ &$9.9444$ &$0.8485$ &$3.1263$ &$-0.0012$ \\
Pd0 &$-1.1955$ &$17.6282$ &$0.3183$ &$11.3094$ &$0.8696$ &$2.9089$ &$-0.0006$ \\
Pd1 &$-1.4098$ &$17.7650$ &$0.7927$ &$9.9991$ &$0.7710$ &$2.9297$ &$-0.0006$ \\
\hline\\[-2ex]
\end{tabular}
}
\end{table}
\section{{\large$\langle j_4\rangle$} Form factors for 5d atoms and ions}
\begin{table}[H]
 \caption{$\langle j_4\rangle$ form factors for the 5d electrons of transition atoms and ions from Hf to Re.\cite{kob:11}}
\label{5dj4}
 \vspace{1ex}
 {\tablesize
\begin{tabular}{llrrrrrrrrr}
\hline
\multicolumn{1}{c}{ Ion}&\multicolumn{1}{c}{ Config}&\multicolumn{1}{c}{ A }&\multicolumn{1}{c}{  a }&\multicolumn{1}{c}{B }&\multicolumn{1}{c}{ b }&\multicolumn{1}{c}{ C }&\multicolumn{1}{c}{ c }&\multicolumn{1}{c}{ D }&\multicolumn{1}{c}{ d }&\multicolumn{1}{c}{E }\\
\hline
Hf2 & 6s05d2 &$-2.5342$ &$43.826$ &$1.8466$ &$10.393$ &$0.7761$ &$4.888$ &$-0.0327$ &$1.589$ &$0.0017$ \\
Hf3 & 6s05d1 &$-2.3574$ &$32.651$ &$1.8717$ &$8.476$ &$0.6367$ &$3.953$ &$-0.1133$ &$2.169$ &$0.0017$ \\
Ta2 & 6s05d3 &$-2.1974$ &$38.294$ &$1.6220$ &$8.838$ &$0.6836$ &$4.212$ &$-0.0539$ &$1.746$ &$0.0016$ \\
Ta3 & 6s05d2 &$-2.0884$ &$29.531$ &$1.7145$ &$7.385$ &$1.1809$ &$2.994$ &$-0.7705$ &$2.577$ &$0.0016$ \\
Ta4 & 6s05d1 &$-2.0226$ &$24.035$ &$1.7084$ &$6.594$ &$1.1799$ &$2.775$ &$-0.8384$ &$2.430$ &$0.0020$ \\
W0 & 6s05d6 &$-2.1307$ &$79.955$ &$1.4055$ &$11.876$ &$0.8808$ &$4.969$ &$-0.0120$ &$0.419$ &$0.0051$ \\
W0 & 6s15d5 &$-1.9667$ &$60.069$ &$1.3446$ &$9.992$ &$0.7861$ &$4.554$ &$-0.0197$ &$1.087$ &$0.0021$ \\
W0 & 6s25d4 &$-1.8575$ &$47.554$ &$1.3868$ &$8.437$ &$0.6537$ &$3.939$ &$-0.0570$ &$1.722$ &$0.0015$ \\
W0 & 6s05d5 &$-2.0231$ &$46.962$ &$1.3937$ &$9.263$ &$0.7420$ &$4.332$ &$-0.0288$ &$1.327$ &$0.0018$ \\
W0 & 6s15d4 &$-1.9122$ &$39.952$ &$1.4385$ &$8.019$ &$0.6336$ &$3.725$ &$-0.0819$ &$1.868$ &$0.0015$ \\
W2 & 6s05d4 &$-1.9355$ &$33.935$ &$1.5020$ &$7.541$ &$0.6453$ &$3.375$ &$-0.1601$ &$2.110$ &$0.0015$ \\
W3 & 6s05d3 &$-1.8752$ &$26.706$ &$1.5440$ &$6.585$ &$5.9785$ &$2.575$ &$-5.6111$ &$2.516$ &$0.0018$ \\
W4 & 6s05d2 &$-1.8309$ &$22.142$ &$1.5605$ &$5.935$ &$1.4898$ &$2.482$ &$-1.1930$ &$2.266$ &$0.0022$ \\
W5 & 6s05d1 &$-1.7958$ &$18.987$ &$1.5913$ &$5.419$ &$2.0498$ &$2.278$ &$-1.8262$ &$2.156$ &$0.0026$ \\
Re0 & 6s05d7 &$-1.8013$ &$63.944$ &$1.1773$ &$9.808$ &$0.7912$ &$4.380$ &$-0.0155$ &$0.795$ &$0.0027$ \\
Re0 & 6s15d6 &$-1.7056$ &$49.628$ &$1.2209$ &$8.231$ &$0.6637$ &$3.836$ &$-0.0443$ &$1.514$ &$0.0015$ \\
Re0 & 6s25d5 &$-1.6402$ &$40.319$ &$1.3192$ &$7.064$ &$0.8659$ &$2.901$ &$-0.4299$ &$2.306$ &$0.0013$ \\
Re0 & 6s05d6 &$-1.7723$ &$40.683$ &$1.2795$ &$7.798$ &$0.6385$ &$3.630$ &$-0.0656$ &$1.686$ &$0.0015$ \\
Re0 & 6s15d5 &$-1.6968$ &$34.939$ &$1.3535$ &$6.851$ &$1.7292$ &$2.681$ &$-1.3118$ &$2.445$ &$0.0014$ \\
Re2 & 6s05d5 &$-1.7305$ &$30.305$ &$1.3808$ &$6.606$ &$1.5787$ &$2.630$ &$-1.1785$ &$2.378$ &$0.0015$ \\
Re3 & 6s05d4 &$-1.6969$ &$24.325$ &$1.4088$ &$5.901$ &$1.5067$ &$2.424$ &$-1.1834$ &$2.209$ &$0.0019$ \\
Re4 & 6s05d3 &$-1.6679$ &$20.454$ &$1.4439$ &$5.357$ &$0.8175$ &$2.321$ &$-0.5685$ &$1.976$ &$0.0024$ \\
Re5 & 6s05d2 &$-1.6427$ &$17.722$ &$1.4880$ &$4.916$ &$0.5419$ &$2.189$ &$-0.3694$ &$1.781$ &$0.0029$ \\
Re6 & 6s05d1 &$-1.6211$ &$15.673$ &$1.5419$ &$4.548$ &$1.9205$ &$1.833$ &$-1.8287$ &$1.766$ &$0.0034$ \\
\hline
\end{tabular}
}
\end{table}
\begin{table}[H]
 \caption{$\langle j_4\rangle$ form factors for the 5d electrons of transition atoms and ions from Os to Au.\cite{kob:11}}
\label{5dj4b} \vspace{1ex}
{\tablesize
\begin{tabular}{llrrrrrrrrr}
\hline
\multicolumn{1}{c}{ Ion}&\multicolumn{1}{c}{ Config}&\multicolumn{1}{c}{ A }&\multicolumn{1}{c}{  a }&\multicolumn{1}{c}{B }&\multicolumn{1}{c}{ b }&\multicolumn{1}{c}{ C }&\multicolumn{1}{c}{ c }&\multicolumn{1}{c}{ D }&\multicolumn{1}{c}{ d }&\multicolumn{1}{c}{E }\\
\hline
Os0 & 6s05d8 &$-1.5677$ &$53.075$ &$1.0631$ &$8.143$ &$0.6808$ &$3.771$ &$-0.0308$ &$1.243$ &$0.0017$ \\
Os0 & 6s15d7 &$-1.5109$ &$42.193$ &$1.1910$ &$6.850$ &$2.4597$ &$2.602$ &$-2.0163$ &$2.444$ &$0.0011$ \\
Os0 & 6s25d6 &$-1.4734$ &$34.814$ &$1.2105$ &$6.165$ &$0.9468$ &$2.534$ &$-0.5773$ &$2.131$ &$0.0015$ \\
Os0 & 6s05d7 &$-1.5777$ &$35.746$ &$1.2222$ &$6.650$ &$1.7260$ &$2.590$ &$-1.2942$ &$2.360$ &$0.0012$ \\
Os0 & 6s15d6 &$-1.5274$ &$30.891$ &$1.2364$ &$6.044$ &$1.4100$ &$2.431$ &$-1.0480$ &$2.188$ &$0.0016$ \\
Os2 & 6s05d6 &$-1.5637$ &$27.292$ &$1.2627$ &$5.880$ &$1.3526$ &$2.393$ &$-1.0026$ &$2.146$ &$0.0017$ \\
Os3 & 6s05d5 &$-1.5453$ &$22.300$ &$1.3058$ &$5.301$ &$0.7208$ &$2.294$ &$-0.4480$ &$1.881$ &$0.0022$ \\
Os4 & 6s05d4 &$-1.5267$ &$18.972$ &$1.3619$ &$4.834$ &$3.0121$ &$1.938$ &$-2.8240$ &$1.877$ &$0.0027$ \\
Os5 & 6s05d3 &$-1.5094$ &$16.573$ &$1.4158$ &$4.458$ &$0.8448$ &$1.797$ &$-0.7349$ &$1.638$ &$0.0033$ \\
Os6 & 6s05d2 &$-1.4938$ &$14.751$ &$1.4678$ &$4.149$ &$0.8091$ &$1.535$ &$-0.7720$ &$1.442$ &$0.0041$ \\
Os7 & 6s05d1 &$-0.0341$ &$37.994$ &$-1.4680$ &$13.159$ &$1.5216$ &$3.898$ &$-0.0308$ &$0.550$ &$0.0083$ \\
Ir0 & 6s05d9 &$-1.3913$ &$45.243$ &$1.0627$ &$6.722$ &$2.5141$ &$2.534$ &$-2.0510$ &$2.383$ &$0.0009$ \\
Ir0 & 6s15d8 &$-1.3605$ &$36.399$ &$1.0953$ &$5.990$ &$1.7223$ &$2.353$ &$-1.3416$ &$2.162$ &$0.0014$ \\
Ir0 & 6s25d7 &$-1.3382$ &$30.628$ &$1.1376$ &$5.420$ &$1.4261$ &$2.160$ &$-1.1282$ &$1.974$ &$0.0019$ \\
Ir0 & 6s05d8 &$-1.4233$ &$31.680$ &$1.1221$ &$5.872$ &$2.2721$ &$2.301$ &$-1.8973$ &$2.163$ &$0.0015$ \\
Ir0 & 6s15d7 &$-1.3875$ &$27.660$ &$1.1508$ &$5.362$ &$0.6586$ &$2.302$ &$-0.3554$ &$1.810$ &$0.0020$ \\
Ir2 & 6s05d7 &$-1.4233$ &$24.796$ &$1.1799$ &$5.246$ &$3.2548$ &$2.072$ &$-2.9649$ &$1.997$ &$0.0020$ \\
Ir3 & 6s05d6 &$-1.4149$ &$20.563$ &$1.2388$ &$4.761$ &$1.1780$ &$1.928$ &$-0.9708$ &$1.761$ &$0.0026$ \\
Ir4 & 6s05d5 &$-1.4039$ &$17.664$ &$1.2993$ &$4.371$ &$3.5599$ &$1.661$ &$-3.4340$ &$1.625$ &$0.0033$ \\
\hline
\end{tabular}
}
\end{table}
\section{{\large$\langle j_4\rangle$} Form factors for 4f electrons of rare earth ions}
\begin{table}[H]
\caption{$\langle j_4\rangle$ form factors for rare earth ions}\vspace{2mm}
\label{rej4}
{\tablesize
\begin{tabular}{lrrrrrrr}
\hline
Ion&
\multicolumn{1}{c}{A}&\multicolumn{1}{c}{a}&
\multicolumn{1}{c}{B}&\multicolumn{1}{c}{b}&
\multicolumn{1}{c}{C}&\multicolumn{1}{c}{c}&\multicolumn{1}{c}{D}\\
\hline\\[-2ex]
Ce2 &$-0.6468$ &$10.5331$ &$0.4052$ &$5.6243$ &$0.3412$ &$1.5346$ &$0.0080$ \\
Nd2 &$-0.5744$ &$10.9304$ &$0.4210$ &$6.1052$ &$0.3124$ &$1.4654$ &$0.0081$ \\
Nd2 &$-0.5416$ &$12.2043$ &$0.3571$ &$6.1695$ &$0.3154$ &$1.4847$ &$0.0098$ \\
Nd3 &$-0.4053$ &$14.0141$ &$0.0329$ &$7.0046$ &$0.3759$ &$1.7074$ &$0.0209$ \\
Sm2 &$-0.4150$ &$14.0570$ &$0.1368$ &$7.0317$ &$0.3272$ &$1.5825$ &$0.0192$ \\
Sm3 &$-0.4288$ &$10.0525$ &$0.1782$ &$5.0191$ &$0.2833$ &$1.2364$ &$0.0088$ \\
Eu2 &$-0.4145$ &$10.1930$ &$0.2447$ &$5.1644$ &$0.2661$ &$1.2054$ &$0.0065$ \\
Eu3 &$-0.4095$ &$10.2113$ &$0.1485$ &$5.1755$ &$0.2720$ &$1.2374$ &$0.0131$ \\
Gd2 &$-0.3824$ &$10.3436$ &$0.1955$ &$5.3057$ &$0.2622$ &$1.2032$ &$0.0097$ \\
Gd3 &$-0.3621$ &$10.3531$ &$0.1016$ &$5.3104$ &$0.2649$ &$1.2185$ &$0.0147$ \\
Tb2 &$-0.3443$ &$10.4686$ &$0.1481$ &$5.4156$ &$0.2575$ &$1.1824$ &$0.0104$ \\
Tb3 &$-0.3228$ &$10.4763$ &$0.0638$ &$5.4189$ &$0.2566$ &$1.1962$ &$0.0159$ \\
Dy2 &$-0.3206$ &$12.0714$ &$0.0904$ &$8.0264$ &$0.2616$ &$1.2296$ &$0.0143$ \\
Dy3 &$-0.2829$ &$9.5247$ &$0.0565$ &$4.4292$ &$0.2437$ &$1.0665$ &$0.0092$ \\
Ho2 &$-0.2976$ &$9.7190$ &$0.1224$ &$4.6345$ &$0.2279$ &$1.0052$ &$0.0063$ \\
Ho3 &$-0.2717$ &$9.7313$ &$0.0474$ &$4.6378$ &$0.2292$ &$1.0473$ &$0.0124$ \\
Er2 &$-0.2975$ &$9.8294$ &$0.1189$ &$4.7406$ &$0.2116$ &$1.0039$ &$0.0117$ \\
Er3 &$-0.2568$ &$9.8339$ &$0.0356$ &$4.7415$ &$0.2172$ &$1.0281$ &$0.0148$ \\
Tm2 &$-0.2677$ &$9.8883$ &$0.0925$ &$4.7838$ &$0.2056$ &$0.9896$ &$0.0124$ \\
Tm3 &$-0.2292$ &$9.8948$ &$0.0124$ &$4.7850$ &$0.2108$ &$1.0071$ &$0.0151$ \\
Yb2 &$-0.2393$ &$9.9469$ &$0.0663$ &$4.8231$ &$0.2009$ &$0.9651$ &$0.0122$ \\
Yb3 &$-0.2121$ &$8.1967$ &$0.0325$ &$3.1533$ &$0.1975$ &$0.8842$ &$0.0093$ \\
Pr3 &$-0.3970$ &$10.9919$ &$0.0818$ &$5.9897$ &$0.3656$ &$1.5021$ &$0.0110$ \\
\hline\\[-2ex]
\end{tabular}
}
\end{table}
\section{{\large$\langle j_4\rangle$} Form factors for 4f electrons of actinide ions}
\begin{table}[H]
\caption{$\langle j_4\rangle$ Form factors for actinide ions}\vspace{2mm}
\label{acj4}
{\tablesize
\begin{tabular}{lrrrrrrr}
\hline
Ion&
\multicolumn{1}{c}{A}&\multicolumn{1}{c}{a}&
\multicolumn{1}{c}{B}&\multicolumn{1}{c}{b}&
\multicolumn{1}{c}{C}&\multicolumn{1}{c}{c}&\multicolumn{1}{c}{D}\\
\hline\\[-2ex]
U3 &$-0.9859$ &$16.6010$ &$0.6116$ &$6.5147$ &$0.6020$ &$2.5970$ &$-0.0010$ \\
U4 &$-1.0540$ &$16.6055$ &$0.4339$ &$6.5119$ &$0.6746$ &$2.5993$ &$-0.0011$ \\
U5 &$-0.9588$ &$16.4851$ &$0.1576$ &$6.4397$ &$0.7785$ &$2.6402$ &$-0.0010$ \\
Np3 &$-0.9029$ &$16.5858$ &$0.4006$ &$6.4699$ &$0.6545$ &$2.5631$ &$-0.0004$ \\
Np4 &$-0.9887$ &$12.4415$ &$0.5918$ &$5.2941$ &$0.5306$ &$2.2625$ &$-0.0021$ \\
Np5 &$-0.8146$ &$16.5809$ &$-0.0055$ &$6.4751$ &$0.7956$ &$2.5623$ &$-0.0004$ \\
Np6 &$-0.6738$ &$16.5531$ &$-0.2297$ &$6.5055$ &$0.8513$ &$2.5528$ &$-0.0003$ \\
Pu3 &$-0.7014$ &$16.3687$ &$-0.1162$ &$6.6971$ &$0.7778$ &$2.4502$ &$0.0000$ \\
Pu4 &$-0.9160$ &$12.2027$ &$0.4891$ &$5.1274$ &$0.5290$ &$2.1487$ &$-0.0022$ \\
Pu5 &$-0.7035$ &$16.3601$ &$-0.0979$ &$6.7057$ &$0.7726$ &$2.4475$ &$0.0000$ \\
Pu6 &$-0.5560$ &$16.3215$ &$-0.3046$ &$6.7685$ &$0.8146$ &$2.4259$ &$0.0001$ \\
Am2 &$-0.7433$ &$16.4163$ &$0.3481$ &$6.7884$ &$0.6014$ &$2.3465$ &$0.0000$ \\
Am3 &$-0.8092$ &$12.8542$ &$0.4161$ &$5.4592$ &$0.5476$ &$2.1721$ &$-0.0011$ \\
Am4 &$-0.8548$ &$12.2257$ &$0.3037$ &$5.9087$ &$0.6173$ &$2.1881$ &$-0.0016$ \\
Am5 &$-0.6538$ &$15.4625$ &$-0.0948$ &$5.9971$ &$0.7295$ &$2.2968$ &$0.0000$ \\
Am6 &$-0.5390$ &$15.4491$ &$-0.2689$ &$6.0169$ &$0.7711$ &$2.2970$ &$0.0002$ \\
Am7 &$-0.4688$ &$12.0193$ &$-0.2692$ &$7.0415$ &$0.7297$ &$2.1638$ &$-0.0011$ \\
\hline\\[-2ex]
\end{tabular}
}
\end{table}
\section{{\large $\langle j_6\rangle$} Form factors for 4f electrons of rare earth ions}
\begin{table}[H]
\caption{$\langle j_6\rangle$ Form factors for rare earth ions}\vspace{2mm}
\label{rej6}
{\tablesize
\begin{tabular}{lrrrrrrr}
\hline
Ion&
\multicolumn{1}{c}{A}&\multicolumn{1}{c}{a}&
\multicolumn{1}{c}{B}&\multicolumn{1}{c}{b}&
\multicolumn{1}{c}{C}&\multicolumn{1}{c}{c}&\multicolumn{1}{c}{D}\\
\hline\\[-2ex]
Ce2 &$-0.1212$ &$7.9940$ &$-0.0639$ &$4.0244$ &$0.1519$ &$1.0957$ &$0.0078$ \\
Nd2 &$-0.1600$ &$8.0086$ &$0.0272$ &$4.0284$ &$0.1104$ &$1.0682$ &$0.0139$ \\
Nd3 &$-0.0416$ &$8.0136$ &$-0.1261$ &$4.0399$ &$0.1400$ &$1.0873$ &$0.0102$ \\
Sm2 &$-0.1428$ &$6.0407$ &$0.0723$ &$2.0329$ &$0.0550$ &$0.5134$ &$0.0081$ \\
Sm3 &$-0.0944$ &$6.0299$ &$-0.0498$ &$2.0743$ &$0.1372$ &$0.6451$ &$-0.0132$ \\
Eu2 &$-0.1252$ &$6.0485$ &$0.0507$ &$2.0852$ &$0.0572$ &$0.6460$ &$0.0132$ \\
Eu3 &$-0.0817$ &$6.0389$ &$-0.0596$ &$2.1198$ &$0.1243$ &$0.7639$ &$-0.0001$ \\
Gd2 &$-0.1351$ &$5.0298$ &$0.0828$ &$2.0248$ &$0.0315$ &$0.5034$ &$0.0187$ \\
Gd3 &$-0.0662$ &$6.0308$ &$-0.0850$ &$2.1542$ &$0.1323$ &$0.8910$ &$0.0048$ \\
Tb2 &$-0.0758$ &$6.0319$ &$-0.0540$ &$2.1583$ &$0.1199$ &$0.8895$ &$0.0051$ \\
Tb3 &$-0.0559$ &$6.0311$ &$-0.1020$ &$2.2365$ &$0.1264$ &$1.1066$ &$0.0167$ \\
Dy2 &$-0.0568$ &$6.0324$ &$-0.1003$ &$2.2396$ &$0.1401$ &$1.1062$ &$0.0109$ \\
Dy3 &$-0.0423$ &$6.0376$ &$-0.1248$ &$2.2437$ &$0.1359$ &$1.2002$ &$0.0188$ \\
Ho2 &$-0.0725$ &$6.0453$ &$-0.0318$ &$2.2428$ &$0.0738$ &$1.2018$ &$0.0252$ \\
Ho3 &$-0.0289$ &$6.0504$ &$-0.1545$ &$2.2305$ &$0.1550$ &$1.2605$ &$0.0177$ \\
Er2 &$-0.0648$ &$6.0559$ &$-0.0515$ &$2.2303$ &$0.0825$ &$1.2638$ &$0.0250$ \\
Er3 &$-0.0110$ &$6.0609$ &$-0.1954$ &$2.2242$ &$0.1818$ &$1.2958$ &$0.0149$ \\
Tm2 &$-0.0842$ &$4.0699$ &$0.0807$ &$0.8492$ &$-0.2087$ &$0.0386$ &$0.2095$ \\
Tm3 &$-0.0727$ &$4.0730$ &$0.0243$ &$0.6888$ &$3.9459$ &$0.0023$ &$-3.9076$ \\
Yb2 &$-0.0739$ &$5.0306$ &$0.0140$ &$2.0300$ &$0.0351$ &$0.5080$ &$0.0174$ \\
Yb3 &$-0.0345$ &$5.0073$ &$-0.0677$ &$2.0198$ &$0.0985$ &$0.5485$ &$-0.0076$ \\
Pr3 &$-0.0224$ &$7.9931$ &$-0.1202$ &$3.9406$ &$0.1299$ &$0.8938$ &$0.0051$ \\
\hline\\[-2ex]
\end{tabular}
}
\end{table}
\section{{\large$\langle j_6\rangle$} Form factors for 4f electrons of actinide ions}
\begin{table}[H]
\caption{$\langle j_6\rangle$ Form factors for actinide ions}\vspace{2mm}
\label{acj6}
{\tablesize
\begin{tabular}{lrrrrrrr}
\hline
Ion&
\multicolumn{1}{c}{A}&\multicolumn{1}{c}{a}&
\multicolumn{1}{c}{B}&\multicolumn{1}{c}{b}&
\multicolumn{1}{c}{C}&\multicolumn{1}{c}{c}&\multicolumn{1}{c}{D}\\
\hline\\[-2ex]
U3 &$-0.3797$ &$9.9525$ &$0.0459$ &$5.0379$ &$0.2748$ &$1.6072$ &$0.0016$ \\
U4 &$-0.1793$ &$11.8961$ &$-0.2269$ &$5.4280$ &$0.3291$ &$1.7008$ &$0.0030$ \\
U5 &$-0.0399$ &$11.8909$ &$-0.3458$ &$5.5803$ &$0.3340$ &$1.6448$ &$0.0029$ \\
Np3 &$-0.2427$ &$11.8444$ &$-0.1129$ &$5.3774$ &$0.2848$ &$1.5676$ &$0.0022$ \\
Np4 &$-0.2436$ &$9.5988$ &$-0.1317$ &$4.1014$ &$0.3029$ &$1.5447$ &$0.0019$ \\
Np5 &$-0.1157$ &$9.5649$ &$-0.2654$ &$4.2599$ &$0.3298$ &$1.5494$ &$0.0025$ \\
Np6 &$-0.0128$ &$9.5692$ &$-0.3611$ &$4.3035$ &$0.3419$ &$1.5406$ &$0.0032$ \\
Pu3 &$-0.0364$ &$9.5721$ &$-0.3181$ &$4.3424$ &$0.3210$ &$1.5233$ &$0.0041$ \\
Pu4 &$-0.2394$ &$7.8367$ &$-0.0785$ &$4.0243$ &$0.2643$ &$1.3776$ &$0.0012$ \\
Pu5 &$-0.1090$ &$7.8188$ &$-0.2243$ &$4.1000$ &$0.2947$ &$1.4040$ &$0.0015$ \\
Pu6 &$-0.0001$ &$7.8196$ &$-0.3354$ &$4.1439$ &$0.3097$ &$1.4027$ &$0.0020$ \\
Am2 &$-0.3176$ &$7.8635$ &$0.0771$ &$4.1611$ &$0.2194$ &$1.3387$ &$0.0018$ \\
Am3 &$-0.3159$ &$6.9821$ &$0.0682$ &$3.9948$ &$0.2141$ &$1.1875$ &$-0.0015$ \\
Am4 &$-0.1787$ &$7.8805$ &$-0.1274$ &$4.0898$ &$0.2565$ &$1.3152$ &$0.0017$ \\
Am5 &$-0.0927$ &$6.0727$ &$-0.2227$ &$3.7840$ &$0.2916$ &$1.3723$ &$0.0026$ \\
Am6 &$0.0152$ &$6.0788$ &$-0.3549$ &$3.8610$ &$0.3125$ &$1.4031$ &$0.0036$ \\
Am7 &$0.1292$ &$6.0816$ &$-0.4689$ &$3.8791$ &$0.3234$ &$1.3934$ &$0.0042$ \\
\hline\\[-2ex]
\end{tabular}
}
\end{table}



\begin{thebibliography}{99}
\bibitem{Marshall}Marshall W and Lovesey S W{\it Theory of thermal neutron
scattering Chapter 6} Oxford University Press (1971)
\bibitem{clementi} Clementi E and Roetti C {\it Atomic Data and Nuclear Data
Tables}{\bf 14} pp 177-478 (1974)
\bibitem{FandD}Freeman A J and Descleaux J P {\it J. Magn. Mag. Mater.} {\bf 12}
pp 11-21 (1979)
\bibitem{DandF}Descleaux J P and Freeman A J {\it J. Magn. Mag. Mater.} {\bf 8} pp 119-129 (1978)
\bibitem{kob:11}Kobayashi K Nagao T and Ito M {\it Acta Cryst.}. {\bf A68}, pp 473-480  (2011)
\bibitem{kob:12}Kobayashi K Nagao T and Ito M {\it Acta Cryst.}  {\bf A68}, pp  589-594 (2012).

\end{thebibliography}
\end{document}
