\documentclass[onecolumn,12pt,a4paper,]{report}
\usepackage{CCSLman,makeidx,color,ifthen,bm}
\makeatletter
\makeatletter
\@ifpackageloaded{tex4ht}
  {\let\iftexforht\@firstoftwo}
  {\let\iftexforht\@secondoftwo}
\makeatother
%
\iftexforht{\usepackage{here}}{\usepackage[style=plaintop]{floatrow}}
\usepackage{hyperref}
\iftexforht{\newcommand{\tablesize}{\normalsize}}{\newcommand{\tablesize}{\footnotesize}}
\usepackage{hyperref}

%
\newenvironment{clist}{\begin{list}{}{\setlength{\labelwidth}{2.5cm}
  \setlength{\itemsep}{-4ex}
  \setlength{\leftmargin}{\labelwidth}
 \addtolength{\leftmargin}{2cm}}}{\end{list}}

\pagestyle{empty}\usepackage{CCSLman}

%\input{versiondate}
%\newcommand{\versiondate}{\version}
%
%
%\html{\ccslstyles{../CCSLstyles.css}}
\title{Analytical approximations to Magnetic Form Factors}
\author{P.J. Brown}
\makeindex
\setlength{\topmargin}{0mm}
\def\kpv{{\bf{q}}}
\def\kp{q}
\def\kpu{\hat{\bf{q}}}
\def\rv{{\bf r}}
\def\ru{\hat{\bf r}}
\def\hf{\frac{1}{2}}
\def\Mv{{\bf M}}
\def\mv{{\bf m}}
\def\stl{$\sin\theta/\lambda$}
\def\inA{\AA{$^{-1}$}}
\def\n{\relax}
\def\p{\par}
\begin{document}
\maketitle
\chapter{Magnetic Form Factors}
\section{Definition of form factors}
The magnetic form factor  $f(\kpv)$ is obtained from the fourier transform of
the  magnetisation distribution of a single magnetic atom. Assuming that it
has a unique magnetisation direction it can be written
\begin{equation}
\Mv\int 
m(\rv)e^{i\kpv\cdot\rv}\,dr=\Mv f(\kpv)
\end{equation}
where \Mv\ gives the magnitude and direction of the moment and $m(\rv)$ is
a normalised scalar function  which describes how the intensity of magnetisation 
varies over the volume of the atom.
When the magnetisation arises from electrons in a single open shell the magnetic
form factor can be calculated from the radial distribution of the electrons in
that shell. The integrals from which the form factors are obtained have the form
\begin{equation}
\left\langle j_L(\kp)\right\rangle= \int 
U^2(r)j_l(\kp r)4\pi r^2\,dr
\end{equation}
The $j_l$ are the spherical Bessel functions defined by
\begin{equation}
j_l(x)={\sqrt{\pi\over2x}}J_{l+\frac12}(x)
\end{equation}
If the open shell has orbital quantum number $l$ the form factor 
for spin moment is 
\begin{equation}
f_s(\kpv)=\frac1{\mathrm{M}_S}\sum_{L=0}^{2l}i^L\left\langle 
j_L(\kp)\right\rangle
\sum_{M=-L}^L
S_{LM}Y^L_M\left({\kpu}\right)
\end{equation}
and that for orbital moment 
\begin{equation}
f_o(\kpv)=\frac1{\mathrm{M}_L}\sum_{L=0,2,\ldots}^{2l}
\left(\left\langle j_L(\kp)\right\rangle +
\left\langle j_{L+2}(\kp)\right\rangle\right)
\sum_{M=-L}^LB_{LM}Y^L_M(\kpu)
\end{equation}
\n
The coefficients $S_{LM}$, $B_{LM}$ have to be computed from the  orbital
wave-function \cite{Marshall}. The total spin moment ${\mathrm{M}_S}$ is  
given by $S_{00}$ 
and the orbital moment  ${\mathrm{M}_L}$ by $B_{00}$. For small $\kp$ the dipole
approximation 
\begin{equation}
f(\kpv)=({\mathrm{L+2S})}\left\langle j_0(\kp)\right\rangle
{\mathrm{L}}\left\langle j_2(\kp)\right\rangle
\end{equation}
can be used.
\\[2ex]
\section{Analytical approximation}
The tables which follow give the coefficients of an analytic approximation 
to the $\langle j_0\rangle$ integrals for the d electrons in ions of the 
\htmlref{3d}{3dj0} 
and \htmlref{4d}{4dj0}  and \htmlref{5d}{5dj0} series, the \htmlref{4f}{re4fj0} and \htmlref{5d}{re5dj0}
electrons of some rare earth ions and  the \htmlref{5f electrons }{acj0} of some actinide ions. 
For the 3d, 4d, 4f and 5f form factors the approximation has the form
\begin{equation}
\langle j_0(s)\rangle = A \exp(-as^2) + B \exp(-bs^2) + C \exp(-cs^2) + D
\end{equation}
For the data in the tables $s = \sin\theta/\lambda$ in units of \inA.
\\[2ex]
For the transition metal series the fits were made with 
form factor integrals  calculated from Hartree-Fock wave-functions \cite{clementi}. 
For
the rare-earth and actinide series the fits were with
Dirac-Fock form factors \cite{FandD,DandF}. 
There are some more recent calculations of form-factors for the 5d electrons
of the third group transition metals and of the rare-earth elements
\cite{kob:11,kob:12}. These authors used an expansion with seven rather than five terms
to fit the calculated form factors:
\begin{eqnarray}  
\langle j_0(s)\rangle& = &A \exp(-as^2) + B \exp(-bs^2) + C \exp(-cs^2)\nonumber\\
& &+ D \exp(-ds^2) + E\\
\end{eqnarray}
In the tables the number 
following the atom symbol indicates the ionisation state of the atom. 
Thus the coefficients following Fe0 are for a neutral iron atom and 
those following Fe2 are for Fe$^{2+}$.
\p
The integrals $\langle j_L\rangle$ with $L\ne0$ are zero when $s=0$ and have been
fitted with the forms 
\begin{equation}
\langle j_{L\ne 0}(s)\rangle = \left(A\exp(-as^2) + B\exp(-bs^2) + 
C\exp(-cs^2) + D\right)s^2
\end{equation}
or 
\begin{eqnarray}  
\langle j_{L\ne 0}(s)\rangle&= &\left(A\exp(-as^2) + B\exp(-bs^2) + C\exp(-cs^2)\right.\nonumber\\
& &  \left. + D\exp(-ds^2) + E \right)s^2
\end{eqnarray}

The second set of tables give the coefficients obtained for  
$\langle j2\rangle$ for \htmlref{3d}{3dj2}, \htmlref{4d}{4dj2} and \htmlref{5d}{5dj2} transition metals, 
\htmlref{4f}{re4fj2} and \htmlref{5d}{re5dj2} electrons of rare earths  and \htmlref{actinides}{acj2};
 $\langle j4\rangle$ for  \htmlref{3d}{3dj4}, \htmlref{4d}{4dj4} and \htmlref{5d}{5dj4} transition metals, 
\htmlref{rare earths}{rej4} and \htmlref{actinides}{acj4};
and $\langle j6\rangle$ for \htmlref{rare earths}{rej6} and \htmlref{actinides}{acj6}. 
\chapter{Tables of Form Factors}
\newpage
\input{fftable}

\begin{thebibliography}{99}
\bibitem{Marshall}Marshall W and Lovesey S W{\it Theory of thermal neutron
scattering Chapter 6} Oxford University Press (1971)
\bibitem{clementi} Clementi E and Roetti C {\it Atomic Data and Nuclear Data
Tables}{\bf 14} pp 177-478 (1974)
\bibitem{FandD}Freeman A J and Descleaux J P {\it J. Magn. Mag. Mater.} {\bf 12}
pp 11-21 (1979)
\bibitem{DandF}Descleaux J P and Freeman A J {\it J. Magn. Mag. Mater.} {\bf 8} pp 119-129 (1978)
\bibitem{kob:11}Kobayashi K Nagao T and Ito M {\it Acta Cryst.}. {\bf A68}, pp 473-480  (2011)
\bibitem{kob:12}Kobayashi K Nagao T and Ito M {\it Acta Cryst.}  {\bf A68}, pp  589-594 (2012).

\end{thebibliography}
\end{document}
