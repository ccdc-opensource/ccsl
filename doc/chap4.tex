%\chapter{INPUT AND OUTPUT OF OTHER DATA}
%\begin{htmlonly}
%\documentclass[a4paper]{CCSLman}
%\usepackage{html,makeidx,color}
%%\chapter{INPUT AND OUTPUT OF OTHER DATA}
%\begin{htmlonly}
%\documentclass[a4paper]{CCSLman}
%\usepackage{html,makeidx,color}
%%\chapter{INPUT AND OUTPUT OF OTHER DATA}
%\begin{htmlonly}
%\documentclass[a4paper]{CCSLman}
%\usepackage{html,makeidx,color}
%%\chapter{INPUT AND OUTPUT OF OTHER DATA}
%\begin{htmlonly}
%\documentclass[a4paper]{CCSLman}
%\usepackage{html,makeidx,color}
%\input{chap4.ptr}
%\end{htmlonly}
%\internal{c2}
%\internal{c3}
%\internal{c1}
%\internal{c5}
%\internal{c6}
%\internal{c7}
\startdocument
%\begin{document}
%\htmladdtonavigation{\htmladdnormallink
%  {\htmladdimg{../icons/appenx.gif}}
%  {../appenx/appendix.html}}
\label{chap:4}
\markboth{Input and Output}{}
\section{Introduction}\markright{Introduction}
Some uses of CCSL require as input only the Crystal Data.  More often, a
file of what the crystallographer usually thinks of as his \ital{data}
is needed as well.  This will typically be a list of items, all similar in
format, e.g. indexed reflection or intensity data.\p
At present such files are usually read by FORTRAN fixed format READ
statements. CCSL free format read routines are gradually being
introduced instead, in cases where the fixed format causes problems. An
example is in GRLSQ, the Least Squares Refinement of a structure for
which it is required to group together several reflections on input, to
compare against one calculated value.  Such data are read by SUBROUTINE
\stlink{i}{INOBGR}, which is required to read an unknown number of integers followed
by a real number (with decimal point).  This is not simple using
FORTRAN READ.\p
Another application of free format is the routine \stlink{r}{RDDATA}, which reads 
$h,k,l$ and a set of other numbers, with the inclusion
of a \ital{comment} facility, so that such a list can be headed by some
identifying title.
\section{Diffraction data files}\markright{Diffraction data files}
Many of the data files treated using CCSL routines will be the results of
diffraction experiments which depend both on the diffraction
geometry, UB matrix, wavelength etc. and on sample environment parameters
such as the temperature and magnetic field. If these parameters can be extracted
from the raw data when it is initially processed, they may be written in the form of a header on the resulting data files. Each line in the header starts with a hash (\#),
followed by a word to identify the parameters on the rest of the line. Subsequent
processing programs may either recognise the word and use the parameters, or write them
unchanged on the output file. The information in the headers allows data sets
collected with different wavelengths, applied fields etc. to be refined
together using the multi-souce options for structure factor least-squares.
%
\section{Input of Least Squares Data}\markright{Input of Least Squares Data}
Wherever possible, the point at which these other data 
(which we will call \ital{reflection data}) are read is accessible to
the user, so that if he has data
in some strange format he can still read them. An example of this is
in the various Least Squares Refinement main programs like \mlink{sflsq}{SFLSQ}, 
\mlink{maglsq}{MAGLSQ},
which actually contain the READ and FORMAT statements for the reflection
data. Several alternative formats are already provided;  there is also a
data input type 0, indicating that the data will be provided, a line at a
time, by successive calls to SUBROUTINE QLSQIN, which the user must provide.\p
In general, any SUBROUTINE QxxxIN is a user-provided input routine for
reflection data.  (The user need not concern himself with special formats
or a routine QLSQIN if he has data in some standard format known to the
system; there is a dummy routine \stlink{q}{QLSQIN}\ in the Library, which will be
superseded by any routine of the same name if the user provides one.)
%
\section{Input of Fourier Data}\markright{Input of Fourier Data}
Input of reflection data for Fourier calculations is less accessible,
being buried in the various Fourier routines like \stlink{f}{FOUR1Z}, 
\stlink{f}{FOUR1D}\ as a
call to \stlink{f}{FOUINP}.  Again an input type 0 input routine \stlink{q}{QFOUIN}\ is
available.  The user would need to write a new QFOUIN if, say, he had
non-centrosymmetric data and was not assuming Friedel's Law.  He would
have to decide what to do with Friedel pairs before presenting them to
the standard Fourier calculations, which do assume Friedel.
%
\section{File Handling and the Use of Input/Output Units}
\markright{File Handling}
The assignment of FORTRAN units is handled by CCSL.
Interactive terminal input and output units, the line-printer and the
plotter are assigned to the units named ITI, ITO, LPT, and IPLO
respectively, and are given values appropriate to the system by INITIL.
They are held in the COMMON /IOUNIT/.\p
Assignment of files to FORTRAN units is done using FUNCTION
\stlink{n}{NOPFIL} to open a file and SUBROUTINE \stlink{c}{CLOFIL} to close it.\p
Fifteen units are required by CCSL and managed by it; if not forbidden by
the particular operating system they are units 20-34.  The user of
standard programs should find that his input and output are taken care
of by calls of NOPFIL already in the system.  A user writing or
modifying a main program will need to know more details, in order to
deal with his own files.\p
{\bf Note} The length of the leaf part of file-names in CCSL is restricted
to 10 characters: a main part of up to 6 characters optionally followed by a period (.)
and and extension of up to 3 characters.
\subsection{Opening a File to be Read}
To open a file to be read by a program in the simplest possible way
the code:
\\ 
\exac{~~~~~~LUN=NOPFIL(1)}\\ 
should be obeyed. The system will respond with the message:\\ 
\exac{Give name of input file}\\ 
to which the user should respond with the file name.  Under VMS on a VAX,
this can
include the full path name and an extension if required but it may be simply the
letters (up to 6) of the file name, in which case the extension .DAT is
added.  LUN will be set to the value of the Fortran unit allocated by 
the system.
The value of LUN should not normally concern the user;  he reads the file
by code such as:\\
\exac{~~~~~~READ (LUN,42)  FRED}\\ 
but if he needs for some reason to open a specifically numbered file he may
use SUBROUTINE \stlink{o}{OPNFIL}, thus:
\begin{verbatim} 

      LUNIT=14
      CALL OPNFIL(LUNIT,1)
\end{verbatim}
NOPFIL can interpret relative directory specifications in UNIX (eg ../) or VMS
(eg [-.]) and absolute file or directory names given as environment variables
(\$$\cdots$) in UNIX or as logical assignments ($\cdots:$) in VMS.
\subsection{Opening an Output File}
To open for writing a file which is not destined for a line printer, the
code:\\[0.2ex]
\exac{~~~~~~LUN=NOPFIL(2)}\\[0.2ex]
should be obeyed.  The file is then written using, for example,\\[0.2ex]
\exac{~~~~~~WRITE (LUN,43) JOAN}\\ 
To open for writing a file which is destined for a line printer, a VMS
user may write:\\[0.2ex]
\exac{~~~~~~LUN=NOPFIL(2002)}\\[0.2ex]
When the file is opened this will use:
\begin{verbatim}
      CARRIAGECONTROL='FORTRAN'
\end{verbatim}
Fortran carriage control is not recognised by UNIX.
\subsection{Options Available when Opening Files}
Other options can be obtained by obeying:\\[0.2ex]
\exac{~~~~~~LUN=NOPFIL(MODE)}\\[0.2ex]
where MODE indicates the type of file to be opened and how to obtain
the file-name.\p 
MODE may have up to 5 digits.The least significant, MODE1, indicates
the file-type; the second digit, MODE2, shows how to obtain the file
name; the third, MODE3, deals with default extensions; the fourth, MODE4,
selects formatted or unformatted files and the fifth, MODE5, selects
sequential or direct access files.\p
Character data such as a message or
the file name are transmitted to NOPFIL via the COMMON /SCRACH/ which
can contain 200 characters, and within NOPFIL is divided into MESSAG 
and NAMFIL of 100 characters each. The maximum length of both these
strings is configurable when building the system. \p
The significance of MODE is as follows:
\begin{list} {} {\setlength{\labelwidth}{3cm}
  \setlength{\parsep}{-1ex}
  \setlength{\leftmargin}{\labelwidth}
 \addtolength{\leftmargin}{1cm}} 
\item[MODE1 \hfill] = 1 for a read file\\
       = 2 for a write file status new\\
       = 3 for a write file status undefined (UNKNOWN)\\
       = 4 for a write file to be modified (APPEND for sequential files)\\
       = 5 for a scratch file.
 
\item[MODE2 \hfill] = 0 give the standard messages for
 read or write files; machine specific
           information like the path name and file extension 
           may be included in
           the user's response.\\
       = 1 use the message in MESSAG; otherwise as MODE2=0\\
       = 2 use the file-name in MESSAG. Report that the file is opened.\\
       = 3 as 2 but do not give file-opened message.\\
       = 4 as 0 but do not give file-opened message.
 
\item[MODE3 \hfill] = 0 use the system default for path and default 
           extension .DAT\\
       = 1 use defaults for extension, disc and path-name found in characters
           1-4, 5-10, and 11-30 respectively of NAMFIL (up to 40
           characters, in common SCRACH after item MESSAG). If the disc or
           path-name is absent, default as system.\\
       = 2 use the file-name exactly as given in response to the request.
 
\item[MODE4 \hfill] = 0 for a formatted file, not expecting FORTRAN 
           carriage control
           characters (and therefore not a line-printer file)\\
       = 1 for an unformatted file\\
       = 2 for a formatted file with FORTRAN carriage control characters,
           such as would be sent to a line printer.
 
\item[MODE5 \hfill] = 0 for a sequential file\\
       = 1 for a direct access file\\
       = 2 for RECORDTYPE='FIXED' (needed for ``GENIE" files).\end{list}
 
\subsection{Exit from NOPFIL}
After a successful exit from the call:\\[0.2ex]
\exac{~~~~~~LUN=NOPFIL(MODE)}\\[0.2ex]
LUN is set to a small positive number, which is the
unit number allocated to the opened file by NOPFIL.  LUN is zero
if no response is obtained (i.e. the RETURN key is pressed) when a
file name is reqested. LUN is negative if the file could not be opened,
although NOPFIL allows the user to try again after typing errors or
other blunders from which it has a sensible way to recover.\p 
The short names (10 characters) are stored in COMMON FILNAM and
may be obtained by declaring the COMMON and obeying:
\\[0.2ex]
\begin{verbatim}         CHARACTER*10 NAME,FILNOM
         NAME=FILNOM(LUN)
\end{verbatim}
\subsection{Closing Files}
When input or output operations are terminated on a particular unit
it should be returned to the system by obeying:
\verb|         CALL CLOFIL(LUN)|
\subsection{Examples of the Use of NOPFIL} 
The main program for a Fourier calculation is given below:
\begin{verbatim}C MINIMAL FOURIER PROGRAM.
      CALL PREFIN
      LUNI= NOPFIL(1)
      CALL RECIP
      CALL OPSYM(2)
      CALL SETFOU
   . . . map-producing code - see various main program examples
      STOP
      END
\end{verbatim}
\ssk
It contains one explicit call to NOPFIL; however the call to PREFIN
provokes several other calls to NOPFIL, one of which sends the message:\\ 
\exac{Give name of Crystal Data File :}\\ 
to which the user must reply by typing the appropriate file name.
Another call to NOPFIL does not result in a message as it just opens
an internal scratch file.  (If the Libary has been generated for RAL rather
than ILL, an early call of NOPFIL asks for the name of the printer 
output file.)\p
The call to \stlink{n}{NOPFIL}\ in the main program sends
the message:\\ 
\exac{Give name of input file :}\\ 
and the user should type the name of the file containing his reflection data.\p
A less trivial example is given in the fragments from the main program ABSMSF
reproduced below. \mlink{absmsf}{ABSMSF}\ is a program which reads a file containing
reflection intensities in the special format produced by the sorting
programs \mlink{arrnge}{ARRNGE}\ and \mlink{arrinc}{ARRINC}.  
The file name has the extension .ARR.
 It then makes 
an absorption correction if required and calculates the 
average of the corrected intensity over all equivalent reflections. Finally
it writes an output file containing
the mean structure factors and their standard deviations.
\\[0.2ex]
\begin{verbatim}      PROGRAM ABSMSF
            .
            .
      DIMENSION LUN(2)
      CHARACTER*56 HEDING
      LOGICAL INC
            .
            .
      COMMON /IOUNIT/LPT,ITI,ITO,IPLO,LUNI,IOUT
      COMMON /REFS/K(3,2),LL(48,2),R(500,2),SCALE(2),INC,II,FF(3,2)
      COMMON/SCRACH/MESSAG,NAMFIL
      CHARACTER *40 MESSAG
      CHARACTER *60 NAMFIL
      DATA HEADNG/'(5X,''h'',4X,''k'',4X,''l'',10X,''Fobs'',7X,
     1''DFobs''/)'/
            .
            .
      IF (INC.EQ.0) GO TO 2
C ALTER THE FORMAT FOR FLOATING POINT OUTPUT:
      HEADNG(2:2)='6'
      HEADNG(9:9)='7'
      HEADNG(16:16)='7'
      HEADNG(23:24)='12'
            .
            .
      NAMFIL='.ARR'
      LUN(1)=NOPFIL(101)
      NAMFIL='.SF '
      LOUT=NOPFIL(102)
            .
            .
      STOP
      END
\end{verbatim}
\ssk
The first call to NOPFIL asks the user for the name of the input file
and uses the default .ARR if no ``." is found in the reply.
The second call creates a new file for output with the same name as
the input file but with extension .SF (structure factors).
\section{Printed Output Files: Use of subroutine \slink{t}{TESTP}}
\markright{Printed Output Files}
There is a SUBROUTINE TESTP which
has been written to facilitate the production of long lists of similar
items with column headings at the top of each page.  Calls to TESTP have
the form:\\ 
\exac{      CALL TESTP(LUN,LCOUNT,NTEST,HEADNG,NLINES)}\\ 
where LUN is the FORTRAN unit on which the list is being written, LCOUNT
is the number of lines printed since the last page throw, NTEST is used
to provoke a page throw if there are less than NTEST lines left on the
page.\p
HEADNG is a character variable containing the page heading (see
the example above and also that 
in the BLOCK DATA sub-program below). NLINES is the number of
lines in the heading. HEADNG need not be in COMMON if data are only
written to the list from the main program. In ABSMSF it can be seen that
HEADNG is modified by the program if INC is .TRUE. (non-integer indices)
to accomodate the longer FORMAT required to print the indices in this case.
\\[0.2ex]
\begin{verbatim}      BLOCK DATA PAGEHD
      COMMON /HEDING/ HEADNG
      CHARACTER*132 HEADNG
      DATA HEADNG/' (''0Seq No       h    k    l   Omega  2Theta   
     1  Nu       R        dR      Peak cps  Bkgd cps      DVM
     2  Date    Time''/)'/
      END
\end{verbatim}\ssk
\section{Graphical Output}\markright{{Graphical Output}}
%
The form of graphical output is very dependent on the hardware used
to produce it and the local software used to drive that hardware.
CCSL addresses this problem by chanelling all calls to the plotter
hardware through a single SUBROUTINE \stlink{p}{PIGLET}.
\p
There is a separate
section of the CCSL Master File which contains various PIGLETs which have
been written to drive different devices, such as Tektronix 
4010 terminals and BENSON plotters. Special PIGLETs for GKS,
PostScript and PGPLOT interfaces are also included. A user who wishes to compile
a program which uses CCSL graphics should extract the most suitable
SUBROUTINE PIGLET from the Master File, modify it if necessary, and
include it in his link statement together with any graphics libraries
called by this particular PIGLET.
\p
In many cases the output will drive a graphics terminal directly
and local system facilities will be used to obtain hard copies.
Output files destined for centralised plotters will usually be
created and named by the local system software. 
\p
When the graphical output is sent directly to a file
by the CCSL graphics interface, for instance when using the
PostScript or PGPLOT PIGLETs the output file will be named by CCSL, usually
with the name of the program, a sequence number and an extension approriate to the type
of output (e.g. .PS for a PostScript file).
\subsection{Plotting in colour} 
16 different colour names are recognised by CCSL using the first 3 letters of their names
(case insensitive), they are:\\[1ex]
\begin{tabular}{rlllll}
\hspace{10mm}Number&\quad&Name&Red&Green&Blue\\
0&&White&1.00&  1.00  &1.00\\
1&&Black &0.00  &0.00  &0.00\\
2&&Red  &1.00  &0.00  &0.00\\
3&&Green  &0.00  &1.00  &0.00\\
4&&Blue&0.00  &0.00  &1.00\\
5&&Cyan&0.00  &1.00  &1.00\\
6&&Magenta&1.00  &0.00  &1.00\\
7&&Yellow&1.00  &1.00  &0.00\\
8&&Orange&1.00  &0.50  &0.00\\
9&&SpringGreen&0.50  &1.00  &0.00\\
10&&SeaGreen&0.00  &1.00  &0.50\\
11&&Aqua&0.00  &0.50  &1.00\\
12&&Purple&0.50  &0.00  &1.00\\
13&&Pink&1.00  &0.00  &0.50\\
14&&DkGrey&0.33  &0.33  &0.33\\
15&&LtGrey&0.66  &0.66  &0.66 \\ 
\end{tabular} \\
The function \stlink{l}{LCOLPG} will convert a colour name to a colour number recognised by
\stlink{p}{PIGLET}.
\subsection{Plotting Symbols}
Symbols in CCSL are plotted using the subroutine \stlink{k}{KANGA3} which recognises 9 different symbol shapes and two fill types: \em{filled} and \em{open}. The full list of symbols
with their names is given below. The names are recognised by case insensitive matching 
of the first 6 letters; and the fill property by case insensitive matching
of the first 4 letters in \em{filled} or \em{open}. \\[1ex]
\begin{tabular}{rlllll}
\hspace{10mm}Number&\quad&Name&Description&Filled/Open\\
1&&square&Square&Yes\\
2&&trianup&Triangle vertex up&Yes\\
3&&triandown&Triangle vertex down&Yes\\
4&&circle&A hexagon&Yes\\
5&&xcross&Cross with diagonal members&No\\
6&&plussign&Cross with hor. and vert. bars&No\\
7&&eggtimer&Xcross with ends closed&Yes\\
8&&bowtie&Eggtimer rotated 90 degreep&Yes\\
9&&diamond&Square rotated 45 degrees&Yes\\
\end{tabular}
The function \stlink{l}{LSYMPG} will convert a symbol name and fill type into a symbol
number for input to \stlink{k}{KANGA3}.

\finchapter
%\end{document}

%\end{htmlonly}
%\internal{c2}
%\internal{c3}
%\internal{c1}
%\internal{c5}
%\internal{c6}
%\internal{c7}
\startdocument
%\begin{document}
%\htmladdtonavigation{\htmladdnormallink
%  {\htmladdimg{../icons/appenx.gif}}
%  {../appenx/appendix.html}}
\label{chap:4}
\markboth{Input and Output}{}
\section{Introduction}\markright{Introduction}
Some uses of CCSL require as input only the Crystal Data.  More often, a
file of what the crystallographer usually thinks of as his \ital{data}
is needed as well.  This will typically be a list of items, all similar in
format, e.g. indexed reflection or intensity data.\p
At present such files are usually read by FORTRAN fixed format READ
statements. CCSL free format read routines are gradually being
introduced instead, in cases where the fixed format causes problems. An
example is in GRLSQ, the Least Squares Refinement of a structure for
which it is required to group together several reflections on input, to
compare against one calculated value.  Such data are read by SUBROUTINE
\stlink{i}{INOBGR}, which is required to read an unknown number of integers followed
by a real number (with decimal point).  This is not simple using
FORTRAN READ.\p
Another application of free format is the routine \stlink{r}{RDDATA}, which reads 
$h,k,l$ and a set of other numbers, with the inclusion
of a \ital{comment} facility, so that such a list can be headed by some
identifying title.
\section{Diffraction data files}\markright{Diffraction data files}
Many of the data files treated using CCSL routines will be the results of
diffraction experiments which depend both on the diffraction
geometry, UB matrix, wavelength etc. and on sample environment parameters
such as the temperature and magnetic field. If these parameters can be extracted
from the raw data when it is initially processed, they may be written in the form of a header on the resulting data files. Each line in the header starts with a hash (\#),
followed by a word to identify the parameters on the rest of the line. Subsequent
processing programs may either recognise the word and use the parameters, or write them
unchanged on the output file. The information in the headers allows data sets
collected with different wavelengths, applied fields etc. to be refined
together using the multi-souce options for structure factor least-squares.
%
\section{Input of Least Squares Data}\markright{Input of Least Squares Data}
Wherever possible, the point at which these other data 
(which we will call \ital{reflection data}) are read is accessible to
the user, so that if he has data
in some strange format he can still read them. An example of this is
in the various Least Squares Refinement main programs like \mlink{sflsq}{SFLSQ}, 
\mlink{maglsq}{MAGLSQ},
which actually contain the READ and FORMAT statements for the reflection
data. Several alternative formats are already provided;  there is also a
data input type 0, indicating that the data will be provided, a line at a
time, by successive calls to SUBROUTINE QLSQIN, which the user must provide.\p
In general, any SUBROUTINE QxxxIN is a user-provided input routine for
reflection data.  (The user need not concern himself with special formats
or a routine QLSQIN if he has data in some standard format known to the
system; there is a dummy routine \stlink{q}{QLSQIN}\ in the Library, which will be
superseded by any routine of the same name if the user provides one.)
%
\section{Input of Fourier Data}\markright{Input of Fourier Data}
Input of reflection data for Fourier calculations is less accessible,
being buried in the various Fourier routines like \stlink{f}{FOUR1Z}, 
\stlink{f}{FOUR1D}\ as a
call to \stlink{f}{FOUINP}.  Again an input type 0 input routine \stlink{q}{QFOUIN}\ is
available.  The user would need to write a new QFOUIN if, say, he had
non-centrosymmetric data and was not assuming Friedel's Law.  He would
have to decide what to do with Friedel pairs before presenting them to
the standard Fourier calculations, which do assume Friedel.
%
\section{File Handling and the Use of Input/Output Units}
\markright{File Handling}
The assignment of FORTRAN units is handled by CCSL.
Interactive terminal input and output units, the line-printer and the
plotter are assigned to the units named ITI, ITO, LPT, and IPLO
respectively, and are given values appropriate to the system by INITIL.
They are held in the COMMON /IOUNIT/.\p
Assignment of files to FORTRAN units is done using FUNCTION
\stlink{n}{NOPFIL} to open a file and SUBROUTINE \stlink{c}{CLOFIL} to close it.\p
Fifteen units are required by CCSL and managed by it; if not forbidden by
the particular operating system they are units 20-34.  The user of
standard programs should find that his input and output are taken care
of by calls of NOPFIL already in the system.  A user writing or
modifying a main program will need to know more details, in order to
deal with his own files.\p
{\bf Note} The length of the leaf part of file-names in CCSL is restricted
to 10 characters: a main part of up to 6 characters optionally followed by a period (.)
and and extension of up to 3 characters.
\subsection{Opening a File to be Read}
To open a file to be read by a program in the simplest possible way
the code:
\\ 
\exac{~~~~~~LUN=NOPFIL(1)}\\ 
should be obeyed. The system will respond with the message:\\ 
\exac{Give name of input file}\\ 
to which the user should respond with the file name.  Under VMS on a VAX,
this can
include the full path name and an extension if required but it may be simply the
letters (up to 6) of the file name, in which case the extension .DAT is
added.  LUN will be set to the value of the Fortran unit allocated by 
the system.
The value of LUN should not normally concern the user;  he reads the file
by code such as:\\
\exac{~~~~~~READ (LUN,42)  FRED}\\ 
but if he needs for some reason to open a specifically numbered file he may
use SUBROUTINE \stlink{o}{OPNFIL}, thus:
\begin{verbatim} 

      LUNIT=14
      CALL OPNFIL(LUNIT,1)
\end{verbatim}
NOPFIL can interpret relative directory specifications in UNIX (eg ../) or VMS
(eg [-.]) and absolute file or directory names given as environment variables
(\$$\cdots$) in UNIX or as logical assignments ($\cdots:$) in VMS.
\subsection{Opening an Output File}
To open for writing a file which is not destined for a line printer, the
code:\\[0.2ex]
\exac{~~~~~~LUN=NOPFIL(2)}\\[0.2ex]
should be obeyed.  The file is then written using, for example,\\[0.2ex]
\exac{~~~~~~WRITE (LUN,43) JOAN}\\ 
To open for writing a file which is destined for a line printer, a VMS
user may write:\\[0.2ex]
\exac{~~~~~~LUN=NOPFIL(2002)}\\[0.2ex]
When the file is opened this will use:
\begin{verbatim}
      CARRIAGECONTROL='FORTRAN'
\end{verbatim}
Fortran carriage control is not recognised by UNIX.
\subsection{Options Available when Opening Files}
Other options can be obtained by obeying:\\[0.2ex]
\exac{~~~~~~LUN=NOPFIL(MODE)}\\[0.2ex]
where MODE indicates the type of file to be opened and how to obtain
the file-name.\p 
MODE may have up to 5 digits.The least significant, MODE1, indicates
the file-type; the second digit, MODE2, shows how to obtain the file
name; the third, MODE3, deals with default extensions; the fourth, MODE4,
selects formatted or unformatted files and the fifth, MODE5, selects
sequential or direct access files.\p
Character data such as a message or
the file name are transmitted to NOPFIL via the COMMON /SCRACH/ which
can contain 200 characters, and within NOPFIL is divided into MESSAG 
and NAMFIL of 100 characters each. The maximum length of both these
strings is configurable when building the system. \p
The significance of MODE is as follows:
\begin{list} {} {\setlength{\labelwidth}{3cm}
  \setlength{\parsep}{-1ex}
  \setlength{\leftmargin}{\labelwidth}
 \addtolength{\leftmargin}{1cm}} 
\item[MODE1 \hfill] = 1 for a read file\\
       = 2 for a write file status new\\
       = 3 for a write file status undefined (UNKNOWN)\\
       = 4 for a write file to be modified (APPEND for sequential files)\\
       = 5 for a scratch file.
 
\item[MODE2 \hfill] = 0 give the standard messages for
 read or write files; machine specific
           information like the path name and file extension 
           may be included in
           the user's response.\\
       = 1 use the message in MESSAG; otherwise as MODE2=0\\
       = 2 use the file-name in MESSAG. Report that the file is opened.\\
       = 3 as 2 but do not give file-opened message.\\
       = 4 as 0 but do not give file-opened message.
 
\item[MODE3 \hfill] = 0 use the system default for path and default 
           extension .DAT\\
       = 1 use defaults for extension, disc and path-name found in characters
           1-4, 5-10, and 11-30 respectively of NAMFIL (up to 40
           characters, in common SCRACH after item MESSAG). If the disc or
           path-name is absent, default as system.\\
       = 2 use the file-name exactly as given in response to the request.
 
\item[MODE4 \hfill] = 0 for a formatted file, not expecting FORTRAN 
           carriage control
           characters (and therefore not a line-printer file)\\
       = 1 for an unformatted file\\
       = 2 for a formatted file with FORTRAN carriage control characters,
           such as would be sent to a line printer.
 
\item[MODE5 \hfill] = 0 for a sequential file\\
       = 1 for a direct access file\\
       = 2 for RECORDTYPE='FIXED' (needed for ``GENIE" files).\end{list}
 
\subsection{Exit from NOPFIL}
After a successful exit from the call:\\[0.2ex]
\exac{~~~~~~LUN=NOPFIL(MODE)}\\[0.2ex]
LUN is set to a small positive number, which is the
unit number allocated to the opened file by NOPFIL.  LUN is zero
if no response is obtained (i.e. the RETURN key is pressed) when a
file name is reqested. LUN is negative if the file could not be opened,
although NOPFIL allows the user to try again after typing errors or
other blunders from which it has a sensible way to recover.\p 
The short names (10 characters) are stored in COMMON FILNAM and
may be obtained by declaring the COMMON and obeying:
\\[0.2ex]
\begin{verbatim}         CHARACTER*10 NAME,FILNOM
         NAME=FILNOM(LUN)
\end{verbatim}
\subsection{Closing Files}
When input or output operations are terminated on a particular unit
it should be returned to the system by obeying:
\verb|         CALL CLOFIL(LUN)|
\subsection{Examples of the Use of NOPFIL} 
The main program for a Fourier calculation is given below:
\begin{verbatim}C MINIMAL FOURIER PROGRAM.
      CALL PREFIN
      LUNI= NOPFIL(1)
      CALL RECIP
      CALL OPSYM(2)
      CALL SETFOU
   . . . map-producing code - see various main program examples
      STOP
      END
\end{verbatim}
\ssk
It contains one explicit call to NOPFIL; however the call to PREFIN
provokes several other calls to NOPFIL, one of which sends the message:\\ 
\exac{Give name of Crystal Data File :}\\ 
to which the user must reply by typing the appropriate file name.
Another call to NOPFIL does not result in a message as it just opens
an internal scratch file.  (If the Libary has been generated for RAL rather
than ILL, an early call of NOPFIL asks for the name of the printer 
output file.)\p
The call to \stlink{n}{NOPFIL}\ in the main program sends
the message:\\ 
\exac{Give name of input file :}\\ 
and the user should type the name of the file containing his reflection data.\p
A less trivial example is given in the fragments from the main program ABSMSF
reproduced below. \mlink{absmsf}{ABSMSF}\ is a program which reads a file containing
reflection intensities in the special format produced by the sorting
programs \mlink{arrnge}{ARRNGE}\ and \mlink{arrinc}{ARRINC}.  
The file name has the extension .ARR.
 It then makes 
an absorption correction if required and calculates the 
average of the corrected intensity over all equivalent reflections. Finally
it writes an output file containing
the mean structure factors and their standard deviations.
\\[0.2ex]
\begin{verbatim}      PROGRAM ABSMSF
            .
            .
      DIMENSION LUN(2)
      CHARACTER*56 HEDING
      LOGICAL INC
            .
            .
      COMMON /IOUNIT/LPT,ITI,ITO,IPLO,LUNI,IOUT
      COMMON /REFS/K(3,2),LL(48,2),R(500,2),SCALE(2),INC,II,FF(3,2)
      COMMON/SCRACH/MESSAG,NAMFIL
      CHARACTER *40 MESSAG
      CHARACTER *60 NAMFIL
      DATA HEADNG/'(5X,''h'',4X,''k'',4X,''l'',10X,''Fobs'',7X,
     1''DFobs''/)'/
            .
            .
      IF (INC.EQ.0) GO TO 2
C ALTER THE FORMAT FOR FLOATING POINT OUTPUT:
      HEADNG(2:2)='6'
      HEADNG(9:9)='7'
      HEADNG(16:16)='7'
      HEADNG(23:24)='12'
            .
            .
      NAMFIL='.ARR'
      LUN(1)=NOPFIL(101)
      NAMFIL='.SF '
      LOUT=NOPFIL(102)
            .
            .
      STOP
      END
\end{verbatim}
\ssk
The first call to NOPFIL asks the user for the name of the input file
and uses the default .ARR if no ``." is found in the reply.
The second call creates a new file for output with the same name as
the input file but with extension .SF (structure factors).
\section{Printed Output Files: Use of subroutine \slink{t}{TESTP}}
\markright{Printed Output Files}
There is a SUBROUTINE TESTP which
has been written to facilitate the production of long lists of similar
items with column headings at the top of each page.  Calls to TESTP have
the form:\\ 
\exac{      CALL TESTP(LUN,LCOUNT,NTEST,HEADNG,NLINES)}\\ 
where LUN is the FORTRAN unit on which the list is being written, LCOUNT
is the number of lines printed since the last page throw, NTEST is used
to provoke a page throw if there are less than NTEST lines left on the
page.\p
HEADNG is a character variable containing the page heading (see
the example above and also that 
in the BLOCK DATA sub-program below). NLINES is the number of
lines in the heading. HEADNG need not be in COMMON if data are only
written to the list from the main program. In ABSMSF it can be seen that
HEADNG is modified by the program if INC is .TRUE. (non-integer indices)
to accomodate the longer FORMAT required to print the indices in this case.
\\[0.2ex]
\begin{verbatim}      BLOCK DATA PAGEHD
      COMMON /HEDING/ HEADNG
      CHARACTER*132 HEADNG
      DATA HEADNG/' (''0Seq No       h    k    l   Omega  2Theta   
     1  Nu       R        dR      Peak cps  Bkgd cps      DVM
     2  Date    Time''/)'/
      END
\end{verbatim}\ssk
\section{Graphical Output}\markright{{Graphical Output}}
%
The form of graphical output is very dependent on the hardware used
to produce it and the local software used to drive that hardware.
CCSL addresses this problem by chanelling all calls to the plotter
hardware through a single SUBROUTINE \stlink{p}{PIGLET}.
\p
There is a separate
section of the CCSL Master File which contains various PIGLETs which have
been written to drive different devices, such as Tektronix 
4010 terminals and BENSON plotters. Special PIGLETs for GKS,
PostScript and PGPLOT interfaces are also included. A user who wishes to compile
a program which uses CCSL graphics should extract the most suitable
SUBROUTINE PIGLET from the Master File, modify it if necessary, and
include it in his link statement together with any graphics libraries
called by this particular PIGLET.
\p
In many cases the output will drive a graphics terminal directly
and local system facilities will be used to obtain hard copies.
Output files destined for centralised plotters will usually be
created and named by the local system software. 
\p
When the graphical output is sent directly to a file
by the CCSL graphics interface, for instance when using the
PostScript or PGPLOT PIGLETs the output file will be named by CCSL, usually
with the name of the program, a sequence number and an extension approriate to the type
of output (e.g. .PS for a PostScript file).
\subsection{Plotting in colour} 
16 different colour names are recognised by CCSL using the first 3 letters of their names
(case insensitive), they are:\\[1ex]
\begin{tabular}{rlllll}
\hspace{10mm}Number&\quad&Name&Red&Green&Blue\\
0&&White&1.00&  1.00  &1.00\\
1&&Black &0.00  &0.00  &0.00\\
2&&Red  &1.00  &0.00  &0.00\\
3&&Green  &0.00  &1.00  &0.00\\
4&&Blue&0.00  &0.00  &1.00\\
5&&Cyan&0.00  &1.00  &1.00\\
6&&Magenta&1.00  &0.00  &1.00\\
7&&Yellow&1.00  &1.00  &0.00\\
8&&Orange&1.00  &0.50  &0.00\\
9&&SpringGreen&0.50  &1.00  &0.00\\
10&&SeaGreen&0.00  &1.00  &0.50\\
11&&Aqua&0.00  &0.50  &1.00\\
12&&Purple&0.50  &0.00  &1.00\\
13&&Pink&1.00  &0.00  &0.50\\
14&&DkGrey&0.33  &0.33  &0.33\\
15&&LtGrey&0.66  &0.66  &0.66 \\ 
\end{tabular} \\
The function \stlink{l}{LCOLPG} will convert a colour name to a colour number recognised by
\stlink{p}{PIGLET}.
\subsection{Plotting Symbols}
Symbols in CCSL are plotted using the subroutine \stlink{k}{KANGA3} which recognises 9 different symbol shapes and two fill types: \em{filled} and \em{open}. The full list of symbols
with their names is given below. The names are recognised by case insensitive matching 
of the first 6 letters; and the fill property by case insensitive matching
of the first 4 letters in \em{filled} or \em{open}. \\[1ex]
\begin{tabular}{rlllll}
\hspace{10mm}Number&\quad&Name&Description&Filled/Open\\
1&&square&Square&Yes\\
2&&trianup&Triangle vertex up&Yes\\
3&&triandown&Triangle vertex down&Yes\\
4&&circle&A hexagon&Yes\\
5&&xcross&Cross with diagonal members&No\\
6&&plussign&Cross with hor. and vert. bars&No\\
7&&eggtimer&Xcross with ends closed&Yes\\
8&&bowtie&Eggtimer rotated 90 degreep&Yes\\
9&&diamond&Square rotated 45 degrees&Yes\\
\end{tabular}
The function \stlink{l}{LSYMPG} will convert a symbol name and fill type into a symbol
number for input to \stlink{k}{KANGA3}.

\finchapter
%\end{document}

%\end{htmlonly}
%\internal{c2}
%\internal{c3}
%\internal{c1}
%\internal{c5}
%\internal{c6}
%\internal{c7}
\startdocument
%\begin{document}
%\htmladdtonavigation{\htmladdnormallink
%  {\htmladdimg{../icons/appenx.gif}}
%  {../appenx/appendix.html}}
\label{chap:4}
\markboth{Input and Output}{}
\section{Introduction}\markright{Introduction}
Some uses of CCSL require as input only the Crystal Data.  More often, a
file of what the crystallographer usually thinks of as his \ital{data}
is needed as well.  This will typically be a list of items, all similar in
format, e.g. indexed reflection or intensity data.\p
At present such files are usually read by FORTRAN fixed format READ
statements. CCSL free format read routines are gradually being
introduced instead, in cases where the fixed format causes problems. An
example is in GRLSQ, the Least Squares Refinement of a structure for
which it is required to group together several reflections on input, to
compare against one calculated value.  Such data are read by SUBROUTINE
\stlink{i}{INOBGR}, which is required to read an unknown number of integers followed
by a real number (with decimal point).  This is not simple using
FORTRAN READ.\p
Another application of free format is the routine \stlink{r}{RDDATA}, which reads 
$h,k,l$ and a set of other numbers, with the inclusion
of a \ital{comment} facility, so that such a list can be headed by some
identifying title.
\section{Diffraction data files}\markright{Diffraction data files}
Many of the data files treated using CCSL routines will be the results of
diffraction experiments which depend both on the diffraction
geometry, UB matrix, wavelength etc. and on sample environment parameters
such as the temperature and magnetic field. If these parameters can be extracted
from the raw data when it is initially processed, they may be written in the form of a header on the resulting data files. Each line in the header starts with a hash (\#),
followed by a word to identify the parameters on the rest of the line. Subsequent
processing programs may either recognise the word and use the parameters, or write them
unchanged on the output file. The information in the headers allows data sets
collected with different wavelengths, applied fields etc. to be refined
together using the multi-souce options for structure factor least-squares.
%
\section{Input of Least Squares Data}\markright{Input of Least Squares Data}
Wherever possible, the point at which these other data 
(which we will call \ital{reflection data}) are read is accessible to
the user, so that if he has data
in some strange format he can still read them. An example of this is
in the various Least Squares Refinement main programs like \mlink{sflsq}{SFLSQ}, 
\mlink{maglsq}{MAGLSQ},
which actually contain the READ and FORMAT statements for the reflection
data. Several alternative formats are already provided;  there is also a
data input type 0, indicating that the data will be provided, a line at a
time, by successive calls to SUBROUTINE QLSQIN, which the user must provide.\p
In general, any SUBROUTINE QxxxIN is a user-provided input routine for
reflection data.  (The user need not concern himself with special formats
or a routine QLSQIN if he has data in some standard format known to the
system; there is a dummy routine \stlink{q}{QLSQIN}\ in the Library, which will be
superseded by any routine of the same name if the user provides one.)
%
\section{Input of Fourier Data}\markright{Input of Fourier Data}
Input of reflection data for Fourier calculations is less accessible,
being buried in the various Fourier routines like \stlink{f}{FOUR1Z}, 
\stlink{f}{FOUR1D}\ as a
call to \stlink{f}{FOUINP}.  Again an input type 0 input routine \stlink{q}{QFOUIN}\ is
available.  The user would need to write a new QFOUIN if, say, he had
non-centrosymmetric data and was not assuming Friedel's Law.  He would
have to decide what to do with Friedel pairs before presenting them to
the standard Fourier calculations, which do assume Friedel.
%
\section{File Handling and the Use of Input/Output Units}
\markright{File Handling}
The assignment of FORTRAN units is handled by CCSL.
Interactive terminal input and output units, the line-printer and the
plotter are assigned to the units named ITI, ITO, LPT, and IPLO
respectively, and are given values appropriate to the system by INITIL.
They are held in the COMMON /IOUNIT/.\p
Assignment of files to FORTRAN units is done using FUNCTION
\stlink{n}{NOPFIL} to open a file and SUBROUTINE \stlink{c}{CLOFIL} to close it.\p
Fifteen units are required by CCSL and managed by it; if not forbidden by
the particular operating system they are units 20-34.  The user of
standard programs should find that his input and output are taken care
of by calls of NOPFIL already in the system.  A user writing or
modifying a main program will need to know more details, in order to
deal with his own files.\p
{\bf Note} The length of the leaf part of file-names in CCSL is restricted
to 10 characters: a main part of up to 6 characters optionally followed by a period (.)
and and extension of up to 3 characters.
\subsection{Opening a File to be Read}
To open a file to be read by a program in the simplest possible way
the code:
\\ 
\exac{~~~~~~LUN=NOPFIL(1)}\\ 
should be obeyed. The system will respond with the message:\\ 
\exac{Give name of input file}\\ 
to which the user should respond with the file name.  Under VMS on a VAX,
this can
include the full path name and an extension if required but it may be simply the
letters (up to 6) of the file name, in which case the extension .DAT is
added.  LUN will be set to the value of the Fortran unit allocated by 
the system.
The value of LUN should not normally concern the user;  he reads the file
by code such as:\\
\exac{~~~~~~READ (LUN,42)  FRED}\\ 
but if he needs for some reason to open a specifically numbered file he may
use SUBROUTINE \stlink{o}{OPNFIL}, thus:
\begin{verbatim} 

      LUNIT=14
      CALL OPNFIL(LUNIT,1)
\end{verbatim}
NOPFIL can interpret relative directory specifications in UNIX (eg ../) or VMS
(eg [-.]) and absolute file or directory names given as environment variables
(\$$\cdots$) in UNIX or as logical assignments ($\cdots:$) in VMS.
\subsection{Opening an Output File}
To open for writing a file which is not destined for a line printer, the
code:\\[0.2ex]
\exac{~~~~~~LUN=NOPFIL(2)}\\[0.2ex]
should be obeyed.  The file is then written using, for example,\\[0.2ex]
\exac{~~~~~~WRITE (LUN,43) JOAN}\\ 
To open for writing a file which is destined for a line printer, a VMS
user may write:\\[0.2ex]
\exac{~~~~~~LUN=NOPFIL(2002)}\\[0.2ex]
When the file is opened this will use:
\begin{verbatim}
      CARRIAGECONTROL='FORTRAN'
\end{verbatim}
Fortran carriage control is not recognised by UNIX.
\subsection{Options Available when Opening Files}
Other options can be obtained by obeying:\\[0.2ex]
\exac{~~~~~~LUN=NOPFIL(MODE)}\\[0.2ex]
where MODE indicates the type of file to be opened and how to obtain
the file-name.\p 
MODE may have up to 5 digits.The least significant, MODE1, indicates
the file-type; the second digit, MODE2, shows how to obtain the file
name; the third, MODE3, deals with default extensions; the fourth, MODE4,
selects formatted or unformatted files and the fifth, MODE5, selects
sequential or direct access files.\p
Character data such as a message or
the file name are transmitted to NOPFIL via the COMMON /SCRACH/ which
can contain 200 characters, and within NOPFIL is divided into MESSAG 
and NAMFIL of 100 characters each. The maximum length of both these
strings is configurable when building the system. \p
The significance of MODE is as follows:
\begin{list} {} {\setlength{\labelwidth}{3cm}
  \setlength{\parsep}{-1ex}
  \setlength{\leftmargin}{\labelwidth}
 \addtolength{\leftmargin}{1cm}} 
\item[MODE1 \hfill] = 1 for a read file\\
       = 2 for a write file status new\\
       = 3 for a write file status undefined (UNKNOWN)\\
       = 4 for a write file to be modified (APPEND for sequential files)\\
       = 5 for a scratch file.
 
\item[MODE2 \hfill] = 0 give the standard messages for
 read or write files; machine specific
           information like the path name and file extension 
           may be included in
           the user's response.\\
       = 1 use the message in MESSAG; otherwise as MODE2=0\\
       = 2 use the file-name in MESSAG. Report that the file is opened.\\
       = 3 as 2 but do not give file-opened message.\\
       = 4 as 0 but do not give file-opened message.
 
\item[MODE3 \hfill] = 0 use the system default for path and default 
           extension .DAT\\
       = 1 use defaults for extension, disc and path-name found in characters
           1-4, 5-10, and 11-30 respectively of NAMFIL (up to 40
           characters, in common SCRACH after item MESSAG). If the disc or
           path-name is absent, default as system.\\
       = 2 use the file-name exactly as given in response to the request.
 
\item[MODE4 \hfill] = 0 for a formatted file, not expecting FORTRAN 
           carriage control
           characters (and therefore not a line-printer file)\\
       = 1 for an unformatted file\\
       = 2 for a formatted file with FORTRAN carriage control characters,
           such as would be sent to a line printer.
 
\item[MODE5 \hfill] = 0 for a sequential file\\
       = 1 for a direct access file\\
       = 2 for RECORDTYPE='FIXED' (needed for ``GENIE" files).\end{list}
 
\subsection{Exit from NOPFIL}
After a successful exit from the call:\\[0.2ex]
\exac{~~~~~~LUN=NOPFIL(MODE)}\\[0.2ex]
LUN is set to a small positive number, which is the
unit number allocated to the opened file by NOPFIL.  LUN is zero
if no response is obtained (i.e. the RETURN key is pressed) when a
file name is reqested. LUN is negative if the file could not be opened,
although NOPFIL allows the user to try again after typing errors or
other blunders from which it has a sensible way to recover.\p 
The short names (10 characters) are stored in COMMON FILNAM and
may be obtained by declaring the COMMON and obeying:
\\[0.2ex]
\begin{verbatim}         CHARACTER*10 NAME,FILNOM
         NAME=FILNOM(LUN)
\end{verbatim}
\subsection{Closing Files}
When input or output operations are terminated on a particular unit
it should be returned to the system by obeying:
\verb|         CALL CLOFIL(LUN)|
\subsection{Examples of the Use of NOPFIL} 
The main program for a Fourier calculation is given below:
\begin{verbatim}C MINIMAL FOURIER PROGRAM.
      CALL PREFIN
      LUNI= NOPFIL(1)
      CALL RECIP
      CALL OPSYM(2)
      CALL SETFOU
   . . . map-producing code - see various main program examples
      STOP
      END
\end{verbatim}
\ssk
It contains one explicit call to NOPFIL; however the call to PREFIN
provokes several other calls to NOPFIL, one of which sends the message:\\ 
\exac{Give name of Crystal Data File :}\\ 
to which the user must reply by typing the appropriate file name.
Another call to NOPFIL does not result in a message as it just opens
an internal scratch file.  (If the Libary has been generated for RAL rather
than ILL, an early call of NOPFIL asks for the name of the printer 
output file.)\p
The call to \stlink{n}{NOPFIL}\ in the main program sends
the message:\\ 
\exac{Give name of input file :}\\ 
and the user should type the name of the file containing his reflection data.\p
A less trivial example is given in the fragments from the main program ABSMSF
reproduced below. \mlink{absmsf}{ABSMSF}\ is a program which reads a file containing
reflection intensities in the special format produced by the sorting
programs \mlink{arrnge}{ARRNGE}\ and \mlink{arrinc}{ARRINC}.  
The file name has the extension .ARR.
 It then makes 
an absorption correction if required and calculates the 
average of the corrected intensity over all equivalent reflections. Finally
it writes an output file containing
the mean structure factors and their standard deviations.
\\[0.2ex]
\begin{verbatim}      PROGRAM ABSMSF
            .
            .
      DIMENSION LUN(2)
      CHARACTER*56 HEDING
      LOGICAL INC
            .
            .
      COMMON /IOUNIT/LPT,ITI,ITO,IPLO,LUNI,IOUT
      COMMON /REFS/K(3,2),LL(48,2),R(500,2),SCALE(2),INC,II,FF(3,2)
      COMMON/SCRACH/MESSAG,NAMFIL
      CHARACTER *40 MESSAG
      CHARACTER *60 NAMFIL
      DATA HEADNG/'(5X,''h'',4X,''k'',4X,''l'',10X,''Fobs'',7X,
     1''DFobs''/)'/
            .
            .
      IF (INC.EQ.0) GO TO 2
C ALTER THE FORMAT FOR FLOATING POINT OUTPUT:
      HEADNG(2:2)='6'
      HEADNG(9:9)='7'
      HEADNG(16:16)='7'
      HEADNG(23:24)='12'
            .
            .
      NAMFIL='.ARR'
      LUN(1)=NOPFIL(101)
      NAMFIL='.SF '
      LOUT=NOPFIL(102)
            .
            .
      STOP
      END
\end{verbatim}
\ssk
The first call to NOPFIL asks the user for the name of the input file
and uses the default .ARR if no ``." is found in the reply.
The second call creates a new file for output with the same name as
the input file but with extension .SF (structure factors).
\section{Printed Output Files: Use of subroutine \slink{t}{TESTP}}
\markright{Printed Output Files}
There is a SUBROUTINE TESTP which
has been written to facilitate the production of long lists of similar
items with column headings at the top of each page.  Calls to TESTP have
the form:\\ 
\exac{      CALL TESTP(LUN,LCOUNT,NTEST,HEADNG,NLINES)}\\ 
where LUN is the FORTRAN unit on which the list is being written, LCOUNT
is the number of lines printed since the last page throw, NTEST is used
to provoke a page throw if there are less than NTEST lines left on the
page.\p
HEADNG is a character variable containing the page heading (see
the example above and also that 
in the BLOCK DATA sub-program below). NLINES is the number of
lines in the heading. HEADNG need not be in COMMON if data are only
written to the list from the main program. In ABSMSF it can be seen that
HEADNG is modified by the program if INC is .TRUE. (non-integer indices)
to accomodate the longer FORMAT required to print the indices in this case.
\\[0.2ex]
\begin{verbatim}      BLOCK DATA PAGEHD
      COMMON /HEDING/ HEADNG
      CHARACTER*132 HEADNG
      DATA HEADNG/' (''0Seq No       h    k    l   Omega  2Theta   
     1  Nu       R        dR      Peak cps  Bkgd cps      DVM
     2  Date    Time''/)'/
      END
\end{verbatim}\ssk
\section{Graphical Output}\markright{{Graphical Output}}
%
The form of graphical output is very dependent on the hardware used
to produce it and the local software used to drive that hardware.
CCSL addresses this problem by chanelling all calls to the plotter
hardware through a single SUBROUTINE \stlink{p}{PIGLET}.
\p
There is a separate
section of the CCSL Master File which contains various PIGLETs which have
been written to drive different devices, such as Tektronix 
4010 terminals and BENSON plotters. Special PIGLETs for GKS,
PostScript and PGPLOT interfaces are also included. A user who wishes to compile
a program which uses CCSL graphics should extract the most suitable
SUBROUTINE PIGLET from the Master File, modify it if necessary, and
include it in his link statement together with any graphics libraries
called by this particular PIGLET.
\p
In many cases the output will drive a graphics terminal directly
and local system facilities will be used to obtain hard copies.
Output files destined for centralised plotters will usually be
created and named by the local system software. 
\p
When the graphical output is sent directly to a file
by the CCSL graphics interface, for instance when using the
PostScript or PGPLOT PIGLETs the output file will be named by CCSL, usually
with the name of the program, a sequence number and an extension approriate to the type
of output (e.g. .PS for a PostScript file).
\subsection{Plotting in colour} 
16 different colour names are recognised by CCSL using the first 3 letters of their names
(case insensitive), they are:\\[1ex]
\begin{tabular}{rlllll}
\hspace{10mm}Number&\quad&Name&Red&Green&Blue\\
0&&White&1.00&  1.00  &1.00\\
1&&Black &0.00  &0.00  &0.00\\
2&&Red  &1.00  &0.00  &0.00\\
3&&Green  &0.00  &1.00  &0.00\\
4&&Blue&0.00  &0.00  &1.00\\
5&&Cyan&0.00  &1.00  &1.00\\
6&&Magenta&1.00  &0.00  &1.00\\
7&&Yellow&1.00  &1.00  &0.00\\
8&&Orange&1.00  &0.50  &0.00\\
9&&SpringGreen&0.50  &1.00  &0.00\\
10&&SeaGreen&0.00  &1.00  &0.50\\
11&&Aqua&0.00  &0.50  &1.00\\
12&&Purple&0.50  &0.00  &1.00\\
13&&Pink&1.00  &0.00  &0.50\\
14&&DkGrey&0.33  &0.33  &0.33\\
15&&LtGrey&0.66  &0.66  &0.66 \\ 
\end{tabular} \\
The function \stlink{l}{LCOLPG} will convert a colour name to a colour number recognised by
\stlink{p}{PIGLET}.
\subsection{Plotting Symbols}
Symbols in CCSL are plotted using the subroutine \stlink{k}{KANGA3} which recognises 9 different symbol shapes and two fill types: \em{filled} and \em{open}. The full list of symbols
with their names is given below. The names are recognised by case insensitive matching 
of the first 6 letters; and the fill property by case insensitive matching
of the first 4 letters in \em{filled} or \em{open}. \\[1ex]
\begin{tabular}{rlllll}
\hspace{10mm}Number&\quad&Name&Description&Filled/Open\\
1&&square&Square&Yes\\
2&&trianup&Triangle vertex up&Yes\\
3&&triandown&Triangle vertex down&Yes\\
4&&circle&A hexagon&Yes\\
5&&xcross&Cross with diagonal members&No\\
6&&plussign&Cross with hor. and vert. bars&No\\
7&&eggtimer&Xcross with ends closed&Yes\\
8&&bowtie&Eggtimer rotated 90 degreep&Yes\\
9&&diamond&Square rotated 45 degrees&Yes\\
\end{tabular}
The function \stlink{l}{LSYMPG} will convert a symbol name and fill type into a symbol
number for input to \stlink{k}{KANGA3}.

\finchapter
%\end{document}

%\end{htmlonly}
%\internal{c2}
%\internal{c3}
%\internal{c1}
%\internal{c5}
%\internal{c6}
%\internal{c7}
\startdocument
%\begin{document}
%\htmladdtonavigation{\htmladdnormallink
%  {\htmladdimg{../icons/appenx.gif}}
%  {../appenx/appendix.html}}
\label{chap:4}
\markboth{Input and Output}{}
\section{Introduction}\markright{Introduction}
Some uses of CCSL require as input only the Crystal Data.  More often, a
file of what the crystallographer usually thinks of as his \ital{data}
is needed as well.  This will typically be a list of items, all similar in
format, e.g. indexed reflection or intensity data.\p
At present such files are usually read by FORTRAN fixed format READ
statements. CCSL free format read routines are gradually being
introduced instead, in cases where the fixed format causes problems. An
example is in GRLSQ, the Least Squares Refinement of a structure for
which it is required to group together several reflections on input, to
compare against one calculated value.  Such data are read by SUBROUTINE
\stlink{i}{INOBGR}, which is required to read an unknown number of integers followed
by a real number (with decimal point).  This is not simple using
FORTRAN READ.\p
Another application of free format is the routine \stlink{r}{RDDATA}, which reads 
$h,k,l$ and a set of other numbers, with the inclusion
of a \ital{comment} facility, so that such a list can be headed by some
identifying title.
\section{Diffraction data files}\markright{Diffraction data files}
Many of the data files treated using CCSL routines will be the results of
diffraction experiments which depend both on the diffraction
geometry, UB matrix, wavelength etc. and on sample environment parameters
such as the temperature and magnetic field. If these parameters can be extracted
from the raw data when it is initially processed, they may be written in the form of a header on the resulting data files. Each line in the header starts with a hash (\#),
followed by a word to identify the parameters on the rest of the line. Subsequent
processing programs may either recognise the word and use the parameters, or write them
unchanged on the output file. The information in the headers allows data sets
collected with different wavelengths, applied fields etc. to be refined
together using the multi-souce options for structure factor least-squares.
%
\section{Input of Least Squares Data}\markright{Input of Least Squares Data}
Wherever possible, the point at which these other data 
(which we will call \ital{reflection data}) are read is accessible to
the user, so that if he has data
in some strange format he can still read them. An example of this is
in the various Least Squares Refinement main programs like \mlink{sflsq}{SFLSQ}, 
\mlink{maglsq}{MAGLSQ},
which actually contain the READ and FORMAT statements for the reflection
data. Several alternative formats are already provided;  there is also a
data input type 0, indicating that the data will be provided, a line at a
time, by successive calls to SUBROUTINE QLSQIN, which the user must provide.\p
In general, any SUBROUTINE QxxxIN is a user-provided input routine for
reflection data.  (The user need not concern himself with special formats
or a routine QLSQIN if he has data in some standard format known to the
system; there is a dummy routine \stlink{q}{QLSQIN}\ in the Library, which will be
superseded by any routine of the same name if the user provides one.)
%
\section{Input of Fourier Data}\markright{Input of Fourier Data}
Input of reflection data for Fourier calculations is less accessible,
being buried in the various Fourier routines like \stlink{f}{FOUR1Z}, 
\stlink{f}{FOUR1D}\ as a
call to \stlink{f}{FOUINP}.  Again an input type 0 input routine \stlink{q}{QFOUIN}\ is
available.  The user would need to write a new QFOUIN if, say, he had
non-centrosymmetric data and was not assuming Friedel's Law.  He would
have to decide what to do with Friedel pairs before presenting them to
the standard Fourier calculations, which do assume Friedel.
%
\section{File Handling and the Use of Input/Output Units}
\markright{File Handling}
The assignment of FORTRAN units is handled by CCSL.
Interactive terminal input and output units, the line-printer and the
plotter are assigned to the units named ITI, ITO, LPT, and IPLO
respectively, and are given values appropriate to the system by INITIL.
They are held in the COMMON /IOUNIT/.\p
Assignment of files to FORTRAN units is done using FUNCTION
\stlink{n}{NOPFIL} to open a file and SUBROUTINE \stlink{c}{CLOFIL} to close it.\p
Fifteen units are required by CCSL and managed by it; if not forbidden by
the particular operating system they are units 20-34.  The user of
standard programs should find that his input and output are taken care
of by calls of NOPFIL already in the system.  A user writing or
modifying a main program will need to know more details, in order to
deal with his own files.\p
{\bf Note} The length of the leaf part of file-names in CCSL is restricted
to 10 characters: a main part of up to 6 characters optionally followed by a period (.)
and and extension of up to 3 characters.
\subsection{Opening a File to be Read}
To open a file to be read by a program in the simplest possible way
the code:
\\ 
\exac{~~~~~~LUN=NOPFIL(1)}\\ 
should be obeyed. The system will respond with the message:\\ 
\exac{Give name of input file}\\ 
to which the user should respond with the file name.  Under VMS on a VAX,
this can
include the full path name and an extension if required but it may be simply the
letters (up to 6) of the file name, in which case the extension .DAT is
added.  LUN will be set to the value of the Fortran unit allocated by 
the system.
The value of LUN should not normally concern the user;  he reads the file
by code such as:\\
\exac{~~~~~~READ (LUN,42)  FRED}\\ 
but if he needs for some reason to open a specifically numbered file he may
use SUBROUTINE \stlink{o}{OPNFIL}, thus:
\begin{verbatim} 

      LUNIT=14
      CALL OPNFIL(LUNIT,1)
\end{verbatim}
NOPFIL can interpret relative directory specifications in UNIX (eg ../) or VMS
(eg [-.]) and absolute file or directory names given as environment variables
(\$$\cdots$) in UNIX or as logical assignments ($\cdots:$) in VMS.
\subsection{Opening an Output File}
To open for writing a file which is not destined for a line printer, the
code:\\[0.2ex]
\exac{~~~~~~LUN=NOPFIL(2)}\\[0.2ex]
should be obeyed.  The file is then written using, for example,\\[0.2ex]
\exac{~~~~~~WRITE (LUN,43) JOAN}\\ 
To open for writing a file which is destined for a line printer, a VMS
user may write:\\[0.2ex]
\exac{~~~~~~LUN=NOPFIL(2002)}\\[0.2ex]
When the file is opened this will use:
\begin{verbatim}
      CARRIAGECONTROL='FORTRAN'
\end{verbatim}
Fortran carriage control is not recognised by UNIX.
\subsection{Options Available when Opening Files}
Other options can be obtained by obeying:\\[0.2ex]
\exac{~~~~~~LUN=NOPFIL(MODE)}\\[0.2ex]
where MODE indicates the type of file to be opened and how to obtain
the file-name.\p 
MODE may have up to 5 digits.The least significant, MODE1, indicates
the file-type; the second digit, MODE2, shows how to obtain the file
name; the third, MODE3, deals with default extensions; the fourth, MODE4,
selects formatted or unformatted files and the fifth, MODE5, selects
sequential or direct access files.\p
Character data such as a message or
the file name are transmitted to NOPFIL via the COMMON /SCRACH/ which
can contain 200 characters, and within NOPFIL is divided into MESSAG 
and NAMFIL of 100 characters each. The maximum length of both these
strings is configurable when building the system. \p
The significance of MODE is as follows:
\begin{list} {} {\setlength{\labelwidth}{3cm}
  \setlength{\parsep}{-1ex}
  \setlength{\leftmargin}{\labelwidth}
 \addtolength{\leftmargin}{1cm}} 
\item[MODE1 \hfill] = 1 for a read file\\
       = 2 for a write file status new\\
       = 3 for a write file status undefined (UNKNOWN)\\
       = 4 for a write file to be modified (APPEND for sequential files)\\
       = 5 for a scratch file.
 
\item[MODE2 \hfill] = 0 give the standard messages for
 read or write files; machine specific
           information like the path name and file extension 
           may be included in
           the user's response.\\
       = 1 use the message in MESSAG; otherwise as MODE2=0\\
       = 2 use the file-name in MESSAG. Report that the file is opened.\\
       = 3 as 2 but do not give file-opened message.\\
       = 4 as 0 but do not give file-opened message.
 
\item[MODE3 \hfill] = 0 use the system default for path and default 
           extension .DAT\\
       = 1 use defaults for extension, disc and path-name found in characters
           1-4, 5-10, and 11-30 respectively of NAMFIL (up to 40
           characters, in common SCRACH after item MESSAG). If the disc or
           path-name is absent, default as system.\\
       = 2 use the file-name exactly as given in response to the request.
 
\item[MODE4 \hfill] = 0 for a formatted file, not expecting FORTRAN 
           carriage control
           characters (and therefore not a line-printer file)\\
       = 1 for an unformatted file\\
       = 2 for a formatted file with FORTRAN carriage control characters,
           such as would be sent to a line printer.
 
\item[MODE5 \hfill] = 0 for a sequential file\\
       = 1 for a direct access file\\
       = 2 for RECORDTYPE='FIXED' (needed for ``GENIE" files).\end{list}
 
\subsection{Exit from NOPFIL}
After a successful exit from the call:\\[0.2ex]
\exac{~~~~~~LUN=NOPFIL(MODE)}\\[0.2ex]
LUN is set to a small positive number, which is the
unit number allocated to the opened file by NOPFIL.  LUN is zero
if no response is obtained (i.e. the RETURN key is pressed) when a
file name is reqested. LUN is negative if the file could not be opened,
although NOPFIL allows the user to try again after typing errors or
other blunders from which it has a sensible way to recover.\p 
The short names (10 characters) are stored in COMMON FILNAM and
may be obtained by declaring the COMMON and obeying:
\\[0.2ex]
\begin{verbatim}         CHARACTER*10 NAME,FILNOM
         NAME=FILNOM(LUN)
\end{verbatim}
\subsection{Closing Files}
When input or output operations are terminated on a particular unit
it should be returned to the system by obeying:
\verb|         CALL CLOFIL(LUN)|
\subsection{Examples of the Use of NOPFIL} 
The main program for a Fourier calculation is given below:
\begin{verbatim}C MINIMAL FOURIER PROGRAM.
      CALL PREFIN
      LUNI= NOPFIL(1)
      CALL RECIP
      CALL OPSYM(2)
      CALL SETFOU
   . . . map-producing code - see various main program examples
      STOP
      END
\end{verbatim}
\ssk
It contains one explicit call to NOPFIL; however the call to PREFIN
provokes several other calls to NOPFIL, one of which sends the message:\\ 
\exac{Give name of Crystal Data File :}\\ 
to which the user must reply by typing the appropriate file name.
Another call to NOPFIL does not result in a message as it just opens
an internal scratch file.  (If the Libary has been generated for RAL rather
than ILL, an early call of NOPFIL asks for the name of the printer 
output file.)\p
The call to \stlink{n}{NOPFIL}\ in the main program sends
the message:\\ 
\exac{Give name of input file :}\\ 
and the user should type the name of the file containing his reflection data.\p
A less trivial example is given in the fragments from the main program ABSMSF
reproduced below. \mlink{absmsf}{ABSMSF}\ is a program which reads a file containing
reflection intensities in the special format produced by the sorting
programs \mlink{arrnge}{ARRNGE}\ and \mlink{arrinc}{ARRINC}.  
The file name has the extension .ARR.
 It then makes 
an absorption correction if required and calculates the 
average of the corrected intensity over all equivalent reflections. Finally
it writes an output file containing
the mean structure factors and their standard deviations.
\\[0.2ex]
\begin{verbatim}      PROGRAM ABSMSF
            .
            .
      DIMENSION LUN(2)
      CHARACTER*56 HEDING
      LOGICAL INC
            .
            .
      COMMON /IOUNIT/LPT,ITI,ITO,IPLO,LUNI,IOUT
      COMMON /REFS/K(3,2),LL(48,2),R(500,2),SCALE(2),INC,II,FF(3,2)
      COMMON/SCRACH/MESSAG,NAMFIL
      CHARACTER *40 MESSAG
      CHARACTER *60 NAMFIL
      DATA HEADNG/'(5X,''h'',4X,''k'',4X,''l'',10X,''Fobs'',7X,
     1''DFobs''/)'/
            .
            .
      IF (INC.EQ.0) GO TO 2
C ALTER THE FORMAT FOR FLOATING POINT OUTPUT:
      HEADNG(2:2)='6'
      HEADNG(9:9)='7'
      HEADNG(16:16)='7'
      HEADNG(23:24)='12'
            .
            .
      NAMFIL='.ARR'
      LUN(1)=NOPFIL(101)
      NAMFIL='.SF '
      LOUT=NOPFIL(102)
            .
            .
      STOP
      END
\end{verbatim}
\ssk
The first call to NOPFIL asks the user for the name of the input file
and uses the default .ARR if no ``." is found in the reply.
The second call creates a new file for output with the same name as
the input file but with extension .SF (structure factors).
\section{Printed Output Files: Use of subroutine \slink{t}{TESTP}}
\markright{Printed Output Files}
There is a SUBROUTINE TESTP which
has been written to facilitate the production of long lists of similar
items with column headings at the top of each page.  Calls to TESTP have
the form:\\ 
\exac{      CALL TESTP(LUN,LCOUNT,NTEST,HEADNG,NLINES)}\\ 
where LUN is the FORTRAN unit on which the list is being written, LCOUNT
is the number of lines printed since the last page throw, NTEST is used
to provoke a page throw if there are less than NTEST lines left on the
page.\p
HEADNG is a character variable containing the page heading (see
the example above and also that 
in the BLOCK DATA sub-program below). NLINES is the number of
lines in the heading. HEADNG need not be in COMMON if data are only
written to the list from the main program. In ABSMSF it can be seen that
HEADNG is modified by the program if INC is .TRUE. (non-integer indices)
to accomodate the longer FORMAT required to print the indices in this case.
\\[0.2ex]
\begin{verbatim}      BLOCK DATA PAGEHD
      COMMON /HEDING/ HEADNG
      CHARACTER*132 HEADNG
      DATA HEADNG/' (''0Seq No       h    k    l   Omega  2Theta   
     1  Nu       R        dR      Peak cps  Bkgd cps      DVM
     2  Date    Time''/)'/
      END
\end{verbatim}\ssk
\section{Graphical Output}\markright{{Graphical Output}}
%
The form of graphical output is very dependent on the hardware used
to produce it and the local software used to drive that hardware.
CCSL addresses this problem by chanelling all calls to the plotter
hardware through a single SUBROUTINE \stlink{p}{PIGLET}.
\p
There is a separate
section of the CCSL Master File which contains various PIGLETs which have
been written to drive different devices, such as Tektronix 
4010 terminals and BENSON plotters. Special PIGLETs for GKS,
PostScript and PGPLOT interfaces are also included. A user who wishes to compile
a program which uses CCSL graphics should extract the most suitable
SUBROUTINE PIGLET from the Master File, modify it if necessary, and
include it in his link statement together with any graphics libraries
called by this particular PIGLET.
\p
In many cases the output will drive a graphics terminal directly
and local system facilities will be used to obtain hard copies.
Output files destined for centralised plotters will usually be
created and named by the local system software. 
\p
When the graphical output is sent directly to a file
by the CCSL graphics interface, for instance when using the
PostScript or PGPLOT PIGLETs the output file will be named by CCSL, usually
with the name of the program, a sequence number and an extension approriate to the type
of output (e.g. .PS for a PostScript file).
\subsection{Plotting in colour} 
16 different colour names are recognised by CCSL using the first 3 letters of their names
(case insensitive), they are:\\[1ex]
\begin{tabular}{rlllll}
\hspace{10mm}Number&\quad&Name&Red&Green&Blue\\
0&&White&1.00&  1.00  &1.00\\
1&&Black &0.00  &0.00  &0.00\\
2&&Red  &1.00  &0.00  &0.00\\
3&&Green  &0.00  &1.00  &0.00\\
4&&Blue&0.00  &0.00  &1.00\\
5&&Cyan&0.00  &1.00  &1.00\\
6&&Magenta&1.00  &0.00  &1.00\\
7&&Yellow&1.00  &1.00  &0.00\\
8&&Orange&1.00  &0.50  &0.00\\
9&&SpringGreen&0.50  &1.00  &0.00\\
10&&SeaGreen&0.00  &1.00  &0.50\\
11&&Aqua&0.00  &0.50  &1.00\\
12&&Purple&0.50  &0.00  &1.00\\
13&&Pink&1.00  &0.00  &0.50\\
14&&DkGrey&0.33  &0.33  &0.33\\
15&&LtGrey&0.66  &0.66  &0.66 \\ 
\end{tabular} \\
The function \stlink{l}{LCOLPG} will convert a colour name to a colour number recognised by
\stlink{p}{PIGLET}.
\subsection{Plotting Symbols}
Symbols in CCSL are plotted using the subroutine \stlink{k}{KANGA3} which recognises 9 different symbol shapes and two fill types: \em{filled} and \em{open}. The full list of symbols
with their names is given below. The names are recognised by case insensitive matching 
of the first 6 letters; and the fill property by case insensitive matching
of the first 4 letters in \em{filled} or \em{open}. \\[1ex]
\begin{tabular}{rlllll}
\hspace{10mm}Number&\quad&Name&Description&Filled/Open\\
1&&square&Square&Yes\\
2&&trianup&Triangle vertex up&Yes\\
3&&triandown&Triangle vertex down&Yes\\
4&&circle&A hexagon&Yes\\
5&&xcross&Cross with diagonal members&No\\
6&&plussign&Cross with hor. and vert. bars&No\\
7&&eggtimer&Xcross with ends closed&Yes\\
8&&bowtie&Eggtimer rotated 90 degreep&Yes\\
9&&diamond&Square rotated 45 degrees&Yes\\
\end{tabular}
The function \stlink{l}{LSYMPG} will convert a symbol name and fill type into a symbol
number for input to \stlink{k}{KANGA3}.

\finchapter
%\end{document}
