%\setcounter{page}{9}
%\setcounter{chapter}{1}
%\chapter{THE STRUCTURE OF THE SYSTEM}
%\begin{htmlonly}
%\usepackage{CCSLman,html,makeidx,color,ifthen,bm}
%%\setcounter{page}{9}
%\setcounter{chapter}{1}
%\chapter{THE STRUCTURE OF THE SYSTEM}
%\begin{htmlonly}
%\usepackage{CCSLman,html,makeidx,color,ifthen,bm}
%%\setcounter{page}{9}
%\setcounter{chapter}{1}
%\chapter{THE STRUCTURE OF THE SYSTEM}
%\begin{htmlonly}
%\usepackage{CCSLman,html,makeidx,color,ifthen,bm}
%%\setcounter{page}{9}
%\setcounter{chapter}{1}
%\chapter{THE STRUCTURE OF THE SYSTEM}
%\begin{htmlonly}
%\usepackage{CCSLman,html,makeidx,color,ifthen,bm}
%\input{chap2.ptr}
%\end{htmlonly}
\internal{c1}
\internal{c3}
\internal{c4}
\internal{c5}
\internal{c6}
\internal{c7}
\startdocument
\label{chap:2}
\markboth{The Structure of the System}{}
\par  
This Chapter is an introduction to the Library as a whole,  the language
used, and the Master File.  A reader interested initially only in the
crystallography could skip to the 
\latexonly{\bd{List of Available Routines}
(page~\pageref{routines:lis})}
\htmlonly{\href{../appenx/libclasslist.html}{List of Available Routines}}.
\section{CCSL As a FORTRAN Library}\markright{CCSL As a FORTRAN Library}
CCSL is a set of FORTRAN routines.  Specifications of all
the routines with detailed descriptions are in Appendix A, in alphabetical
order of routine.  Later in this chapter they are listed in groups under
various headings, to give the user a broad view of what is available.
\p
A FORTRAN job always has a \ital{main} program, which usually calls routines. A
CCSL user is expected to provide the main program, and then to link it with
CCSL compiled as a searchable library.  (The Master File contains some  
main programs as well as the Library).
\p 
Running a FORTRAN job varies from one computer to another, but
where a FORTRAN compiler exists there should be some sort of facilities
for:
\p 
\begin{list} {} {\setlength{\labelwidth}{2 cm}
  \setlength{\parsep}{-1ex}
  \setlength{\leftmargin}{\labelwidth}
 \addtolength{\leftmargin}{5mm}}
\item[(a) \hfill] compiling each routine of CCSL separately, and holding the set of
       such compiled routines as a Library, and
\item[(b) \hfill] when given a CCSL main program, scanning this Library and
       extracting any routines referred to by the main program (and all
       routines referred to by them, and so on) so that the program can be
       run.\end{list}
The point of compiling separate routines into a Library is that not
every run needs every routine;  the user is unlikely to call for, say,
absorption corrections in the same run as a plotted Fourier.  So he
wishes to load and obey only those FORTRAN routines he actually needs in the
program.
\p 
\subsection{Language}
CCSL is written almost entirely in FORTRAN 77, (ANSI standard X3.9-1978). The
reference for what is standard F77 was taken from the VAX-11
FORTRAN Language Reference Manual (AA-D034C-TE) trying to ignore the
extensions printed in blue. In update 4.21 (2011) the code was
upgraded so as to generate no warnings when compiled with 
\href{http://gcc.gnu.org/fortran/}{gfortran}.
to 
\p 
\subsection{Portability}
There will always be problems trying to make the FORTRAN portable.
The areas which are particularly troublesome in portable programs are:
\p 
\begin{list} {} {\setlength{\labelwidth}{2 cm}
  \setlength{\parsep}{-1ex}
  \setlength{\leftmargin}{\labelwidth}
 \addtolength{\leftmargin}{5mm}}
\item[\ \ 1) \hfill] input/output
\item[\ \ 2) \hfill] use of graphical output
\item[\ \ 3) \hfill] what the consequences of an error are\end{list}
\par  
but there are other areas, which tend only to become apparent when the
system is actually implemented on a particular machine.
\p
We are aware of the problems of portability, but nevertheless we insist
that CCSL must be portable.  Some of the FORTRAN used may seem pedestrian
as a result. 
\p 
The best way to get a view of CCSL's FORTRAN is to print out part of
the Library.
It is intended to be understandable as far as possible.
\p 
\section{The Master File}\markright{The Master File}
The complete set of CCSL routines and main programs is held in a \ital{Master
File}. When a new user who wishes to install CCSL on his computer,
he should obtain a Master File and a set of 
\iftexforht{\href{http://www.perl.com}{\ital{perl}}}{\ital{perl}}
 scripts which can be used to
generate a version of the Library and a set of main programs 
to run on his computer.
\p 
It would be possible to provide a FORTRAN Library which could be
compiled directly.  However, one of the features of the \ital{Master File}
format is that it contains symbolic parameters which allow the dimensions 
of arrays to be customised. For example, if the user discovers a need for,
say, 100 atomic positions, when only 50 are allowed in the standard CCSL,
he can remake his Library to cater for this.
\p 
It is useful to hold all the code (all routines and all programs)
together, even though the resulting file is large. This 
ensures that when a library and a set of programs is generated, all copies 
of a particular labelled COMMON block are identical;  getting them
out of step is a frequent
source of error.  All the COMMON blocks are listed in alphabetical order
of their names at the start of the Master File, many with symbolic
dimensions for their arrays (see also 
\iftexforht{\href{../appenx/appenc.html}{ Appendix C}}{Appendix C)}.
\p 
The Master File also contains all the alternative code for areas which
are system or device dependent.  Such code has been classified into three types
\begin{enumerate}
\item The strictness with which the F77 standard is imposed.\\
the present options are:
\begin{description}
\item[LAX] Allow very common extensions of the standard, in particular \$
FORMAT processing.
\item[PICKY] Enforce the standard rigidly
\end{description}
\item Operating system dependent code.
\begin{description}
\item[UNIX] Code and system functions understood only by UNIX operating 
systems eg SETENV, and case dependent file-names.
\item[VMS] The same for VMS. eg The READONLY option must be given to open
existing files.
\end{description}  
\item Site preferences, of historical origin. After its inception in the
now defunct Crystallographic laboratory of the Cavendish Laboratory Cambridge, 
the library was developed in parallel at the Institut Laue Langevin 
(ILL) and at the Rutherford Laboratory (RAL) where different preferences
developed.
\begin{description}
\item[ILL] The default extension for Crystal Data files is ".cry". Listing files
are called by the name of the program with extension ".lis".
\item[RAL] The default extension for Crystal Data files is ".ccl". The user
is asked to name his Listing files.
\end{description}
\end{enumerate}
The start of option dependent lines of code are flagged in the master file 
{\bf CS} followed by one of
the above \ital{option words}. The line will be unflagged so that
it is included in the FORTRAN compilation only if the \ital{option word} is amongst the
chosen options. A minus sign(-) preceeding the \ital{option word} indicates that
the line should be included only of the \ital{option} is {\bf not} chosen.
\par 
More details of the Master file and the scripts provided to manipulate it are
given in \iftexforht{href{../appenx/appene.html}{Appendix E}}{Appendix E}. 
\p
\subsection{Sequence of Routines}
The Master File contains a sequence of routines in alphabetical order of
their names.  This ordering is usually the most convenient for reference,
but is unsuitable for a library on machines which have
one-pass loaders. 
In order to make it possible  to generate a \ital{sorted} list of routines,
which may be correctly linked by a one-pass loader. The necessary 
information is stored in the first (comment) line of each routine, which reads:
``C~LEVEL" followed by an integer.  Those routines which call nothing else from
CCSL are LEVEL~1;  those only calling level 1 are LEVEL~2, and so on.
The sorted version of the Library should have the routines of the highest
level first, then all the next level, and so on down to LEVEL~1.
\p
{bf Note} added in the April 2013 edition\\
One pass loaders are not used with gfortran and so the LEVELS given to recently
added modules need no longer be strictly correct although the C LEVEL line
must still be present.
\p 
\section{Routines Available}\markright{Routines Available}
Most CCSL routines can be assigned to one of the three broad categories
introduced in the small sample main program of \htmlref{Chapter 1}{chap:1}.  These are:
\p
\begin{list} {} {\setlength{\labelwidth}{20mm}
  \setlength{\parsep}{-1ex}
  \setlength{\leftmargin}{\labelwidth}
 \addtolength{\leftmargin}{5mm}}
 \label{chap2:groups}
\item[A. \hfill] \ital{Setting-up} routines, concerned with input of information and 
     preparation for subsequent calculation;
\item[B. \hfill] \ital{Crystallographic} routines, doing specific calculations such as a
       Fourier section, a structure factor or an absorption correction; 
\item[C. \hfill] \ital{Utility} routines, which are of more general use, being small
       routines to perform tasks such as the solution of spherical triangles,
       matrix multiplication or a vector product.\end{list}
\p
The boundaries between categories are not rigid.  In general, category C
routines can be used in many contexts and do not include labelled COMMON
(other than /IOUNIT/ which associates FORTRAN input/output units with
physical devices).  Category B are useful in more specific contexts;
various output routines are included.  Category A includes all routines
which read from the Crystal Data; most of these routines are used just once
at the start of a job.
\pn
Within each category there is a wide range of material.  Some attempt at
further classification follows.  Some routines are of obvious use to
users who write their own main programs, whereas some belong internally
to CCSL.
\p 
All the routines available at the time of writing are listed here.  The
user should also consult the separate Appendix A, which gives more detail
and may be more up-to-date.
%******************************************************************
% Note: The subroutine list is created from the index to Appendix A
% by the perl script mkindex.  PJB
%******************************************************************
%
%}%\end{latexonly}
\iftexforht{\applink{see:Classified List of Subroutines\\}
{../appenx/appendix.html}{appa}
\href{../appenx/libalfa.html}{Alphabetic List of Subroutines}}
{\section{Classified List of Subroutines \label{routines:lis}}
\markright{Classified List of Subroutines}
The page numbers given refer to AppendixA\\[0.5ex]
\input{libclasslist}
}
\finchapter
%\end{document}

%\end{htmlonly}
\internal{c1}
\internal{c3}
\internal{c4}
\internal{c5}
\internal{c6}
\internal{c7}
\startdocument
\label{chap:2}
\markboth{The Structure of the System}{}
\par  
This Chapter is an introduction to the Library as a whole,  the language
used, and the Master File.  A reader interested initially only in the
crystallography could skip to the 
\latexonly{\bd{List of Available Routines}
(page~\pageref{routines:lis})}
\htmlonly{\href{../appenx/libclasslist.html}{List of Available Routines}}.
\section{CCSL As a FORTRAN Library}\markright{CCSL As a FORTRAN Library}
CCSL is a set of FORTRAN routines.  Specifications of all
the routines with detailed descriptions are in Appendix A, in alphabetical
order of routine.  Later in this chapter they are listed in groups under
various headings, to give the user a broad view of what is available.
\p
A FORTRAN job always has a \ital{main} program, which usually calls routines. A
CCSL user is expected to provide the main program, and then to link it with
CCSL compiled as a searchable library.  (The Master File contains some  
main programs as well as the Library).
\p 
Running a FORTRAN job varies from one computer to another, but
where a FORTRAN compiler exists there should be some sort of facilities
for:
\p 
\begin{list} {} {\setlength{\labelwidth}{2 cm}
  \setlength{\parsep}{-1ex}
  \setlength{\leftmargin}{\labelwidth}
 \addtolength{\leftmargin}{5mm}}
\item[(a) \hfill] compiling each routine of CCSL separately, and holding the set of
       such compiled routines as a Library, and
\item[(b) \hfill] when given a CCSL main program, scanning this Library and
       extracting any routines referred to by the main program (and all
       routines referred to by them, and so on) so that the program can be
       run.\end{list}
The point of compiling separate routines into a Library is that not
every run needs every routine;  the user is unlikely to call for, say,
absorption corrections in the same run as a plotted Fourier.  So he
wishes to load and obey only those FORTRAN routines he actually needs in the
program.
\p 
\subsection{Language}
CCSL is written almost entirely in FORTRAN 77, (ANSI standard X3.9-1978). The
reference for what is standard F77 was taken from the VAX-11
FORTRAN Language Reference Manual (AA-D034C-TE) trying to ignore the
extensions printed in blue. In update 4.21 (2011) the code was
upgraded so as to generate no warnings when compiled with 
\href{http://gcc.gnu.org/fortran/}{gfortran}.
to 
\p 
\subsection{Portability}
There will always be problems trying to make the FORTRAN portable.
The areas which are particularly troublesome in portable programs are:
\p 
\begin{list} {} {\setlength{\labelwidth}{2 cm}
  \setlength{\parsep}{-1ex}
  \setlength{\leftmargin}{\labelwidth}
 \addtolength{\leftmargin}{5mm}}
\item[\ \ 1) \hfill] input/output
\item[\ \ 2) \hfill] use of graphical output
\item[\ \ 3) \hfill] what the consequences of an error are\end{list}
\par  
but there are other areas, which tend only to become apparent when the
system is actually implemented on a particular machine.
\p
We are aware of the problems of portability, but nevertheless we insist
that CCSL must be portable.  Some of the FORTRAN used may seem pedestrian
as a result. 
\p 
The best way to get a view of CCSL's FORTRAN is to print out part of
the Library.
It is intended to be understandable as far as possible.
\p 
\section{The Master File}\markright{The Master File}
The complete set of CCSL routines and main programs is held in a \ital{Master
File}. When a new user who wishes to install CCSL on his computer,
he should obtain a Master File and a set of 
\iftexforht{\href{http://www.perl.com}{\ital{perl}}}{\ital{perl}}
 scripts which can be used to
generate a version of the Library and a set of main programs 
to run on his computer.
\p 
It would be possible to provide a FORTRAN Library which could be
compiled directly.  However, one of the features of the \ital{Master File}
format is that it contains symbolic parameters which allow the dimensions 
of arrays to be customised. For example, if the user discovers a need for,
say, 100 atomic positions, when only 50 are allowed in the standard CCSL,
he can remake his Library to cater for this.
\p 
It is useful to hold all the code (all routines and all programs)
together, even though the resulting file is large. This 
ensures that when a library and a set of programs is generated, all copies 
of a particular labelled COMMON block are identical;  getting them
out of step is a frequent
source of error.  All the COMMON blocks are listed in alphabetical order
of their names at the start of the Master File, many with symbolic
dimensions for their arrays (see also 
\iftexforht{\href{../appenx/appenc.html}{ Appendix C}}{Appendix C)}.
\p 
The Master File also contains all the alternative code for areas which
are system or device dependent.  Such code has been classified into three types
\begin{enumerate}
\item The strictness with which the F77 standard is imposed.\\
the present options are:
\begin{description}
\item[LAX] Allow very common extensions of the standard, in particular \$
FORMAT processing.
\item[PICKY] Enforce the standard rigidly
\end{description}
\item Operating system dependent code.
\begin{description}
\item[UNIX] Code and system functions understood only by UNIX operating 
systems eg SETENV, and case dependent file-names.
\item[VMS] The same for VMS. eg The READONLY option must be given to open
existing files.
\end{description}  
\item Site preferences, of historical origin. After its inception in the
now defunct Crystallographic laboratory of the Cavendish Laboratory Cambridge, 
the library was developed in parallel at the Institut Laue Langevin 
(ILL) and at the Rutherford Laboratory (RAL) where different preferences
developed.
\begin{description}
\item[ILL] The default extension for Crystal Data files is ".cry". Listing files
are called by the name of the program with extension ".lis".
\item[RAL] The default extension for Crystal Data files is ".ccl". The user
is asked to name his Listing files.
\end{description}
\end{enumerate}
The start of option dependent lines of code are flagged in the master file 
{\bf CS} followed by one of
the above \ital{option words}. The line will be unflagged so that
it is included in the FORTRAN compilation only if the \ital{option word} is amongst the
chosen options. A minus sign(-) preceeding the \ital{option word} indicates that
the line should be included only of the \ital{option} is {\bf not} chosen.
\par 
More details of the Master file and the scripts provided to manipulate it are
given in \iftexforht{href{../appenx/appene.html}{Appendix E}}{Appendix E}. 
\p
\subsection{Sequence of Routines}
The Master File contains a sequence of routines in alphabetical order of
their names.  This ordering is usually the most convenient for reference,
but is unsuitable for a library on machines which have
one-pass loaders. 
In order to make it possible  to generate a \ital{sorted} list of routines,
which may be correctly linked by a one-pass loader. The necessary 
information is stored in the first (comment) line of each routine, which reads:
``C~LEVEL" followed by an integer.  Those routines which call nothing else from
CCSL are LEVEL~1;  those only calling level 1 are LEVEL~2, and so on.
The sorted version of the Library should have the routines of the highest
level first, then all the next level, and so on down to LEVEL~1.
\p
{bf Note} added in the April 2013 edition\\
One pass loaders are not used with gfortran and so the LEVELS given to recently
added modules need no longer be strictly correct although the C LEVEL line
must still be present.
\p 
\section{Routines Available}\markright{Routines Available}
Most CCSL routines can be assigned to one of the three broad categories
introduced in the small sample main program of \htmlref{Chapter 1}{chap:1}.  These are:
\p
\begin{list} {} {\setlength{\labelwidth}{20mm}
  \setlength{\parsep}{-1ex}
  \setlength{\leftmargin}{\labelwidth}
 \addtolength{\leftmargin}{5mm}}
 \label{chap2:groups}
\item[A. \hfill] \ital{Setting-up} routines, concerned with input of information and 
     preparation for subsequent calculation;
\item[B. \hfill] \ital{Crystallographic} routines, doing specific calculations such as a
       Fourier section, a structure factor or an absorption correction; 
\item[C. \hfill] \ital{Utility} routines, which are of more general use, being small
       routines to perform tasks such as the solution of spherical triangles,
       matrix multiplication or a vector product.\end{list}
\p
The boundaries between categories are not rigid.  In general, category C
routines can be used in many contexts and do not include labelled COMMON
(other than /IOUNIT/ which associates FORTRAN input/output units with
physical devices).  Category B are useful in more specific contexts;
various output routines are included.  Category A includes all routines
which read from the Crystal Data; most of these routines are used just once
at the start of a job.
\pn
Within each category there is a wide range of material.  Some attempt at
further classification follows.  Some routines are of obvious use to
users who write their own main programs, whereas some belong internally
to CCSL.
\p 
All the routines available at the time of writing are listed here.  The
user should also consult the separate Appendix A, which gives more detail
and may be more up-to-date.
%******************************************************************
% Note: The subroutine list is created from the index to Appendix A
% by the perl script mkindex.  PJB
%******************************************************************
%
%}%\end{latexonly}
\iftexforht{\applink{see:Classified List of Subroutines\\}
{../appenx/appendix.html}{appa}
\href{../appenx/libalfa.html}{Alphabetic List of Subroutines}}
{\section{Classified List of Subroutines \label{routines:lis}}
\markright{Classified List of Subroutines}
The page numbers given refer to AppendixA\\[0.5ex]
\input{libclasslist}
}
\finchapter
%\end{document}

%\end{htmlonly}
\internal{c1}
\internal{c3}
\internal{c4}
\internal{c5}
\internal{c6}
\internal{c7}
\startdocument
\label{chap:2}
\markboth{The Structure of the System}{}
\par  
This Chapter is an introduction to the Library as a whole,  the language
used, and the Master File.  A reader interested initially only in the
crystallography could skip to the 
\latexonly{\bd{List of Available Routines}
(page~\pageref{routines:lis})}
\htmlonly{\href{../appenx/libclasslist.html}{List of Available Routines}}.
\section{CCSL As a FORTRAN Library}\markright{CCSL As a FORTRAN Library}
CCSL is a set of FORTRAN routines.  Specifications of all
the routines with detailed descriptions are in Appendix A, in alphabetical
order of routine.  Later in this chapter they are listed in groups under
various headings, to give the user a broad view of what is available.
\p
A FORTRAN job always has a \ital{main} program, which usually calls routines. A
CCSL user is expected to provide the main program, and then to link it with
CCSL compiled as a searchable library.  (The Master File contains some  
main programs as well as the Library).
\p 
Running a FORTRAN job varies from one computer to another, but
where a FORTRAN compiler exists there should be some sort of facilities
for:
\p 
\begin{list} {} {\setlength{\labelwidth}{2 cm}
  \setlength{\parsep}{-1ex}
  \setlength{\leftmargin}{\labelwidth}
 \addtolength{\leftmargin}{5mm}}
\item[(a) \hfill] compiling each routine of CCSL separately, and holding the set of
       such compiled routines as a Library, and
\item[(b) \hfill] when given a CCSL main program, scanning this Library and
       extracting any routines referred to by the main program (and all
       routines referred to by them, and so on) so that the program can be
       run.\end{list}
The point of compiling separate routines into a Library is that not
every run needs every routine;  the user is unlikely to call for, say,
absorption corrections in the same run as a plotted Fourier.  So he
wishes to load and obey only those FORTRAN routines he actually needs in the
program.
\p 
\subsection{Language}
CCSL is written almost entirely in FORTRAN 77, (ANSI standard X3.9-1978). The
reference for what is standard F77 was taken from the VAX-11
FORTRAN Language Reference Manual (AA-D034C-TE) trying to ignore the
extensions printed in blue. In update 4.21 (2011) the code was
upgraded so as to generate no warnings when compiled with 
\href{http://gcc.gnu.org/fortran/}{gfortran}.
to 
\p 
\subsection{Portability}
There will always be problems trying to make the FORTRAN portable.
The areas which are particularly troublesome in portable programs are:
\p 
\begin{list} {} {\setlength{\labelwidth}{2 cm}
  \setlength{\parsep}{-1ex}
  \setlength{\leftmargin}{\labelwidth}
 \addtolength{\leftmargin}{5mm}}
\item[\ \ 1) \hfill] input/output
\item[\ \ 2) \hfill] use of graphical output
\item[\ \ 3) \hfill] what the consequences of an error are\end{list}
\par  
but there are other areas, which tend only to become apparent when the
system is actually implemented on a particular machine.
\p
We are aware of the problems of portability, but nevertheless we insist
that CCSL must be portable.  Some of the FORTRAN used may seem pedestrian
as a result. 
\p 
The best way to get a view of CCSL's FORTRAN is to print out part of
the Library.
It is intended to be understandable as far as possible.
\p 
\section{The Master File}\markright{The Master File}
The complete set of CCSL routines and main programs is held in a \ital{Master
File}. When a new user who wishes to install CCSL on his computer,
he should obtain a Master File and a set of 
\iftexforht{\href{http://www.perl.com}{\ital{perl}}}{\ital{perl}}
 scripts which can be used to
generate a version of the Library and a set of main programs 
to run on his computer.
\p 
It would be possible to provide a FORTRAN Library which could be
compiled directly.  However, one of the features of the \ital{Master File}
format is that it contains symbolic parameters which allow the dimensions 
of arrays to be customised. For example, if the user discovers a need for,
say, 100 atomic positions, when only 50 are allowed in the standard CCSL,
he can remake his Library to cater for this.
\p 
It is useful to hold all the code (all routines and all programs)
together, even though the resulting file is large. This 
ensures that when a library and a set of programs is generated, all copies 
of a particular labelled COMMON block are identical;  getting them
out of step is a frequent
source of error.  All the COMMON blocks are listed in alphabetical order
of their names at the start of the Master File, many with symbolic
dimensions for their arrays (see also 
\iftexforht{\href{../appenx/appenc.html}{ Appendix C}}{Appendix C)}.
\p 
The Master File also contains all the alternative code for areas which
are system or device dependent.  Such code has been classified into three types
\begin{enumerate}
\item The strictness with which the F77 standard is imposed.\\
the present options are:
\begin{description}
\item[LAX] Allow very common extensions of the standard, in particular \$
FORMAT processing.
\item[PICKY] Enforce the standard rigidly
\end{description}
\item Operating system dependent code.
\begin{description}
\item[UNIX] Code and system functions understood only by UNIX operating 
systems eg SETENV, and case dependent file-names.
\item[VMS] The same for VMS. eg The READONLY option must be given to open
existing files.
\end{description}  
\item Site preferences, of historical origin. After its inception in the
now defunct Crystallographic laboratory of the Cavendish Laboratory Cambridge, 
the library was developed in parallel at the Institut Laue Langevin 
(ILL) and at the Rutherford Laboratory (RAL) where different preferences
developed.
\begin{description}
\item[ILL] The default extension for Crystal Data files is ".cry". Listing files
are called by the name of the program with extension ".lis".
\item[RAL] The default extension for Crystal Data files is ".ccl". The user
is asked to name his Listing files.
\end{description}
\end{enumerate}
The start of option dependent lines of code are flagged in the master file 
{\bf CS} followed by one of
the above \ital{option words}. The line will be unflagged so that
it is included in the FORTRAN compilation only if the \ital{option word} is amongst the
chosen options. A minus sign(-) preceeding the \ital{option word} indicates that
the line should be included only of the \ital{option} is {\bf not} chosen.
\par 
More details of the Master file and the scripts provided to manipulate it are
given in \iftexforht{href{../appenx/appene.html}{Appendix E}}{Appendix E}. 
\p
\subsection{Sequence of Routines}
The Master File contains a sequence of routines in alphabetical order of
their names.  This ordering is usually the most convenient for reference,
but is unsuitable for a library on machines which have
one-pass loaders. 
In order to make it possible  to generate a \ital{sorted} list of routines,
which may be correctly linked by a one-pass loader. The necessary 
information is stored in the first (comment) line of each routine, which reads:
``C~LEVEL" followed by an integer.  Those routines which call nothing else from
CCSL are LEVEL~1;  those only calling level 1 are LEVEL~2, and so on.
The sorted version of the Library should have the routines of the highest
level first, then all the next level, and so on down to LEVEL~1.
\p
{bf Note} added in the April 2013 edition\\
One pass loaders are not used with gfortran and so the LEVELS given to recently
added modules need no longer be strictly correct although the C LEVEL line
must still be present.
\p 
\section{Routines Available}\markright{Routines Available}
Most CCSL routines can be assigned to one of the three broad categories
introduced in the small sample main program of \htmlref{Chapter 1}{chap:1}.  These are:
\p
\begin{list} {} {\setlength{\labelwidth}{20mm}
  \setlength{\parsep}{-1ex}
  \setlength{\leftmargin}{\labelwidth}
 \addtolength{\leftmargin}{5mm}}
 \label{chap2:groups}
\item[A. \hfill] \ital{Setting-up} routines, concerned with input of information and 
     preparation for subsequent calculation;
\item[B. \hfill] \ital{Crystallographic} routines, doing specific calculations such as a
       Fourier section, a structure factor or an absorption correction; 
\item[C. \hfill] \ital{Utility} routines, which are of more general use, being small
       routines to perform tasks such as the solution of spherical triangles,
       matrix multiplication or a vector product.\end{list}
\p
The boundaries between categories are not rigid.  In general, category C
routines can be used in many contexts and do not include labelled COMMON
(other than /IOUNIT/ which associates FORTRAN input/output units with
physical devices).  Category B are useful in more specific contexts;
various output routines are included.  Category A includes all routines
which read from the Crystal Data; most of these routines are used just once
at the start of a job.
\pn
Within each category there is a wide range of material.  Some attempt at
further classification follows.  Some routines are of obvious use to
users who write their own main programs, whereas some belong internally
to CCSL.
\p 
All the routines available at the time of writing are listed here.  The
user should also consult the separate Appendix A, which gives more detail
and may be more up-to-date.
%******************************************************************
% Note: The subroutine list is created from the index to Appendix A
% by the perl script mkindex.  PJB
%******************************************************************
%
%}%\end{latexonly}
\iftexforht{\applink{see:Classified List of Subroutines\\}
{../appenx/appendix.html}{appa}
\href{../appenx/libalfa.html}{Alphabetic List of Subroutines}}
{\section{Classified List of Subroutines \label{routines:lis}}
\markright{Classified List of Subroutines}
The page numbers given refer to AppendixA\\[0.5ex]
\input{libclasslist}
}
\finchapter
%\end{document}

%\end{htmlonly}
\internal{c1}
\internal{c3}
\internal{c4}
\internal{c5}
\internal{c6}
\internal{c7}
\startdocument
\label{chap:2}
\markboth{The Structure of the System}{}
\par  
This Chapter is an introduction to the Library as a whole,  the language
used, and the Master File.  A reader interested initially only in the
crystallography could skip to the 
\latexonly{\bd{List of Available Routines}
(page~\pageref{routines:lis})}
\htmlonly{\href{../appenx/libclasslist.html}{List of Available Routines}}.
\section{CCSL As a FORTRAN Library}\markright{CCSL As a FORTRAN Library}
CCSL is a set of FORTRAN routines.  Specifications of all
the routines with detailed descriptions are in Appendix A, in alphabetical
order of routine.  Later in this chapter they are listed in groups under
various headings, to give the user a broad view of what is available.
\p
A FORTRAN job always has a \ital{main} program, which usually calls routines. A
CCSL user is expected to provide the main program, and then to link it with
CCSL compiled as a searchable library.  (The Master File contains some  
main programs as well as the Library).
\p 
Running a FORTRAN job varies from one computer to another, but
where a FORTRAN compiler exists there should be some sort of facilities
for:
\p 
\begin{list} {} {\setlength{\labelwidth}{2 cm}
  \setlength{\parsep}{-1ex}
  \setlength{\leftmargin}{\labelwidth}
 \addtolength{\leftmargin}{5mm}}
\item[(a) \hfill] compiling each routine of CCSL separately, and holding the set of
       such compiled routines as a Library, and
\item[(b) \hfill] when given a CCSL main program, scanning this Library and
       extracting any routines referred to by the main program (and all
       routines referred to by them, and so on) so that the program can be
       run.\end{list}
The point of compiling separate routines into a Library is that not
every run needs every routine;  the user is unlikely to call for, say,
absorption corrections in the same run as a plotted Fourier.  So he
wishes to load and obey only those FORTRAN routines he actually needs in the
program.
\p 
\subsection{Language}
CCSL is written almost entirely in FORTRAN 77, (ANSI standard X3.9-1978). The
reference for what is standard F77 was taken from the VAX-11
FORTRAN Language Reference Manual (AA-D034C-TE) trying to ignore the
extensions printed in blue. In update 4.21 (2011) the code was
upgraded so as to generate no warnings when compiled with 
\href{http://gcc.gnu.org/fortran/}{gfortran}.
to 
\p 
\subsection{Portability}
There will always be problems trying to make the FORTRAN portable.
The areas which are particularly troublesome in portable programs are:
\p 
\begin{list} {} {\setlength{\labelwidth}{2 cm}
  \setlength{\parsep}{-1ex}
  \setlength{\leftmargin}{\labelwidth}
 \addtolength{\leftmargin}{5mm}}
\item[\ \ 1) \hfill] input/output
\item[\ \ 2) \hfill] use of graphical output
\item[\ \ 3) \hfill] what the consequences of an error are\end{list}
\par  
but there are other areas, which tend only to become apparent when the
system is actually implemented on a particular machine.
\p
We are aware of the problems of portability, but nevertheless we insist
that CCSL must be portable.  Some of the FORTRAN used may seem pedestrian
as a result. 
\p 
The best way to get a view of CCSL's FORTRAN is to print out part of
the Library.
It is intended to be understandable as far as possible.
\p 
\section{The Master File}\markright{The Master File}
The complete set of CCSL routines and main programs is held in a \ital{Master
File}. When a new user who wishes to install CCSL on his computer,
he should obtain a Master File and a set of 
\iftexforht{\href{http://www.perl.com}{\ital{perl}}}{\ital{perl}}
 scripts which can be used to
generate a version of the Library and a set of main programs 
to run on his computer.
\p 
It would be possible to provide a FORTRAN Library which could be
compiled directly.  However, one of the features of the \ital{Master File}
format is that it contains symbolic parameters which allow the dimensions 
of arrays to be customised. For example, if the user discovers a need for,
say, 100 atomic positions, when only 50 are allowed in the standard CCSL,
he can remake his Library to cater for this.
\p 
It is useful to hold all the code (all routines and all programs)
together, even though the resulting file is large. This 
ensures that when a library and a set of programs is generated, all copies 
of a particular labelled COMMON block are identical;  getting them
out of step is a frequent
source of error.  All the COMMON blocks are listed in alphabetical order
of their names at the start of the Master File, many with symbolic
dimensions for their arrays (see also 
\iftexforht{\href{../appenx/appenc.html}{ Appendix C}}{Appendix C)}.
\p 
The Master File also contains all the alternative code for areas which
are system or device dependent.  Such code has been classified into three types
\begin{enumerate}
\item The strictness with which the F77 standard is imposed.\\
the present options are:
\begin{description}
\item[LAX] Allow very common extensions of the standard, in particular \$
FORMAT processing.
\item[PICKY] Enforce the standard rigidly
\end{description}
\item Operating system dependent code.
\begin{description}
\item[UNIX] Code and system functions understood only by UNIX operating 
systems eg SETENV, and case dependent file-names.
\item[VMS] The same for VMS. eg The READONLY option must be given to open
existing files.
\end{description}  
\item Site preferences, of historical origin. After its inception in the
now defunct Crystallographic laboratory of the Cavendish Laboratory Cambridge, 
the library was developed in parallel at the Institut Laue Langevin 
(ILL) and at the Rutherford Laboratory (RAL) where different preferences
developed.
\begin{description}
\item[ILL] The default extension for Crystal Data files is ".cry". Listing files
are called by the name of the program with extension ".lis".
\item[RAL] The default extension for Crystal Data files is ".ccl". The user
is asked to name his Listing files.
\end{description}
\end{enumerate}
The start of option dependent lines of code are flagged in the master file 
{\bf CS} followed by one of
the above \ital{option words}. The line will be unflagged so that
it is included in the FORTRAN compilation only if the \ital{option word} is amongst the
chosen options. A minus sign(-) preceeding the \ital{option word} indicates that
the line should be included only of the \ital{option} is {\bf not} chosen.
\par 
More details of the Master file and the scripts provided to manipulate it are
given in \iftexforht{href{../appenx/appene.html}{Appendix E}}{Appendix E}. 
\p
\subsection{Sequence of Routines}
The Master File contains a sequence of routines in alphabetical order of
their names.  This ordering is usually the most convenient for reference,
but is unsuitable for a library on machines which have
one-pass loaders. 
In order to make it possible  to generate a \ital{sorted} list of routines,
which may be correctly linked by a one-pass loader. The necessary 
information is stored in the first (comment) line of each routine, which reads:
``C~LEVEL" followed by an integer.  Those routines which call nothing else from
CCSL are LEVEL~1;  those only calling level 1 are LEVEL~2, and so on.
The sorted version of the Library should have the routines of the highest
level first, then all the next level, and so on down to LEVEL~1.
\p
{bf Note} added in the April 2013 edition\\
One pass loaders are not used with gfortran and so the LEVELS given to recently
added modules need no longer be strictly correct although the C LEVEL line
must still be present.
\p 
\section{Routines Available}\markright{Routines Available}
Most CCSL routines can be assigned to one of the three broad categories
introduced in the small sample main program of \htmlref{Chapter 1}{chap:1}.  These are:
\p
\begin{list} {} {\setlength{\labelwidth}{20mm}
  \setlength{\parsep}{-1ex}
  \setlength{\leftmargin}{\labelwidth}
 \addtolength{\leftmargin}{5mm}}
 \label{chap2:groups}
\item[A. \hfill] \ital{Setting-up} routines, concerned with input of information and 
     preparation for subsequent calculation;
\item[B. \hfill] \ital{Crystallographic} routines, doing specific calculations such as a
       Fourier section, a structure factor or an absorption correction; 
\item[C. \hfill] \ital{Utility} routines, which are of more general use, being small
       routines to perform tasks such as the solution of spherical triangles,
       matrix multiplication or a vector product.\end{list}
\p
The boundaries between categories are not rigid.  In general, category C
routines can be used in many contexts and do not include labelled COMMON
(other than /IOUNIT/ which associates FORTRAN input/output units with
physical devices).  Category B are useful in more specific contexts;
various output routines are included.  Category A includes all routines
which read from the Crystal Data; most of these routines are used just once
at the start of a job.
\pn
Within each category there is a wide range of material.  Some attempt at
further classification follows.  Some routines are of obvious use to
users who write their own main programs, whereas some belong internally
to CCSL.
\p 
All the routines available at the time of writing are listed here.  The
user should also consult the separate Appendix A, which gives more detail
and may be more up-to-date.
%******************************************************************
% Note: The subroutine list is created from the index to Appendix A
% by the perl script mkindex.  PJB
%******************************************************************
%
%}%\end{latexonly}
\iftexforht{\applink{see:Classified List of Subroutines\\}
{../appenx/appendix.html}{appa}
\href{../appenx/libalfa.html}{Alphabetic List of Subroutines}}
{\section{Classified List of Subroutines \label{routines:lis}}
\markright{Classified List of Subroutines}
The page numbers given refer to AppendixA\\[0.5ex]
\input{libclasslist}
}
\finchapter
%\end{document}
