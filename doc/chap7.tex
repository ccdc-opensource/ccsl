%\documentclass[onecolumn,twoside,11pt,a4paper]{report}
%\usepackage{CCSLman,html,makeidx,color,ifthen,bm}
%\begin{document}
%\chapter{USING THE SYSTEM}
%\begin{htmlonly}
%\usepackage{CCSLman,html,makeidx,color,ifthen,bm}
%%\documentclass[onecolumn,twoside,11pt,a4paper]{report}
%\usepackage{CCSLman,html,makeidx,color,ifthen,bm}
%\begin{document}
%\chapter{USING THE SYSTEM}
%\begin{htmlonly}
%\usepackage{CCSLman,html,makeidx,color,ifthen,bm}
%%\documentclass[onecolumn,twoside,11pt,a4paper]{report}
%\usepackage{CCSLman,html,makeidx,color,ifthen,bm}
%\begin{document}
%\chapter{USING THE SYSTEM}
%\begin{htmlonly}
%\usepackage{CCSLman,html,makeidx,color,ifthen,bm}
%%\documentclass[onecolumn,twoside,11pt,a4paper]{report}
%\usepackage{CCSLman,html,makeidx,color,ifthen,bm}
%\begin{document}
%\chapter{USING THE SYSTEM}
%\begin{htmlonly}
%\usepackage{CCSLman,html,makeidx,color,ifthen,bm}
%\input{chap7.ptr}
%\end{htmlonly}
%\internal{c2}
%\internal{c3}
%\internal{c4}
%\internal{c5}
%\internal{c6}
%\internal{c1}
\startdocument
\label{chap:7}
\markboth{Using the System}{}
We assume first that the user has access to a computer or network
on which CCSL has already been compiled and made into a Library.\p
Any discussion of the details of running jobs must naturally refer to
specific items such as file names, which differ on different computers.
Many current users of CCSL use  LINUX or UNIX
and the examples in general refer to them. Users of other operating
systems such as WINDOWS or DOS should take them only as a rough guide. 
\section{Running \emph{Black Box} Jobs}\markright{Running \emph{Black Box} Jobs}
A standard main program such as \mlink{arrnge}{ARRNGE}\ (which groups together the
results of single crystal diffraction measurements) will have been
precompiled and linked using CCSL as a search Library.  An executable file
such as ARRNGE.EXE should therefore exist.
The user wishing to run this as it stands, as a \ital{black box}, first puts
his Crystal Data into some file, named, for example, TESTAR.CRY.
\pn
He would also have files containing data to arrange, called, 
say, \mbox{refln2.c5}, \mbox{refln1.c5}
and \mbox{refln3.c5}, and a list of the numbers of reflections to 
reject in, say, \mbox{badref.rej}.\p
The following sequence would run the program.
The dialogue in italics is to be typed by the user.\p
\begin{list} {} {\setlength{\labelwidth}{25mm}
  \setlength{\parsep}{-1ex}
  \setlength{\leftmargin}{5mm}}
\item\texttt{2\% arrnge}
\item\exac{Give name of Crystal data file  }\ital{TESTAR}\\
\item\exac{Give name of file containing rejection list   }\ital{badref}\\
\item\exac{Give name of reflection file   }\ital{refln1.c5}\\
\item\exac{Give name of next reflection file (or c/r to end)  }\ital{refln2.c5}\\
\item\exac{Give name of next reflection file (or c/r to end)  }\ital{refln3.c5}\\
\item\exac{Give name of next reflection file (or c/r to end)}\\
\ms 
\item\exac{Sort started   . . .   Sorted 1485 records}\\ 
\ms 
\item\exac{Give name for output file of sorted reflections   }\ital{arrout}\ssp
 \exac{3\%}
\end{list} 
 
In general, the user supplies file names on request.  The file extension
for Crystal Data files defaults to .cry or .ccl . General data 
files have default extension .dat although some default extensions are
associated with particular data types. (For example, for \mbox{arrnge,} .fli is
the default for DTYP=1, meaning
polarised neutron flipping ratios). The extension .c5 is not one of
these, hence the .c5 file
extensions above were typed explicitly.  Exactly which defaults have been
used is under the control of the writer of the main program, but
routine \stlink{n}{NOPFIL}\ should in general allow the user to try again if he types
a file name it cannot handle.\p
\section{The \emph{Listing}\ File}\markright{The \emph{Listing}\ File}
Every job will open one file which is intended to be read by a human.
When using the ILL CCSL options this file is given the name of the 
main program and extension .lis. On a UNIX system there may
be an extra line at the start of the program dialogue saying:\\
\exac{Existing file arrnge.lis will be overwritten OK? (Y/N) :}\\
This will occur if \exac{arrnge} has been run before in the same directory. 
It reminds the user that UNIX will overwrite the listing file from 
the previous run and that, if this is not what is wanted, he should exit 
from the program and save the listing with a different name. 
With the RAL options, the user will 
be asked to name the listing file, and if he responds by typing RETURN the file-name 
will default as above.  So at RAL, there will be an extra line at the start of
the dialogue saying :\\
\exac{Give name for Output  }
\p
The listing file, which is on logical unit LPT,
is opened by a call of OPNFIL, from SUBROUTINE \stlink{i}{INITIL}, usually from
the call of \stlink{p}{PREFIN}\ which occurs near the start of most CCSL programs.  If
the user wishes to open a specific unit number, say 8, he must set
\exac{LPT=8} before the \exac{CALL~PREFIN}.
\pn
If he does not do this OPNFIL (via NOPFIL) will choose its own unit number.
%
\section{Running New or Changed Programs}
\markright{Running New or Changed Programs}
If a new main program is written, or if new versions of CCSL routines
are made, the progam must be recompiled and relinked.
Suppose a new program prog.f has been made which calls the new or modified
subroutines su=b1 and sub2
For LINUX/UNIX a sample  Makefile could be :\pn
\begin{verbatim} 
FC=gfortran -Wall -m32
FFLAGS= -g -Wall -m32
CCLIB=$(CCSLIB)
PREQ=sub1.o sub2.o 
prog: prog.o  $(PREQ) $(CCLIB)
	$(FC)  prog.o  $(PREQ)  $(CCLIB) -o prog
\end{verbatim}
where CCLIB is an environment variable set to the file path to the  CCSL Object Library 
(libmk4.a) file.
\ssk
 Graphics Libraries may also need to be included in the link step.
\pn 
% 
\section{Use of Scratch COMMON}\markright{Use of Scratch COMMON}
Two areas of labelled scratch COMMON are used in CCSL routines,
and can with certain restrictions be made use of by user programs;
they are called /SCRAT/ and /SCRACH/.\p
COMMON /SCRAT/ has a maximum length
within CCSL of 10201 items (integers or real numbers).\p
COMMON /SCRACH/ is a character buffer to
contain at least 80 characters, and usually 200.\p
It is used extensively by the card reading routines.
The CHARACTER *80 item ICARD in COMMMON /SCRACH/ is used as
the buffer in which the routine \stlink{c}{CARDIN}\ places a character string from the
Crystal Data, and where all the CCSL free format subroutines expect to find it. 
It is this character
string which is interpreted by the various INPUTx routines, and which
could be interpreted by user programs. The subroutine \stlink{a}{ASK}\ which reads a line of
input from the terminal puts it into this common area.\p 
The character string NAMFIL, normally of length 100, is used to store complete
file paths. If these are very long the symbolic variable name FNAM may need
to be made greater than 100.
In general it must not be assumed that either of the scratch common
areas will be unchanged by CCSL routines;  the inexperienced user should 
avoid them.
\section{Main Programs Currently Available}
\markright{Main Programs Currently Available}
A list follows of 
\iftexforht{\href{../appenx/maiapx.html}{CCSL main programs}}
{CCSL main programs}%
\ currently used at ILL, and available in the 
Master File. They have been classified into the same
\htmlref{groups}{chap2:groups} as the subroutines
in chapter 2. The existing programs reflect, rather strongly, the interests of 
the present
author. Many of them have been written specially to deal with the reduction
and evaluation of data obtained using the single crystal diffractometers at ILL.
\iftexforht{href{../appenx/appendix.html\#appd}{Appendix D}}{Appendix D}
  describes these programs and 
how to use them in more detail. 
\htmlonly{An% 
\hyperref{../appenx/maialfa.html}{alphabetically ordered index}
is also given.}\\[1ex]
\latexonly{\begin{center}{\bbold Index of Main Programs}\\
The page numbers given refer to AppendixD\\[1ex]\end{center}
\input {maiclasslist}%
}%
\ms 
More main programs are being added all the time.  Although some of these
are for applications specific to their authors, they may well help the
new user tackle some related problem.\p
The user should should be able to obtain the totality of CCSL code
(Library routines and main programs) in the form of
FORTRAN source files. He may find it useful to work with printouts of these files, but should
be cautious about printing out the whole Library, which is long.\\[1cm] 
\subsection{Test Data}
For most of the MAIN programs in general use, data are available against which
a new compilation may be tested. They  may be downloaded individually,  or as 
a single compressed archive testdata.tar.gz
from the \href{http://forge.epn-campus.eu/projects/sxtalsoft/repository/show/CCSL/testdata}
{epn-campus repository}
.\\ The files are arranged in
sub-directories with the same name as that of the programs to which they 
refer. For  each set of test data (eg \exac{xxx}) There is normally a crystal
data file  \exac{xxx.cry} and perhaps an input data file \exac{xxx.ext} (where
\exac{ext} may be a default extension expected by the program). There will
always be a listing file \exac{pname\_xxx.lis},  where \exac{pname} is the
program name. There may be other output files named \exac{xxx.yyy} and
identifable from the extension \exac{yyy}. For example the sflsq (structure
factor least squares directory contains the files for structure factor
refinement of LiCoPO4
\begin{description}
\item      [\exac{lcp.cry} ] the crystal data file 
\item      [\exac{lcp.sfe}]  the observations file containing the extinction
                    constants (calculated by EXTCAL)
\item      [ \exac{sflsq\_lcp.lis}]        the output listing
\item      [\exac{lcp.new} ]  a crystal data file containing the 
 refined parameters
For each set of data there is a file \exac{\_xxx} (or simply \exac{r} if
there is only one test data set for the program) The program can be
tested by running it so that it takes its input from the \_xxx file eg
with the command \exac{pname < \_xxx}.
\p
There is an executable  perl script "CCSLtests" in the CCS\_PERL
directory which will run all the tests specified in the file tes\_list
and report on the results. The \exac{\_xxx} response files for graphics
programs may need to be modified to reflect local options. \end{description}

\section{Conditions of Use}%
\subsection{Disclaimer}%
This software is distributed in the hope that it will be useful
but \emph{without any warranty}. The author(s) do not accept responsibility 
to anyone for the consequences of using it or for whether it serves 
any particular purpose or works at all.  
\\ 
\subsection{Copying}%
Use and copying of this software and the preparation of derivative
works based on it are permitted, so long as the following
conditions are met:
\begin{itemize}
\item 
The copyright notice and this entire notice are included intact
and prominently carried on all copies and supporting documentation.
\item 
No fees or compensation are charged for use, copies, or
access to this software. You may charge a nominal
distribution fee for the physical act of transferring a
copy, but you may not charge for the program itself. 
\item 
If you modify this software, you must cause the modified
file(s) to carry notices describing the changes, who made 
the changes, and the date
of those changes.
\end{itemize} 
\subsection{Authors}
The CCSL library was written by J.C. Matthewman and P.J. Brown with
contributions from W.I.F. David, J.B. Forsyth J.H. Matthewman, and J.P. Wright.\\[2ex] %
Copyright \copyright 1999--2001. All rights reserved.
\\[3ex]
\latexonly{\begin{tabular}{lllll}
\qquad&P.J. Brown  &\qquad&Institut Laue Langevin&\qquad e-mail: brownj(At)ill.fr\\
&\today &&38042 Grenoble Cedex\\
&&&{France.}\\
\end{tabular}
}%\end{latexonly}                                 
\finchapter
%S\end{document}


%\end{htmlonly}
%\internal{c2}
%\internal{c3}
%\internal{c4}
%\internal{c5}
%\internal{c6}
%\internal{c1}
\startdocument
\label{chap:7}
\markboth{Using the System}{}
We assume first that the user has access to a computer or network
on which CCSL has already been compiled and made into a Library.\p
Any discussion of the details of running jobs must naturally refer to
specific items such as file names, which differ on different computers.
Many current users of CCSL use  LINUX or UNIX
and the examples in general refer to them. Users of other operating
systems such as WINDOWS or DOS should take them only as a rough guide. 
\section{Running \emph{Black Box} Jobs}\markright{Running \emph{Black Box} Jobs}
A standard main program such as \mlink{arrnge}{ARRNGE}\ (which groups together the
results of single crystal diffraction measurements) will have been
precompiled and linked using CCSL as a search Library.  An executable file
such as ARRNGE.EXE should therefore exist.
The user wishing to run this as it stands, as a \ital{black box}, first puts
his Crystal Data into some file, named, for example, TESTAR.CRY.
\pn
He would also have files containing data to arrange, called, 
say, \mbox{refln2.c5}, \mbox{refln1.c5}
and \mbox{refln3.c5}, and a list of the numbers of reflections to 
reject in, say, \mbox{badref.rej}.\p
The following sequence would run the program.
The dialogue in italics is to be typed by the user.\p
\begin{list} {} {\setlength{\labelwidth}{25mm}
  \setlength{\parsep}{-1ex}
  \setlength{\leftmargin}{5mm}}
\item\texttt{2\% arrnge}
\item\exac{Give name of Crystal data file  }\ital{TESTAR}\\
\item\exac{Give name of file containing rejection list   }\ital{badref}\\
\item\exac{Give name of reflection file   }\ital{refln1.c5}\\
\item\exac{Give name of next reflection file (or c/r to end)  }\ital{refln2.c5}\\
\item\exac{Give name of next reflection file (or c/r to end)  }\ital{refln3.c5}\\
\item\exac{Give name of next reflection file (or c/r to end)}\\
\ms 
\item\exac{Sort started   . . .   Sorted 1485 records}\\ 
\ms 
\item\exac{Give name for output file of sorted reflections   }\ital{arrout}\ssp
 \exac{3\%}
\end{list} 
 
In general, the user supplies file names on request.  The file extension
for Crystal Data files defaults to .cry or .ccl . General data 
files have default extension .dat although some default extensions are
associated with particular data types. (For example, for \mbox{arrnge,} .fli is
the default for DTYP=1, meaning
polarised neutron flipping ratios). The extension .c5 is not one of
these, hence the .c5 file
extensions above were typed explicitly.  Exactly which defaults have been
used is under the control of the writer of the main program, but
routine \stlink{n}{NOPFIL}\ should in general allow the user to try again if he types
a file name it cannot handle.\p
\section{The \emph{Listing}\ File}\markright{The \emph{Listing}\ File}
Every job will open one file which is intended to be read by a human.
When using the ILL CCSL options this file is given the name of the 
main program and extension .lis. On a UNIX system there may
be an extra line at the start of the program dialogue saying:\\
\exac{Existing file arrnge.lis will be overwritten OK? (Y/N) :}\\
This will occur if \exac{arrnge} has been run before in the same directory. 
It reminds the user that UNIX will overwrite the listing file from 
the previous run and that, if this is not what is wanted, he should exit 
from the program and save the listing with a different name. 
With the RAL options, the user will 
be asked to name the listing file, and if he responds by typing RETURN the file-name 
will default as above.  So at RAL, there will be an extra line at the start of
the dialogue saying :\\
\exac{Give name for Output  }
\p
The listing file, which is on logical unit LPT,
is opened by a call of OPNFIL, from SUBROUTINE \stlink{i}{INITIL}, usually from
the call of \stlink{p}{PREFIN}\ which occurs near the start of most CCSL programs.  If
the user wishes to open a specific unit number, say 8, he must set
\exac{LPT=8} before the \exac{CALL~PREFIN}.
\pn
If he does not do this OPNFIL (via NOPFIL) will choose its own unit number.
%
\section{Running New or Changed Programs}
\markright{Running New or Changed Programs}
If a new main program is written, or if new versions of CCSL routines
are made, the progam must be recompiled and relinked.
Suppose a new program prog.f has been made which calls the new or modified
subroutines su=b1 and sub2
For LINUX/UNIX a sample  Makefile could be :\pn
\begin{verbatim} 
FC=gfortran -Wall -m32
FFLAGS= -g -Wall -m32
CCLIB=$(CCSLIB)
PREQ=sub1.o sub2.o 
prog: prog.o  $(PREQ) $(CCLIB)
	$(FC)  prog.o  $(PREQ)  $(CCLIB) -o prog
\end{verbatim}
where CCLIB is an environment variable set to the file path to the  CCSL Object Library 
(libmk4.a) file.
\ssk
 Graphics Libraries may also need to be included in the link step.
\pn 
% 
\section{Use of Scratch COMMON}\markright{Use of Scratch COMMON}
Two areas of labelled scratch COMMON are used in CCSL routines,
and can with certain restrictions be made use of by user programs;
they are called /SCRAT/ and /SCRACH/.\p
COMMON /SCRAT/ has a maximum length
within CCSL of 10201 items (integers or real numbers).\p
COMMON /SCRACH/ is a character buffer to
contain at least 80 characters, and usually 200.\p
It is used extensively by the card reading routines.
The CHARACTER *80 item ICARD in COMMMON /SCRACH/ is used as
the buffer in which the routine \stlink{c}{CARDIN}\ places a character string from the
Crystal Data, and where all the CCSL free format subroutines expect to find it. 
It is this character
string which is interpreted by the various INPUTx routines, and which
could be interpreted by user programs. The subroutine \stlink{a}{ASK}\ which reads a line of
input from the terminal puts it into this common area.\p 
The character string NAMFIL, normally of length 100, is used to store complete
file paths. If these are very long the symbolic variable name FNAM may need
to be made greater than 100.
In general it must not be assumed that either of the scratch common
areas will be unchanged by CCSL routines;  the inexperienced user should 
avoid them.
\section{Main Programs Currently Available}
\markright{Main Programs Currently Available}
A list follows of 
\iftexforht{\href{../appenx/maiapx.html}{CCSL main programs}}
{CCSL main programs}%
\ currently used at ILL, and available in the 
Master File. They have been classified into the same
\htmlref{groups}{chap2:groups} as the subroutines
in chapter 2. The existing programs reflect, rather strongly, the interests of 
the present
author. Many of them have been written specially to deal with the reduction
and evaluation of data obtained using the single crystal diffractometers at ILL.
\iftexforht{href{../appenx/appendix.html\#appd}{Appendix D}}{Appendix D}
  describes these programs and 
how to use them in more detail. 
\htmlonly{An% 
\hyperref{../appenx/maialfa.html}{alphabetically ordered index}
is also given.}\\[1ex]
\latexonly{\begin{center}{\bbold Index of Main Programs}\\
The page numbers given refer to AppendixD\\[1ex]\end{center}
\input {maiclasslist}%
}%
\ms 
More main programs are being added all the time.  Although some of these
are for applications specific to their authors, they may well help the
new user tackle some related problem.\p
The user should should be able to obtain the totality of CCSL code
(Library routines and main programs) in the form of
FORTRAN source files. He may find it useful to work with printouts of these files, but should
be cautious about printing out the whole Library, which is long.\\[1cm] 
\subsection{Test Data}
For most of the MAIN programs in general use, data are available against which
a new compilation may be tested. They  may be downloaded individually,  or as 
a single compressed archive testdata.tar.gz
from the \href{http://forge.epn-campus.eu/projects/sxtalsoft/repository/show/CCSL/testdata}
{epn-campus repository}
.\\ The files are arranged in
sub-directories with the same name as that of the programs to which they 
refer. For  each set of test data (eg \exac{xxx}) There is normally a crystal
data file  \exac{xxx.cry} and perhaps an input data file \exac{xxx.ext} (where
\exac{ext} may be a default extension expected by the program). There will
always be a listing file \exac{pname\_xxx.lis},  where \exac{pname} is the
program name. There may be other output files named \exac{xxx.yyy} and
identifable from the extension \exac{yyy}. For example the sflsq (structure
factor least squares directory contains the files for structure factor
refinement of LiCoPO4
\begin{description}
\item      [\exac{lcp.cry} ] the crystal data file 
\item      [\exac{lcp.sfe}]  the observations file containing the extinction
                    constants (calculated by EXTCAL)
\item      [ \exac{sflsq\_lcp.lis}]        the output listing
\item      [\exac{lcp.new} ]  a crystal data file containing the 
 refined parameters
For each set of data there is a file \exac{\_xxx} (or simply \exac{r} if
there is only one test data set for the program) The program can be
tested by running it so that it takes its input from the \_xxx file eg
with the command \exac{pname < \_xxx}.
\p
There is an executable  perl script "CCSLtests" in the CCS\_PERL
directory which will run all the tests specified in the file tes\_list
and report on the results. The \exac{\_xxx} response files for graphics
programs may need to be modified to reflect local options. \end{description}

\section{Conditions of Use}%
\subsection{Disclaimer}%
This software is distributed in the hope that it will be useful
but \emph{without any warranty}. The author(s) do not accept responsibility 
to anyone for the consequences of using it or for whether it serves 
any particular purpose or works at all.  
\\ 
\subsection{Copying}%
Use and copying of this software and the preparation of derivative
works based on it are permitted, so long as the following
conditions are met:
\begin{itemize}
\item 
The copyright notice and this entire notice are included intact
and prominently carried on all copies and supporting documentation.
\item 
No fees or compensation are charged for use, copies, or
access to this software. You may charge a nominal
distribution fee for the physical act of transferring a
copy, but you may not charge for the program itself. 
\item 
If you modify this software, you must cause the modified
file(s) to carry notices describing the changes, who made 
the changes, and the date
of those changes.
\end{itemize} 
\subsection{Authors}
The CCSL library was written by J.C. Matthewman and P.J. Brown with
contributions from W.I.F. David, J.B. Forsyth J.H. Matthewman, and J.P. Wright.\\[2ex] %
Copyright \copyright 1999--2001. All rights reserved.
\\[3ex]
\latexonly{\begin{tabular}{lllll}
\qquad&P.J. Brown  &\qquad&Institut Laue Langevin&\qquad e-mail: brownj(At)ill.fr\\
&\today &&38042 Grenoble Cedex\\
&&&{France.}\\
\end{tabular}
}%\end{latexonly}                                 
\finchapter
%S\end{document}


%\end{htmlonly}
%\internal{c2}
%\internal{c3}
%\internal{c4}
%\internal{c5}
%\internal{c6}
%\internal{c1}
\startdocument
\label{chap:7}
\markboth{Using the System}{}
We assume first that the user has access to a computer or network
on which CCSL has already been compiled and made into a Library.\p
Any discussion of the details of running jobs must naturally refer to
specific items such as file names, which differ on different computers.
Many current users of CCSL use  LINUX or UNIX
and the examples in general refer to them. Users of other operating
systems such as WINDOWS or DOS should take them only as a rough guide. 
\section{Running \emph{Black Box} Jobs}\markright{Running \emph{Black Box} Jobs}
A standard main program such as \mlink{arrnge}{ARRNGE}\ (which groups together the
results of single crystal diffraction measurements) will have been
precompiled and linked using CCSL as a search Library.  An executable file
such as ARRNGE.EXE should therefore exist.
The user wishing to run this as it stands, as a \ital{black box}, first puts
his Crystal Data into some file, named, for example, TESTAR.CRY.
\pn
He would also have files containing data to arrange, called, 
say, \mbox{refln2.c5}, \mbox{refln1.c5}
and \mbox{refln3.c5}, and a list of the numbers of reflections to 
reject in, say, \mbox{badref.rej}.\p
The following sequence would run the program.
The dialogue in italics is to be typed by the user.\p
\begin{list} {} {\setlength{\labelwidth}{25mm}
  \setlength{\parsep}{-1ex}
  \setlength{\leftmargin}{5mm}}
\item\texttt{2\% arrnge}
\item\exac{Give name of Crystal data file  }\ital{TESTAR}\\
\item\exac{Give name of file containing rejection list   }\ital{badref}\\
\item\exac{Give name of reflection file   }\ital{refln1.c5}\\
\item\exac{Give name of next reflection file (or c/r to end)  }\ital{refln2.c5}\\
\item\exac{Give name of next reflection file (or c/r to end)  }\ital{refln3.c5}\\
\item\exac{Give name of next reflection file (or c/r to end)}\\
\ms 
\item\exac{Sort started   . . .   Sorted 1485 records}\\ 
\ms 
\item\exac{Give name for output file of sorted reflections   }\ital{arrout}\ssp
 \exac{3\%}
\end{list} 
 
In general, the user supplies file names on request.  The file extension
for Crystal Data files defaults to .cry or .ccl . General data 
files have default extension .dat although some default extensions are
associated with particular data types. (For example, for \mbox{arrnge,} .fli is
the default for DTYP=1, meaning
polarised neutron flipping ratios). The extension .c5 is not one of
these, hence the .c5 file
extensions above were typed explicitly.  Exactly which defaults have been
used is under the control of the writer of the main program, but
routine \stlink{n}{NOPFIL}\ should in general allow the user to try again if he types
a file name it cannot handle.\p
\section{The \emph{Listing}\ File}\markright{The \emph{Listing}\ File}
Every job will open one file which is intended to be read by a human.
When using the ILL CCSL options this file is given the name of the 
main program and extension .lis. On a UNIX system there may
be an extra line at the start of the program dialogue saying:\\
\exac{Existing file arrnge.lis will be overwritten OK? (Y/N) :}\\
This will occur if \exac{arrnge} has been run before in the same directory. 
It reminds the user that UNIX will overwrite the listing file from 
the previous run and that, if this is not what is wanted, he should exit 
from the program and save the listing with a different name. 
With the RAL options, the user will 
be asked to name the listing file, and if he responds by typing RETURN the file-name 
will default as above.  So at RAL, there will be an extra line at the start of
the dialogue saying :\\
\exac{Give name for Output  }
\p
The listing file, which is on logical unit LPT,
is opened by a call of OPNFIL, from SUBROUTINE \stlink{i}{INITIL}, usually from
the call of \stlink{p}{PREFIN}\ which occurs near the start of most CCSL programs.  If
the user wishes to open a specific unit number, say 8, he must set
\exac{LPT=8} before the \exac{CALL~PREFIN}.
\pn
If he does not do this OPNFIL (via NOPFIL) will choose its own unit number.
%
\section{Running New or Changed Programs}
\markright{Running New or Changed Programs}
If a new main program is written, or if new versions of CCSL routines
are made, the progam must be recompiled and relinked.
Suppose a new program prog.f has been made which calls the new or modified
subroutines su=b1 and sub2
For LINUX/UNIX a sample  Makefile could be :\pn
\begin{verbatim} 
FC=gfortran -Wall -m32
FFLAGS= -g -Wall -m32
CCLIB=$(CCSLIB)
PREQ=sub1.o sub2.o 
prog: prog.o  $(PREQ) $(CCLIB)
	$(FC)  prog.o  $(PREQ)  $(CCLIB) -o prog
\end{verbatim}
where CCLIB is an environment variable set to the file path to the  CCSL Object Library 
(libmk4.a) file.
\ssk
 Graphics Libraries may also need to be included in the link step.
\pn 
% 
\section{Use of Scratch COMMON}\markright{Use of Scratch COMMON}
Two areas of labelled scratch COMMON are used in CCSL routines,
and can with certain restrictions be made use of by user programs;
they are called /SCRAT/ and /SCRACH/.\p
COMMON /SCRAT/ has a maximum length
within CCSL of 10201 items (integers or real numbers).\p
COMMON /SCRACH/ is a character buffer to
contain at least 80 characters, and usually 200.\p
It is used extensively by the card reading routines.
The CHARACTER *80 item ICARD in COMMMON /SCRACH/ is used as
the buffer in which the routine \stlink{c}{CARDIN}\ places a character string from the
Crystal Data, and where all the CCSL free format subroutines expect to find it. 
It is this character
string which is interpreted by the various INPUTx routines, and which
could be interpreted by user programs. The subroutine \stlink{a}{ASK}\ which reads a line of
input from the terminal puts it into this common area.\p 
The character string NAMFIL, normally of length 100, is used to store complete
file paths. If these are very long the symbolic variable name FNAM may need
to be made greater than 100.
In general it must not be assumed that either of the scratch common
areas will be unchanged by CCSL routines;  the inexperienced user should 
avoid them.
\section{Main Programs Currently Available}
\markright{Main Programs Currently Available}
A list follows of 
\iftexforht{\href{../appenx/maiapx.html}{CCSL main programs}}
{CCSL main programs}%
\ currently used at ILL, and available in the 
Master File. They have been classified into the same
\htmlref{groups}{chap2:groups} as the subroutines
in chapter 2. The existing programs reflect, rather strongly, the interests of 
the present
author. Many of them have been written specially to deal with the reduction
and evaluation of data obtained using the single crystal diffractometers at ILL.
\iftexforht{href{../appenx/appendix.html\#appd}{Appendix D}}{Appendix D}
  describes these programs and 
how to use them in more detail. 
\htmlonly{An% 
\hyperref{../appenx/maialfa.html}{alphabetically ordered index}
is also given.}\\[1ex]
\latexonly{\begin{center}{\bbold Index of Main Programs}\\
The page numbers given refer to AppendixD\\[1ex]\end{center}
\input {maiclasslist}%
}%
\ms 
More main programs are being added all the time.  Although some of these
are for applications specific to their authors, they may well help the
new user tackle some related problem.\p
The user should should be able to obtain the totality of CCSL code
(Library routines and main programs) in the form of
FORTRAN source files. He may find it useful to work with printouts of these files, but should
be cautious about printing out the whole Library, which is long.\\[1cm] 
\subsection{Test Data}
For most of the MAIN programs in general use, data are available against which
a new compilation may be tested. They  may be downloaded individually,  or as 
a single compressed archive testdata.tar.gz
from the \href{http://forge.epn-campus.eu/projects/sxtalsoft/repository/show/CCSL/testdata}
{epn-campus repository}
.\\ The files are arranged in
sub-directories with the same name as that of the programs to which they 
refer. For  each set of test data (eg \exac{xxx}) There is normally a crystal
data file  \exac{xxx.cry} and perhaps an input data file \exac{xxx.ext} (where
\exac{ext} may be a default extension expected by the program). There will
always be a listing file \exac{pname\_xxx.lis},  where \exac{pname} is the
program name. There may be other output files named \exac{xxx.yyy} and
identifable from the extension \exac{yyy}. For example the sflsq (structure
factor least squares directory contains the files for structure factor
refinement of LiCoPO4
\begin{description}
\item      [\exac{lcp.cry} ] the crystal data file 
\item      [\exac{lcp.sfe}]  the observations file containing the extinction
                    constants (calculated by EXTCAL)
\item      [ \exac{sflsq\_lcp.lis}]        the output listing
\item      [\exac{lcp.new} ]  a crystal data file containing the 
 refined parameters
For each set of data there is a file \exac{\_xxx} (or simply \exac{r} if
there is only one test data set for the program) The program can be
tested by running it so that it takes its input from the \_xxx file eg
with the command \exac{pname < \_xxx}.
\p
There is an executable  perl script "CCSLtests" in the CCS\_PERL
directory which will run all the tests specified in the file tes\_list
and report on the results. The \exac{\_xxx} response files for graphics
programs may need to be modified to reflect local options. \end{description}

\section{Conditions of Use}%
\subsection{Disclaimer}%
This software is distributed in the hope that it will be useful
but \emph{without any warranty}. The author(s) do not accept responsibility 
to anyone for the consequences of using it or for whether it serves 
any particular purpose or works at all.  
\\ 
\subsection{Copying}%
Use and copying of this software and the preparation of derivative
works based on it are permitted, so long as the following
conditions are met:
\begin{itemize}
\item 
The copyright notice and this entire notice are included intact
and prominently carried on all copies and supporting documentation.
\item 
No fees or compensation are charged for use, copies, or
access to this software. You may charge a nominal
distribution fee for the physical act of transferring a
copy, but you may not charge for the program itself. 
\item 
If you modify this software, you must cause the modified
file(s) to carry notices describing the changes, who made 
the changes, and the date
of those changes.
\end{itemize} 
\subsection{Authors}
The CCSL library was written by J.C. Matthewman and P.J. Brown with
contributions from W.I.F. David, J.B. Forsyth J.H. Matthewman, and J.P. Wright.\\[2ex] %
Copyright \copyright 1999--2001. All rights reserved.
\\[3ex]
\latexonly{\begin{tabular}{lllll}
\qquad&P.J. Brown  &\qquad&Institut Laue Langevin&\qquad e-mail: brownj(At)ill.fr\\
&\today &&38042 Grenoble Cedex\\
&&&{France.}\\
\end{tabular}
}%\end{latexonly}                                 
\finchapter
%S\end{document}


%\end{htmlonly}
%\internal{c2}
%\internal{c3}
%\internal{c4}
%\internal{c5}
%\internal{c6}
%\internal{c1}
\startdocument
\label{chap:7}
\markboth{Using the System}{}
We assume first that the user has access to a computer or network
on which CCSL has already been compiled and made into a Library.\p
Any discussion of the details of running jobs must naturally refer to
specific items such as file names, which differ on different computers.
Many current users of CCSL use  LINUX or UNIX
and the examples in general refer to them. Users of other operating
systems such as WINDOWS or DOS should take them only as a rough guide. 
\section{Running \emph{Black Box} Jobs}\markright{Running \emph{Black Box} Jobs}
A standard main program such as \mlink{arrnge}{ARRNGE}\ (which groups together the
results of single crystal diffraction measurements) will have been
precompiled and linked using CCSL as a search Library.  An executable file
such as ARRNGE.EXE should therefore exist.
The user wishing to run this as it stands, as a \ital{black box}, first puts
his Crystal Data into some file, named, for example, TESTAR.CRY.
\pn
He would also have files containing data to arrange, called, 
say, \mbox{refln2.c5}, \mbox{refln1.c5}
and \mbox{refln3.c5}, and a list of the numbers of reflections to 
reject in, say, \mbox{badref.rej}.\p
The following sequence would run the program.
The dialogue in italics is to be typed by the user.\p
\begin{list} {} {\setlength{\labelwidth}{25mm}
  \setlength{\parsep}{-1ex}
  \setlength{\leftmargin}{5mm}}
\item\texttt{2\% arrnge}
\item\exac{Give name of Crystal data file  }\ital{TESTAR}\\
\item\exac{Give name of file containing rejection list   }\ital{badref}\\
\item\exac{Give name of reflection file   }\ital{refln1.c5}\\
\item\exac{Give name of next reflection file (or c/r to end)  }\ital{refln2.c5}\\
\item\exac{Give name of next reflection file (or c/r to end)  }\ital{refln3.c5}\\
\item\exac{Give name of next reflection file (or c/r to end)}\\
\ms 
\item\exac{Sort started   . . .   Sorted 1485 records}\\ 
\ms 
\item\exac{Give name for output file of sorted reflections   }\ital{arrout}\ssp
 \exac{3\%}
\end{list} 
 
In general, the user supplies file names on request.  The file extension
for Crystal Data files defaults to .cry or .ccl . General data 
files have default extension .dat although some default extensions are
associated with particular data types. (For example, for \mbox{arrnge,} .fli is
the default for DTYP=1, meaning
polarised neutron flipping ratios). The extension .c5 is not one of
these, hence the .c5 file
extensions above were typed explicitly.  Exactly which defaults have been
used is under the control of the writer of the main program, but
routine \stlink{n}{NOPFIL}\ should in general allow the user to try again if he types
a file name it cannot handle.\p
\section{The \emph{Listing}\ File}\markright{The \emph{Listing}\ File}
Every job will open one file which is intended to be read by a human.
When using the ILL CCSL options this file is given the name of the 
main program and extension .lis. On a UNIX system there may
be an extra line at the start of the program dialogue saying:\\
\exac{Existing file arrnge.lis will be overwritten OK? (Y/N) :}\\
This will occur if \exac{arrnge} has been run before in the same directory. 
It reminds the user that UNIX will overwrite the listing file from 
the previous run and that, if this is not what is wanted, he should exit 
from the program and save the listing with a different name. 
With the RAL options, the user will 
be asked to name the listing file, and if he responds by typing RETURN the file-name 
will default as above.  So at RAL, there will be an extra line at the start of
the dialogue saying :\\
\exac{Give name for Output  }
\p
The listing file, which is on logical unit LPT,
is opened by a call of OPNFIL, from SUBROUTINE \stlink{i}{INITIL}, usually from
the call of \stlink{p}{PREFIN}\ which occurs near the start of most CCSL programs.  If
the user wishes to open a specific unit number, say 8, he must set
\exac{LPT=8} before the \exac{CALL~PREFIN}.
\pn
If he does not do this OPNFIL (via NOPFIL) will choose its own unit number.
%
\section{Running New or Changed Programs}
\markright{Running New or Changed Programs}
If a new main program is written, or if new versions of CCSL routines
are made, the progam must be recompiled and relinked.
Suppose a new program prog.f has been made which calls the new or modified
subroutines su=b1 and sub2
For LINUX/UNIX a sample  Makefile could be :\pn
\begin{verbatim} 
FC=gfortran -Wall -m32
FFLAGS= -g -Wall -m32
CCLIB=$(CCSLIB)
PREQ=sub1.o sub2.o 
prog: prog.o  $(PREQ) $(CCLIB)
	$(FC)  prog.o  $(PREQ)  $(CCLIB) -o prog
\end{verbatim}
where CCLIB is an environment variable set to the file path to the  CCSL Object Library 
(libmk4.a) file.
\ssk
 Graphics Libraries may also need to be included in the link step.
\pn 
% 
\section{Use of Scratch COMMON}\markright{Use of Scratch COMMON}
Two areas of labelled scratch COMMON are used in CCSL routines,
and can with certain restrictions be made use of by user programs;
they are called /SCRAT/ and /SCRACH/.\p
COMMON /SCRAT/ has a maximum length
within CCSL of 10201 items (integers or real numbers).\p
COMMON /SCRACH/ is a character buffer to
contain at least 80 characters, and usually 200.\p
It is used extensively by the card reading routines.
The CHARACTER *80 item ICARD in COMMMON /SCRACH/ is used as
the buffer in which the routine \stlink{c}{CARDIN}\ places a character string from the
Crystal Data, and where all the CCSL free format subroutines expect to find it. 
It is this character
string which is interpreted by the various INPUTx routines, and which
could be interpreted by user programs. The subroutine \stlink{a}{ASK}\ which reads a line of
input from the terminal puts it into this common area.\p 
The character string NAMFIL, normally of length 100, is used to store complete
file paths. If these are very long the symbolic variable name FNAM may need
to be made greater than 100.
In general it must not be assumed that either of the scratch common
areas will be unchanged by CCSL routines;  the inexperienced user should 
avoid them.
\section{Main Programs Currently Available}
\markright{Main Programs Currently Available}
A list follows of 
\iftexforht{\href{../appenx/maiapx.html}{CCSL main programs}}
{CCSL main programs}%
\ currently used at ILL, and available in the 
Master File. They have been classified into the same
\htmlref{groups}{chap2:groups} as the subroutines
in chapter 2. The existing programs reflect, rather strongly, the interests of 
the present
author. Many of them have been written specially to deal with the reduction
and evaluation of data obtained using the single crystal diffractometers at ILL.
\iftexforht{href{../appenx/appendix.html\#appd}{Appendix D}}{Appendix D}
  describes these programs and 
how to use them in more detail. 
\htmlonly{An% 
\hyperref{../appenx/maialfa.html}{alphabetically ordered index}
is also given.}\\[1ex]
\latexonly{\begin{center}{\bbold Index of Main Programs}\\
The page numbers given refer to AppendixD\\[1ex]\end{center}
\input {maiclasslist}%
}%
\ms 
More main programs are being added all the time.  Although some of these
are for applications specific to their authors, they may well help the
new user tackle some related problem.\p
The user should should be able to obtain the totality of CCSL code
(Library routines and main programs) in the form of
FORTRAN source files. He may find it useful to work with printouts of these files, but should
be cautious about printing out the whole Library, which is long.\\[1cm] 
\subsection{Test Data}
For most of the MAIN programs in general use, data are available against which
a new compilation may be tested. They  may be downloaded individually,  or as 
a single compressed archive testdata.tar.gz
from the \href{http://forge.epn-campus.eu/projects/sxtalsoft/repository/show/CCSL/testdata}
{epn-campus repository}
.\\ The files are arranged in
sub-directories with the same name as that of the programs to which they 
refer. For  each set of test data (eg \exac{xxx}) There is normally a crystal
data file  \exac{xxx.cry} and perhaps an input data file \exac{xxx.ext} (where
\exac{ext} may be a default extension expected by the program). There will
always be a listing file \exac{pname\_xxx.lis},  where \exac{pname} is the
program name. There may be other output files named \exac{xxx.yyy} and
identifable from the extension \exac{yyy}. For example the sflsq (structure
factor least squares directory contains the files for structure factor
refinement of LiCoPO4
\begin{description}
\item      [\exac{lcp.cry} ] the crystal data file 
\item      [\exac{lcp.sfe}]  the observations file containing the extinction
                    constants (calculated by EXTCAL)
\item      [ \exac{sflsq\_lcp.lis}]        the output listing
\item      [\exac{lcp.new} ]  a crystal data file containing the 
 refined parameters
For each set of data there is a file \exac{\_xxx} (or simply \exac{r} if
there is only one test data set for the program) The program can be
tested by running it so that it takes its input from the \_xxx file eg
with the command \exac{pname < \_xxx}.
\p
There is an executable  perl script "CCSLtests" in the CCS\_PERL
directory which will run all the tests specified in the file tes\_list
and report on the results. The \exac{\_xxx} response files for graphics
programs may need to be modified to reflect local options. \end{description}

\section{Conditions of Use}%
\subsection{Disclaimer}%
This software is distributed in the hope that it will be useful
but \emph{without any warranty}. The author(s) do not accept responsibility 
to anyone for the consequences of using it or for whether it serves 
any particular purpose or works at all.  
\\ 
\subsection{Copying}%
Use and copying of this software and the preparation of derivative
works based on it are permitted, so long as the following
conditions are met:
\begin{itemize}
\item 
The copyright notice and this entire notice are included intact
and prominently carried on all copies and supporting documentation.
\item 
No fees or compensation are charged for use, copies, or
access to this software. You may charge a nominal
distribution fee for the physical act of transferring a
copy, but you may not charge for the program itself. 
\item 
If you modify this software, you must cause the modified
file(s) to carry notices describing the changes, who made 
the changes, and the date
of those changes.
\end{itemize} 
\subsection{Authors}
The CCSL library was written by J.C. Matthewman and P.J. Brown with
contributions from W.I.F. David, J.B. Forsyth J.H. Matthewman, and J.P. Wright.\\[2ex] %
Copyright \copyright 1999--2001. All rights reserved.
\\[3ex]
\latexonly{\begin{tabular}{lllll}
\qquad&P.J. Brown  &\qquad&Institut Laue Langevin&\qquad e-mail: brownj(At)ill.fr\\
&\today &&38042 Grenoble Cedex\\
&&&{France.}\\
\end{tabular}
}%\end{latexonly}                                 
\finchapter
%S\end{document}

