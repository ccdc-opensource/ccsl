%\setcounter{page}{0}
%\setcounter{chapter}{0}
%\chapter{INTRODUCTION TO CCSL}
%\begin{htmlonly}
%\usepackage{CCSLman,html,makeidx,color,ifthen,bm}
%%\setcounter{page}{0}
%\setcounter{chapter}{0}
%\chapter{INTRODUCTION TO CCSL}
%\begin{htmlonly}
%\usepackage{CCSLman,html,makeidx,color,ifthen,bm}
%%\setcounter{page}{0}
%\setcounter{chapter}{0}
%\chapter{INTRODUCTION TO CCSL}
%\begin{htmlonly}
%\usepackage{CCSLman,html,makeidx,color,ifthen,bm}
%%\setcounter{page}{0}
%\setcounter{chapter}{0}
%\chapter{INTRODUCTION TO CCSL}
%\begin{htmlonly}
%\usepackage{CCSLman,html,makeidx,color,ifthen,bm}
%\input{chap1.ptr}
%\end{htmlonly}
%\internal{c2}
%\internal{c3}
%\internal{c4}
%\internal{c5}
%\internal{c6}
%\internal{c7}
%\startdocument
\label{chap:1}%\hyperlink{chap:1}
\markboth{Introduction to CCSL}{}
This Manual describes the Cambridge Crystallography
Subroutine Library  (``CCSL").
It describes the material in the Master File of the program and is intended
for distribution to users at sites where CCSL is installed.
\pn 
\section{Content of the Manual}\markright{Content of the Manual} 
The Manual is intended both as an introduction to the Library for new
users and as an \ital{aide memoire} for more experienced users.
Some familiarity with the use of FORTRAN is assumed, including the use of a
computer to make and edit files and to compile, link and run FORTRAN jobs.
\pn
% 
\subsection{Scope} 
The manual indicates how to use all members of CCSL except those 
specifically written
for Profile Refinement, for which separate documentation is available.
\pn
% 
\subsection{Changes}
% 
This revision of the Manual describes the system as of \versiondate. 
Inevitably, some of the information will change as the library develops.  New
editions of the Manual will reflect important changes in the system. \pn
% 
\subsection{Content of the chapters} 
This Chapter ends with two sample jobs, followed through in enough
detail to introduce various necessary ideas.  The other chapters describe
different aspects of the Library and need not necessarily be read in
sequence.
\pn
\htmlref{Chapter 2}{chap:2} gives an overall view of the Library.  It should also help the
user to discover what is provided, and to begin to appreciate the system as
a whole.  At the end is a classified list of all the available routines.
\pn
\htmlref{Chapter 3}{chap:3}
 describes the \ital{Crystal Data File},  which contains the
 data describing the crystal structure, essentially, all data other than lists
of reflections,  observations etc.  The bulk of the chapter provides detailed
descriptions of the formats in which these data must be provided.
\pn
\htmlref{Chapter 4}{chap:4}
describes the input of other data, such as lists of 
observations for Least Squares Refinement, etc.  There is also a
section on the creation and use of input/output files.
\pn 
\htmlref{Chapter 5}{chap:5}
introduces the Least Squares Refinement facilities which the system provides,
and explains some of the terminology for users who want to refine
something a little different from what is already available.
\pn 
\htmlref{Chapter 6}{chap:6}
explains how the CCSL deals with magnetic structures.
\pn
\htmlref{Chapter 7}{chap:7}
describes how to run jobs, with reference to operating
systems under which CCSL has been implemented.  A list of some currently
available main programs is given.
\pn  
\subsection{Content of the appendices} 
The Appendices collect together lists of information useful for reference,
to obtain further understanding of the system, or for writing new programs.
They will usually be available separately from the Manual.
\pn 
\iftexforht{\href{../appenx/libapx.html}{Appendix A}}{Appendix A} lists the routines 
alphabetically, with explanation of:\pn
\begin{varindent}{3 cm}prerequisite calls or settings,\\
what is altered or output by the routine,\\
what the routine is used for, and, briefly, how it works.\\
\end{varindent}
\pn 
For each routine Appendix A lists which other
routines it calls, and which  routines call it. It also catalogues the COMMON
blocks that it declares and the items that it accesses from each. This
information was originally provided in Appendix B which is now obsolete.
\p 
\iftexforht{\href{../appenx/appenc.html}{Appendix C}}{Appendix C} lists all the labelled COMMON, with brief descriptions of 
their content. It also gives a list of the symbolic parameters (with their default values), which can be used 
to re-dimension items in these blocks
\p 
\iftexforht{\href{../appenx/maiapx.html}{Appendix D}}{Appendix D} describes some main programs which
are available and how to use them. It also gives information about them similar
to that provided for the library subroutines in Appendix A. 
A new user may at first find what he wants here, but will
probably move on to developing his own main programs from the existing ones.
\pn 
All the material for the Appendices is contained within CCSL itself, and
there are programs available to create the Appendices from the Master File.
\p
\section{An introductory example}\markright{An introductory example}
CCSL is a Library of FORTRAN routines intended to be used for standard
crystallographic calculations.  The routines may be thought of as
tools with which the crystallographer can make his own programs to do
exactly the computations he wants.
\pn
It has been written with the needs of the non-standard crystallographer
(or the solid state physicist working with crystalline materials) in mind.
It is not intended simply for the solution or refinement of ordinary crystal
structures, for which other systems may be more suitable.
\pn
Let us first take a simple but realistic small example.  The main program:
\bgs
%
\ttfamily\begin{verbatim}
      LOGICAL NOMORE
      DIMENSION H(3),K(3)
      COMMON /IOUNIT/LPT,ITI,ITO,IPLO,LUNI,IOUT
      CALL PREFIN('SIMPLE')
      CALL SYMOP
      CALL OPSYM(1)
      CALL SYMUNI
      CALL RECIP
      CALL SETGEN(0.2)
      NSUM=0
      WRITE (LPT,2000)
2000  FORMAT (//'   h     k    l  Multiplicity')
C
   1  CALL GETGEN(H,NOMORE)
      IF (NOMORE) GO TO 2
      MULT=MULBOX(H)
      NSUM=NSUM+MULT
      CALL INDFIX(H,K)
      IF (MULT .NE. 0) WRITE (LPT,2001) K,MULT
 2001 FORMAT(3I5,I8)
      GO TO 1
C
   2  WRITE (LPT,2002)  NSUM
2002  FORMAT (/' Number of reflections inside sphere =',I5)
      STOP
      END
\end{verbatim}\rm 
\bgs 
if supplied with a Crystal Data File named SAMPLE.CRY which says:\\
\ttfamily\begin{verbatim}
C 7.07107   7.07107   8.66025
S     Y,     Y-X,  1/6+Z
S    -Y,     -X,   1/6-Z
\end{verbatim}\rm 
\bgs
\par 
and run as a FORTRAN job, given CCSL as a Library, produces the 
\htlink{output}{output on page~}{simple.lis}{simple:out}.
\p
\section{Explanation of the Example}\markright{Explanation of the Example}
The user will naturally need some knowledge of FORTRAN.  Throughout the manual,
words which are parts of the FORTRAN programs will be written in
capitals, as they usually are in the actual source code.  
\pn 
This main program should be recognisable to the user as FORTRAN, most of
which he can interpret.  The nine names which refer to routines of CCSL
are:
\pn 
   \stlink{p}{PREFIN}, \stlink{s}{SYMOP}, \stlink{s}{SYMUNI}, \stlink{r}{RECIP}\ and 
   \stlink{s}{SETGEN} (all for input), \stlink{o}{OPSYM}\
for symmetry output and
   GETGEN, MULBOX and INDFIX inside a loop generating $h,k,l$ values.
\pn 
The routines  \stlink{g}{GETGEN}\ and \stlink{m}{MULBOX} are of special interest.  
GETGEN fills the
array H with the ``next" set of reflection indices $h,k,l$, eventually
generating all reciprocal lattice vectors in
the whole of an asymmetric unit of reciprocal space.  MULBOX returns the
value of the multiplicity of the reflection with indices $h,k,l$.
\pn
GETGEN is a FORTRAN SUBROUTINE (invoked by the word CALL), whereas MULBOX
is a FORTRAN FUNCTION, in which a variable called MULBOX is set to the
multiplicity.
Every member of the CCSL except for a few BLOCK DATA SUBPROGRAMS 
is either a SUBROUTINE or a FUNCTION ;
for ease of discussion we use the word ``routine" to mean ``either
SUBROUTINE or FUNCTION".
\pn 
GETGEN and MULBOX, then, are examples of \ital{crystallographic}\ routines of
CCSL.  There are a variety of others, such as LOGICAL FUNCTION ISPABS(H)
which determines whether $h,k,l$\/ in array H is or is not a space group absence,
COMPLEX FUNCTION FCALC(H) which calculates a complex Structure Factor,
and SUBROUTINE FOUR1Z which calculates one layer of a Fourier map.
\pn 
\stlink{i}{INDFIX}\ is a small \ital{utility} routine of CCSL which takes the 3 numbers in
the array H and ``fixes" them (converts them from REAL numbers to INTEGERS)
into the integer array K.  Utility routines exist for any operation which
the authors have become tired of writing several times, such as matrix
multiplication, finding a number in a given list, vector products, packing
small integers into bigger ones, etc.
\pn 
Except for OPSYM, the remaining CCSL routines in the example
(PREFIN, SYMOP, RECIP,
SYMUNI and SETGEN) are all involved in the input of data necessary to
perform the later calculations and are examples of \ital{setting-up routines}.
\pn 
\begin{list} {} {\setlength{\labelwidth}{2 cm}
  \setlength{\parsep}{-1ex}
  \setlength{\leftmargin}{\labelwidth}}
\item[\stlink{s}{SYMOP} \hfill] reads and interprets the data cards starting \bd{S} for the space group
symmetry.
\item[\stlink{r}{RECIP} \hfill] reads the \cd{C} for the unit cell parameters.
\item[\stlink{s}{SYMUNI} \hfill] makes a reciprocal space asymmetric unit, using the space group
information.
\item[\stlink{s}{SETGEN} \hfill] sets up the system ready for repeated use of GETGEN.  It needs
to know the space group, the asymmetric unit and the cell parameters,
so SYMOP, SYMUNI and RECIP must all have been obeyed before SETGEN.
\item[\stlink{o}{OPSYM}(1) \hfill] writes out the space group so found, in real space terms. 
 (OPSYM(2) would use reciprocal space).\end{list}
\pn 
This leaves \stlink{p}{PREFIN}.  All CCSL programs which read a Crystal Data File
should CALL PREFIN with the
name of the program as argument at an early
stage.  It initialises various aspects of the system, and puts the crystal
data into a form from which it can subsequently be read and interpreted
as often as required. 
\pn 
The remainder of the main program is ordinary FORTRAN.  Note that
although the Library does the work of producing $h,k,l$ values and
multiplicities, it is the main program which controls what is done with
them and how they are printed. The labelled COMMON /IOUNIT/ has to be
declared in the main program to direct the output of indices 
to the logical unit LPT which has been set up in PREFIN to be a printable file.
\p\pagebreak[8]
{\label{simple:out}
\hyperdef{}{simple.lis}{simple.lis}
\begin{verbatim}

                    * * * S I M P L E * * *

         Cambridge Crystallography Subroutine Library     Mark 4.0

                    Job run at 15:59 on 19-AUG-92

 Crystal data file  SAMPLE.CRY opened

 Data read by PREFIN from file SAMPLE.CRY
        1 card   labelled C
        2 cards  labelled S

 Non-centrosymmetric space group with 12 operator(s)

 The group is generated by the 2 element(s) numbered:  2  7

 Friedel's law NOT assumed - hkl distinct from -h-k-l

 General equivalent positions are:
       0                   0                   0       + 

  1      x                   y                   z       
  2      y                  -x+y                1/6+z    
  3     -x+y                -x                  1/3+z    
  4     -x                  -y                  1/2+z    
  5     -y                   x-y                2/3+z    
  6      x-y                 x                  5/6+z    
  7     -y                  -x                  1/6-z    
  8     -x                  -x+y                1/3-z    
  9      x-y                -y                  -z       
 10     -x+y                 y                  1/2-z    
 11      x                   x-y                5/6-z    
 12      y                   x                  2/3-z    

 Indices  13  11  10 to be used for typical reflection inside asymmetric unit

 Asymmetric unit has  2 plane(s):
             k>=0        
             h>=k        

 Symmetry constraints on lattice parameters    b   =  a  
                                             alpha = 90
                                              beta = 90
                                             gamma = 120

 Real cell          7.0711    7.0711    8.6602   90.00   90.00  120.00
 Volume =   375.0001

 Reciprocal cell    0.1633    0.1633    0.1155   90.00   90.00   60.00
 Volume =   0.2667E-02

 Real cell quadratic products:
 A (=a  sqrd)     B            C   D (=b c cos alpha)  E          F 
     50.00003    50.00003    74.99992     0.00000     0.00000   -25.00002

 Reciprocal cell quadratic products:
 A*(=a* sqrd)     B*          C*  D*(=b*c*cos alpha*)  E*         F*
      0.02667     0.02667     0.01333     0.00000     0.00000     0.01333

 Matrices for transformation of vectors to orthogonal axes
 Real space:
    6.1237    0.0000    0.0000
   -3.5355    7.0711    0.0000
    0.0000    0.0000    8.6602

 Reciprocal space:
    0.1633    0.0816    0.0000
    0.0000    0.1414    0.0000
    0.0000    0.0000    0.1155

   h     k    l  Multiplicity
    1    1    2      12
    1    1    1      12
    1    1    0       6
    1    0    0       6
    2    0    0       6
    0    0   -1       2
    1    0   -1      12
    2    0   -1      12
    0    0   -2       2
    1    0   -2      12
    2    0   -2      12
    0    0   -3       2
    1    0   -3      12

 Number of reflections inside sphere =  108
\end{verbatim}|
\par  
\goodbreak
\section{Comments on the example output}
\markright{Comments on the example output}
The output charts the progress of the program.  PREFIN notes one \cd{C} 
and two \cds{S} (without interpreting the data on them yet).  SYMOP
detects the lack of a centre of symmetry, and counts 12 operators in
total in the group.  In passing it looks for an \bd{I} (``Instructions") card
which could say, e.g.,
\pn\begin{verbatim}
I FRIE  1
\end{verbatim}\rm 
\par 
and would ask to assume Friedel's Law.  It does not find one, and says so.
\pn 
OPSYM(1) lists the operators (this is separate from SYMOP because once
the user is happy with his space group he may well not want to list all
operators for every run).
\pn 
SYMUNI notes that it has not been given a \bd{U} (``Unit") card with typical
$h,k,l$, so it uses (13,11,10), as being a general reflection with
$h > k > l$.  It finds a suitable asymmetric unit and prints it out.
\pn 
RECIP determines the relationships
imposed on the cell parameters by the symmetry and prints them out.  
In the example, which has hexagonal symmetry, these relationships imply that 
only values of cell sides a and c are needed from the \cd{C} (the value of b
is present on the sample Crystal Data but need not be).  It reads and
interprets the \cd{C}, filling in various useful quantities connected
with the cell parameters. Not all these quantities are printed.  It is
possible to make RECIP print more if more detail is needed.
\pn 
SETGEN then devises a scheme for traversing the whole of the asymmetric
unit.  The argument 0.2 is the maximum value of sin$\theta/\lambda$
required. Again, nothing is printed, but it could be if it is needed.
\pn 
The remaining output is generated by the WRITE statements in the user's
main program.
\pn 
Note that one of the arguments of the CALL of GETGEN is a LOGICAL
variable, NOMORE, which is eventually set to be .TRUE. when there are
no more $h,k,l$ values.  This must be declared LOGICAL at the head of
the main program.  Similarly, because arrays H and K are used, these
must be DIMENSIONed at the start, as H(3), K(3).
\pn 
In more complicated main programs it may be necessary to have access to
variables, such as the cell parameters, used in the Library routines.  Such
variables are held in labelled COMMON blocks.  The cell parameters are in
\pn\ttfamily\begin{verbatim}
      COMMON /CELPAR/CELL(3,3,2),V(2),ORTH(3,3,2),CPARS(6,2),KCPARS(6),
     & CELESD(6,6,2),CELLSD(6,6,2),SDCELL,LSQCEL,PRODSD,KOM4
      LOGICAL SDCELL,PRODSD,LSQCEL
\end{verbatim}\rm
\par 
where the array CELL holds both real and reciprocal space sides and angles.
If the main program wants to refer to CELL, it should declare
COMMON /CELPAR/ at its head, copying it from the Library.
\pn 
The user will see from the discussion above that it will be easier to
read this Manual if he has available a printout of the FORTRAN source
code of the Library, although this is very bulky and a copy of Appendix
A may suffice.
\pn 
There are, of course, certain typical computations which many users will
require often, and these give rise to standard main programs using CCSL.
To the user, such a main program is, for his particular computer, made
into a complete job which he can regard as a `black box'.  He feeds his
data into this box as if he were using a more conventional system of
programs.
\pn 
A new user will probably approach CCSL this way.  It is quite likely,
though, that soon the black box will not do exactly what he wants, and
he will move on to using CCSL in the way it is intended, writing his own
programs, or modifying existing programs, to fit his problem precisely.
\p
% 
\section{A More Elaborate Example}\markright{A More Elaborate Example}
% 
Taking the sample program as a basis, let us expand it to illustrate the
sort of features to expect in CCSL.  Suppose we have data which describe
a crystallographic structure, and we want to generate indices as above,
but then to take an asymmetric unit full of reflections, sort them in
ascending order of $\sin\theta/\lambda$, and list them with calculated
structure factors.
\pn
The expanded program below uses in addition from CCSL:\p
\begin{list} {} {\setlength{\labelwidth}{60mm}
  \setlength{\parsep}{-1ex}
  \setlength{\leftmargin}{\labelwidth}} 
\item[crystallographic routines \hfill] 
\stlink{l}{LATABS}\ - is it a lattice absence?\\
\stlink{f}{FCALC}\  - form a complex structure factor\\
\stlink{i}{ISPABS}\ - is it a space group absence?
\item[utility routines \hfill] 
\stlink{v}{VCTMOD}\ - vector length (for $\sin\theta/\lambda$)\\
\stlink{i}{INDFLO}\ - convert integers into real numbers\\
\stlink{s}{SORTX}\  - sort real numbers in ascending order\\
\stlink{e}{ERRMES}\ - send an error message if necessary
\item[input (\ital{setting-up}) routines \hfill] 
\stlink{a}{ATOPOS}\ - read a set of atomic positions\\
\stlink{s}{SETFOR}\ - read a set of form factors\\
\stlink{s}{SETANI}\ - read anisotropic temperature factors\end{list}
There are more declarations at the head of the program.  The FUNCTION
FCALC is COMPLEX, so both it and the variable to which the program sets
it must be declared COMPLEX.  ISPCE and ISTAR must be declared
CHARACTER *1.
\pn 
ISPABS and LATABS are LOGICAL FUNCTIONs, and must be so declared.
SORTX sorts within arrays in store, so these must be DIMENSIONed. 
\goodbreak \pn
The following, then, is the larger example main program.  It should
by now be fairly clear to the reader what the various parts of it are,
and how they fit together.
\p
\ttfamily\begin{verbatim}
      COMPLEX FC,FCALC
      LOGICAL NOMORE,ISPABS,LATABS
      DIMENSION H(3),K(3,1000),SINTH(1000),IPNT(1000)
      COMMON /IOUNIT/LPT,ITI,ITO,IPLO,LUNI,IOUT
      CHARACTER *1 ISPCE,ISTAR
      DATA ISPCE,ISTAR/' ','*'/
C
      CALL PREFIN('GETSF')
      WRITE (ITO,2002)
2002  FORMAT (' Value of sin theta/lambda max? ')
      READ (ITI,1000) S
1000  FORMAT (F10.4)
      CALL SYMOP
      CALL OPSYM(1)
      CALL OPSYM(2)
      CALL RECIP
      CALL ATOPOS
      CALL SETFOR
      CALL SETANI
      CALL SYMUNI
      WRITE (LPT,2000)
2000  FORMAT (////'  Sorted structure factors - * indicates space',
     1 'group absence:'/'     No.    h   k   l    Mult     s',
     2'            A            B          FcMod')
      CALL SETGEN(S)
C COMPLAIN AND STOP IF THERE WERE ERRORS IN THE INPUT
      CALL ERRMES(0,0,' to GETSF')
      NSUM=0
      N=0
   1  CALL GETGEN(H,NOMORE)
      IF (NOMORE) GO TO 2
      IF (LATABS(H)) GO TO 1
      MULT=MULBOX(H)
      IF (MULT .EQ. 0) GO TO 1
      N=N+1
      NSUM=NSUM+MULT
      SINTH(N)=VCTMOD(0.5,H,2)
      CALL INDFIX(H,K(1,N))
      GO TO 1
C
C SORT THE ARRAY IPNT, SO THAT IT POINTS TO THE ELEMENTS 
C OF SINTH IN SEQUENCE:  
   2  CALL SORTX(SINTH,IPNT,N)
      DO 3 I=1,N
      J=IPNT(I)
      CALL INDFLO(H,K(1,J))
      FC=FCALC(H)
      A=REAL(FC)
      B=AIMAG(FC)
      FCMOD=SQRT(A*A+B*B)
      IC=ISPCE
      IF (ISPABS(H)) IC=ISTAR
      M=MULBOX(H)
      WRITE (LPT,2001) IC,I,(K(L,J),L=1,3),M,SINTH(J),A,B,FCMOD
2001  FORMAT (' ',A1,I5,2X,3I4,2X,I5,F10.5,3X,3F12.5)
   3  CONTINUE
      WRITE (LPT,100)  NSUM,S
 100  FORMAT (/'  Total number of reflections inside sphere',
     1'=',I4/' S max=',F10.4)
      STOP
      END
\end{verbatim}\rm 
\par 
The initial dialogue which has been given as a WRITE and a READ statement for
simplicity could be more elegantly rendered as:
\pn\ttfamily\begin{verbatim}
      CALL ASK('Value of sin theta/lambda max?')
      CALL RDREAL(S,1,IP,80,IE)
\end{verbatim}\rm
\par 
because the routine ASK puts out the given message on the screen, and reads in a
line from the keyboard ready for routine RDREAL to read the value of S.
\pn
Similarly, the WRITE statement with FORMAT 2000 could be replaced by:
\pn\ttfamily\begin{verbatim}
      CALL CENTRE(LPT,4,'Sorted structure factors - * indicates '//
     1 'space group absence;',80)
      CALL MESS(LPT,0,'    No.    h   k   l    Mult     s'//
     2 '            A            B          FcMod')
\end{verbatim}\rm
%
\finchapter
%\end{document}

%\end{htmlonly}
%\internal{c2}
%\internal{c3}
%\internal{c4}
%\internal{c5}
%\internal{c6}
%\internal{c7}
%\startdocument
\label{chap:1}%\hyperlink{chap:1}
\markboth{Introduction to CCSL}{}
This Manual describes the Cambridge Crystallography
Subroutine Library  (``CCSL").
It describes the material in the Master File of the program and is intended
for distribution to users at sites where CCSL is installed.
\pn 
\section{Content of the Manual}\markright{Content of the Manual} 
The Manual is intended both as an introduction to the Library for new
users and as an \ital{aide memoire} for more experienced users.
Some familiarity with the use of FORTRAN is assumed, including the use of a
computer to make and edit files and to compile, link and run FORTRAN jobs.
\pn
% 
\subsection{Scope} 
The manual indicates how to use all members of CCSL except those 
specifically written
for Profile Refinement, for which separate documentation is available.
\pn
% 
\subsection{Changes}
% 
This revision of the Manual describes the system as of \versiondate. 
Inevitably, some of the information will change as the library develops.  New
editions of the Manual will reflect important changes in the system. \pn
% 
\subsection{Content of the chapters} 
This Chapter ends with two sample jobs, followed through in enough
detail to introduce various necessary ideas.  The other chapters describe
different aspects of the Library and need not necessarily be read in
sequence.
\pn
\htmlref{Chapter 2}{chap:2} gives an overall view of the Library.  It should also help the
user to discover what is provided, and to begin to appreciate the system as
a whole.  At the end is a classified list of all the available routines.
\pn
\htmlref{Chapter 3}{chap:3}
 describes the \ital{Crystal Data File},  which contains the
 data describing the crystal structure, essentially, all data other than lists
of reflections,  observations etc.  The bulk of the chapter provides detailed
descriptions of the formats in which these data must be provided.
\pn
\htmlref{Chapter 4}{chap:4}
describes the input of other data, such as lists of 
observations for Least Squares Refinement, etc.  There is also a
section on the creation and use of input/output files.
\pn 
\htmlref{Chapter 5}{chap:5}
introduces the Least Squares Refinement facilities which the system provides,
and explains some of the terminology for users who want to refine
something a little different from what is already available.
\pn 
\htmlref{Chapter 6}{chap:6}
explains how the CCSL deals with magnetic structures.
\pn
\htmlref{Chapter 7}{chap:7}
describes how to run jobs, with reference to operating
systems under which CCSL has been implemented.  A list of some currently
available main programs is given.
\pn  
\subsection{Content of the appendices} 
The Appendices collect together lists of information useful for reference,
to obtain further understanding of the system, or for writing new programs.
They will usually be available separately from the Manual.
\pn 
\iftexforht{\href{../appenx/libapx.html}{Appendix A}}{Appendix A} lists the routines 
alphabetically, with explanation of:\pn
\begin{varindent}{3 cm}prerequisite calls or settings,\\
what is altered or output by the routine,\\
what the routine is used for, and, briefly, how it works.\\
\end{varindent}
\pn 
For each routine Appendix A lists which other
routines it calls, and which  routines call it. It also catalogues the COMMON
blocks that it declares and the items that it accesses from each. This
information was originally provided in Appendix B which is now obsolete.
\p 
\iftexforht{\href{../appenx/appenc.html}{Appendix C}}{Appendix C} lists all the labelled COMMON, with brief descriptions of 
their content. It also gives a list of the symbolic parameters (with their default values), which can be used 
to re-dimension items in these blocks
\p 
\iftexforht{\href{../appenx/maiapx.html}{Appendix D}}{Appendix D} describes some main programs which
are available and how to use them. It also gives information about them similar
to that provided for the library subroutines in Appendix A. 
A new user may at first find what he wants here, but will
probably move on to developing his own main programs from the existing ones.
\pn 
All the material for the Appendices is contained within CCSL itself, and
there are programs available to create the Appendices from the Master File.
\p
\section{An introductory example}\markright{An introductory example}
CCSL is a Library of FORTRAN routines intended to be used for standard
crystallographic calculations.  The routines may be thought of as
tools with which the crystallographer can make his own programs to do
exactly the computations he wants.
\pn
It has been written with the needs of the non-standard crystallographer
(or the solid state physicist working with crystalline materials) in mind.
It is not intended simply for the solution or refinement of ordinary crystal
structures, for which other systems may be more suitable.
\pn
Let us first take a simple but realistic small example.  The main program:
\bgs
%
\ttfamily\begin{verbatim}
      LOGICAL NOMORE
      DIMENSION H(3),K(3)
      COMMON /IOUNIT/LPT,ITI,ITO,IPLO,LUNI,IOUT
      CALL PREFIN('SIMPLE')
      CALL SYMOP
      CALL OPSYM(1)
      CALL SYMUNI
      CALL RECIP
      CALL SETGEN(0.2)
      NSUM=0
      WRITE (LPT,2000)
2000  FORMAT (//'   h     k    l  Multiplicity')
C
   1  CALL GETGEN(H,NOMORE)
      IF (NOMORE) GO TO 2
      MULT=MULBOX(H)
      NSUM=NSUM+MULT
      CALL INDFIX(H,K)
      IF (MULT .NE. 0) WRITE (LPT,2001) K,MULT
 2001 FORMAT(3I5,I8)
      GO TO 1
C
   2  WRITE (LPT,2002)  NSUM
2002  FORMAT (/' Number of reflections inside sphere =',I5)
      STOP
      END
\end{verbatim}\rm 
\bgs 
if supplied with a Crystal Data File named SAMPLE.CRY which says:\\
\ttfamily\begin{verbatim}
C 7.07107   7.07107   8.66025
S     Y,     Y-X,  1/6+Z
S    -Y,     -X,   1/6-Z
\end{verbatim}\rm 
\bgs
\par 
and run as a FORTRAN job, given CCSL as a Library, produces the 
\htlink{output}{output on page~}{simple.lis}{simple:out}.
\p
\section{Explanation of the Example}\markright{Explanation of the Example}
The user will naturally need some knowledge of FORTRAN.  Throughout the manual,
words which are parts of the FORTRAN programs will be written in
capitals, as they usually are in the actual source code.  
\pn 
This main program should be recognisable to the user as FORTRAN, most of
which he can interpret.  The nine names which refer to routines of CCSL
are:
\pn 
   \stlink{p}{PREFIN}, \stlink{s}{SYMOP}, \stlink{s}{SYMUNI}, \stlink{r}{RECIP}\ and 
   \stlink{s}{SETGEN} (all for input), \stlink{o}{OPSYM}\
for symmetry output and
   GETGEN, MULBOX and INDFIX inside a loop generating $h,k,l$ values.
\pn 
The routines  \stlink{g}{GETGEN}\ and \stlink{m}{MULBOX} are of special interest.  
GETGEN fills the
array H with the ``next" set of reflection indices $h,k,l$, eventually
generating all reciprocal lattice vectors in
the whole of an asymmetric unit of reciprocal space.  MULBOX returns the
value of the multiplicity of the reflection with indices $h,k,l$.
\pn
GETGEN is a FORTRAN SUBROUTINE (invoked by the word CALL), whereas MULBOX
is a FORTRAN FUNCTION, in which a variable called MULBOX is set to the
multiplicity.
Every member of the CCSL except for a few BLOCK DATA SUBPROGRAMS 
is either a SUBROUTINE or a FUNCTION ;
for ease of discussion we use the word ``routine" to mean ``either
SUBROUTINE or FUNCTION".
\pn 
GETGEN and MULBOX, then, are examples of \ital{crystallographic}\ routines of
CCSL.  There are a variety of others, such as LOGICAL FUNCTION ISPABS(H)
which determines whether $h,k,l$\/ in array H is or is not a space group absence,
COMPLEX FUNCTION FCALC(H) which calculates a complex Structure Factor,
and SUBROUTINE FOUR1Z which calculates one layer of a Fourier map.
\pn 
\stlink{i}{INDFIX}\ is a small \ital{utility} routine of CCSL which takes the 3 numbers in
the array H and ``fixes" them (converts them from REAL numbers to INTEGERS)
into the integer array K.  Utility routines exist for any operation which
the authors have become tired of writing several times, such as matrix
multiplication, finding a number in a given list, vector products, packing
small integers into bigger ones, etc.
\pn 
Except for OPSYM, the remaining CCSL routines in the example
(PREFIN, SYMOP, RECIP,
SYMUNI and SETGEN) are all involved in the input of data necessary to
perform the later calculations and are examples of \ital{setting-up routines}.
\pn 
\begin{list} {} {\setlength{\labelwidth}{2 cm}
  \setlength{\parsep}{-1ex}
  \setlength{\leftmargin}{\labelwidth}}
\item[\stlink{s}{SYMOP} \hfill] reads and interprets the data cards starting \bd{S} for the space group
symmetry.
\item[\stlink{r}{RECIP} \hfill] reads the \cd{C} for the unit cell parameters.
\item[\stlink{s}{SYMUNI} \hfill] makes a reciprocal space asymmetric unit, using the space group
information.
\item[\stlink{s}{SETGEN} \hfill] sets up the system ready for repeated use of GETGEN.  It needs
to know the space group, the asymmetric unit and the cell parameters,
so SYMOP, SYMUNI and RECIP must all have been obeyed before SETGEN.
\item[\stlink{o}{OPSYM}(1) \hfill] writes out the space group so found, in real space terms. 
 (OPSYM(2) would use reciprocal space).\end{list}
\pn 
This leaves \stlink{p}{PREFIN}.  All CCSL programs which read a Crystal Data File
should CALL PREFIN with the
name of the program as argument at an early
stage.  It initialises various aspects of the system, and puts the crystal
data into a form from which it can subsequently be read and interpreted
as often as required. 
\pn 
The remainder of the main program is ordinary FORTRAN.  Note that
although the Library does the work of producing $h,k,l$ values and
multiplicities, it is the main program which controls what is done with
them and how they are printed. The labelled COMMON /IOUNIT/ has to be
declared in the main program to direct the output of indices 
to the logical unit LPT which has been set up in PREFIN to be a printable file.
\p\pagebreak[8]
{\label{simple:out}
\hyperdef{}{simple.lis}{simple.lis}
\begin{verbatim}

                    * * * S I M P L E * * *

         Cambridge Crystallography Subroutine Library     Mark 4.0

                    Job run at 15:59 on 19-AUG-92

 Crystal data file  SAMPLE.CRY opened

 Data read by PREFIN from file SAMPLE.CRY
        1 card   labelled C
        2 cards  labelled S

 Non-centrosymmetric space group with 12 operator(s)

 The group is generated by the 2 element(s) numbered:  2  7

 Friedel's law NOT assumed - hkl distinct from -h-k-l

 General equivalent positions are:
       0                   0                   0       + 

  1      x                   y                   z       
  2      y                  -x+y                1/6+z    
  3     -x+y                -x                  1/3+z    
  4     -x                  -y                  1/2+z    
  5     -y                   x-y                2/3+z    
  6      x-y                 x                  5/6+z    
  7     -y                  -x                  1/6-z    
  8     -x                  -x+y                1/3-z    
  9      x-y                -y                  -z       
 10     -x+y                 y                  1/2-z    
 11      x                   x-y                5/6-z    
 12      y                   x                  2/3-z    

 Indices  13  11  10 to be used for typical reflection inside asymmetric unit

 Asymmetric unit has  2 plane(s):
             k>=0        
             h>=k        

 Symmetry constraints on lattice parameters    b   =  a  
                                             alpha = 90
                                              beta = 90
                                             gamma = 120

 Real cell          7.0711    7.0711    8.6602   90.00   90.00  120.00
 Volume =   375.0001

 Reciprocal cell    0.1633    0.1633    0.1155   90.00   90.00   60.00
 Volume =   0.2667E-02

 Real cell quadratic products:
 A (=a  sqrd)     B            C   D (=b c cos alpha)  E          F 
     50.00003    50.00003    74.99992     0.00000     0.00000   -25.00002

 Reciprocal cell quadratic products:
 A*(=a* sqrd)     B*          C*  D*(=b*c*cos alpha*)  E*         F*
      0.02667     0.02667     0.01333     0.00000     0.00000     0.01333

 Matrices for transformation of vectors to orthogonal axes
 Real space:
    6.1237    0.0000    0.0000
   -3.5355    7.0711    0.0000
    0.0000    0.0000    8.6602

 Reciprocal space:
    0.1633    0.0816    0.0000
    0.0000    0.1414    0.0000
    0.0000    0.0000    0.1155

   h     k    l  Multiplicity
    1    1    2      12
    1    1    1      12
    1    1    0       6
    1    0    0       6
    2    0    0       6
    0    0   -1       2
    1    0   -1      12
    2    0   -1      12
    0    0   -2       2
    1    0   -2      12
    2    0   -2      12
    0    0   -3       2
    1    0   -3      12

 Number of reflections inside sphere =  108
\end{verbatim}|
\par  
\goodbreak
\section{Comments on the example output}
\markright{Comments on the example output}
The output charts the progress of the program.  PREFIN notes one \cd{C} 
and two \cds{S} (without interpreting the data on them yet).  SYMOP
detects the lack of a centre of symmetry, and counts 12 operators in
total in the group.  In passing it looks for an \bd{I} (``Instructions") card
which could say, e.g.,
\pn\begin{verbatim}
I FRIE  1
\end{verbatim}\rm 
\par 
and would ask to assume Friedel's Law.  It does not find one, and says so.
\pn 
OPSYM(1) lists the operators (this is separate from SYMOP because once
the user is happy with his space group he may well not want to list all
operators for every run).
\pn 
SYMUNI notes that it has not been given a \bd{U} (``Unit") card with typical
$h,k,l$, so it uses (13,11,10), as being a general reflection with
$h > k > l$.  It finds a suitable asymmetric unit and prints it out.
\pn 
RECIP determines the relationships
imposed on the cell parameters by the symmetry and prints them out.  
In the example, which has hexagonal symmetry, these relationships imply that 
only values of cell sides a and c are needed from the \cd{C} (the value of b
is present on the sample Crystal Data but need not be).  It reads and
interprets the \cd{C}, filling in various useful quantities connected
with the cell parameters. Not all these quantities are printed.  It is
possible to make RECIP print more if more detail is needed.
\pn 
SETGEN then devises a scheme for traversing the whole of the asymmetric
unit.  The argument 0.2 is the maximum value of sin$\theta/\lambda$
required. Again, nothing is printed, but it could be if it is needed.
\pn 
The remaining output is generated by the WRITE statements in the user's
main program.
\pn 
Note that one of the arguments of the CALL of GETGEN is a LOGICAL
variable, NOMORE, which is eventually set to be .TRUE. when there are
no more $h,k,l$ values.  This must be declared LOGICAL at the head of
the main program.  Similarly, because arrays H and K are used, these
must be DIMENSIONed at the start, as H(3), K(3).
\pn 
In more complicated main programs it may be necessary to have access to
variables, such as the cell parameters, used in the Library routines.  Such
variables are held in labelled COMMON blocks.  The cell parameters are in
\pn\ttfamily\begin{verbatim}
      COMMON /CELPAR/CELL(3,3,2),V(2),ORTH(3,3,2),CPARS(6,2),KCPARS(6),
     & CELESD(6,6,2),CELLSD(6,6,2),SDCELL,LSQCEL,PRODSD,KOM4
      LOGICAL SDCELL,PRODSD,LSQCEL
\end{verbatim}\rm
\par 
where the array CELL holds both real and reciprocal space sides and angles.
If the main program wants to refer to CELL, it should declare
COMMON /CELPAR/ at its head, copying it from the Library.
\pn 
The user will see from the discussion above that it will be easier to
read this Manual if he has available a printout of the FORTRAN source
code of the Library, although this is very bulky and a copy of Appendix
A may suffice.
\pn 
There are, of course, certain typical computations which many users will
require often, and these give rise to standard main programs using CCSL.
To the user, such a main program is, for his particular computer, made
into a complete job which he can regard as a `black box'.  He feeds his
data into this box as if he were using a more conventional system of
programs.
\pn 
A new user will probably approach CCSL this way.  It is quite likely,
though, that soon the black box will not do exactly what he wants, and
he will move on to using CCSL in the way it is intended, writing his own
programs, or modifying existing programs, to fit his problem precisely.
\p
% 
\section{A More Elaborate Example}\markright{A More Elaborate Example}
% 
Taking the sample program as a basis, let us expand it to illustrate the
sort of features to expect in CCSL.  Suppose we have data which describe
a crystallographic structure, and we want to generate indices as above,
but then to take an asymmetric unit full of reflections, sort them in
ascending order of $\sin\theta/\lambda$, and list them with calculated
structure factors.
\pn
The expanded program below uses in addition from CCSL:\p
\begin{list} {} {\setlength{\labelwidth}{60mm}
  \setlength{\parsep}{-1ex}
  \setlength{\leftmargin}{\labelwidth}} 
\item[crystallographic routines \hfill] 
\stlink{l}{LATABS}\ - is it a lattice absence?\\
\stlink{f}{FCALC}\  - form a complex structure factor\\
\stlink{i}{ISPABS}\ - is it a space group absence?
\item[utility routines \hfill] 
\stlink{v}{VCTMOD}\ - vector length (for $\sin\theta/\lambda$)\\
\stlink{i}{INDFLO}\ - convert integers into real numbers\\
\stlink{s}{SORTX}\  - sort real numbers in ascending order\\
\stlink{e}{ERRMES}\ - send an error message if necessary
\item[input (\ital{setting-up}) routines \hfill] 
\stlink{a}{ATOPOS}\ - read a set of atomic positions\\
\stlink{s}{SETFOR}\ - read a set of form factors\\
\stlink{s}{SETANI}\ - read anisotropic temperature factors\end{list}
There are more declarations at the head of the program.  The FUNCTION
FCALC is COMPLEX, so both it and the variable to which the program sets
it must be declared COMPLEX.  ISPCE and ISTAR must be declared
CHARACTER *1.
\pn 
ISPABS and LATABS are LOGICAL FUNCTIONs, and must be so declared.
SORTX sorts within arrays in store, so these must be DIMENSIONed. 
\goodbreak \pn
The following, then, is the larger example main program.  It should
by now be fairly clear to the reader what the various parts of it are,
and how they fit together.
\p
\ttfamily\begin{verbatim}
      COMPLEX FC,FCALC
      LOGICAL NOMORE,ISPABS,LATABS
      DIMENSION H(3),K(3,1000),SINTH(1000),IPNT(1000)
      COMMON /IOUNIT/LPT,ITI,ITO,IPLO,LUNI,IOUT
      CHARACTER *1 ISPCE,ISTAR
      DATA ISPCE,ISTAR/' ','*'/
C
      CALL PREFIN('GETSF')
      WRITE (ITO,2002)
2002  FORMAT (' Value of sin theta/lambda max? ')
      READ (ITI,1000) S
1000  FORMAT (F10.4)
      CALL SYMOP
      CALL OPSYM(1)
      CALL OPSYM(2)
      CALL RECIP
      CALL ATOPOS
      CALL SETFOR
      CALL SETANI
      CALL SYMUNI
      WRITE (LPT,2000)
2000  FORMAT (////'  Sorted structure factors - * indicates space',
     1 'group absence:'/'     No.    h   k   l    Mult     s',
     2'            A            B          FcMod')
      CALL SETGEN(S)
C COMPLAIN AND STOP IF THERE WERE ERRORS IN THE INPUT
      CALL ERRMES(0,0,' to GETSF')
      NSUM=0
      N=0
   1  CALL GETGEN(H,NOMORE)
      IF (NOMORE) GO TO 2
      IF (LATABS(H)) GO TO 1
      MULT=MULBOX(H)
      IF (MULT .EQ. 0) GO TO 1
      N=N+1
      NSUM=NSUM+MULT
      SINTH(N)=VCTMOD(0.5,H,2)
      CALL INDFIX(H,K(1,N))
      GO TO 1
C
C SORT THE ARRAY IPNT, SO THAT IT POINTS TO THE ELEMENTS 
C OF SINTH IN SEQUENCE:  
   2  CALL SORTX(SINTH,IPNT,N)
      DO 3 I=1,N
      J=IPNT(I)
      CALL INDFLO(H,K(1,J))
      FC=FCALC(H)
      A=REAL(FC)
      B=AIMAG(FC)
      FCMOD=SQRT(A*A+B*B)
      IC=ISPCE
      IF (ISPABS(H)) IC=ISTAR
      M=MULBOX(H)
      WRITE (LPT,2001) IC,I,(K(L,J),L=1,3),M,SINTH(J),A,B,FCMOD
2001  FORMAT (' ',A1,I5,2X,3I4,2X,I5,F10.5,3X,3F12.5)
   3  CONTINUE
      WRITE (LPT,100)  NSUM,S
 100  FORMAT (/'  Total number of reflections inside sphere',
     1'=',I4/' S max=',F10.4)
      STOP
      END
\end{verbatim}\rm 
\par 
The initial dialogue which has been given as a WRITE and a READ statement for
simplicity could be more elegantly rendered as:
\pn\ttfamily\begin{verbatim}
      CALL ASK('Value of sin theta/lambda max?')
      CALL RDREAL(S,1,IP,80,IE)
\end{verbatim}\rm
\par 
because the routine ASK puts out the given message on the screen, and reads in a
line from the keyboard ready for routine RDREAL to read the value of S.
\pn
Similarly, the WRITE statement with FORMAT 2000 could be replaced by:
\pn\ttfamily\begin{verbatim}
      CALL CENTRE(LPT,4,'Sorted structure factors - * indicates '//
     1 'space group absence;',80)
      CALL MESS(LPT,0,'    No.    h   k   l    Mult     s'//
     2 '            A            B          FcMod')
\end{verbatim}\rm
%
\finchapter
%\end{document}

%\end{htmlonly}
%\internal{c2}
%\internal{c3}
%\internal{c4}
%\internal{c5}
%\internal{c6}
%\internal{c7}
%\startdocument
\label{chap:1}%\hyperlink{chap:1}
\markboth{Introduction to CCSL}{}
This Manual describes the Cambridge Crystallography
Subroutine Library  (``CCSL").
It describes the material in the Master File of the program and is intended
for distribution to users at sites where CCSL is installed.
\pn 
\section{Content of the Manual}\markright{Content of the Manual} 
The Manual is intended both as an introduction to the Library for new
users and as an \ital{aide memoire} for more experienced users.
Some familiarity with the use of FORTRAN is assumed, including the use of a
computer to make and edit files and to compile, link and run FORTRAN jobs.
\pn
% 
\subsection{Scope} 
The manual indicates how to use all members of CCSL except those 
specifically written
for Profile Refinement, for which separate documentation is available.
\pn
% 
\subsection{Changes}
% 
This revision of the Manual describes the system as of \versiondate. 
Inevitably, some of the information will change as the library develops.  New
editions of the Manual will reflect important changes in the system. \pn
% 
\subsection{Content of the chapters} 
This Chapter ends with two sample jobs, followed through in enough
detail to introduce various necessary ideas.  The other chapters describe
different aspects of the Library and need not necessarily be read in
sequence.
\pn
\htmlref{Chapter 2}{chap:2} gives an overall view of the Library.  It should also help the
user to discover what is provided, and to begin to appreciate the system as
a whole.  At the end is a classified list of all the available routines.
\pn
\htmlref{Chapter 3}{chap:3}
 describes the \ital{Crystal Data File},  which contains the
 data describing the crystal structure, essentially, all data other than lists
of reflections,  observations etc.  The bulk of the chapter provides detailed
descriptions of the formats in which these data must be provided.
\pn
\htmlref{Chapter 4}{chap:4}
describes the input of other data, such as lists of 
observations for Least Squares Refinement, etc.  There is also a
section on the creation and use of input/output files.
\pn 
\htmlref{Chapter 5}{chap:5}
introduces the Least Squares Refinement facilities which the system provides,
and explains some of the terminology for users who want to refine
something a little different from what is already available.
\pn 
\htmlref{Chapter 6}{chap:6}
explains how the CCSL deals with magnetic structures.
\pn
\htmlref{Chapter 7}{chap:7}
describes how to run jobs, with reference to operating
systems under which CCSL has been implemented.  A list of some currently
available main programs is given.
\pn  
\subsection{Content of the appendices} 
The Appendices collect together lists of information useful for reference,
to obtain further understanding of the system, or for writing new programs.
They will usually be available separately from the Manual.
\pn 
\iftexforht{\href{../appenx/libapx.html}{Appendix A}}{Appendix A} lists the routines 
alphabetically, with explanation of:\pn
\begin{varindent}{3 cm}prerequisite calls or settings,\\
what is altered or output by the routine,\\
what the routine is used for, and, briefly, how it works.\\
\end{varindent}
\pn 
For each routine Appendix A lists which other
routines it calls, and which  routines call it. It also catalogues the COMMON
blocks that it declares and the items that it accesses from each. This
information was originally provided in Appendix B which is now obsolete.
\p 
\iftexforht{\href{../appenx/appenc.html}{Appendix C}}{Appendix C} lists all the labelled COMMON, with brief descriptions of 
their content. It also gives a list of the symbolic parameters (with their default values), which can be used 
to re-dimension items in these blocks
\p 
\iftexforht{\href{../appenx/maiapx.html}{Appendix D}}{Appendix D} describes some main programs which
are available and how to use them. It also gives information about them similar
to that provided for the library subroutines in Appendix A. 
A new user may at first find what he wants here, but will
probably move on to developing his own main programs from the existing ones.
\pn 
All the material for the Appendices is contained within CCSL itself, and
there are programs available to create the Appendices from the Master File.
\p
\section{An introductory example}\markright{An introductory example}
CCSL is a Library of FORTRAN routines intended to be used for standard
crystallographic calculations.  The routines may be thought of as
tools with which the crystallographer can make his own programs to do
exactly the computations he wants.
\pn
It has been written with the needs of the non-standard crystallographer
(or the solid state physicist working with crystalline materials) in mind.
It is not intended simply for the solution or refinement of ordinary crystal
structures, for which other systems may be more suitable.
\pn
Let us first take a simple but realistic small example.  The main program:
\bgs
%
\ttfamily\begin{verbatim}
      LOGICAL NOMORE
      DIMENSION H(3),K(3)
      COMMON /IOUNIT/LPT,ITI,ITO,IPLO,LUNI,IOUT
      CALL PREFIN('SIMPLE')
      CALL SYMOP
      CALL OPSYM(1)
      CALL SYMUNI
      CALL RECIP
      CALL SETGEN(0.2)
      NSUM=0
      WRITE (LPT,2000)
2000  FORMAT (//'   h     k    l  Multiplicity')
C
   1  CALL GETGEN(H,NOMORE)
      IF (NOMORE) GO TO 2
      MULT=MULBOX(H)
      NSUM=NSUM+MULT
      CALL INDFIX(H,K)
      IF (MULT .NE. 0) WRITE (LPT,2001) K,MULT
 2001 FORMAT(3I5,I8)
      GO TO 1
C
   2  WRITE (LPT,2002)  NSUM
2002  FORMAT (/' Number of reflections inside sphere =',I5)
      STOP
      END
\end{verbatim}\rm 
\bgs 
if supplied with a Crystal Data File named SAMPLE.CRY which says:\\
\ttfamily\begin{verbatim}
C 7.07107   7.07107   8.66025
S     Y,     Y-X,  1/6+Z
S    -Y,     -X,   1/6-Z
\end{verbatim}\rm 
\bgs
\par 
and run as a FORTRAN job, given CCSL as a Library, produces the 
\htlink{output}{output on page~}{simple.lis}{simple:out}.
\p
\section{Explanation of the Example}\markright{Explanation of the Example}
The user will naturally need some knowledge of FORTRAN.  Throughout the manual,
words which are parts of the FORTRAN programs will be written in
capitals, as they usually are in the actual source code.  
\pn 
This main program should be recognisable to the user as FORTRAN, most of
which he can interpret.  The nine names which refer to routines of CCSL
are:
\pn 
   \stlink{p}{PREFIN}, \stlink{s}{SYMOP}, \stlink{s}{SYMUNI}, \stlink{r}{RECIP}\ and 
   \stlink{s}{SETGEN} (all for input), \stlink{o}{OPSYM}\
for symmetry output and
   GETGEN, MULBOX and INDFIX inside a loop generating $h,k,l$ values.
\pn 
The routines  \stlink{g}{GETGEN}\ and \stlink{m}{MULBOX} are of special interest.  
GETGEN fills the
array H with the ``next" set of reflection indices $h,k,l$, eventually
generating all reciprocal lattice vectors in
the whole of an asymmetric unit of reciprocal space.  MULBOX returns the
value of the multiplicity of the reflection with indices $h,k,l$.
\pn
GETGEN is a FORTRAN SUBROUTINE (invoked by the word CALL), whereas MULBOX
is a FORTRAN FUNCTION, in which a variable called MULBOX is set to the
multiplicity.
Every member of the CCSL except for a few BLOCK DATA SUBPROGRAMS 
is either a SUBROUTINE or a FUNCTION ;
for ease of discussion we use the word ``routine" to mean ``either
SUBROUTINE or FUNCTION".
\pn 
GETGEN and MULBOX, then, are examples of \ital{crystallographic}\ routines of
CCSL.  There are a variety of others, such as LOGICAL FUNCTION ISPABS(H)
which determines whether $h,k,l$\/ in array H is or is not a space group absence,
COMPLEX FUNCTION FCALC(H) which calculates a complex Structure Factor,
and SUBROUTINE FOUR1Z which calculates one layer of a Fourier map.
\pn 
\stlink{i}{INDFIX}\ is a small \ital{utility} routine of CCSL which takes the 3 numbers in
the array H and ``fixes" them (converts them from REAL numbers to INTEGERS)
into the integer array K.  Utility routines exist for any operation which
the authors have become tired of writing several times, such as matrix
multiplication, finding a number in a given list, vector products, packing
small integers into bigger ones, etc.
\pn 
Except for OPSYM, the remaining CCSL routines in the example
(PREFIN, SYMOP, RECIP,
SYMUNI and SETGEN) are all involved in the input of data necessary to
perform the later calculations and are examples of \ital{setting-up routines}.
\pn 
\begin{list} {} {\setlength{\labelwidth}{2 cm}
  \setlength{\parsep}{-1ex}
  \setlength{\leftmargin}{\labelwidth}}
\item[\stlink{s}{SYMOP} \hfill] reads and interprets the data cards starting \bd{S} for the space group
symmetry.
\item[\stlink{r}{RECIP} \hfill] reads the \cd{C} for the unit cell parameters.
\item[\stlink{s}{SYMUNI} \hfill] makes a reciprocal space asymmetric unit, using the space group
information.
\item[\stlink{s}{SETGEN} \hfill] sets up the system ready for repeated use of GETGEN.  It needs
to know the space group, the asymmetric unit and the cell parameters,
so SYMOP, SYMUNI and RECIP must all have been obeyed before SETGEN.
\item[\stlink{o}{OPSYM}(1) \hfill] writes out the space group so found, in real space terms. 
 (OPSYM(2) would use reciprocal space).\end{list}
\pn 
This leaves \stlink{p}{PREFIN}.  All CCSL programs which read a Crystal Data File
should CALL PREFIN with the
name of the program as argument at an early
stage.  It initialises various aspects of the system, and puts the crystal
data into a form from which it can subsequently be read and interpreted
as often as required. 
\pn 
The remainder of the main program is ordinary FORTRAN.  Note that
although the Library does the work of producing $h,k,l$ values and
multiplicities, it is the main program which controls what is done with
them and how they are printed. The labelled COMMON /IOUNIT/ has to be
declared in the main program to direct the output of indices 
to the logical unit LPT which has been set up in PREFIN to be a printable file.
\p\pagebreak[8]
{\label{simple:out}
\hyperdef{}{simple.lis}{simple.lis}
\begin{verbatim}

                    * * * S I M P L E * * *

         Cambridge Crystallography Subroutine Library     Mark 4.0

                    Job run at 15:59 on 19-AUG-92

 Crystal data file  SAMPLE.CRY opened

 Data read by PREFIN from file SAMPLE.CRY
        1 card   labelled C
        2 cards  labelled S

 Non-centrosymmetric space group with 12 operator(s)

 The group is generated by the 2 element(s) numbered:  2  7

 Friedel's law NOT assumed - hkl distinct from -h-k-l

 General equivalent positions are:
       0                   0                   0       + 

  1      x                   y                   z       
  2      y                  -x+y                1/6+z    
  3     -x+y                -x                  1/3+z    
  4     -x                  -y                  1/2+z    
  5     -y                   x-y                2/3+z    
  6      x-y                 x                  5/6+z    
  7     -y                  -x                  1/6-z    
  8     -x                  -x+y                1/3-z    
  9      x-y                -y                  -z       
 10     -x+y                 y                  1/2-z    
 11      x                   x-y                5/6-z    
 12      y                   x                  2/3-z    

 Indices  13  11  10 to be used for typical reflection inside asymmetric unit

 Asymmetric unit has  2 plane(s):
             k>=0        
             h>=k        

 Symmetry constraints on lattice parameters    b   =  a  
                                             alpha = 90
                                              beta = 90
                                             gamma = 120

 Real cell          7.0711    7.0711    8.6602   90.00   90.00  120.00
 Volume =   375.0001

 Reciprocal cell    0.1633    0.1633    0.1155   90.00   90.00   60.00
 Volume =   0.2667E-02

 Real cell quadratic products:
 A (=a  sqrd)     B            C   D (=b c cos alpha)  E          F 
     50.00003    50.00003    74.99992     0.00000     0.00000   -25.00002

 Reciprocal cell quadratic products:
 A*(=a* sqrd)     B*          C*  D*(=b*c*cos alpha*)  E*         F*
      0.02667     0.02667     0.01333     0.00000     0.00000     0.01333

 Matrices for transformation of vectors to orthogonal axes
 Real space:
    6.1237    0.0000    0.0000
   -3.5355    7.0711    0.0000
    0.0000    0.0000    8.6602

 Reciprocal space:
    0.1633    0.0816    0.0000
    0.0000    0.1414    0.0000
    0.0000    0.0000    0.1155

   h     k    l  Multiplicity
    1    1    2      12
    1    1    1      12
    1    1    0       6
    1    0    0       6
    2    0    0       6
    0    0   -1       2
    1    0   -1      12
    2    0   -1      12
    0    0   -2       2
    1    0   -2      12
    2    0   -2      12
    0    0   -3       2
    1    0   -3      12

 Number of reflections inside sphere =  108
\end{verbatim}|
\par  
\goodbreak
\section{Comments on the example output}
\markright{Comments on the example output}
The output charts the progress of the program.  PREFIN notes one \cd{C} 
and two \cds{S} (without interpreting the data on them yet).  SYMOP
detects the lack of a centre of symmetry, and counts 12 operators in
total in the group.  In passing it looks for an \bd{I} (``Instructions") card
which could say, e.g.,
\pn\begin{verbatim}
I FRIE  1
\end{verbatim}\rm 
\par 
and would ask to assume Friedel's Law.  It does not find one, and says so.
\pn 
OPSYM(1) lists the operators (this is separate from SYMOP because once
the user is happy with his space group he may well not want to list all
operators for every run).
\pn 
SYMUNI notes that it has not been given a \bd{U} (``Unit") card with typical
$h,k,l$, so it uses (13,11,10), as being a general reflection with
$h > k > l$.  It finds a suitable asymmetric unit and prints it out.
\pn 
RECIP determines the relationships
imposed on the cell parameters by the symmetry and prints them out.  
In the example, which has hexagonal symmetry, these relationships imply that 
only values of cell sides a and c are needed from the \cd{C} (the value of b
is present on the sample Crystal Data but need not be).  It reads and
interprets the \cd{C}, filling in various useful quantities connected
with the cell parameters. Not all these quantities are printed.  It is
possible to make RECIP print more if more detail is needed.
\pn 
SETGEN then devises a scheme for traversing the whole of the asymmetric
unit.  The argument 0.2 is the maximum value of sin$\theta/\lambda$
required. Again, nothing is printed, but it could be if it is needed.
\pn 
The remaining output is generated by the WRITE statements in the user's
main program.
\pn 
Note that one of the arguments of the CALL of GETGEN is a LOGICAL
variable, NOMORE, which is eventually set to be .TRUE. when there are
no more $h,k,l$ values.  This must be declared LOGICAL at the head of
the main program.  Similarly, because arrays H and K are used, these
must be DIMENSIONed at the start, as H(3), K(3).
\pn 
In more complicated main programs it may be necessary to have access to
variables, such as the cell parameters, used in the Library routines.  Such
variables are held in labelled COMMON blocks.  The cell parameters are in
\pn\ttfamily\begin{verbatim}
      COMMON /CELPAR/CELL(3,3,2),V(2),ORTH(3,3,2),CPARS(6,2),KCPARS(6),
     & CELESD(6,6,2),CELLSD(6,6,2),SDCELL,LSQCEL,PRODSD,KOM4
      LOGICAL SDCELL,PRODSD,LSQCEL
\end{verbatim}\rm
\par 
where the array CELL holds both real and reciprocal space sides and angles.
If the main program wants to refer to CELL, it should declare
COMMON /CELPAR/ at its head, copying it from the Library.
\pn 
The user will see from the discussion above that it will be easier to
read this Manual if he has available a printout of the FORTRAN source
code of the Library, although this is very bulky and a copy of Appendix
A may suffice.
\pn 
There are, of course, certain typical computations which many users will
require often, and these give rise to standard main programs using CCSL.
To the user, such a main program is, for his particular computer, made
into a complete job which he can regard as a `black box'.  He feeds his
data into this box as if he were using a more conventional system of
programs.
\pn 
A new user will probably approach CCSL this way.  It is quite likely,
though, that soon the black box will not do exactly what he wants, and
he will move on to using CCSL in the way it is intended, writing his own
programs, or modifying existing programs, to fit his problem precisely.
\p
% 
\section{A More Elaborate Example}\markright{A More Elaborate Example}
% 
Taking the sample program as a basis, let us expand it to illustrate the
sort of features to expect in CCSL.  Suppose we have data which describe
a crystallographic structure, and we want to generate indices as above,
but then to take an asymmetric unit full of reflections, sort them in
ascending order of $\sin\theta/\lambda$, and list them with calculated
structure factors.
\pn
The expanded program below uses in addition from CCSL:\p
\begin{list} {} {\setlength{\labelwidth}{60mm}
  \setlength{\parsep}{-1ex}
  \setlength{\leftmargin}{\labelwidth}} 
\item[crystallographic routines \hfill] 
\stlink{l}{LATABS}\ - is it a lattice absence?\\
\stlink{f}{FCALC}\  - form a complex structure factor\\
\stlink{i}{ISPABS}\ - is it a space group absence?
\item[utility routines \hfill] 
\stlink{v}{VCTMOD}\ - vector length (for $\sin\theta/\lambda$)\\
\stlink{i}{INDFLO}\ - convert integers into real numbers\\
\stlink{s}{SORTX}\  - sort real numbers in ascending order\\
\stlink{e}{ERRMES}\ - send an error message if necessary
\item[input (\ital{setting-up}) routines \hfill] 
\stlink{a}{ATOPOS}\ - read a set of atomic positions\\
\stlink{s}{SETFOR}\ - read a set of form factors\\
\stlink{s}{SETANI}\ - read anisotropic temperature factors\end{list}
There are more declarations at the head of the program.  The FUNCTION
FCALC is COMPLEX, so both it and the variable to which the program sets
it must be declared COMPLEX.  ISPCE and ISTAR must be declared
CHARACTER *1.
\pn 
ISPABS and LATABS are LOGICAL FUNCTIONs, and must be so declared.
SORTX sorts within arrays in store, so these must be DIMENSIONed. 
\goodbreak \pn
The following, then, is the larger example main program.  It should
by now be fairly clear to the reader what the various parts of it are,
and how they fit together.
\p
\ttfamily\begin{verbatim}
      COMPLEX FC,FCALC
      LOGICAL NOMORE,ISPABS,LATABS
      DIMENSION H(3),K(3,1000),SINTH(1000),IPNT(1000)
      COMMON /IOUNIT/LPT,ITI,ITO,IPLO,LUNI,IOUT
      CHARACTER *1 ISPCE,ISTAR
      DATA ISPCE,ISTAR/' ','*'/
C
      CALL PREFIN('GETSF')
      WRITE (ITO,2002)
2002  FORMAT (' Value of sin theta/lambda max? ')
      READ (ITI,1000) S
1000  FORMAT (F10.4)
      CALL SYMOP
      CALL OPSYM(1)
      CALL OPSYM(2)
      CALL RECIP
      CALL ATOPOS
      CALL SETFOR
      CALL SETANI
      CALL SYMUNI
      WRITE (LPT,2000)
2000  FORMAT (////'  Sorted structure factors - * indicates space',
     1 'group absence:'/'     No.    h   k   l    Mult     s',
     2'            A            B          FcMod')
      CALL SETGEN(S)
C COMPLAIN AND STOP IF THERE WERE ERRORS IN THE INPUT
      CALL ERRMES(0,0,' to GETSF')
      NSUM=0
      N=0
   1  CALL GETGEN(H,NOMORE)
      IF (NOMORE) GO TO 2
      IF (LATABS(H)) GO TO 1
      MULT=MULBOX(H)
      IF (MULT .EQ. 0) GO TO 1
      N=N+1
      NSUM=NSUM+MULT
      SINTH(N)=VCTMOD(0.5,H,2)
      CALL INDFIX(H,K(1,N))
      GO TO 1
C
C SORT THE ARRAY IPNT, SO THAT IT POINTS TO THE ELEMENTS 
C OF SINTH IN SEQUENCE:  
   2  CALL SORTX(SINTH,IPNT,N)
      DO 3 I=1,N
      J=IPNT(I)
      CALL INDFLO(H,K(1,J))
      FC=FCALC(H)
      A=REAL(FC)
      B=AIMAG(FC)
      FCMOD=SQRT(A*A+B*B)
      IC=ISPCE
      IF (ISPABS(H)) IC=ISTAR
      M=MULBOX(H)
      WRITE (LPT,2001) IC,I,(K(L,J),L=1,3),M,SINTH(J),A,B,FCMOD
2001  FORMAT (' ',A1,I5,2X,3I4,2X,I5,F10.5,3X,3F12.5)
   3  CONTINUE
      WRITE (LPT,100)  NSUM,S
 100  FORMAT (/'  Total number of reflections inside sphere',
     1'=',I4/' S max=',F10.4)
      STOP
      END
\end{verbatim}\rm 
\par 
The initial dialogue which has been given as a WRITE and a READ statement for
simplicity could be more elegantly rendered as:
\pn\ttfamily\begin{verbatim}
      CALL ASK('Value of sin theta/lambda max?')
      CALL RDREAL(S,1,IP,80,IE)
\end{verbatim}\rm
\par 
because the routine ASK puts out the given message on the screen, and reads in a
line from the keyboard ready for routine RDREAL to read the value of S.
\pn
Similarly, the WRITE statement with FORMAT 2000 could be replaced by:
\pn\ttfamily\begin{verbatim}
      CALL CENTRE(LPT,4,'Sorted structure factors - * indicates '//
     1 'space group absence;',80)
      CALL MESS(LPT,0,'    No.    h   k   l    Mult     s'//
     2 '            A            B          FcMod')
\end{verbatim}\rm
%
\finchapter
%\end{document}

%\end{htmlonly}
%\internal{c2}
%\internal{c3}
%\internal{c4}
%\internal{c5}
%\internal{c6}
%\internal{c7}
%\startdocument
\label{chap:1}%\hyperlink{chap:1}
\markboth{Introduction to CCSL}{}
This Manual describes the Cambridge Crystallography
Subroutine Library  (``CCSL").
It describes the material in the Master File of the program and is intended
for distribution to users at sites where CCSL is installed.
\pn 
\section{Content of the Manual}\markright{Content of the Manual} 
The Manual is intended both as an introduction to the Library for new
users and as an \ital{aide memoire} for more experienced users.
Some familiarity with the use of FORTRAN is assumed, including the use of a
computer to make and edit files and to compile, link and run FORTRAN jobs.
\pn
% 
\subsection{Scope} 
The manual indicates how to use all members of CCSL except those 
specifically written
for Profile Refinement, for which separate documentation is available.
\pn
% 
\subsection{Changes}
% 
This revision of the Manual describes the system as of \versiondate. 
Inevitably, some of the information will change as the library develops.  New
editions of the Manual will reflect important changes in the system. \pn
% 
\subsection{Content of the chapters} 
This Chapter ends with two sample jobs, followed through in enough
detail to introduce various necessary ideas.  The other chapters describe
different aspects of the Library and need not necessarily be read in
sequence.
\pn
\htmlref{Chapter 2}{chap:2} gives an overall view of the Library.  It should also help the
user to discover what is provided, and to begin to appreciate the system as
a whole.  At the end is a classified list of all the available routines.
\pn
\htmlref{Chapter 3}{chap:3}
 describes the \ital{Crystal Data File},  which contains the
 data describing the crystal structure, essentially, all data other than lists
of reflections,  observations etc.  The bulk of the chapter provides detailed
descriptions of the formats in which these data must be provided.
\pn
\htmlref{Chapter 4}{chap:4}
describes the input of other data, such as lists of 
observations for Least Squares Refinement, etc.  There is also a
section on the creation and use of input/output files.
\pn 
\htmlref{Chapter 5}{chap:5}
introduces the Least Squares Refinement facilities which the system provides,
and explains some of the terminology for users who want to refine
something a little different from what is already available.
\pn 
\htmlref{Chapter 6}{chap:6}
explains how the CCSL deals with magnetic structures.
\pn
\htmlref{Chapter 7}{chap:7}
describes how to run jobs, with reference to operating
systems under which CCSL has been implemented.  A list of some currently
available main programs is given.
\pn  
\subsection{Content of the appendices} 
The Appendices collect together lists of information useful for reference,
to obtain further understanding of the system, or for writing new programs.
They will usually be available separately from the Manual.
\pn 
\iftexforht{\href{../appenx/libapx.html}{Appendix A}}{Appendix A} lists the routines 
alphabetically, with explanation of:\pn
\begin{varindent}{3 cm}prerequisite calls or settings,\\
what is altered or output by the routine,\\
what the routine is used for, and, briefly, how it works.\\
\end{varindent}
\pn 
For each routine Appendix A lists which other
routines it calls, and which  routines call it. It also catalogues the COMMON
blocks that it declares and the items that it accesses from each. This
information was originally provided in Appendix B which is now obsolete.
\p 
\iftexforht{\href{../appenx/appenc.html}{Appendix C}}{Appendix C} lists all the labelled COMMON, with brief descriptions of 
their content. It also gives a list of the symbolic parameters (with their default values), which can be used 
to re-dimension items in these blocks
\p 
\iftexforht{\href{../appenx/maiapx.html}{Appendix D}}{Appendix D} describes some main programs which
are available and how to use them. It also gives information about them similar
to that provided for the library subroutines in Appendix A. 
A new user may at first find what he wants here, but will
probably move on to developing his own main programs from the existing ones.
\pn 
All the material for the Appendices is contained within CCSL itself, and
there are programs available to create the Appendices from the Master File.
\p
\section{An introductory example}\markright{An introductory example}
CCSL is a Library of FORTRAN routines intended to be used for standard
crystallographic calculations.  The routines may be thought of as
tools with which the crystallographer can make his own programs to do
exactly the computations he wants.
\pn
It has been written with the needs of the non-standard crystallographer
(or the solid state physicist working with crystalline materials) in mind.
It is not intended simply for the solution or refinement of ordinary crystal
structures, for which other systems may be more suitable.
\pn
Let us first take a simple but realistic small example.  The main program:
\bgs
%
\ttfamily\begin{verbatim}
      LOGICAL NOMORE
      DIMENSION H(3),K(3)
      COMMON /IOUNIT/LPT,ITI,ITO,IPLO,LUNI,IOUT
      CALL PREFIN('SIMPLE')
      CALL SYMOP
      CALL OPSYM(1)
      CALL SYMUNI
      CALL RECIP
      CALL SETGEN(0.2)
      NSUM=0
      WRITE (LPT,2000)
2000  FORMAT (//'   h     k    l  Multiplicity')
C
   1  CALL GETGEN(H,NOMORE)
      IF (NOMORE) GO TO 2
      MULT=MULBOX(H)
      NSUM=NSUM+MULT
      CALL INDFIX(H,K)
      IF (MULT .NE. 0) WRITE (LPT,2001) K,MULT
 2001 FORMAT(3I5,I8)
      GO TO 1
C
   2  WRITE (LPT,2002)  NSUM
2002  FORMAT (/' Number of reflections inside sphere =',I5)
      STOP
      END
\end{verbatim}\rm 
\bgs 
if supplied with a Crystal Data File named SAMPLE.CRY which says:\\
\ttfamily\begin{verbatim}
C 7.07107   7.07107   8.66025
S     Y,     Y-X,  1/6+Z
S    -Y,     -X,   1/6-Z
\end{verbatim}\rm 
\bgs
\par 
and run as a FORTRAN job, given CCSL as a Library, produces the 
\htlink{output}{output on page~}{simple.lis}{simple:out}.
\p
\section{Explanation of the Example}\markright{Explanation of the Example}
The user will naturally need some knowledge of FORTRAN.  Throughout the manual,
words which are parts of the FORTRAN programs will be written in
capitals, as they usually are in the actual source code.  
\pn 
This main program should be recognisable to the user as FORTRAN, most of
which he can interpret.  The nine names which refer to routines of CCSL
are:
\pn 
   \stlink{p}{PREFIN}, \stlink{s}{SYMOP}, \stlink{s}{SYMUNI}, \stlink{r}{RECIP}\ and 
   \stlink{s}{SETGEN} (all for input), \stlink{o}{OPSYM}\
for symmetry output and
   GETGEN, MULBOX and INDFIX inside a loop generating $h,k,l$ values.
\pn 
The routines  \stlink{g}{GETGEN}\ and \stlink{m}{MULBOX} are of special interest.  
GETGEN fills the
array H with the ``next" set of reflection indices $h,k,l$, eventually
generating all reciprocal lattice vectors in
the whole of an asymmetric unit of reciprocal space.  MULBOX returns the
value of the multiplicity of the reflection with indices $h,k,l$.
\pn
GETGEN is a FORTRAN SUBROUTINE (invoked by the word CALL), whereas MULBOX
is a FORTRAN FUNCTION, in which a variable called MULBOX is set to the
multiplicity.
Every member of the CCSL except for a few BLOCK DATA SUBPROGRAMS 
is either a SUBROUTINE or a FUNCTION ;
for ease of discussion we use the word ``routine" to mean ``either
SUBROUTINE or FUNCTION".
\pn 
GETGEN and MULBOX, then, are examples of \ital{crystallographic}\ routines of
CCSL.  There are a variety of others, such as LOGICAL FUNCTION ISPABS(H)
which determines whether $h,k,l$\/ in array H is or is not a space group absence,
COMPLEX FUNCTION FCALC(H) which calculates a complex Structure Factor,
and SUBROUTINE FOUR1Z which calculates one layer of a Fourier map.
\pn 
\stlink{i}{INDFIX}\ is a small \ital{utility} routine of CCSL which takes the 3 numbers in
the array H and ``fixes" them (converts them from REAL numbers to INTEGERS)
into the integer array K.  Utility routines exist for any operation which
the authors have become tired of writing several times, such as matrix
multiplication, finding a number in a given list, vector products, packing
small integers into bigger ones, etc.
\pn 
Except for OPSYM, the remaining CCSL routines in the example
(PREFIN, SYMOP, RECIP,
SYMUNI and SETGEN) are all involved in the input of data necessary to
perform the later calculations and are examples of \ital{setting-up routines}.
\pn 
\begin{list} {} {\setlength{\labelwidth}{2 cm}
  \setlength{\parsep}{-1ex}
  \setlength{\leftmargin}{\labelwidth}}
\item[\stlink{s}{SYMOP} \hfill] reads and interprets the data cards starting \bd{S} for the space group
symmetry.
\item[\stlink{r}{RECIP} \hfill] reads the \cd{C} for the unit cell parameters.
\item[\stlink{s}{SYMUNI} \hfill] makes a reciprocal space asymmetric unit, using the space group
information.
\item[\stlink{s}{SETGEN} \hfill] sets up the system ready for repeated use of GETGEN.  It needs
to know the space group, the asymmetric unit and the cell parameters,
so SYMOP, SYMUNI and RECIP must all have been obeyed before SETGEN.
\item[\stlink{o}{OPSYM}(1) \hfill] writes out the space group so found, in real space terms. 
 (OPSYM(2) would use reciprocal space).\end{list}
\pn 
This leaves \stlink{p}{PREFIN}.  All CCSL programs which read a Crystal Data File
should CALL PREFIN with the
name of the program as argument at an early
stage.  It initialises various aspects of the system, and puts the crystal
data into a form from which it can subsequently be read and interpreted
as often as required. 
\pn 
The remainder of the main program is ordinary FORTRAN.  Note that
although the Library does the work of producing $h,k,l$ values and
multiplicities, it is the main program which controls what is done with
them and how they are printed. The labelled COMMON /IOUNIT/ has to be
declared in the main program to direct the output of indices 
to the logical unit LPT which has been set up in PREFIN to be a printable file.
\p\pagebreak[8]
{\label{simple:out}
\hyperdef{}{simple.lis}{simple.lis}
\begin{verbatim}

                    * * * S I M P L E * * *

         Cambridge Crystallography Subroutine Library     Mark 4.0

                    Job run at 15:59 on 19-AUG-92

 Crystal data file  SAMPLE.CRY opened

 Data read by PREFIN from file SAMPLE.CRY
        1 card   labelled C
        2 cards  labelled S

 Non-centrosymmetric space group with 12 operator(s)

 The group is generated by the 2 element(s) numbered:  2  7

 Friedel's law NOT assumed - hkl distinct from -h-k-l

 General equivalent positions are:
       0                   0                   0       + 

  1      x                   y                   z       
  2      y                  -x+y                1/6+z    
  3     -x+y                -x                  1/3+z    
  4     -x                  -y                  1/2+z    
  5     -y                   x-y                2/3+z    
  6      x-y                 x                  5/6+z    
  7     -y                  -x                  1/6-z    
  8     -x                  -x+y                1/3-z    
  9      x-y                -y                  -z       
 10     -x+y                 y                  1/2-z    
 11      x                   x-y                5/6-z    
 12      y                   x                  2/3-z    

 Indices  13  11  10 to be used for typical reflection inside asymmetric unit

 Asymmetric unit has  2 plane(s):
             k>=0        
             h>=k        

 Symmetry constraints on lattice parameters    b   =  a  
                                             alpha = 90
                                              beta = 90
                                             gamma = 120

 Real cell          7.0711    7.0711    8.6602   90.00   90.00  120.00
 Volume =   375.0001

 Reciprocal cell    0.1633    0.1633    0.1155   90.00   90.00   60.00
 Volume =   0.2667E-02

 Real cell quadratic products:
 A (=a  sqrd)     B            C   D (=b c cos alpha)  E          F 
     50.00003    50.00003    74.99992     0.00000     0.00000   -25.00002

 Reciprocal cell quadratic products:
 A*(=a* sqrd)     B*          C*  D*(=b*c*cos alpha*)  E*         F*
      0.02667     0.02667     0.01333     0.00000     0.00000     0.01333

 Matrices for transformation of vectors to orthogonal axes
 Real space:
    6.1237    0.0000    0.0000
   -3.5355    7.0711    0.0000
    0.0000    0.0000    8.6602

 Reciprocal space:
    0.1633    0.0816    0.0000
    0.0000    0.1414    0.0000
    0.0000    0.0000    0.1155

   h     k    l  Multiplicity
    1    1    2      12
    1    1    1      12
    1    1    0       6
    1    0    0       6
    2    0    0       6
    0    0   -1       2
    1    0   -1      12
    2    0   -1      12
    0    0   -2       2
    1    0   -2      12
    2    0   -2      12
    0    0   -3       2
    1    0   -3      12

 Number of reflections inside sphere =  108
\end{verbatim}|
\par  
\goodbreak
\section{Comments on the example output}
\markright{Comments on the example output}
The output charts the progress of the program.  PREFIN notes one \cd{C} 
and two \cds{S} (without interpreting the data on them yet).  SYMOP
detects the lack of a centre of symmetry, and counts 12 operators in
total in the group.  In passing it looks for an \bd{I} (``Instructions") card
which could say, e.g.,
\pn\begin{verbatim}
I FRIE  1
\end{verbatim}\rm 
\par 
and would ask to assume Friedel's Law.  It does not find one, and says so.
\pn 
OPSYM(1) lists the operators (this is separate from SYMOP because once
the user is happy with his space group he may well not want to list all
operators for every run).
\pn 
SYMUNI notes that it has not been given a \bd{U} (``Unit") card with typical
$h,k,l$, so it uses (13,11,10), as being a general reflection with
$h > k > l$.  It finds a suitable asymmetric unit and prints it out.
\pn 
RECIP determines the relationships
imposed on the cell parameters by the symmetry and prints them out.  
In the example, which has hexagonal symmetry, these relationships imply that 
only values of cell sides a and c are needed from the \cd{C} (the value of b
is present on the sample Crystal Data but need not be).  It reads and
interprets the \cd{C}, filling in various useful quantities connected
with the cell parameters. Not all these quantities are printed.  It is
possible to make RECIP print more if more detail is needed.
\pn 
SETGEN then devises a scheme for traversing the whole of the asymmetric
unit.  The argument 0.2 is the maximum value of sin$\theta/\lambda$
required. Again, nothing is printed, but it could be if it is needed.
\pn 
The remaining output is generated by the WRITE statements in the user's
main program.
\pn 
Note that one of the arguments of the CALL of GETGEN is a LOGICAL
variable, NOMORE, which is eventually set to be .TRUE. when there are
no more $h,k,l$ values.  This must be declared LOGICAL at the head of
the main program.  Similarly, because arrays H and K are used, these
must be DIMENSIONed at the start, as H(3), K(3).
\pn 
In more complicated main programs it may be necessary to have access to
variables, such as the cell parameters, used in the Library routines.  Such
variables are held in labelled COMMON blocks.  The cell parameters are in
\pn\ttfamily\begin{verbatim}
      COMMON /CELPAR/CELL(3,3,2),V(2),ORTH(3,3,2),CPARS(6,2),KCPARS(6),
     & CELESD(6,6,2),CELLSD(6,6,2),SDCELL,LSQCEL,PRODSD,KOM4
      LOGICAL SDCELL,PRODSD,LSQCEL
\end{verbatim}\rm
\par 
where the array CELL holds both real and reciprocal space sides and angles.
If the main program wants to refer to CELL, it should declare
COMMON /CELPAR/ at its head, copying it from the Library.
\pn 
The user will see from the discussion above that it will be easier to
read this Manual if he has available a printout of the FORTRAN source
code of the Library, although this is very bulky and a copy of Appendix
A may suffice.
\pn 
There are, of course, certain typical computations which many users will
require often, and these give rise to standard main programs using CCSL.
To the user, such a main program is, for his particular computer, made
into a complete job which he can regard as a `black box'.  He feeds his
data into this box as if he were using a more conventional system of
programs.
\pn 
A new user will probably approach CCSL this way.  It is quite likely,
though, that soon the black box will not do exactly what he wants, and
he will move on to using CCSL in the way it is intended, writing his own
programs, or modifying existing programs, to fit his problem precisely.
\p
% 
\section{A More Elaborate Example}\markright{A More Elaborate Example}
% 
Taking the sample program as a basis, let us expand it to illustrate the
sort of features to expect in CCSL.  Suppose we have data which describe
a crystallographic structure, and we want to generate indices as above,
but then to take an asymmetric unit full of reflections, sort them in
ascending order of $\sin\theta/\lambda$, and list them with calculated
structure factors.
\pn
The expanded program below uses in addition from CCSL:\p
\begin{list} {} {\setlength{\labelwidth}{60mm}
  \setlength{\parsep}{-1ex}
  \setlength{\leftmargin}{\labelwidth}} 
\item[crystallographic routines \hfill] 
\stlink{l}{LATABS}\ - is it a lattice absence?\\
\stlink{f}{FCALC}\  - form a complex structure factor\\
\stlink{i}{ISPABS}\ - is it a space group absence?
\item[utility routines \hfill] 
\stlink{v}{VCTMOD}\ - vector length (for $\sin\theta/\lambda$)\\
\stlink{i}{INDFLO}\ - convert integers into real numbers\\
\stlink{s}{SORTX}\  - sort real numbers in ascending order\\
\stlink{e}{ERRMES}\ - send an error message if necessary
\item[input (\ital{setting-up}) routines \hfill] 
\stlink{a}{ATOPOS}\ - read a set of atomic positions\\
\stlink{s}{SETFOR}\ - read a set of form factors\\
\stlink{s}{SETANI}\ - read anisotropic temperature factors\end{list}
There are more declarations at the head of the program.  The FUNCTION
FCALC is COMPLEX, so both it and the variable to which the program sets
it must be declared COMPLEX.  ISPCE and ISTAR must be declared
CHARACTER *1.
\pn 
ISPABS and LATABS are LOGICAL FUNCTIONs, and must be so declared.
SORTX sorts within arrays in store, so these must be DIMENSIONed. 
\goodbreak \pn
The following, then, is the larger example main program.  It should
by now be fairly clear to the reader what the various parts of it are,
and how they fit together.
\p
\ttfamily\begin{verbatim}
      COMPLEX FC,FCALC
      LOGICAL NOMORE,ISPABS,LATABS
      DIMENSION H(3),K(3,1000),SINTH(1000),IPNT(1000)
      COMMON /IOUNIT/LPT,ITI,ITO,IPLO,LUNI,IOUT
      CHARACTER *1 ISPCE,ISTAR
      DATA ISPCE,ISTAR/' ','*'/
C
      CALL PREFIN('GETSF')
      WRITE (ITO,2002)
2002  FORMAT (' Value of sin theta/lambda max? ')
      READ (ITI,1000) S
1000  FORMAT (F10.4)
      CALL SYMOP
      CALL OPSYM(1)
      CALL OPSYM(2)
      CALL RECIP
      CALL ATOPOS
      CALL SETFOR
      CALL SETANI
      CALL SYMUNI
      WRITE (LPT,2000)
2000  FORMAT (////'  Sorted structure factors - * indicates space',
     1 'group absence:'/'     No.    h   k   l    Mult     s',
     2'            A            B          FcMod')
      CALL SETGEN(S)
C COMPLAIN AND STOP IF THERE WERE ERRORS IN THE INPUT
      CALL ERRMES(0,0,' to GETSF')
      NSUM=0
      N=0
   1  CALL GETGEN(H,NOMORE)
      IF (NOMORE) GO TO 2
      IF (LATABS(H)) GO TO 1
      MULT=MULBOX(H)
      IF (MULT .EQ. 0) GO TO 1
      N=N+1
      NSUM=NSUM+MULT
      SINTH(N)=VCTMOD(0.5,H,2)
      CALL INDFIX(H,K(1,N))
      GO TO 1
C
C SORT THE ARRAY IPNT, SO THAT IT POINTS TO THE ELEMENTS 
C OF SINTH IN SEQUENCE:  
   2  CALL SORTX(SINTH,IPNT,N)
      DO 3 I=1,N
      J=IPNT(I)
      CALL INDFLO(H,K(1,J))
      FC=FCALC(H)
      A=REAL(FC)
      B=AIMAG(FC)
      FCMOD=SQRT(A*A+B*B)
      IC=ISPCE
      IF (ISPABS(H)) IC=ISTAR
      M=MULBOX(H)
      WRITE (LPT,2001) IC,I,(K(L,J),L=1,3),M,SINTH(J),A,B,FCMOD
2001  FORMAT (' ',A1,I5,2X,3I4,2X,I5,F10.5,3X,3F12.5)
   3  CONTINUE
      WRITE (LPT,100)  NSUM,S
 100  FORMAT (/'  Total number of reflections inside sphere',
     1'=',I4/' S max=',F10.4)
      STOP
      END
\end{verbatim}\rm 
\par 
The initial dialogue which has been given as a WRITE and a READ statement for
simplicity could be more elegantly rendered as:
\pn\ttfamily\begin{verbatim}
      CALL ASK('Value of sin theta/lambda max?')
      CALL RDREAL(S,1,IP,80,IE)
\end{verbatim}\rm
\par 
because the routine ASK puts out the given message on the screen, and reads in a
line from the keyboard ready for routine RDREAL to read the value of S.
\pn
Similarly, the WRITE statement with FORMAT 2000 could be replaced by:
\pn\ttfamily\begin{verbatim}
      CALL CENTRE(LPT,4,'Sorted structure factors - * indicates '//
     1 'space group absence;',80)
      CALL MESS(LPT,0,'    No.    h   k   l    Mult     s'//
     2 '            A            B          FcMod')
\end{verbatim}\rm
%
\finchapter
%\end{document}
