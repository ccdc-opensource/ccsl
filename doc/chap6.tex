%\documentclass[onecolumn,twoside,11pt,a4paper]{report}
%\usepackage{CCSLman,html,makeidx,color,ifthen,bm}
%\chapter{MAGNETIC STRUCTURES AND STRUCTURE FACTORS}
%\begin{htmlonly}
%\usepackage{CCSLman,html,makeidx,color,ifthen,bm}
%%\documentclass[onecolumn,twoside,11pt,a4paper]{report}
%\usepackage{CCSLman,html,makeidx,color,ifthen,bm}
%\chapter{MAGNETIC STRUCTURES AND STRUCTURE FACTORS}
%\begin{htmlonly}
%\usepackage{CCSLman,html,makeidx,color,ifthen,bm}
%%\documentclass[onecolumn,twoside,11pt,a4paper]{report}
%\usepackage{CCSLman,html,makeidx,color,ifthen,bm}
%\chapter{MAGNETIC STRUCTURES AND STRUCTURE FACTORS}
%\begin{htmlonly}
%\usepackage{CCSLman,html,makeidx,color,ifthen,bm}
%%\documentclass[onecolumn,twoside,11pt,a4paper]{report}
%\usepackage{CCSLman,html,makeidx,color,ifthen,bm}
%\chapter{MAGNETIC STRUCTURES AND STRUCTURE FACTORS}
%\begin{htmlonly}
%\usepackage{CCSLman,html,makeidx,color,ifthen,bm}
%\input{chap6.ptr}
%\end{htmlonly}
%\internal{c2}
%\internal{c3}
%\internal{c4}
%\internal{c5}
%\internal{c1}
%\internal{c7}
\startdocument
%\htmladdtonavigation{\htmladdnormallink
%  {\htmladdimg{../.icons/appenx.gif}}
%  {../appenx/appendix}}
\label{chap:6}
\markboth{Magnetic Structures and Structure Factors}{}
\section{Description of a Magnetic Structure}
\markright{Description of a Magnetic Structure}
In CCSL magnetic structures are described using a propagation vector to define  
the periodicity, and a magnetic space group to define the relative 
orientations of spins on different sub-lattices within the non-magnetic 
unit cell. This approach has the advantage that only the minimum 
information need be given and multiple unit cells are not required.
The propagation vector \pv\ is defined so that
the ordered moment $\Sv$ on a magnetic sublattice is related to that 
${\Sv}_l$ in 
another unit cell at vector distance \lat\space(a lattice vector) from it by
\[{\Sv}_l= {\Sv}\,\mbox{\rm fn}(\pv\cdot\mathbf{l})\]
where $\mbox{\rm fn}(x)$ is a periodic function of $x$ such that  
$\mbox{\rm fn}(x)=\mbox{\rm fn}(n+x)$ for
all integer $n$.\p
The magnetic space group must be congruent with the crystallographic space
group or one of its sub-groups.
Each of the elements of the magnetic group acts on the magnetic moment with 
the rotation and translation appropriate to the corresponding element in the 
crystallographic group.
This may be followed by the operation of time 
inversion in which case the element is \ital{primed};
otherwise it is \ital{unprimed}.\p
If $\tilde R_s$ and ${\bf t}_s$ are the rotation and translation operators
associated with one of the elements in the magnetic group and $\tilde T_s$
the corresponding time reversal operator (1 or $-$1 depending on whether 
time-reversal is invoked) then
magnetic moment 
\Sv\ at vector distance \rv\ from the origin of the unit
cell implies magnetic moment 
\[{\Sv}_s=\tilde T_s\tilde R_s{\Sv}\quad\mbox{at}\quad
\tilde R_s{\bf r}+{\bf t}_s\]
There is one such relationship for each of the elements in the magnetic 
group. One must
remember that magnetic moment is an axial vector so that \ital{improper}
rotations introduce an additional inversion.
\p
The information needed to describe a magnetic structure is given on 
\cds{Q}
which are fully described in \hyperlink{card:Q}{chapter 3}.
\section{Types of Magnetic Order}
\markright{Types of Magnetic Order}
Eight types of magnetic order are recognised. 
They are defined on \hyperlink{card:Q}{\cds{Q STYP}} by 
the words:
\p
\begin{list} {} {\setlength{\labelwidth}{30mm}
  \setlength{\parsep}{-1ex}
  \setlength{\leftmargin}{\labelwidth}
 \addtolength{\leftmargin}{10mm}}
\item[\bd{FERO} \hfill] Ferromagnetic and unmagnetised so all possible 
domains are equally populated.
\item[\bd{FERA} \hfill] Ferromagnetic and magnetised in the direction of the z 
diffractometer axis.
\item[\bd{FANI} \hfill] Anisotropic ferromagnetic the direction of magnetisation is in the plane
containing both the magnetic field direction and the easy axis 
diffractometer axis.
\item[\bd{ANTI} \hfill] A simple commensurate antiferromagnetic structure.
\item[\bd{AMOD} \hfill] A possibly incommensurate amplitude modulated structure; this 
differs from \bd{ANTI} in that the relative phases of the modulation at the 
different symmetrically related sub-lattices may need to be defined.
\item[\bd{HELI} \hfill] A possibly incommensurate helical structure. 
It differs from \bd{AMOD} 
because the magnitude and direction of both spin components which define the 
helical envelope must be defined. They are necessarily perpendicular to one 
another.
\item[\bd{INCM} \hfill] An incommensurate structure  defined by the complex Fourier
components of the magnetisation distribution, one for each partner in the irreduciuble
representation of the magnetic symmetry group. 
\item[\bd{PARA} \hfill] Paramagnetic magnetised in the direction of the z 
diffractometer axis (only recognised by \mlink{chilsq}{CHILSQ}.\end{list}
\p
\section{Symmetry constraints on magnetic parameters}
\markright{Symmetry constraints}
As with other crystallographic parameters such as site multiplicities, anisotropic
temperature factors, etc, CCSL takes care of the constraints imposed by symmetry on
the magnetic parameters. Such constraints may apply to the propagation vector, the
magnetic moment directions and the magnetic moments themselves. The propagation vector
must be invariant under the symmetry operations of the reciprocal lattice, if it is
not, either the cell symmetry is reduced, leading to configuration domains, or the
structure is a multi-q one. Propagation vectors corresponding to special points in the
Brillouin zone are fixed, those on symmetry  axes or planes are constrained so as to
continue to lie on them. The magnetic moments on atoms which are at special positions
are constrained to have the point group symmetries of those positions. For instance,
the magnetic moment of  an atom which lies on an axis of order higher than 2 is
constrained to be parallel to the axis. For an atom on a mirror plane the moment is
parallel to the plane if the reflection operation is combined with time-reversal, and
perpendicular if it is not. Atoms on centres of symmetry which are combined with
time-reversal can carry no moment. The subroutine \stlink{m}{MAGCON}\ determines the
symmetry constraints on the magnetic  moments and prints them on the listing file. If
the value of the magnetic moment is determined to be fixed by symmetry it must be
zero.
\p
\p 
\section{Magnetic Structure Factors}
\markright{Magnetic Structure Factors}
The magnetic structure factor \Mk\ can be defined as 
the Fourier transform of
the magnetisation distribution.\p
The intensity of neutron scattering is proportional to the square
of the component of \Mk\ perpendicular to the scattering
vector \kv. This is the \ital{magnetic interaction\ vector} $\mathbf{Q(k)}$. Thus
\[\Qk=\kv\times\Mk\times\kv\]
Using the description of a magnetic structure just given the unit cell magnetic 
structure factor can be written:
\begin{eqnarray*}
\Mk & = &\frac1N\sum_{n=-\infty}^\infty Q(n) \sum_l^{\mathit{crystal}}
\exp \imath[(\kv+n\pv)\cdot\lat]\\
&\times&\sum_i^{\mathit{magnetic\ atoms}}\ \sum_p^{\mathit{group}}
\ \tilde T_p{\bf M}_{i}
f_i(\kv)\exp \imath[\kv\cdot(\tilde R_p{\bf r}_i+{\bf t}_p)+\phi_i]
\end{eqnarray*}
In this equation 
$\tilde T_p$ and $\tilde R_p$ represent rotations imposed by the elements of the space group on the magnetic 
moments $\Mv_i$ and position vectors $\rv_i$ of the atoms in the unit cell, $f_i(\kv)$ is the magnetic form factor 
and $\phi_i$ the phase factor for the $i$th atom. $Q(n)$ is  the Fourier transform of the
modulating function $\mbox{\rm fn}(x)$. The sum over the lattice vectors is
zero unless $\kv=\rl\pm n\pv$ where \rl\ is a  reciprocal lattice vector. 

The projection of the unit cell magnetic structure factor on the plane perpendicular to the scattering vector is
the  quantity calculated by the magnetic structure factor routines 
\stlink{f}{FMCALC}\ and \stlink{l}{LMCALC}.
\p
For the magnetic structure type INCM the parameters used are the complex fourier components $\Sv_{si}(\qv)$ of the 
 magnetisation distribution for partner $s$ of the relevant irreducible representation. The unit cell magnetic structure factor for partner $s'$ is 
 then
\begin{eqnarray*}&&\Mv(\pv)=\frac1N \sum_{n=-\infty}^\infty Q(n) \sum_l^{\mathit{crystal}}
\exp \imath[(\kv+n\pv)\cdot\lat]\\
\times&&\sum^{partners}_{s}\ \sum^{magnetic\ atoms}_{i}\ \sum^{group}_{p}f_i(k)\sigma_{s'} \tilde {O}_{sip}S_{si}\exp\imath(\pv\cdot(\tilde R_p\rv_i+{\bf t}_p)\pm\phi_{p})
\end{eqnarray*}
$\sigma_s$ is the complex order parameter for partner $s'$ and $\tilde {O}_{sip}$ defines 
the symmetry operation
of the element $p$ of the representation on $\Sv_{si}(\pv)$. The irreducible representation  being used is
 defined by the MSYM card and the  $\tilde {O}_{sip}$ operators by combining this information with the 
 symmetry properties of the propagation vector (subroutines \stlink{p}{PROPER} and \stlink{c}{CPVSYM}). 
For the INCM structure type the magnetic interaction vectors are obtained using  the subroutines
\stlink{c}{CPVCAL}\ and \stlink{l}{LPVCAL}.
\p
At
present CCSL only recognises sinusoidal modulations so that $p(n)=0$ unless
$n=1$. 
\
%
\section{Magnetic Domains}\markright{Magnetic Domains}
Whenever the magnetic symmetry of a structure is less than the symmetry of 
the nuclear structure magnetic domains can occur.
Different parts of 
the crystal conform to one or other of the possible domains.
The different types that occur are: configuration domains, 180\degrees
domains, orientation domains and chirality domains. 
\p
Configuration domains
occur when the propagation vector \pv\ is not invariant under one or 
more of the symmetry operations of the space group. Magnetic structure
factors are calculated for the configuration domain whose propagation
vector is given on the \cd{Q PROP}. If magnetic atoms which were equivalent
under the full symmetry, are found to be inequivalent with the reduced 
configuration symmetry CCSL will raise an error. In this case the symmetry 
cards 
and atomic positions must be adjusted to conform to the configuration symmetry.
\p
180\degrees\ domains occur for all 
structures for which $\pv=0$; the magnetic structure factors for pairs of 
180\degrees\ domains are reversed in direction. 
\p
Orientation domains occur
whenever the magnetic group has symmetry lower than the configurational 
symmetry i.e. when there are one or more \cds{Q NSYM}. 
\p
The subroutines \stlink{f}{FMCALC}\ and \stlink{l}{LMCALC}\  calculate 
magnetic interaction vectors for all the 
orientation domains and store them in COMMON/QCAL/. They also calculate
the mean squared interaction vector averaged over all such orientation
domains assuming equal populations. This is the quantity used in
the \mlink{maglsq}{MAGLSQ}\ magnetic structure refinement with REFI=1 or 2 and printed out as $F_m^2$\ 
by the magnetic structure factor program \mlink{getmsf}{GETMSF}. 
\p
A further type of orientation domain occurs when the magnetic structure cannot be 
described using all the operators of the parent space-group even using NSYM 
operations. This happens when one or more of
the symmetry operators of the space group operates differently on the magnetic moments 
of different magnetic atoms. In this case the magnetic structure factors must be calculated
using the sub-group which omits these operators. The associated domains can be described as {\em
magnetic twin domains} related by twin matrices corresponding to the rotational parts of the missing operators. 
\p
In magnetic structure calculations the contributions from different 
twin domains can be included by using twin matrices given on \cds{R TMAT}.  
 \p
 In cases where chirality domains occur their interaction vectors complex 
conjugate to one another; they are not now (Mark 4.4) included separately in 
/QCAL/.
\subsection{Numbering of magnetic domains}\label{magdoms}
The $N_m$ orientation domains generated by NSYM operations are numbered 
by $n_m$ in the order in which the operators appear in the magnetic symmetry table. Then 
if there are $N_t$ twin matrices given on $N_t$ \cds{R TMAT} numbered $n_t$ with
 ($1 <=n_t<=N_t$) and $N_c$ chirality domains ($N_c= 1$ or 2) numbered $n_c$. Then
 the number given to the domain $n_t,n_m,n_c$ is 
 \[N_d(n_t,n_m,n_c)=N_mN_c(n_t-1) + N_c(n_m-1)+n_c
\]
 If 180\degrees\ domains can be distinguished the numbers given to a pair of such domains 
 differ by $N_tN_mN_c$.

\finchapter
%\end{document}

%\end{htmlonly}
%\internal{c2}
%\internal{c3}
%\internal{c4}
%\internal{c5}
%\internal{c1}
%\internal{c7}
\startdocument
%\htmladdtonavigation{\htmladdnormallink
%  {\htmladdimg{../.icons/appenx.gif}}
%  {../appenx/appendix}}
\label{chap:6}
\markboth{Magnetic Structures and Structure Factors}{}
\section{Description of a Magnetic Structure}
\markright{Description of a Magnetic Structure}
In CCSL magnetic structures are described using a propagation vector to define  
the periodicity, and a magnetic space group to define the relative 
orientations of spins on different sub-lattices within the non-magnetic 
unit cell. This approach has the advantage that only the minimum 
information need be given and multiple unit cells are not required.
The propagation vector \pv\ is defined so that
the ordered moment $\Sv$ on a magnetic sublattice is related to that 
${\Sv}_l$ in 
another unit cell at vector distance \lat\space(a lattice vector) from it by
\[{\Sv}_l= {\Sv}\,\mbox{\rm fn}(\pv\cdot\mathbf{l})\]
where $\mbox{\rm fn}(x)$ is a periodic function of $x$ such that  
$\mbox{\rm fn}(x)=\mbox{\rm fn}(n+x)$ for
all integer $n$.\p
The magnetic space group must be congruent with the crystallographic space
group or one of its sub-groups.
Each of the elements of the magnetic group acts on the magnetic moment with 
the rotation and translation appropriate to the corresponding element in the 
crystallographic group.
This may be followed by the operation of time 
inversion in which case the element is \ital{primed};
otherwise it is \ital{unprimed}.\p
If $\tilde R_s$ and ${\bf t}_s$ are the rotation and translation operators
associated with one of the elements in the magnetic group and $\tilde T_s$
the corresponding time reversal operator (1 or $-$1 depending on whether 
time-reversal is invoked) then
magnetic moment 
\Sv\ at vector distance \rv\ from the origin of the unit
cell implies magnetic moment 
\[{\Sv}_s=\tilde T_s\tilde R_s{\Sv}\quad\mbox{at}\quad
\tilde R_s{\bf r}+{\bf t}_s\]
There is one such relationship for each of the elements in the magnetic 
group. One must
remember that magnetic moment is an axial vector so that \ital{improper}
rotations introduce an additional inversion.
\p
The information needed to describe a magnetic structure is given on 
\cds{Q}
which are fully described in \hyperlink{card:Q}{chapter 3}.
\section{Types of Magnetic Order}
\markright{Types of Magnetic Order}
Eight types of magnetic order are recognised. 
They are defined on \hyperlink{card:Q}{\cds{Q STYP}} by 
the words:
\p
\begin{list} {} {\setlength{\labelwidth}{30mm}
  \setlength{\parsep}{-1ex}
  \setlength{\leftmargin}{\labelwidth}
 \addtolength{\leftmargin}{10mm}}
\item[\bd{FERO} \hfill] Ferromagnetic and unmagnetised so all possible 
domains are equally populated.
\item[\bd{FERA} \hfill] Ferromagnetic and magnetised in the direction of the z 
diffractometer axis.
\item[\bd{FANI} \hfill] Anisotropic ferromagnetic the direction of magnetisation is in the plane
containing both the magnetic field direction and the easy axis 
diffractometer axis.
\item[\bd{ANTI} \hfill] A simple commensurate antiferromagnetic structure.
\item[\bd{AMOD} \hfill] A possibly incommensurate amplitude modulated structure; this 
differs from \bd{ANTI} in that the relative phases of the modulation at the 
different symmetrically related sub-lattices may need to be defined.
\item[\bd{HELI} \hfill] A possibly incommensurate helical structure. 
It differs from \bd{AMOD} 
because the magnitude and direction of both spin components which define the 
helical envelope must be defined. They are necessarily perpendicular to one 
another.
\item[\bd{INCM} \hfill] An incommensurate structure  defined by the complex Fourier
components of the magnetisation distribution, one for each partner in the irreduciuble
representation of the magnetic symmetry group. 
\item[\bd{PARA} \hfill] Paramagnetic magnetised in the direction of the z 
diffractometer axis (only recognised by \mlink{chilsq}{CHILSQ}.\end{list}
\p
\section{Symmetry constraints on magnetic parameters}
\markright{Symmetry constraints}
As with other crystallographic parameters such as site multiplicities, anisotropic
temperature factors, etc, CCSL takes care of the constraints imposed by symmetry on
the magnetic parameters. Such constraints may apply to the propagation vector, the
magnetic moment directions and the magnetic moments themselves. The propagation vector
must be invariant under the symmetry operations of the reciprocal lattice, if it is
not, either the cell symmetry is reduced, leading to configuration domains, or the
structure is a multi-q one. Propagation vectors corresponding to special points in the
Brillouin zone are fixed, those on symmetry  axes or planes are constrained so as to
continue to lie on them. The magnetic moments on atoms which are at special positions
are constrained to have the point group symmetries of those positions. For instance,
the magnetic moment of  an atom which lies on an axis of order higher than 2 is
constrained to be parallel to the axis. For an atom on a mirror plane the moment is
parallel to the plane if the reflection operation is combined with time-reversal, and
perpendicular if it is not. Atoms on centres of symmetry which are combined with
time-reversal can carry no moment. The subroutine \stlink{m}{MAGCON}\ determines the
symmetry constraints on the magnetic  moments and prints them on the listing file. If
the value of the magnetic moment is determined to be fixed by symmetry it must be
zero.
\p
\p 
\section{Magnetic Structure Factors}
\markright{Magnetic Structure Factors}
The magnetic structure factor \Mk\ can be defined as 
the Fourier transform of
the magnetisation distribution.\p
The intensity of neutron scattering is proportional to the square
of the component of \Mk\ perpendicular to the scattering
vector \kv. This is the \ital{magnetic interaction\ vector} $\mathbf{Q(k)}$. Thus
\[\Qk=\kv\times\Mk\times\kv\]
Using the description of a magnetic structure just given the unit cell magnetic 
structure factor can be written:
\begin{eqnarray*}
\Mk & = &\frac1N\sum_{n=-\infty}^\infty Q(n) \sum_l^{\mathit{crystal}}
\exp \imath[(\kv+n\pv)\cdot\lat]\\
&\times&\sum_i^{\mathit{magnetic\ atoms}}\ \sum_p^{\mathit{group}}
\ \tilde T_p{\bf M}_{i}
f_i(\kv)\exp \imath[\kv\cdot(\tilde R_p{\bf r}_i+{\bf t}_p)+\phi_i]
\end{eqnarray*}
In this equation 
$\tilde T_p$ and $\tilde R_p$ represent rotations imposed by the elements of the space group on the magnetic 
moments $\Mv_i$ and position vectors $\rv_i$ of the atoms in the unit cell, $f_i(\kv)$ is the magnetic form factor 
and $\phi_i$ the phase factor for the $i$th atom. $Q(n)$ is  the Fourier transform of the
modulating function $\mbox{\rm fn}(x)$. The sum over the lattice vectors is
zero unless $\kv=\rl\pm n\pv$ where \rl\ is a  reciprocal lattice vector. 

The projection of the unit cell magnetic structure factor on the plane perpendicular to the scattering vector is
the  quantity calculated by the magnetic structure factor routines 
\stlink{f}{FMCALC}\ and \stlink{l}{LMCALC}.
\p
For the magnetic structure type INCM the parameters used are the complex fourier components $\Sv_{si}(\qv)$ of the 
 magnetisation distribution for partner $s$ of the relevant irreducible representation. The unit cell magnetic structure factor for partner $s'$ is 
 then
\begin{eqnarray*}&&\Mv(\pv)=\frac1N \sum_{n=-\infty}^\infty Q(n) \sum_l^{\mathit{crystal}}
\exp \imath[(\kv+n\pv)\cdot\lat]\\
\times&&\sum^{partners}_{s}\ \sum^{magnetic\ atoms}_{i}\ \sum^{group}_{p}f_i(k)\sigma_{s'} \tilde {O}_{sip}S_{si}\exp\imath(\pv\cdot(\tilde R_p\rv_i+{\bf t}_p)\pm\phi_{p})
\end{eqnarray*}
$\sigma_s$ is the complex order parameter for partner $s'$ and $\tilde {O}_{sip}$ defines 
the symmetry operation
of the element $p$ of the representation on $\Sv_{si}(\pv)$. The irreducible representation  being used is
 defined by the MSYM card and the  $\tilde {O}_{sip}$ operators by combining this information with the 
 symmetry properties of the propagation vector (subroutines \stlink{p}{PROPER} and \stlink{c}{CPVSYM}). 
For the INCM structure type the magnetic interaction vectors are obtained using  the subroutines
\stlink{c}{CPVCAL}\ and \stlink{l}{LPVCAL}.
\p
At
present CCSL only recognises sinusoidal modulations so that $p(n)=0$ unless
$n=1$. 
\
%
\section{Magnetic Domains}\markright{Magnetic Domains}
Whenever the magnetic symmetry of a structure is less than the symmetry of 
the nuclear structure magnetic domains can occur.
Different parts of 
the crystal conform to one or other of the possible domains.
The different types that occur are: configuration domains, 180\degrees
domains, orientation domains and chirality domains. 
\p
Configuration domains
occur when the propagation vector \pv\ is not invariant under one or 
more of the symmetry operations of the space group. Magnetic structure
factors are calculated for the configuration domain whose propagation
vector is given on the \cd{Q PROP}. If magnetic atoms which were equivalent
under the full symmetry, are found to be inequivalent with the reduced 
configuration symmetry CCSL will raise an error. In this case the symmetry 
cards 
and atomic positions must be adjusted to conform to the configuration symmetry.
\p
180\degrees\ domains occur for all 
structures for which $\pv=0$; the magnetic structure factors for pairs of 
180\degrees\ domains are reversed in direction. 
\p
Orientation domains occur
whenever the magnetic group has symmetry lower than the configurational 
symmetry i.e. when there are one or more \cds{Q NSYM}. 
\p
The subroutines \stlink{f}{FMCALC}\ and \stlink{l}{LMCALC}\  calculate 
magnetic interaction vectors for all the 
orientation domains and store them in COMMON/QCAL/. They also calculate
the mean squared interaction vector averaged over all such orientation
domains assuming equal populations. This is the quantity used in
the \mlink{maglsq}{MAGLSQ}\ magnetic structure refinement with REFI=1 or 2 and printed out as $F_m^2$\ 
by the magnetic structure factor program \mlink{getmsf}{GETMSF}. 
\p
A further type of orientation domain occurs when the magnetic structure cannot be 
described using all the operators of the parent space-group even using NSYM 
operations. This happens when one or more of
the symmetry operators of the space group operates differently on the magnetic moments 
of different magnetic atoms. In this case the magnetic structure factors must be calculated
using the sub-group which omits these operators. The associated domains can be described as {\em
magnetic twin domains} related by twin matrices corresponding to the rotational parts of the missing operators. 
\p
In magnetic structure calculations the contributions from different 
twin domains can be included by using twin matrices given on \cds{R TMAT}.  
 \p
 In cases where chirality domains occur their interaction vectors complex 
conjugate to one another; they are not now (Mark 4.4) included separately in 
/QCAL/.
\subsection{Numbering of magnetic domains}\label{magdoms}
The $N_m$ orientation domains generated by NSYM operations are numbered 
by $n_m$ in the order in which the operators appear in the magnetic symmetry table. Then 
if there are $N_t$ twin matrices given on $N_t$ \cds{R TMAT} numbered $n_t$ with
 ($1 <=n_t<=N_t$) and $N_c$ chirality domains ($N_c= 1$ or 2) numbered $n_c$. Then
 the number given to the domain $n_t,n_m,n_c$ is 
 \[N_d(n_t,n_m,n_c)=N_mN_c(n_t-1) + N_c(n_m-1)+n_c
\]
 If 180\degrees\ domains can be distinguished the numbers given to a pair of such domains 
 differ by $N_tN_mN_c$.

\finchapter
%\end{document}

%\end{htmlonly}
%\internal{c2}
%\internal{c3}
%\internal{c4}
%\internal{c5}
%\internal{c1}
%\internal{c7}
\startdocument
%\htmladdtonavigation{\htmladdnormallink
%  {\htmladdimg{../.icons/appenx.gif}}
%  {../appenx/appendix}}
\label{chap:6}
\markboth{Magnetic Structures and Structure Factors}{}
\section{Description of a Magnetic Structure}
\markright{Description of a Magnetic Structure}
In CCSL magnetic structures are described using a propagation vector to define  
the periodicity, and a magnetic space group to define the relative 
orientations of spins on different sub-lattices within the non-magnetic 
unit cell. This approach has the advantage that only the minimum 
information need be given and multiple unit cells are not required.
The propagation vector \pv\ is defined so that
the ordered moment $\Sv$ on a magnetic sublattice is related to that 
${\Sv}_l$ in 
another unit cell at vector distance \lat\space(a lattice vector) from it by
\[{\Sv}_l= {\Sv}\,\mbox{\rm fn}(\pv\cdot\mathbf{l})\]
where $\mbox{\rm fn}(x)$ is a periodic function of $x$ such that  
$\mbox{\rm fn}(x)=\mbox{\rm fn}(n+x)$ for
all integer $n$.\p
The magnetic space group must be congruent with the crystallographic space
group or one of its sub-groups.
Each of the elements of the magnetic group acts on the magnetic moment with 
the rotation and translation appropriate to the corresponding element in the 
crystallographic group.
This may be followed by the operation of time 
inversion in which case the element is \ital{primed};
otherwise it is \ital{unprimed}.\p
If $\tilde R_s$ and ${\bf t}_s$ are the rotation and translation operators
associated with one of the elements in the magnetic group and $\tilde T_s$
the corresponding time reversal operator (1 or $-$1 depending on whether 
time-reversal is invoked) then
magnetic moment 
\Sv\ at vector distance \rv\ from the origin of the unit
cell implies magnetic moment 
\[{\Sv}_s=\tilde T_s\tilde R_s{\Sv}\quad\mbox{at}\quad
\tilde R_s{\bf r}+{\bf t}_s\]
There is one such relationship for each of the elements in the magnetic 
group. One must
remember that magnetic moment is an axial vector so that \ital{improper}
rotations introduce an additional inversion.
\p
The information needed to describe a magnetic structure is given on 
\cds{Q}
which are fully described in \hyperlink{card:Q}{chapter 3}.
\section{Types of Magnetic Order}
\markright{Types of Magnetic Order}
Eight types of magnetic order are recognised. 
They are defined on \hyperlink{card:Q}{\cds{Q STYP}} by 
the words:
\p
\begin{list} {} {\setlength{\labelwidth}{30mm}
  \setlength{\parsep}{-1ex}
  \setlength{\leftmargin}{\labelwidth}
 \addtolength{\leftmargin}{10mm}}
\item[\bd{FERO} \hfill] Ferromagnetic and unmagnetised so all possible 
domains are equally populated.
\item[\bd{FERA} \hfill] Ferromagnetic and magnetised in the direction of the z 
diffractometer axis.
\item[\bd{FANI} \hfill] Anisotropic ferromagnetic the direction of magnetisation is in the plane
containing both the magnetic field direction and the easy axis 
diffractometer axis.
\item[\bd{ANTI} \hfill] A simple commensurate antiferromagnetic structure.
\item[\bd{AMOD} \hfill] A possibly incommensurate amplitude modulated structure; this 
differs from \bd{ANTI} in that the relative phases of the modulation at the 
different symmetrically related sub-lattices may need to be defined.
\item[\bd{HELI} \hfill] A possibly incommensurate helical structure. 
It differs from \bd{AMOD} 
because the magnitude and direction of both spin components which define the 
helical envelope must be defined. They are necessarily perpendicular to one 
another.
\item[\bd{INCM} \hfill] An incommensurate structure  defined by the complex Fourier
components of the magnetisation distribution, one for each partner in the irreduciuble
representation of the magnetic symmetry group. 
\item[\bd{PARA} \hfill] Paramagnetic magnetised in the direction of the z 
diffractometer axis (only recognised by \mlink{chilsq}{CHILSQ}.\end{list}
\p
\section{Symmetry constraints on magnetic parameters}
\markright{Symmetry constraints}
As with other crystallographic parameters such as site multiplicities, anisotropic
temperature factors, etc, CCSL takes care of the constraints imposed by symmetry on
the magnetic parameters. Such constraints may apply to the propagation vector, the
magnetic moment directions and the magnetic moments themselves. The propagation vector
must be invariant under the symmetry operations of the reciprocal lattice, if it is
not, either the cell symmetry is reduced, leading to configuration domains, or the
structure is a multi-q one. Propagation vectors corresponding to special points in the
Brillouin zone are fixed, those on symmetry  axes or planes are constrained so as to
continue to lie on them. The magnetic moments on atoms which are at special positions
are constrained to have the point group symmetries of those positions. For instance,
the magnetic moment of  an atom which lies on an axis of order higher than 2 is
constrained to be parallel to the axis. For an atom on a mirror plane the moment is
parallel to the plane if the reflection operation is combined with time-reversal, and
perpendicular if it is not. Atoms on centres of symmetry which are combined with
time-reversal can carry no moment. The subroutine \stlink{m}{MAGCON}\ determines the
symmetry constraints on the magnetic  moments and prints them on the listing file. If
the value of the magnetic moment is determined to be fixed by symmetry it must be
zero.
\p
\p 
\section{Magnetic Structure Factors}
\markright{Magnetic Structure Factors}
The magnetic structure factor \Mk\ can be defined as 
the Fourier transform of
the magnetisation distribution.\p
The intensity of neutron scattering is proportional to the square
of the component of \Mk\ perpendicular to the scattering
vector \kv. This is the \ital{magnetic interaction\ vector} $\mathbf{Q(k)}$. Thus
\[\Qk=\kv\times\Mk\times\kv\]
Using the description of a magnetic structure just given the unit cell magnetic 
structure factor can be written:
\begin{eqnarray*}
\Mk & = &\frac1N\sum_{n=-\infty}^\infty Q(n) \sum_l^{\mathit{crystal}}
\exp \imath[(\kv+n\pv)\cdot\lat]\\
&\times&\sum_i^{\mathit{magnetic\ atoms}}\ \sum_p^{\mathit{group}}
\ \tilde T_p{\bf M}_{i}
f_i(\kv)\exp \imath[\kv\cdot(\tilde R_p{\bf r}_i+{\bf t}_p)+\phi_i]
\end{eqnarray*}
In this equation 
$\tilde T_p$ and $\tilde R_p$ represent rotations imposed by the elements of the space group on the magnetic 
moments $\Mv_i$ and position vectors $\rv_i$ of the atoms in the unit cell, $f_i(\kv)$ is the magnetic form factor 
and $\phi_i$ the phase factor for the $i$th atom. $Q(n)$ is  the Fourier transform of the
modulating function $\mbox{\rm fn}(x)$. The sum over the lattice vectors is
zero unless $\kv=\rl\pm n\pv$ where \rl\ is a  reciprocal lattice vector. 

The projection of the unit cell magnetic structure factor on the plane perpendicular to the scattering vector is
the  quantity calculated by the magnetic structure factor routines 
\stlink{f}{FMCALC}\ and \stlink{l}{LMCALC}.
\p
For the magnetic structure type INCM the parameters used are the complex fourier components $\Sv_{si}(\qv)$ of the 
 magnetisation distribution for partner $s$ of the relevant irreducible representation. The unit cell magnetic structure factor for partner $s'$ is 
 then
\begin{eqnarray*}&&\Mv(\pv)=\frac1N \sum_{n=-\infty}^\infty Q(n) \sum_l^{\mathit{crystal}}
\exp \imath[(\kv+n\pv)\cdot\lat]\\
\times&&\sum^{partners}_{s}\ \sum^{magnetic\ atoms}_{i}\ \sum^{group}_{p}f_i(k)\sigma_{s'} \tilde {O}_{sip}S_{si}\exp\imath(\pv\cdot(\tilde R_p\rv_i+{\bf t}_p)\pm\phi_{p})
\end{eqnarray*}
$\sigma_s$ is the complex order parameter for partner $s'$ and $\tilde {O}_{sip}$ defines 
the symmetry operation
of the element $p$ of the representation on $\Sv_{si}(\pv)$. The irreducible representation  being used is
 defined by the MSYM card and the  $\tilde {O}_{sip}$ operators by combining this information with the 
 symmetry properties of the propagation vector (subroutines \stlink{p}{PROPER} and \stlink{c}{CPVSYM}). 
For the INCM structure type the magnetic interaction vectors are obtained using  the subroutines
\stlink{c}{CPVCAL}\ and \stlink{l}{LPVCAL}.
\p
At
present CCSL only recognises sinusoidal modulations so that $p(n)=0$ unless
$n=1$. 
\
%
\section{Magnetic Domains}\markright{Magnetic Domains}
Whenever the magnetic symmetry of a structure is less than the symmetry of 
the nuclear structure magnetic domains can occur.
Different parts of 
the crystal conform to one or other of the possible domains.
The different types that occur are: configuration domains, 180\degrees
domains, orientation domains and chirality domains. 
\p
Configuration domains
occur when the propagation vector \pv\ is not invariant under one or 
more of the symmetry operations of the space group. Magnetic structure
factors are calculated for the configuration domain whose propagation
vector is given on the \cd{Q PROP}. If magnetic atoms which were equivalent
under the full symmetry, are found to be inequivalent with the reduced 
configuration symmetry CCSL will raise an error. In this case the symmetry 
cards 
and atomic positions must be adjusted to conform to the configuration symmetry.
\p
180\degrees\ domains occur for all 
structures for which $\pv=0$; the magnetic structure factors for pairs of 
180\degrees\ domains are reversed in direction. 
\p
Orientation domains occur
whenever the magnetic group has symmetry lower than the configurational 
symmetry i.e. when there are one or more \cds{Q NSYM}. 
\p
The subroutines \stlink{f}{FMCALC}\ and \stlink{l}{LMCALC}\  calculate 
magnetic interaction vectors for all the 
orientation domains and store them in COMMON/QCAL/. They also calculate
the mean squared interaction vector averaged over all such orientation
domains assuming equal populations. This is the quantity used in
the \mlink{maglsq}{MAGLSQ}\ magnetic structure refinement with REFI=1 or 2 and printed out as $F_m^2$\ 
by the magnetic structure factor program \mlink{getmsf}{GETMSF}. 
\p
A further type of orientation domain occurs when the magnetic structure cannot be 
described using all the operators of the parent space-group even using NSYM 
operations. This happens when one or more of
the symmetry operators of the space group operates differently on the magnetic moments 
of different magnetic atoms. In this case the magnetic structure factors must be calculated
using the sub-group which omits these operators. The associated domains can be described as {\em
magnetic twin domains} related by twin matrices corresponding to the rotational parts of the missing operators. 
\p
In magnetic structure calculations the contributions from different 
twin domains can be included by using twin matrices given on \cds{R TMAT}.  
 \p
 In cases where chirality domains occur their interaction vectors complex 
conjugate to one another; they are not now (Mark 4.4) included separately in 
/QCAL/.
\subsection{Numbering of magnetic domains}\label{magdoms}
The $N_m$ orientation domains generated by NSYM operations are numbered 
by $n_m$ in the order in which the operators appear in the magnetic symmetry table. Then 
if there are $N_t$ twin matrices given on $N_t$ \cds{R TMAT} numbered $n_t$ with
 ($1 <=n_t<=N_t$) and $N_c$ chirality domains ($N_c= 1$ or 2) numbered $n_c$. Then
 the number given to the domain $n_t,n_m,n_c$ is 
 \[N_d(n_t,n_m,n_c)=N_mN_c(n_t-1) + N_c(n_m-1)+n_c
\]
 If 180\degrees\ domains can be distinguished the numbers given to a pair of such domains 
 differ by $N_tN_mN_c$.

\finchapter
%\end{document}

%\end{htmlonly}
%\internal{c2}
%\internal{c3}
%\internal{c4}
%\internal{c5}
%\internal{c1}
%\internal{c7}
\startdocument
%\htmladdtonavigation{\htmladdnormallink
%  {\htmladdimg{../.icons/appenx.gif}}
%  {../appenx/appendix}}
\label{chap:6}
\markboth{Magnetic Structures and Structure Factors}{}
\section{Description of a Magnetic Structure}
\markright{Description of a Magnetic Structure}
In CCSL magnetic structures are described using a propagation vector to define  
the periodicity, and a magnetic space group to define the relative 
orientations of spins on different sub-lattices within the non-magnetic 
unit cell. This approach has the advantage that only the minimum 
information need be given and multiple unit cells are not required.
The propagation vector \pv\ is defined so that
the ordered moment $\Sv$ on a magnetic sublattice is related to that 
${\Sv}_l$ in 
another unit cell at vector distance \lat\space(a lattice vector) from it by
\[{\Sv}_l= {\Sv}\,\mbox{\rm fn}(\pv\cdot\mathbf{l})\]
where $\mbox{\rm fn}(x)$ is a periodic function of $x$ such that  
$\mbox{\rm fn}(x)=\mbox{\rm fn}(n+x)$ for
all integer $n$.\p
The magnetic space group must be congruent with the crystallographic space
group or one of its sub-groups.
Each of the elements of the magnetic group acts on the magnetic moment with 
the rotation and translation appropriate to the corresponding element in the 
crystallographic group.
This may be followed by the operation of time 
inversion in which case the element is \ital{primed};
otherwise it is \ital{unprimed}.\p
If $\tilde R_s$ and ${\bf t}_s$ are the rotation and translation operators
associated with one of the elements in the magnetic group and $\tilde T_s$
the corresponding time reversal operator (1 or $-$1 depending on whether 
time-reversal is invoked) then
magnetic moment 
\Sv\ at vector distance \rv\ from the origin of the unit
cell implies magnetic moment 
\[{\Sv}_s=\tilde T_s\tilde R_s{\Sv}\quad\mbox{at}\quad
\tilde R_s{\bf r}+{\bf t}_s\]
There is one such relationship for each of the elements in the magnetic 
group. One must
remember that magnetic moment is an axial vector so that \ital{improper}
rotations introduce an additional inversion.
\p
The information needed to describe a magnetic structure is given on 
\cds{Q}
which are fully described in \hyperlink{card:Q}{chapter 3}.
\section{Types of Magnetic Order}
\markright{Types of Magnetic Order}
Eight types of magnetic order are recognised. 
They are defined on \hyperlink{card:Q}{\cds{Q STYP}} by 
the words:
\p
\begin{list} {} {\setlength{\labelwidth}{30mm}
  \setlength{\parsep}{-1ex}
  \setlength{\leftmargin}{\labelwidth}
 \addtolength{\leftmargin}{10mm}}
\item[\bd{FERO} \hfill] Ferromagnetic and unmagnetised so all possible 
domains are equally populated.
\item[\bd{FERA} \hfill] Ferromagnetic and magnetised in the direction of the z 
diffractometer axis.
\item[\bd{FANI} \hfill] Anisotropic ferromagnetic the direction of magnetisation is in the plane
containing both the magnetic field direction and the easy axis 
diffractometer axis.
\item[\bd{ANTI} \hfill] A simple commensurate antiferromagnetic structure.
\item[\bd{AMOD} \hfill] A possibly incommensurate amplitude modulated structure; this 
differs from \bd{ANTI} in that the relative phases of the modulation at the 
different symmetrically related sub-lattices may need to be defined.
\item[\bd{HELI} \hfill] A possibly incommensurate helical structure. 
It differs from \bd{AMOD} 
because the magnitude and direction of both spin components which define the 
helical envelope must be defined. They are necessarily perpendicular to one 
another.
\item[\bd{INCM} \hfill] An incommensurate structure  defined by the complex Fourier
components of the magnetisation distribution, one for each partner in the irreduciuble
representation of the magnetic symmetry group. 
\item[\bd{PARA} \hfill] Paramagnetic magnetised in the direction of the z 
diffractometer axis (only recognised by \mlink{chilsq}{CHILSQ}.\end{list}
\p
\section{Symmetry constraints on magnetic parameters}
\markright{Symmetry constraints}
As with other crystallographic parameters such as site multiplicities, anisotropic
temperature factors, etc, CCSL takes care of the constraints imposed by symmetry on
the magnetic parameters. Such constraints may apply to the propagation vector, the
magnetic moment directions and the magnetic moments themselves. The propagation vector
must be invariant under the symmetry operations of the reciprocal lattice, if it is
not, either the cell symmetry is reduced, leading to configuration domains, or the
structure is a multi-q one. Propagation vectors corresponding to special points in the
Brillouin zone are fixed, those on symmetry  axes or planes are constrained so as to
continue to lie on them. The magnetic moments on atoms which are at special positions
are constrained to have the point group symmetries of those positions. For instance,
the magnetic moment of  an atom which lies on an axis of order higher than 2 is
constrained to be parallel to the axis. For an atom on a mirror plane the moment is
parallel to the plane if the reflection operation is combined with time-reversal, and
perpendicular if it is not. Atoms on centres of symmetry which are combined with
time-reversal can carry no moment. The subroutine \stlink{m}{MAGCON}\ determines the
symmetry constraints on the magnetic  moments and prints them on the listing file. If
the value of the magnetic moment is determined to be fixed by symmetry it must be
zero.
\p
\p 
\section{Magnetic Structure Factors}
\markright{Magnetic Structure Factors}
The magnetic structure factor \Mk\ can be defined as 
the Fourier transform of
the magnetisation distribution.\p
The intensity of neutron scattering is proportional to the square
of the component of \Mk\ perpendicular to the scattering
vector \kv. This is the \ital{magnetic interaction\ vector} $\mathbf{Q(k)}$. Thus
\[\Qk=\kv\times\Mk\times\kv\]
Using the description of a magnetic structure just given the unit cell magnetic 
structure factor can be written:
\begin{eqnarray*}
\Mk & = &\frac1N\sum_{n=-\infty}^\infty Q(n) \sum_l^{\mathit{crystal}}
\exp \imath[(\kv+n\pv)\cdot\lat]\\
&\times&\sum_i^{\mathit{magnetic\ atoms}}\ \sum_p^{\mathit{group}}
\ \tilde T_p{\bf M}_{i}
f_i(\kv)\exp \imath[\kv\cdot(\tilde R_p{\bf r}_i+{\bf t}_p)+\phi_i]
\end{eqnarray*}
In this equation 
$\tilde T_p$ and $\tilde R_p$ represent rotations imposed by the elements of the space group on the magnetic 
moments $\Mv_i$ and position vectors $\rv_i$ of the atoms in the unit cell, $f_i(\kv)$ is the magnetic form factor 
and $\phi_i$ the phase factor for the $i$th atom. $Q(n)$ is  the Fourier transform of the
modulating function $\mbox{\rm fn}(x)$. The sum over the lattice vectors is
zero unless $\kv=\rl\pm n\pv$ where \rl\ is a  reciprocal lattice vector. 

The projection of the unit cell magnetic structure factor on the plane perpendicular to the scattering vector is
the  quantity calculated by the magnetic structure factor routines 
\stlink{f}{FMCALC}\ and \stlink{l}{LMCALC}.
\p
For the magnetic structure type INCM the parameters used are the complex fourier components $\Sv_{si}(\qv)$ of the 
 magnetisation distribution for partner $s$ of the relevant irreducible representation. The unit cell magnetic structure factor for partner $s'$ is 
 then
\begin{eqnarray*}&&\Mv(\pv)=\frac1N \sum_{n=-\infty}^\infty Q(n) \sum_l^{\mathit{crystal}}
\exp \imath[(\kv+n\pv)\cdot\lat]\\
\times&&\sum^{partners}_{s}\ \sum^{magnetic\ atoms}_{i}\ \sum^{group}_{p}f_i(k)\sigma_{s'} \tilde {O}_{sip}S_{si}\exp\imath(\pv\cdot(\tilde R_p\rv_i+{\bf t}_p)\pm\phi_{p})
\end{eqnarray*}
$\sigma_s$ is the complex order parameter for partner $s'$ and $\tilde {O}_{sip}$ defines 
the symmetry operation
of the element $p$ of the representation on $\Sv_{si}(\pv)$. The irreducible representation  being used is
 defined by the MSYM card and the  $\tilde {O}_{sip}$ operators by combining this information with the 
 symmetry properties of the propagation vector (subroutines \stlink{p}{PROPER} and \stlink{c}{CPVSYM}). 
For the INCM structure type the magnetic interaction vectors are obtained using  the subroutines
\stlink{c}{CPVCAL}\ and \stlink{l}{LPVCAL}.
\p
At
present CCSL only recognises sinusoidal modulations so that $p(n)=0$ unless
$n=1$. 
\
%
\section{Magnetic Domains}\markright{Magnetic Domains}
Whenever the magnetic symmetry of a structure is less than the symmetry of 
the nuclear structure magnetic domains can occur.
Different parts of 
the crystal conform to one or other of the possible domains.
The different types that occur are: configuration domains, 180\degrees
domains, orientation domains and chirality domains. 
\p
Configuration domains
occur when the propagation vector \pv\ is not invariant under one or 
more of the symmetry operations of the space group. Magnetic structure
factors are calculated for the configuration domain whose propagation
vector is given on the \cd{Q PROP}. If magnetic atoms which were equivalent
under the full symmetry, are found to be inequivalent with the reduced 
configuration symmetry CCSL will raise an error. In this case the symmetry 
cards 
and atomic positions must be adjusted to conform to the configuration symmetry.
\p
180\degrees\ domains occur for all 
structures for which $\pv=0$; the magnetic structure factors for pairs of 
180\degrees\ domains are reversed in direction. 
\p
Orientation domains occur
whenever the magnetic group has symmetry lower than the configurational 
symmetry i.e. when there are one or more \cds{Q NSYM}. 
\p
The subroutines \stlink{f}{FMCALC}\ and \stlink{l}{LMCALC}\  calculate 
magnetic interaction vectors for all the 
orientation domains and store them in COMMON/QCAL/. They also calculate
the mean squared interaction vector averaged over all such orientation
domains assuming equal populations. This is the quantity used in
the \mlink{maglsq}{MAGLSQ}\ magnetic structure refinement with REFI=1 or 2 and printed out as $F_m^2$\ 
by the magnetic structure factor program \mlink{getmsf}{GETMSF}. 
\p
A further type of orientation domain occurs when the magnetic structure cannot be 
described using all the operators of the parent space-group even using NSYM 
operations. This happens when one or more of
the symmetry operators of the space group operates differently on the magnetic moments 
of different magnetic atoms. In this case the magnetic structure factors must be calculated
using the sub-group which omits these operators. The associated domains can be described as {\em
magnetic twin domains} related by twin matrices corresponding to the rotational parts of the missing operators. 
\p
In magnetic structure calculations the contributions from different 
twin domains can be included by using twin matrices given on \cds{R TMAT}.  
 \p
 In cases where chirality domains occur their interaction vectors complex 
conjugate to one another; they are not now (Mark 4.4) included separately in 
/QCAL/.
\subsection{Numbering of magnetic domains}\label{magdoms}
The $N_m$ orientation domains generated by NSYM operations are numbered 
by $n_m$ in the order in which the operators appear in the magnetic symmetry table. Then 
if there are $N_t$ twin matrices given on $N_t$ \cds{R TMAT} numbered $n_t$ with
 ($1 <=n_t<=N_t$) and $N_c$ chirality domains ($N_c= 1$ or 2) numbered $n_c$. Then
 the number given to the domain $n_t,n_m,n_c$ is 
 \[N_d(n_t,n_m,n_c)=N_mN_c(n_t-1) + N_c(n_m-1)+n_c
\]
 If 180\degrees\ domains can be distinguished the numbers given to a pair of such domains 
 differ by $N_tN_mN_c$.

\finchapter
%\end{document}
